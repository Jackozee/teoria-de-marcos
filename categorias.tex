\chapter{Teoría de categorías}
%\section*{(SESIÓN 1: 7 SEP)}

\section{Categorías}
Una categoría  $\mathcal{C}$ consiste de
\begin{itemize}
  \item una colección $\Ob\cal C$, cuyos elementos son llamados los
  objetos de $\cal C$;
  \item para cada par de objetos $A,B\in\Ob\cal C$, una
    colección $\cal C(A,B)$, cuyos elementos se llaman
    \emph{morfismos (o flechas) de $A$ en $B$}, de
    modo que cada $\cal C(A,B)$ es disjunto de los demás;
    es decir; si $A\neq A'$ o si $B\neq B'$, entonces
    $\cal C(A,B)\cap\cal C(A',B')=\emptyset$;
  \item para cualesquiera objetos $A,B,C\in\Ob\cal C$, una
    función
    \begin{align*}
        \circ:\cal C(B,C)\times\cal C(A,B)&\to\cal C(A,C) \\
        (f,g)&\mapsto f\circ g\equiv fg 
    \end{align*}
    denominada 'composición';
    \item para cada objeto $A\in\Ob\cal C$,
    una flecha distinguida $\id_A\in\cal C(A,A)$ llamada
      \emph{la flecha identidad} de $A$;
\end{itemize}
sujetos a las condiciones:
\begin{itemize}
  \item asociatividad: si $f\in\mathcal{C}(A,B),g\in\mathcal{C}(B,C),h\in\mathcal{C}(C,D)$, entonces $(h\circ g)\circ f = h\circ(g\circ f)$.
  \item identidad: para todo $f\in\mathcal{C}(A,B)$, se cumple
  que $f\id_A=f=\id_B f$.
\end{itemize}

\begin{obs}
Dependiendo del autor, el conjunto de morfismos, que aquí
denotamos como $\cal C(A,B)$, se puede escribir como
$\mathrm{Hom}(A,B)$,
$\mathrm{Hom}_{\cal C}(A,B)$, $(A,B)$, $[A,B]$, etc.
Además, pueden exigir que cada $\cal C(A,B)$ sea un conjunto,
pero para los propósitos de estas notas, esta clase puede no ser
cardinable.
\end{obs}

\begin{exa}\leavevmode
\begin{itemize}
    \item $\mathbb{Z}$ puede ser visto como categoría de, al
    menos, dos formas diferentes:
        \begin{itemize}
            \item Tomando $\Ob(\cal Z)=\{\bullet\}$ y
            $\cal
            Z(\bullet,\bullet)=\{\bullet\overset{n}{\to}\bullet\mid
            n\in\Z\}$, tomando la composición como la
            multiplicación usual en $\Z$ (nótese que aquí
            la composición es conmutativa),
            y la identidad es $1\in\Z$ ($1m=m=m1$
            se cumple trivialmente).
            \item Pensemos a $\Z$ con el orden dado por la divisibilidad, $(\Z, \leq_{div})$, donde $n\leq_{div}m$ si y sólo si $m=nk$ para algún $k\in \Z$.
            Para esta categoría, $\Ob(\Z)=\Z$ y
            $\Z(n,m)=\{(n,m)\}$ si y sólo si $n\leq_{div}m$.
        \end{itemize}
        \item En general, para cualquier conjunto parcialmente ordenado $(A,\leq)$ se puede pensar la categoría $\mathcal{C}$ dada por $\Ob\cal C=A$ y 
        \[
        \cal C(a,b) =
        \begin{cases}
            \{(a,b)\}, & m\leq n \\
            \emptyset, & m\nleq n.
        \end{cases}
        \]
        \item Similarmente, cualquier monoide $M$ se puede
        entender como una categoría con un solo objeto $\bullet$
        y definiendo $\cal C(\bullet,\bullet)=M$ y definiendo la
        composición como la operación del monoide.
        \item La categoría de conjuntos $\Con$, donde $\Ob(\Con)$
        consta de todos los conjuntos, $\Con(A,B)$ son las
        funciones de conjuntos $A\to B$ y utilizando la
        composición usual.
        \item Dado un campo $k$, se tiene la categoría $\Vect_k$,
        cuyos objetos son los espacios vectoriales sobre $k$, y
        las flechas son las transformaciones $k-$lineales.
        \item La categoría $\Top$, tomando como objetos los
        espacios topológicos, y las flechas son las funciones
        continuas.
        \item La categoría de campos $\Fld$, cuyos objetos son
        campos y sus morfismos son morfismos de anillos
        (preservan suma, multiplicación y mandan el $1$ al $1$).
        \item La categoría de extensiones de campos.
        $\Fld^\to$.
        Los objetos de $\Fld^\to$ son las
        extensiones de campos, esto es, los morfismos de $\Fld$.
      Dados $(k\xto f L),(k'\xto gL')\in\Ob(\Fld^\to)$,
      los morfismos de $f$ a $g$ en la categoría $\Fld^\to$
      son morfismos de extensiones; esto es:
      pares de morfismos de campos que son compatibles con
      las extensiones:
      \[
        \Fld^\to(f,g)
        = \left\{
          (u,v)\in \Fld(k,k')\times \Fld(L,L')
          \middle|
          \begin{tikzcd}
            L \ar[r,"v"'] & L' \\
            k \ar[u,"f"'] \ar[r,"u"] & k' \ar[u,"g"]
            \ar[equal,from=1-1,to=2-2,shorten=5mm]
          \end{tikzcd}
        \right\}
      .\]
    Dados $f,g,h\in\Ob(\Fld^\to)$, la composición se hereda de la
    composición en $\Fld$:
    \begin{align*}
      \Fld^\to(f,g)\times\Fld^\to(g,h) &\to \Fld^\to(f,h) \\
      ((u,v) , (x,y)) &\mapsto (x,y)(u,v)=(xu,yv).
    \end{align*}
    En un diagrama:
    \[
        \begin{tikzcd}
          L \ar[r,"v"'] \ar[rr,bend left,"yv"] & L' \ar[r,"y"'] & L''\\
          k \ar[u,"f"'] \ar[r,"u"] \ar[rr,bend right,"xu"']
          & k' \ar[u,"g"] \ar[r,"x"] & k''\ar[u,"h"]
          \ar[equal,from=1-1,to=2-2,shorten=5mm]
          \ar[equal,from=1-2,to=2-3,shorten=5mm]
        \end{tikzcd}
    .\]
    La conmutatividad del rectángulo exterior se hereda de la
    conmutatividad de los dos cuadrados:
      \[
          hxu = ygu = yvf
      .\]
    La asociatividad de la composición se hereda de la
    asociatividad de la composición de morfismos de campos
    \[
        \begin{tikzcd}
          L \ar[r,"v"'] & L' \ar[r,"y"'] & L'' \ar[r,"b"'] & L'''\\
          k \ar[u,"f"'] \ar[r,"u"]
          & k' \ar[u,"g"] \ar[r,"x"]
          & k''\ar[u,"h"] \ar[r,"a"]
          & k''' \ar[u,"j"]
          \ar[equal,from=1-1,to=2-2,shorten=5mm]
          \ar[equal,from=1-2,to=2-3,shorten=5mm]
          \ar[equal,from=1-3,to=2-4,shorten=5mm]
        \end{tikzcd}
    .\]
    Esto es:
    \begin{align*}
        ((a,b)(x,y))(u,v)
        &= (ax,by)(u,v) \\
        &= (axu,byv) \\
        &= (a,b)(xu,yv) \\
        &= (a,b)((x,y)(u,v))
    \end{align*}
    La identidad de la extensión $f:k\to L$ es
    $\id_f = (\id_k,\id_L):f\to f$.
    En efecto, dadas $(u,v):f\to g$ y $(c,d):h\to f$, tenemos
    \begin{align*}
        (u,v)(\id_k,\id_L) &= (u \id_k,v \id_L) = (u,v) \\
        (\id_k,\id_L)(c,d) &= (\id_k c,\id_L d) = (c,d).
    \end{align*}
\end{itemize}
\end{exa}


\section{Funtores}
    Para comparar los objetos de una categoría, utilizamos las
    flechas de la categoría, y ahora, además, para comparar
    categorías, definimos el concepto de funtor.
    Por ejemplo, si tenemos una flecha entre espacios topológicos
    , esta flecha es una función continua, lo que significa que
    respeta la estructura,en el sentido de que la premiagen de
    abiertos es abierta; un morfismo de grupos respeta la
    estructura del grupo; un morfismo de anillos respeta la
    estructura del anillo, etc.
    Con esto en mente, tenemos la definición de funtor:
    \begin{defn}[Funtor]
        Para dos categorías $\mathcal{C,D}$, un funtor $F:\mathcal{C\to D}$ actua sobre los objetos y las flechas, de manera que consiste de lo siguiente:
        \item Una asignación 
            \begin{align*}
                Ob(\mathcal{C}) & \longrightarrow Ob(\mathcal{D}) \\
                A & \to F(A)
            \end{align*}
        \item Para cada $A,A'\in \mathcal{C}$, una asignación
            \begin{align*}
                \mathcal{C}(A,A') & \longrightarrow D(FA,FA') \\
                f & \to F(f) \\
            \end{align*}
        Que además cumple que $F(id_A)=id_{FA}$. 
    \end{defn}
    
    \begin{exa}
    \item Dada cualquier categoría $\cal C$, el functor identidad
        $\id_{\cal C}:\cal C\to\cal C$ manda todo objeto a sí mismo y todo
        morfismo a sí mismo.
    \item En la categoría de grupos abelianos $\Z\ds\Mod$, para
    un grupo abeliano $G$,     definimos a $TG$ como el grupo de
    torsión, que consta de todos los elementos de $G$ de orden
    finito. Si $G=TG$ decimos que $G$ es de torsión, y si $TG=0$
    decimos que $G$ es libre de torsión.
    \end{exa}
    
    \begin{exe}%[Armando $\checkmark$]
        Mostrar que la construcción del grupo de torsión
        es un endo-funtor $T:\Z\ds\Mod\to\Z\ds\Mod$.
        \begin{sol}
    Primero, sean $a,b\in T_G$ arbitrarios, entonces, existen $n,m\in\Z^+$ tales que
    \begin{equation*}
        a^n=e,\ b^m=e,\ e\in G \text{ el neutro de } G.
    \end{equation*}
    Si consideramos a $p:=mn$, tenemos que 
    \begin{align*}
        (ab)^p & = a^p b^p \\
               & = (a^n)^m (b^m)^n \\
               & = e^m e^n \\
               & = e \cdot e = e
    \end{align*}
    con $p=mn\in\Z^+$. Ya que $a,b$ son arbitrarios, se sigue que $ab\in T_G$ y en consecuencia $T_G$ es cerrado bajo el producto de $G$.\\
    Ahora, sea $c\in T_G$ arbitrario, entonces existe $k\in\Z^+$ tal que $c^k=e$. Ahora nótese que:
    \begin{align*}
        cc^{-1} & = e \\
        (cc^{-1})^k & = e^k = e \\
        \underbrace{c^k}_{=e}(c^{-1})^k & = e \\
        (c^{-1})^k & = e,\quad k\in\Z^+.
    \end{align*}
    Así, como $c$ es arbitrario, obtenemos que $c^{-1}\in T_G$ y por ende $T_G$ contiene a los inversos.\\
    Con lo anterior, concluimos que $T_G$ es un subgrupo de $G$ y por lo tanto, ya que $G$ es arbitrario, el functor $T$ manda objetos de $\Z-\mathrm{Mod}$ en objetos de $\Z-\mathrm{Mod}$.
    Continuando, para una función $(A\overset{f}{\to}B)\in \Z-\mathrm{Mod}$, nótese que $(T_A\overset{Tf}{\to}T_B)$ se define cómo $Tf=f\mid_{T_A}$, la restricción de $f$ en $T_A$.
    Ahora, para $a\in T_A$ arbitrario, se cumple que $a^n=e$ para algún $n\in\Z^+$. Luego:
    \begin{align*}
        f(a^n) = f(e) & = e \\
        f\left(\underbrace{a\cdot a \cdots a}_{n \text{ veces}}\right) & = e \\
        \text{Como $f\in Ab$ ,} \underbrace{f(a)f(a)\cdots f(a)}_{n\text{ veces}} & = e \\
        \left(f(a)\right)^n & = e\\
    \end{align*}
    Así, $(Tf)(a)=f(a)\in T_B$, y cómo $a$ es arbitrario, se tiene que $Tf:T_A\to T_B$ está bien definido.\\
    Luego, nótese que $(Tf)(e)=f(e)=e$, y se tiene que $e^1 = e$, por tanto, $Tf$ preserva al neutro.\\
    Con esto, $Tf$ es un morfismo de Grupos Abelianos.\\
    Finalmente, es claro que $T1_A=1_{T_A}$, por lo tanto $T$ preserva morfismos, y en conclusión, es un endo-functor de $\Z-\mathrm{Mod}$. 
\end{sol}
    \end{exe}

Adicionalmente, los functores deben respetar la composición de las categorías en una de dos maneras, que determinan la varianza del functor.
\begin{defn}[Functor covariante y contravariante]
    Dados dos objetos $A,B\in\mathcal{C}$, un functor F es covariante si
    \begin{equation*}
        F(A\overset{f}{\to}B)=FA\overset{Ff}{\to}FB
    \end{equation*}
    y es contravariante si
    \begin{equation*}
        F(A\overset{f}{\to}B)=FA\overset{Ff}{\leftarrow}FB
    \end{equation*}
    para todo $f\in\mathcal{C}(A,B)$.
    Con esto, se tiene que, para cualesquiera dos flechas $f,g$ compatibles (que se pueden componer), se tiene que un functor $F$ cumple que:
    \begin{equation*}
        F(gf)=F(g)F(f)
    \end{equation*}
    si es covariante, y 
    \begin{equation*}
        F(gf)=F(f)F(g)
    \end{equation*}
    si es contravariante.
\end{defn}
Finalmente, introducimos unas propiedades adicionales de los functores.
\begin{defn}[Fidelidad]
    Un functor $F$ es fiel si, para cada $A,A'\in\mathcal{C}$ la función 
    \begin{align*}
        \mathcal{C}(A,A') & \to \mathcal{D}(FA,FA')\\
        f & \to F(f)
    \end{align*}
    es inyectiva.
\end{defn}
\begin{defn}[Plenitud]
    Un functor $F$ es pleno si, para cada $A,A'\in\mathcal{C}$ la función 
    \begin{align*}
        \mathcal{C}(A,A') & \to \mathcal{D}(FA,FA')\\
        f & \to F(f)
    \end{align*}
    es suprayectiva.
\end{defn}

\section{Transformaciones naturales}
    Ahora que podemos comparar categorías mediante los functores, queremos herramientas que nos permitan comparar dos functores entre dos categorías. Para cumplir esta idea, se creó el concepto de transformación natural.
    \begin{defn}[Transformación natural]
        Dados dos functores $F,G:\mathcal{C\to D}$, una transformación natural $\alpha:F\to G$ es una familia $(F(A)\overset{\alpha_A}{\to}G(A))_{A\in\mathcal{C}}$ de flechas de $\mathcal{D}$ tal que, para toda flecha $(A\overset{f}{\to}A')\in\mathcal{C}$, se cumple que:
        \begin{equation*}
            G(f)\circ\alpha_A = \alpha_{A'}\circ F(f).
        \end{equation*}
        Aquí, los $(\alpha_\bullet)_{A\in\mathcal{C}}$ se llaman los componentes de la transformación.
    \end{defn}
    Y además, podemos componer transformaciones naturales:\\
    Si $F,G,H:\mathcal{C\to D}$ son functores, y $\alpha:F\to G$,$\beta:G\to H$ son transformaciones naturales entre los functores, entonces, la composición $\beta\alpha:F\to H$ es una transformación natural que va del functor $F$ al functor $H$.
    
    Estas transformaciones se llaman 'naturales' porque precisamente aparecen de forma 'natural' en el ámbito matemático.
    
    Támbien, se denota por $[\mathcal{C,D}]$ a la clase de todas las transformaciones naturales de $\mathcal{C}$ en $\mathcal{D}$, y támbien se le suele llamar la 'exponenciación', y se denota $\mathcal{D^C}$.
    
    \begin{defn}[Isomorfismo natural]
        Se dice que una transformación natural es un isomorfismo natural precisamente cuando es un isomorfismo en $\mathcal{D^C}$.
    \end{defn}
    \begin{exe}%[Yareli $\checkmark$ ]
    (*) Probar que la exponenciación es una categoría.
\end{exe}
\begin{proof}
  Consideramos la clase de los funtores $F\colon C\to D$. Para las categorías $C,D$ fijas.\\
Consideramos las transformaciones naturales como morfismos.\\
Sean $F,G,H,I:C\to D$ funtores y $\alpha\colon F\to G$, $\beta\colon G\to H$ y $\gamma\colon H\to I$ transformaciones naturales. Tenemos que el siguiente diagrama

\[
\begin{tikzcd}
F(A) \arrow[d, "F(f)"'] \arrow[r, "\alpha_A"] & G(A) \arrow[d, "G(f)" description] \arrow[r, "\beta_A"] & H(A) \arrow[d, "H(f)" description] \arrow[r, "\gamma_A"] & I(A) \arrow[d, "I(f)"] \\
F(A') \arrow[r, "\alpha_{A'}"']               & G(A') \arrow[r, "\beta_{A'}"']                          & H(A') \arrow[r, "\gamma_{A'}"']                          & I(A')                 
\end{tikzcd}
\]

conmuta. Es decir, para cada $A\in C$,
\[\alpha_{A'}\circ F(f)=G(f)\circ \alpha_A \qquad \beta_{A'}\circ G(f)=H(f)\circ \beta_A \qquad \gamma_{A'}\circ H(f)=I(f)\circ \gamma_A.\]
\begin{itemize}
\item[\blacksmiley{}] \textit{La composición es una transformación natural.}\\ Definimos $\beta\alpha\colon F\to H$ a la familia $(F(A)\xrightarrow{\beta_A\circ \alpha_A} H(A))$ de $D$.\\
Sea $(A\xrightarrow{f}A')\in C$. Tenemos que
\begin{align*}
\beta_{A'}\circ(\alpha_{A'}\circ F(f))&=\beta_{A'}\circ(G(f)\circ \alpha_A)\\
&=(\beta_{A'}\circ G(f))\circ \alpha_A\\
&=(H(f)\circ\beta_A)\circ\alpha_A.
\end{align*}
Por lo que $(\beta_{A'}\circ\alpha_{A'})\circ F(f)=H(f)\circ(\beta_A\circ\alpha_A)$, es decir, el siguiente diagrama
\[
\begin{tikzcd}
F(A) \arrow[d, "F(f)"'] \arrow[r, "\beta_A\circ \alpha_A"] & H(A) \arrow[d, "H(f)"] \\
F(A') \arrow[r, "\beta_{A'}\circ\alpha_{A'}"']             & H(A')                 
\end{tikzcd}
\]
conmuta.
\item[\blacksmiley{}] \textit{La composición es asociativa.}\\
Sabemos que $\gamma\circ (\beta\circ\alpha)=(\gamma\circ\beta)\circ\alpha$ ya que $\gamma_A\circ(\beta_A\circ\alpha_A)=(\gamma_A\circ\beta_A)\circ\alpha_A$ para cada $(A\xrightarrow{f} A')\in C$.
\item[\blacksmiley{}] \textit{$I_F$ es una transformación natural.}\\
Sea $F\colon C\to D$ un funtor. Definimos $I_F\colon F\to F$ como la familia $(F(A)\xrightarrow{I_A}F(A))\in D$, donde $I_A$ es el morfismo identidad de $F(A)$, para cada $A\in C$.\\
Sea $(A\xrightarrow{f} A')\in D$. Sabemos que $I_{A'}\circ F(f)=F(f)=F(f)\circ I_A$. Por lo que el diagrama
\[
\begin{tikzcd}
F(A) \arrow[d, "F(f)"'] \arrow[r, "I_A"] & F(A) \arrow[d, "F(f)"] \\
F(A') \arrow[r, "I_{A'}"']               & F(A')                 
\end{tikzcd}
\]
conmuta. Es decir, $I_F$ es una transformación natural.
\item[\blacksmiley{}] \textit{Identidad.}\\
Sean $\alpha\colon F\to G$ y $\beta\colon H\to F$ transformaciones lineales, sabemos que $\alpha\circ I_F=\alpha$ y $I_F\circ \beta=\beta$, ya que $\alpha_A\circ I_A=\alpha_A$ y $I_A\circ\beta_A=\beta_A$, para cada $A\in C$.
\end{itemize}
\end{proof}

Además, dados dos functores $F,G\in[\mathcal{C,D}]$, $F(A)\simeq G(A)$ es la naturalidad en $A$ si $F,G$ son naturalmente isomorfos.

Ahora un ejemplo:
Sean $V\in\Vect_k^{<\infty}$ de la categoría de $k-$espacios vectoriales de dimensión finita, y consideramos su espacio dual $V^*=\Hom(V,k)$, esto es una transformación natural $()^*:\Vect_k^{<\infty}\to\Vect_k^{<\infty}$, y además se tiene un iso-natural $V\overset{\alpha_V}{\simeq} V^{**}$, dado por la evaluación en $v$, $v\in V$:
    \begin{align*}
        \alpha_V(v)\in V^{**} & = \Hom(V^*,k) \\
        \alpha_V(v)(\phi) & = \phi(v)
    \end{align*}
Esto determina una transformación natural $\alpha:\id\to(\bullet)^{**}$, que además es un isomorfismo canónico (se llama así ya que esta dado de forma 'no-arbitraria').

Con estos conceptos, se puede construir la definición de categorías equivalentes.

La idea es como sigue: Queremos que $\mathcal{C\equiv D}$ signifique que existen dos functores $F:\mathcal{C\to D}$, $G:\mathcal{C\leftarrow D}$ tales que $GF\equiv 1_\mathcal{C}$ y $FG\equiv 1_\mathcal{D}$. Esto quiere decir que hay isomorfismos naturales $\eta:1_\mathcal{C}\to Gf$ y $\varepsilon: FG\to1_\mathcal{D}$.

\begin{defn}
    Un functor $F\in [\mathcal{C,D}]$ es escencialmente suprayectivo en objetos si, para todo $D\in\mathcal{D}$, existe un $A\in\mathcal{D}$ tal que $F(A)\simeq D$.
\end{defn}

Con esta definición, podemos observar que un functor es una equivalencia si y sólo si es pleno, fiel y esencialmente suprayectivo en objetos.
\begin{exe}%[Alfredo $\checkmark$ ]
    (**) Probar que un funtor es una equivalencia ssi es
      esencialmente suprayectivo y fielmente pleno.
\end{exe}
\begin{sol}
    Primero probaremos el siguiente lema.
    \begin{lemma}
        Si $\alpha:F\to G$ es un isomorfismo natural entre dos funtores
        $F,G:\cal C\to\cal D$, cada componente $\alpha_A:FA\to GA$ de
        $\alpha$ es un isomorfismo en $\cal D$.
    \end{lemma}
    \begin{proof}
        Como $\alpha:F\to G$ es un isomorfismo, existe una trasformación
        natual $\alpha^{-1}:G\to F$ tal que
        \begin{align*}
            \alpha\alpha^{-1} &= \id_G
                & \alpha\alpha^{-1} &= \id_F
        \end{align*}
        Esto significa que, al fijarnos en
        las componentes en cualquier objeto $A$ de $\cal C$, tenemos
        \begin{align*}
                \alpha_A(\alpha^{-1})_A
                = (\alpha\alpha^{-1})_A
                &= (\id_G)_A
                = \id_{GA} \\
                (\alpha^{-1})_A\alpha_A
                =(\alpha^{-1}\alpha)_A
                &= (\id_{F})_A
                = \id_{FA}
        \end{align*}
        Luego, $\alpha_A$ es un isomorfismo con inverso
        $(\alpha_A)^{-1}=(\alpha^{-1})_A$.
        En particular, la notación $\alpha^{-1}_A$ no es ambigua.
    \end{proof}
    Ahora sí, continuamos con el ejercicio.
    Sea $F:\cal C \to \cal D$ una equivalencia.
    Entonces hay otro funtor $G:\cal D\to \cal C$ e isomorfismos
    \begin{align*}
        \epsilon : FG &\simeq \id_{\cal D} & \eta : \id_{\cal C} &\simeq GF.
    \end{align*}
    \begin{itemize}
        \item 
        Primero probaremos que $F$ es fiel.
        Supongamos que $f,g:A\to B$ son morfismos de $\cal C$ tales que
        $Ff = Fg$.
        Aplicando $G$, obtenemos $GFf=GFg$.
        Luego, como $\eta$ es transformación natural, tenemos los siguientes
        diagramas conmutativos
        \[
        \begin{tikzcd}
            GFA \ar[d,"GFf"'] & A \ar[l,"\eta_A"',"\sim"] \ar[d,"f"] \\
            GFB & B \ar[l,"\eta_B","\sim"']
            \com{1-1}{2-2}
        \end{tikzcd}
        \hspace{20mm}
        \begin{tikzcd}
            GFA \ar[d,"GFg"'] & A \ar[l,"\eta_A"',"\sim"] \ar[d,"g"] \\
            GFB & B \ar[l,"\eta_B","\sim"']
            \com{1-1}{2-2}
        \end{tikzcd}
        ,\]
        donde $\eta_A$ y $\eta_B$ son isomorfismos, por el lema que probamos.
        Recordando que $GFf=Gfg$, tenemos
        \begin{align*}
            f
            &= (\eta_B^{-1})(GFf)(\eta_A) \\
            &= (\eta_B^{-1})(GFg)(\eta_A) \\
            &= g,
        \end{align*}
        como se quería.
        
        \item
        Ahora veremos que $F$ es pleno.
        Sea $g:FA\to FB$ un morfismo en $\cal D$.
        Queremos construir un morfismo $f:A\to B$ tal que $Ff=g$.
        Como $G$ es el inverso de $F$ (salvo iso),
        el candidato natural sería $Gg$.
        El problema es que este es un
        morfismo de $GFA$ en $GFB$.
        Podemos intentar arrerglar esto recordando que los componentes de
        la transformación $\eta:\id_{\cal C} \to GF$ son isomorfimos.
        Consideramos la composición $h=(\eta_B^{-1})(Gg)(\eta_A)$:
        \[
        \begin{tikzcd}
            GFA \ar[d,"Gg"'] & A \ar[l,"\eta_A"',"\sim"] 
            \ar[d,"h",dotted]\\
            GFB & B \ar[l,"\eta_B","\sim"']
        \end{tikzcd}
        .\]
        Aplicando $F$, tenemos
        \[
        \begin{tikzcd}
            FA \ar[d,"g"]
                & \ar[l,"\epsilon_{FA}"',"\sim"] FGFA \ar[d,"FGg"']
                & FA \ar[l,"F\eta_A"',"\sim"]  \ar[d,"Fh"]\\
            FB  & FGFB \ar[l,"\epsilon_{FB}","\sim"']
                & FB \ar[l,"F\eta_B","\sim"']
        \end{tikzcd}
        ,\]
        lo cual nos dice que $Fh$ y $g$ difieren por un isomorfismo.
        Tendríamos $Fh=g$ si fuera el caso que
        $(\epsilon_{FA})(F\eta_A)=\id_{FA}$
        y que $(\epsilon_{FB})(F\eta_B)=\id_{FB}$.
        Sin embargo, esto no es cierto, en general.
        
        Para remediar esto, en lugar de tomar $h$, tomamos la composición
        $f$ como
        \[
        \begin{tikzcd}
            GFA \ar[d,"Gg"']
            & GFGFA \ar[l,"GF\eta_A^{-1}"']
            & GFA \ar[l,"G\epsilon_{FA}^{-1}"']
            & A \ar[l,"\eta_A"']
            \ar[d,"f",dotted]\\
            GFB
            & GFGFB \ar[l,"GF\eta_B^{-1}"']
            & GFB \ar[l,"G\epsilon_{FB}^{-1}"']
            & B \ar[l,"\eta_B"']
        \end{tikzcd}
        ,\]
        de modo que, al aplicar $F$, tenemos
        \[
        \begin{tikzcd}
            FA \ar[d,"g"]
            & FGFA \ar[d,"FGg"'] \ar[l,"\epsilon_{FA}"']
            & FGFGFA \ar[l,"FGF\eta_A^{-1}"']
            & FGFA \ar[l,"FG\epsilon_{FA}^{-1}"']
            & FA \ar[l,"F\eta_A"']
            \ar[d,"Ff",dotted]\\
            FB
            & FGFB \ar[l,"\epsilon_{FB}"']
            & FGFGFB \ar[l,"FGF\eta_B^{-1}"']
            & FGFB \ar[l,"FG\epsilon_{FB}^{-1}"']
            & FB \ar[l,"F\eta_B"']
        \end{tikzcd}
        .\]
        La situación puede parecer peor, pero la naturalidad nos salva.
        Agregando arriba y abajo los cuadrados conmutativos que nos
        da la condición de naturalidad, tenemos
        \[
        \begin{tikzcd}
            & FA \ar[equal,dl]
            & FGFA \ar[l,"F\eta_A^{-1}"']
            & FA \ar[l,"\epsilon_{FA}^{-1}"']
            \\
            FA \ar[d,"g"]
            & FGFA \ar[d,"FGg"'] \ar[u,"\epsilon_{FA}"]\ar[l,"\epsilon_{FA}"']
            & FGFGFA \ar[l,"FGF\eta_A^{-1}"'] \ar[u,"\epsilon_{FGFA}"]
            & FGFA \ar[l,"FG\epsilon_{FA}^{-1}"'] \ar[u,"\epsilon_{FA}"]
            & FA \ar[l,"F\eta_A"']
            \ar[d,"Ff",dotted]
            \\
            FB
            & FGFB \ar[l,"\epsilon_{FB}"']\ar[d,"\epsilon_{FB}"']
            & FGFGFB \ar[l,"FGF\eta_B^{-1}"'] \ar[d,"\epsilon_{FGFB}"]
            & FGFB \ar[l,"FG\epsilon_{FB}^{-1}"'] \ar[d,"\epsilon_{FB}"]
            & FB \ar[l,"F\eta_B"']
            \\
            & FB \ar[equal,ul]
            & FGFB \ar[l,"F\eta_B^{-1}"]
            & FB \ar[l,"\epsilon_{FB}^{-1}"]
        \end{tikzcd}
        .\]
        Siguiendo el camino exterior, obtenemos que
        \[
            (F\eta_B^{-1})(\epsilon_{FB}^{-1})(\epsilon_{FB})(F\eta_B)Ff
            = g(F\eta_A^{-1})(\epsilon_{FA}^{-1})(\epsilon_{FA})(F\eta_A)
        .\]
        Es decir, $Ff=g$.
        
        \item
        Con el lema que probamos, es fácil ver que $F$ es
        esencialmente suprayectivo.
        En efecto, 
        para cualquier objeto $B$ de $\cal D$, el componente
        $\epsilon_B:FGB\to B$ de $\epsilon:FG\to\id_{\cal D}$ es un
        isomorfismo, así que $B$ es isomorfo a un objeto en la
        imagen de $F$.
    \end{itemize}
    Ahora la otra implicación.
    Supongamos que $F:\cal C\to\cal D$ es fielmente pleno
    y esencialmente suprayectivo.
    Queremos definir un funtor $G:\cal C\to \cal D$ que haga de
    inverso de $F$ (salvo iso).
    Como $F$ es esencialmente suprayectivo, el axioma de
    elección nos permite elegir, para cada objeto $B$ de $\cal D$,
    un objeto $GB$ de $\cal C$ y un isomorfismo
    $\epsilon_B:FGB\xrightarrow{\sim} B$.
    
    Falta definir la acción de $G$ en morfismos.
    Dado $g:B_1\to B_2$ en $\cal D$, definimos
    $f=(\epsilon_{B_2})^{-1}g(\epsilon_{B_1}):FGB_1\to FGB_2$, de tal
    modo que
    \[
        \begin{tikzcd}
            B_1 \ar[d,"g"']
            & FGB_1 \ar[l,"\epsilon_{B_1}"'] \ar[d,"f",dotted] \\
            B_2 & FGB_2 \ar[l,"\epsilon_{B_2}"]
            \com{1-1}{2-2}
        \end{tikzcd}
    .\]
    Como $F$ es fielmente pleno, podemos definir a $Gg:GB_1\to GB_2$
    como el único morfismo que satisface $FGg=f:FGB_1\to FGB_2$.
    
    Veremos que $G$ es un funtor.
    Si tomamos morfismos
    \[
        \begin{tikzcd}
            B_1 \ar[d,"g"'] \\
            B_2 \ar[d,"h"'] \\
            B_3
        \end{tikzcd}
    \]
    entonces, por definición, $G(hg)$, $Gh$ y $Gg$
    son los únicos morfismos tales que los diagramas
    \[
        \begin{tikzcd}
            B_1 \ar[d,"g"'] & FGB_1 \ar[l,"\epsilon_{B_1}"'] \ar[dd,"FG(hg)"] \\
            B_2 \ar[d,"h"'] \\
            B_3 & FGB_2 \ar[l,"\epsilon_{B_3}"']
            \com{1-1}{3-2}
        \end{tikzcd}
        \hspace{10mm}
        \begin{tikzcd}
            B_1 \ar[d,"g"'] & FGB_1 \ar[l,"\epsilon_{B_1}"'] \ar[d,"FGg"] \\
            B_2 \ar[d,"h"'] & FGB_2 \ar[l,"\epsilon_{B_2}"'] \ar[d,"FGh"] \\
            B_3 & FGB_2 \ar[l,"\epsilon_{B_3}"']
            \com{1-1}{2-2} \com{2-1}{3-2}
        \end{tikzcd}
    .\]
    Se sigue que $FG(hg)=(FGh)(FGg)$.
    Por funtorialidad de $F$, esto es $FG(hg)=F((Gh)(Gg))$.
    Luego, como $F$ es fiel, tenemos $G(hg)=(Gh)(Gg)$.
    
    Por otro lado, tomando el morfismo identidad
    \[
        \begin{tikzcd}
            B \ar[d,"\id_B"'] \\
            B
        \end{tikzcd}
    \]
    tenemos que $G\id_B$ es el único morfismo $GB\to GB$ que hace conmutar
    el diagrama
    \[
        \begin{tikzcd}
            B \ar[d,"\id_B"'] & FGB \ar[l,"\epsilon_B"'] \ar[d,"FG\id_B"] \\
            B & FGB \ar[l,"\epsilon_B"']
            \com{1-1}{2-2}
        \end{tikzcd}
    \]
    Por lo tanto, $FG\id_B=\id_{FGB}=F\id_{GB}$.
    Como $F$ es fiel, esto implica que $G\id_B =\id_{GB}$.
    
    Por definición de la acción de $G$ en morfismos, para cualquier morfismo
    $g:B_1\to B_2$ en $\cal D$ el diagrama
    \[
        \begin{tikzcd}
            B_1 \ar[d,"g"']
            & FGB_1 \ar[l,"\epsilon_{B_1}"'] \ar[d,"FGg"] \\
            B_2 & FGB_2 \ar[l,"\epsilon_{B_2}"]
            \com{1-1}{2-2}
        \end{tikzcd}
    \]
    es conmutativo.
    Esto significa que la familia de morfismos $(\epsilon_B:FGB\to B)_{B\in\Ob\cal D}$
    es una transformación natural
    \[
        \epsilon : FG\to \id_{\cal D}
    .\]
    Como cada $\epsilon_B$ es un isomorfismo y el diagrama anterior
    es conmutativo, se sigue que $\epsilon$ es un
    isomorfismo natural, cuya inversa
    $\epsilon:\id_{\cal D}\to FG$ tiene componentes dadas por
    \[
        (\epsilon^{-1})_B = (\epsilon_B)^{-1}:B\to FGB
    \]
    para cada objeto $B$ de $\cal D$.
    
    Resta construir un isomorfismo natural $\eta:\id_{\cal C}\to GF$.
    Sea $A$ un objeto de $\cal C$.
    Como $F$ es fielmente pleno y $\epsilon:FG\to\id_{\cal D}$ es un
    isomorfismo, podemos definir $\eta_A$ como el único morfismo
    $\eta_A:A\to GFA$ tal que $F\eta_A=\epsilon_{FA}^{-1}:FA\to FGFA$.
    
    Dado que $F\eta_A$ es un isomorfismo (con inverso $\epsilon_{FA}$),
    se sigue que cada $\eta_A$ es un isomorfismo, cuyo inverso $\eta_A^{-1}$ es el
    único morfismo $\eta_A:A\to GFA$ tal que $F\eta_A^{-1}=\epsilon_{FA}:FGFA\to FA$.
    En efecto, si $f:GFA\to A$ es tal que $Ff=\epsilon$, entonces
    \begin{align*}
        F(\eta_Af)
        &=(F\eta_A)(Ff)=\epsilon_{FA}^{-1}\epsilon_{FA}=\id_{FGFA}=F\id_{GFA} \\
        F(f\eta_A)
        &=(Ff)(F\eta_A)=\epsilon_{FA}\epsilon_{FA}^{-1}=\id_{FA}=F\id_A
    \end{align*}
    así que $\eta_Af=\id_{GFA}$ y $f\eta_A=\id_A$, pues $F$ es fiel,
    por lo cual $f=\eta_A^{-1}$.
    
    Finalmente, observemos que $\eta$ es una transformación natural.
    En efecto, para cualquier morfismo $f:A_1\to A_2$ en $\cal C$,
    $Ff:FA_1\to FA_2$ es un morfismo en $\cal D$, por lo cual el diagrama
    \[
        \begin{tikzcd}
            FA_1 \ar[d,"Ff"']
            & FGFA_1 \ar[l,"\epsilon_{FA_1}"'] \ar[d,"FGFf"] \\
            FA_2 & FGFA_2 \ar[l,"\epsilon_{FA_2}"]
            \com{1-1}{2-2}
        \end{tikzcd}
    \]
    es conmutativo.
    Como observamos antes, $F\eta_A^{-1}=\epsilon_A$ para cualquier $A$, así
    que esto es
    \[
        \begin{tikzcd}
            FA_1 \ar[d,"Ff"']
            & FGFA_1 \ar[l,"F\eta_{A_1}^{-1}"'] \ar[d,"FGFf"] \\
            FA_2 & FGFA_2 \ar[l,"F\eta_{A_2}^{-1}"]
            \com{1-1}{2-2}
        \end{tikzcd}
    .\]
    Es decir,
    \[
        F(f\eta_{A_1}^{-1})
        =(Ff)(F\eta_{A_1}^{-1})
        =(F\eta_{A_2}^{-1})(FGFf)
        =F(\eta_{A_2}^{-1}GFf)
    ,\]
    de modo que $f\eta_{A_1}^{-1}=\eta_{A_2}^{-1}GFf$, pues $F$ es fiel.
    Luego, $\eta_{A_2}f=(GFf)\eta_{A_1}$.
    Es decir, el diagrama
    \[
        \begin{tikzcd}
            A_1 \ar[d,"f"'] \ar[r,"\eta_{A_1}"]
            & GFA_1 \ar[d,"GFf"] \\
            A_2 \ar[r,"\eta_{A_2}"'] & GFA_2
            \com{1-2}{2-1}
        \end{tikzcd}
    \]
    es conmutativo.
    Esta es la condición de naturalidad.
\end{sol}

\section{Adjunciones}
\label{ss:adjunciones}

En general, dadas dos categorías y dos functores entre ellas, es muy díficil saber a priori cuando estos functores forman una equivalencia, por eso se buscó las condiciones mínimas que se pueden exigir para tratar a dichas categorías 'como si fueran equivalentes'. De esta idea surgió el concepto de adjunción.
\begin{defn}
    Dadas dos categorías $\mathcal{C,D}$ y dos functores $F:\mathcal{C\to D}$, $G:\mathcal{C\to D}$, diremos que $F$ es el adjunto izquierdo de $G$, y que $G$ es el adjunto derecho de $F$, denotado $F\dashv G$, si $\mathcal{D}(F(A),B)\simeq \mathcal{C}(A,G(B))$, y esto es natural en $A$ y en $B$.
    Es decir, para cualesquiera morfismos $f:A'\to A$
    en $\cal C$ y $g:B\to B'$ en $\cal D$,
    el siguiente diagrama es conmutativo:
    \[
        \begin{tikzcd}
            \cal D(FA,B)
                \ar[d,"g\circ-\circ Ff"']
                \ar[r,"\sim",shift left]
            & \cal C(A,GB) 
                \ar[d,"Gg\circ-\circ f"]
                \ar[l,"\sim",shift left] \\
            \cal D(FA',B')
                \ar[r,"\sim",shift left]
            & \cal C(A',GB')
                \ar[l,"\sim",shift left]
        \end{tikzcd}
    \]
\end{defn}
La definición anterior quiere decir que hay un iso natural entre los functores:

Si $A\in\mathcal{C}, B\in\mathcal{D}$,
entonces existe una correspondencia biyectiva entre las flechas
$FA\to B$ y $A\to GB$.
Es decir, a cualesquiera morfismos
$(FA\xto p B)\in\cal{D}$
y $(A\xto q GB)\in\mathcal{C}$
les corresponden unas únicas flechas
$(A\xto{\Bar{p}}GB)$
y $(FA\xto{\Bar{q}}B)$,
respectivamente,
tales que $\Bar{\Bar{p}}=p$ y $\Bar{\Bar{q}}=q$.
Esto nos lleva al axioma de naturalidad,
el cual es equivalente a la conmutatividad
del diagrama de arriba.

\begin{axiom}[De naturalidad]
    \leavevmode
    \begin{enumerate}
        \item
        Poniendo $A=A'$ y $f=\id_A:A\to A$ en el diagrama,
        obtenemos la condición de que,
        para cualquier morfismo $p:FA\to B$,
        se tenga $(Gg)\bar p=\ol{gp}$:
        \[
            \Big(GB' \lar{Gg} GB \lar{\bar p} A\Big)
            =
            \ol{\Big(B' \lar g B \lar p FA \Big)}
        .\]
        \item
        Poniendo $B=B'$ y $g=\id_B:B\to B$ en el diagrama,
        obtenemos la condición de que,
        para cualquier morfismo $q:A\to GB$,
        se tenga $\bar q(Ff)=\ol{qf}$:
        \[
            \Big(B\lar{\bar q} FA\lar{Ff}FA'\Big)
            =
            \ol{\Big(GB\lar q A\lar f A'\Big)}
        .\]
    \end{enumerate}
\end{axiom}

\begin{exa}
\begin{itemize}
    \item En álgebra surge mucho el ejemplo $free\dashv forget$. Tomando las categorias $\Vect_k$ y $\Con$, tenemos el functor de olvidar $u:\Vect_k\to\Con$, definido por $u((V,+,\cdot))= V$ (olvida la estructura del espacio y lo considera como conjunto), y el functor libre $F:\Con\to\Vect_k$, definido como $F(U)$ es el espacio libre generado por $U$, y estos functores forman una adjunción.
    
    Para mostrar esto, tomamos $S\in\Con$, $V\in\Vect_k$, y la flecha $g:F(S)\to V$.
    Definamos $\Bar{g}:S\to u(V)=V$, como $\Bar{g}(s)=g(s)$ para cada $s\in S$, lo que implica que tenemos la flecha $\Vect_k(FS,V)\to\Con(S,uV)$ que manda cada $g$ a $\Bar{g}$.
    Ahora sea $f\in\Con(S,uV)$. Entonces definimos $\Bar{f}$:
    \begin{align*}
        \Bar{f}:FS & \to V \\
        \Bar{f}\left(\sum_{s\in S}\lambda_ss\right)=\sum_{s\in S}\lambda_s f(s)
    \end{align*}
    \item Un caso similar se da con grupos ('mismo' functor de olvidar, y el functor que asigna el grupo libre del conjunto).
    \item El funtor del espacio vectorial libre es el
      adjunto izquierdo del de olvidar.
      Tenemos el funtor olvidadizo $\Vect_k\to\Con$ que para cada
      espacio vectorial $V$ “olvida” su estructura y le asocia el
      conjunto subyacente. A cada conjunto $X$ se puede asociar
      el espacio vectorial $k\langle X\rangle$ generado por $X$;
      es decir, el espacio cuya base corresponde a los elementos
      de $X$. En este caso toda aplicación lineal $f:k\langle
      X\rangle\to V$ se define de manera única por los imágenes
      de los elementos de la base.
      \[\begin{tikzcd}
    	{X} & {k\langle X\rangle} \\
    	{V}
    	\arrow["{f}"', from=1-1, to=2-1]
    	\arrow[from=1-1, to=1-2, hook]
    	\arrow["{\exists !}"', from=2-1, to=1-2, dotted]
    \end{tikzcd}\]
    Esto nos da una biyección natural
    $$\Vect_k(k\<X\>, V)
    \cong
    \Con(X,V)$$
    Luego, el funtor $k\<-\>$ es adjunto izquierdo del
    funtor olvidar $\Vect_k\to\Con$.
    
    \item
    Tomando los grupos abelianos $\Z-\mathrm{Mod}$, existe una adjunción con $\Grp$, $F\dashv u$, donde $u$ es el functor de inclusión (olvida que es abeliano), y $F$ asigna a cada grupo $G$ su respectiva 'abelianización', dada por $G/[G,G]$,
    donde $[G,G]=\langle xyx^{-1}y^{-1}\mid x,y\in G\rangle$
    es el subgrupo conmutador.
    \item Sean $A,B,C\in\Con$, y consideremos $f\in\Con(A\times B, C)$, entonces
    \begin{align*}
        f: A\times B & \to C \\
        (a,b) & \to f(a,b)
    \end{align*}
    Si fijamos $a$, consideramos la función:
    \begin{align*}
        f_A: A & \to \Con(B,C)\\
        a & \to f_A(a): B\to C \\
        &\quad = f(a,-)(b)=f(a,b)\in C
    \end{align*}
    Es decir, podemos definir una función de conjuntos
    \begin{align*}
        \Con(A\times B, C) & \xto{\widehat{(-)}} \Con(A,\Con(B,C)) \\
        f&\mapsto \hat{f_{\bullet}}
    \end{align*}
    y recíprocamente, tomamos una función $f:A\to\Con(B,C)$, entonces queremos definir una función
    \begin{align*}
        \Bar{f}:A\times B & \to C \\
        \Bar{f}(a,b) &= f(a)(b)
    \end{align*}
    Así, definimos la función 
    \begin{align*}
        \Con(A,\Con(B,C))
        & \xto{\ol{(-)}}
        \Con(A\times B, C)
    \end{align*}
    y notemos que $\ol{(-)}$ y $\widehat{(-)}$ son inversas una de la otra, por tanto son una biyección, y tenemos una adjunción dada por
    \begin{equation*}
        \Con(A\times B, C)\simeq\Con(A,\Con(B,C))
    \end{equation*}
    donde $\Con(C,B)=:C^B$ se le llama la exponenciación de $C$ en $B$, y los functores son
    \begin{equation*}
        {-}\times B:\Con\to \Con
    \end{equation*}
    y el otro functor es 'exponenciar':
    \begin{equation*}
        ({-})^B:\Con\to \Con
    \end{equation*}
\end{itemize}
\end{exa}
    
Recordemos que cuando dos categorías tiene una adjunción:
\begin{center}
\begin{tikzcd}
    \mathcal{A} \arrow[dd, "F"', bend right] \\
    \dashv \\
    \mathcal{B} \arrow[uu, "G"', bend right]
\end{tikzcd}
\end{center}
Se cumple que $\mathcal{B}(FA,B)\simeq\mathcal{A}(A,GB)$, y supongamos que tenemos la configuración:
\begin{equation*}
    FA\overset{g}{\longrightarrow}B\overset{q}{\longrightarrow}B'
\end{equation*}
entonces, la composición $qg$ le corresponde una única flecha:
\begin{equation*}
    \overline{qg}:A\longrightarrow GB'
\end{equation*}
o támbien podemos considerar solamente la única flecha de $g$:
\begin{equation*}
    \overline{g}:A\longrightarrow GB
\end{equation*}
y luego componerla con $G(q):GB\to GB'$ para obtener una nueva flecha
\begin{center}
    \begin{tikzcd}
A \arrow[r, "\overline{g}"] \arrow[rd, "G(q)\circ\overline{g}"', bend right] & GB \arrow[d, "G(q)"] \\
                                                                             & GB'                 
\end{tikzcd}
\end{center}
y la compatibilidad dice que deberiamos obtener la misma flecha, es decir $\overline{qg}=G(q)\circ\overline{g}$.
Ahora, nótese que para cualquier $A\in\mathcal{A}$, se tiene una flecha:
\begin{equation*}
    A\underset{\eta_A}{\longrightarrow}GFA
\end{equation*}
y por lo anterior, esta flecha es la que le corresponde a la identidad, $\eta_A=\overline{1_{FA}}$, y tenemos el caso para  cada $B\in\mathcal{B}$, es decir
\begin{align*}
    \panth{A\underset{\eta_A}{\longrightarrow}GFA} & = \overline{\panth{F(A)\overset{1_{FA}}{\longrightarrow}F(A)}} \\
    \panth{FGB\underset{\varepsilon_B}{\longrightarrow}B} & = \overline{\panth{G(B)\overset{1_{GB}}{\longrightarrow}G(B)}}
\end{align*}
entonces, esto define las transformaciones naturales en cada componente:
\begin{align*}
    \eta_\bullet: 1_\mathcal{A}\longrightarrow GF \\
    \varepsilon_\bullet: FG\longrightarrow 1_\mathcal{B}
\end{align*}
que son, respectivamente, la \textbf{unidad} y la \textbf{co-unidad} de adjunción.

Por ejemplo, tomemos la adjunción:
\begin{center}
    \begin{tikzcd}
\mathrm{Vect}_k \arrow[dd, "u"', bend right] \\
\vdash                               \\
\mathrm{Set} \arrow[uu, "F"', bend right]   
\end{tikzcd}
\end{center}
del functor que olvida $u$ y el functor de la realización libre $F$.
Aquí, la unidad y la co-unidad son:
\begin{align*}
    & \eta_S : S\longrightarrow uF(S)=\pangle{\sum\lambda_s s\mid s\in S} \\
    & \varepsilon_V: Fu(V)\longrightarrow V.
\end{align*}

Ahora, consideremos una adjunción $F\dashv G$ entre categorías, con su unidad y co-unidad $\eta,\varepsilon$, entonces los diagramas
\[
\begin{tikzcd}
    F \arrow[r, "F\eta"] \arrow[rd, "1_F"'] & FGF \arrow[d, "\varepsilon F"] & G \arrow[r, "\eta G"] \arrow[rd, "1_G"'] & GFG \arrow[d, "G\varepsilon"] \\
     & F & & G                            
\end{tikzcd}
\]
tienen que conmutar, pues si evaluamos la unidad en cada objeto $\panth{A\overset{\eta_A}{\to}GFA}\in\cal{A}$ y le aplicamos $F$, tenemos $FA\overset{F\eta_A}{\to}FGFA$.
Por otro lado, buscando al transpuesto de $GFA\overset{1_{GFA}}{\to}GFA$, obtenemos
\begin{align*}
    \overline{1_{GFA}}: FGFA\to FA
\end{align*}
pero esto es precisamente la co-unidad $\eta_{FA}$, y al componerlo con $F\eta_A$
\begin{center}
    \begin{tikzcd}
FA \arrow[r, "F\eta_A"] \arrow[rd, bend right] & FGFA \arrow[d, "\varepsilon_{FA}"] \\
                                               & FA                                
\end{tikzcd}
\end{center}
Entonces, por definición de transposiciones, obtenemos que $\overline{\eta_A}=\overline{\overline{1_{FA}}} = 1_{FA}$, y la conmutatividad del otro diagrama se prueba de manera análoga.

Esto nos dice que, para cualquier adjunción 
\begin{center}
    \begin{tikzcd}
\mathcal{A} \arrow[dd, "F"', bend right] \\
\dashv                           \\
\mathcal{B} \arrow[uu, "G"', bend right]
\end{tikzcd}
\end{center}
con unidad $\eta$ y co-unidad $\varepsilon$, se cumple que la transposición es inducida por $\eta$ como $\overline{g}=G(g)\eta_A$, para cualquier $g\in\mathcal{B}\panth{FA,B}$, y dualmente por $\varepsilon$ como $\overline{f}=\varepsilon_BF(f)$, para cualquier $f\in\mathcal{A}\panth{A,GB}$.
\begin{exe}%[Yareli $\checkmark$ ]
  Las transposiciones inducidas por la
  unidad y counidad coinciden con las transposiciones
  originales.
\end{exe}
\begin{proof}
    Recordemos que, por compatibilidad:
    \begin{align*}
        \left(\ol{F(A)\xto g B \xto q B'}\right)
        &=
        \left(A\xto{\ol g} G(B) \xto{G(q)}G(B')\right) \\
        \left(\ol{A'\xto p A \xto f G(B)}\right)
        &=
        \left(F(A')\xto{F(p)} F(A) \xto{\ol f}B\right),
    \end{align*}
    es decir, $\ol{qg}=G(q)\ol g$ y $\ol{fp}=\ol f F(p)$.\\
    Debemos demostrar que $\overline{g}=G(g)\eta_A$ y $\overline{f}=\epsilon_BF(f)$ para $g\colon F(A)\to B$ y $f\colon A\to G(B)$.\\
Tenemos que
\begin{align*}
G(g)\eta_A&=G(g)\overline{\id}_{F(A)}\\
&=\overline{g\circ \id}_{F(A)}\\
&=\overline{g}.
\end{align*}
Además, 
\begin{align*}
\epsilon_B F(f)&=\overline{\id}_{G(B)}F(f)\\
&=\overline{\id_{F(A)}\circ f}\\
&=\overline{f}.
\end{align*}
\end{proof}


