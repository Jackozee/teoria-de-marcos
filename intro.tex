La idea de la teoría de marcos no es nueva.
A fines de la década de 1950, en un seminario de Ehresmann,
en París \cite{Gattungen}, Bénabou
presentó la idea de estudiar un espacio topológico a partir
de las propiedades de su topología, vista como un conjunto ordenado.
Más adelante, Dowker y Papert Strauss escribieron un famoso artículo
\cite{Quotients}
en el que retoman la idea de Bénabou, introducen el término ``marco''
y desarrollan muchas ideas modernas.
Posteriormente, Isbell publica un artículo \cite{atomless-parts}
donde introduce
la categoría de locales como la categoría opuesta a la de marcos.
Observa que muchas nociones topológicas se pueden entender
a través de las propiedades del marco asociado,
y estudia la adjunción entre la categoría de locales y la categoría
de espacios topológicos
\[
    \begin{tikzcd}
        \Top \ar[d,"\cal O"'{name=izq},shift right=2] \\
        \Frm^\op \ar[u,"\pt"'{name=der},shift right=2]
        \adj{izq}{der}
    \end{tikzcd}
\]
la cual no es una equivalencia,
aunque sí induce una equivalencia (dual) entre las subcategorías
$\Sp\hookrightarrow\Frm$ y $\Sob\hookrightarrow\Top$,
donde los objetos de $\Sob$ y $\Sp$ son los llamados espacios sobrios
y marcos espaciales, respectivamente.
Esta equivalencia constituye una generalización de la dualidad
de Stone clásica.

También veremos que la categoría de marcos incluye,
como subcategoría plena, a la categoría $\CBA$ de álgebras booleanas
completas y morfismos completos y, sin embargo, $\CBA$ no es reflexiva
en $\Frm$.
Un problema interesante es determinar qué marcos tienen reflexión booleana.

El estudio de los marcos también se relaciona con la teoría de toposes
y el estudio de retículas más generales que se encuentran
al estudiar anillos y sus categorías de módulos.

Otro atractivo de la teoría de marcos es que se puede hablar
de propiedades topológicas en términos puramente reticulares y,
en este caso, se comportan de manera un poco distinta:
Peter Johnstone demostró, en \cite{johnstone1981tychonoff}, que el teorema de Tychonoff
para marcos no requiere el uso del axioma de elección, (ver también \cite{banaschewski1988another} y \cite{kvrivz1985constructive})
en contraste con el caso de espacios topológicos.
Más adelante, Johnstone publicó su monografía Stone spaces,
donde expuso de manera sistemática todo lo que se sabía en ese momento
al respecto del tema (posiblemente y muy seguramente es uno de los textos donde se expone toda la diversificación de la teoría de marcos también vale la pena mencionar que en tiempos modernos parte de esa diversificación se puede ver en el libro mas reciente \cite{}).

Finalmente, mencionaremos que una herramienta central
en la teoría es el concepto de \emph{núcleo} en un marco,
cuyo estudio sistemático fue introducido por Harold Simmons \cite{simmons1978framework}
y su estudiante Macnab \cite{macnab198110}.

Existen, principalmente, dos libros que tratan este tema.
El libro Stone spaces, de Johnstone \cite{johnstone1986stone},
y el libro Frames and locales, de Picado y Pultr \cite{PicadoPultr}.
En teoría iba a haber un tercer libro, escrito por Harold Simmons ,
aunque este proyecto no se pudo completar, debido a su
lamentable fallecimiento en 2018.
Parte de estas notas se basan en el libro inconcluso de Simmons. 

%\subsubsection*{Temario: La interacción sin puntos del álgebra y la topología.}
%\begin{enumerate}[label=\Roman*.]
%  \item Preliminares de la teoría de marcos. Aspectos básicos.
%  \begin{itemize}
%      \item Cocientes.
%      \item Completaciones ordenadas (si hay tiempo).
%  \end{itemize}
%  \item El ensamble de un marco.
%  \begin{itemize}
%      \item Derivadas, prenúcleos, estables y núcleos.
%      \item Funtorialidad de $N:\Frm\to\Frm$.
%      \item Representaciones: cálculos en $NA$.
%  \end{itemize}
%  \item El espacio de puntos de un marco.
%  \item Lo que gusten y lo que alcance.
%\end{enumerate}
%
%En la parte IV podemos ver un poco de el problema de reflexión
%booleana, los axiomas de
%separación en $\Frm$, espacios espectrales, gavillas en un marco,
%etc. según lo que se desee y el tiempo que haya.
%
%\newpage
