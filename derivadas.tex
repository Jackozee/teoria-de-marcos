\chapter{Derivadas, estables, prenúcleos y núcleos}

%\section*{(SESIÓN 10: 14 OCT)}

En el capítulo anterior definimos los núcleos de un marco y vimos
éstos están en correspondencia con los cocientes del marco.
Ahora veremos que los núcleos de un marco tienen una importancia
central en la estructura del marco base.
El teorema principal es que el conjunto $NA$ de los núcleos de un
marco $A$ es, de nuevo, un marco, llamado el
ensamble de $A$ y, de hecho $A$ es un submarco de $NA$.

Además de los núcleos, arriba definimos los operadores cerradura
y vimos que los núcleos son algunos operadores cerradura.
En nuestro estudio sistemático del ensamble, usaremos otras
familias de operadores monótonos.

\section{Derivadas, derivadas estables, prenúcleos, operadores cerradura y núcleos}

\begin{defn}[Derivadas]
  Una inflación o derivada en $A$ es una función $f:A\to A$ que
  es monótona e infla.
  Es decir:
  \begin{itemize}
    \item $a\leq b$ implica $f(a) \leq f(b)$.
    \item $a\leq f(a)$.
  \end{itemize}
  Usaremos la notación $DA$ para el conjunto de todas las
  derivadas en $A$.
\end{defn}
Notemos que un operador cerradura (definición
\ref{def:operador-cerradura}) es una derivada idempotente, y
recordemos que usamos la notación $CA$ para el conjunto de los 
operadores cerradura en $A$.

\begin{defn}[Derivadas estables]
  Una derivada $f\in DA$ es \emph{estable} si
  \[
    f(x)\inf y \leq f(x\inf y)
  \]
  para cualesquiera $x,y\in A$.
  Denotamos como $SA$ al conjunto de derivadas estables en $A$.
\end{defn}

\begin{defn}[Prenúcleos]
  Un prenúcleo sobre $A$ es una derivada $f\in DA$ tal que
  \[
    f(x)\inf f(y) \leq f(x\inf y)
  .\]
  Notemos que la otra desigualdad se cumple para cualquier
  función monótona, de modo que los prenúcleos separan ínfimos
  binarios.
  Usaremos la notación $PA$ para referirnos al conjunto de
  prenúcleos de $A$.
  Además, cualquier prenúcleo es estable, ya que $y\leq f(y)$ implica
  \begin{align*}
    f(x) \inf y
    &\leq f(x) \inf f(y) \\
    &\leq f(x\inf y) & \text{ pues $f\in PA$}
  \end{align*}
\end{defn}

En resumen tenemos las contenciones
\begin{align*}
  NA \subseteq PA \subseteq SA \subseteq DA \\
  NA \subseteq CA \subseteq DA.
\end{align*}
También es claro que $NA=PA\cap CA$. Es decir, un núcleo es
exactamente un prenúcleo idempotente.
De hecho, tenemos $NA=SA\cap CA$.
Estos conjuntos son, en sí mismos, conjuntos parcialmente
ordenados, donde el orden está dado puntualmente.
Esto es, dadas dos funciones $f,g:A\to A$, decimos que $f\leq g$
si, y solo si,
\[
  \forall x\in A,\; f(x)\leq g(x)
.\]

Notemos que las derivadas $DA$, las estables $SA$ y los
prenúcleos $PA$ son cerrados bajo composición.
Es decir:
\begin{itemize}
  \item si $f,g\in DA$, entonces $fg,gf\in DA$,
  \item si $f,g\in SA$, entonces $fg,gf\in SA$,
  \item si $f,g\in PA$, entonces $fg,gf\in PA$.
\end{itemize}
Por otro lado, las cerraduras $CA$ y los núcleos $NA$
no lo son, en general.

Veremos que cada uno de estos conjuntos tiene más
estructura que la de conjunto parcialmente ordenado.

\section{Ínfimos en familias de derivadas}

Sea $J\subseteq DA$.
Definimos la función $\Inf J:A\to A$ como
\[
  (\Inf J)(a) = \Inf\{f(a) \mid f\in J\}
.\]
Afirmamos que $\Inf J$ es una derivada y, de hecho, es el ínfimo
de $J$ en $DA$:
\begin{itemize}
  \item
  Si $a\leq b\in A$, entonces $f(a)\leq f(b)$ para cada $f\in J$.
  Luego, cada elemento de $\{f(b)\mid f\in J\}$ está acotado
  inferiormente por un elemento de $\{f(a)\mid f\in J\}$, por lo
  cual
  \[
    \Inf\{f(a)\mid f\in J\} \leq \Inf\{f(b)\mid f\in J\}
  .\]
  \item
  Similarmente $\Inf J$ infla.
  \item
  Si $h\in DA$ es una derivada que está por debajo de cada
  elemento $f\in J$.
  Esto es, para cualesquiera $a\in A$ y $f\in J$ se tiene
  $h(a)\leq f(a)$.
  Luego, $h(a)$ es cota inferior de $\{f(a) \mid f\in J\}$, por
  lo cual $h(a)\leq (\Inf J)(a)$.
\end{itemize}

De hecho, se puede probar lo siguiente:
\begin{exe}%[Dante $\checkmark$ ]
    \leavevmode
  \begin{enumerate}
    \item si $J\subseteq SA$, entonces $\Inf J\in SA$,
    \item si $J\subseteq PA$, entonces $\Inf J\in PA$,
    \item si $J\subseteq CA$, entonces $\Inf J\in CA$,
    \item si $J\subseteq NA$, entonces $\Inf J\in NA$.
  \end{enumerate}
\end{exe}
\begin{sol}
    \begin{enumerate}
        \item Sean $x,y \ \in A$.  Así, 
        \begin{align*}
            \Inf J(x)\inf y&=\Inf\{j(x):j\in J\}\inf y\\
            &\leq j(x)\inf y\\
            &\leq j(x\inf y) \ \forall j \ \in J
        \end{align*}
        Por lo que $\Inf J(x)\inf y$ es una cota inferior del conjunto $\{j(x\inf y) : j \ \in J\}$. Así, como $\Inf J(x\inf y)$ es el ínfimo de $\{j(x\inf y) : j \ \in J\}$, se cumple que $\Inf J(x)\inf y\leq \Inf J(x\inf y)$, y $\inf J \ \in SA$.
        \item Sean $x, y \ \in A$. 
        \begin{align*}
            \Inf J(x)\inf \Inf J(y)&=\Inf\{j(x):j \ \in J\} \inf \Inf \{j(y) : j \ \in J\}\\
            &\leq j(x) \ \inf j(y) \\
            &\leq j(x\inf y) \ \forall j \ \in J
        \end{align*}
        por lo que $J(x)\inf \Inf J(y)$ es una cota inferior del conjunto $\{j(x\inf y): j \in J\}$. Así, se cumple que $\Inf J(x) \inf \Inf J(y)\leq \Inf J(x\inf y)$, y entonces $\Inf J \ \in PA$.
        \item Nótese que si $j\in J$, $j$ infla, y se cumple que 
        $$f(x)\leq j(f(x)) \ \forall j,f \in J$$
        Por lo tanto, 
        $$\Inf J(x)\leq f(\Inf J(x)) \ \forall x\in A, f\in J$$
        y $\Inf J$ es cota inferior del conjunto $\{f(\Inf J(x)):f\in J\}$.
        \\
        Ahora bien, sea $f\in DA$ tal que $f(x) \leq j(\Inf J(x)) \ \forall x\in A, j\in J$. Entonces, como $j(j(x))=j(x) \ \forall j\in J$, ocurre que 
        $$f(x)\leq j(x) \ \forall j\in J$$
        $$\Rightarrow f(x)\leq \Inf J(x) ' \forall x\in A$$
        Por lo anterior, $\Inf J(\Inf J(x))=\Inf J(x)$, y $\Inf J \in CA$.
        \item sean $x,y\in A$. Así, 
        \begin{align*}
            \Inf J(x\inf y)&=\Inf\{j(x\inf y):j\in J\}\\
            &=\Inf\{j(x)\inf j(y):j\in J\}\\
            &=\big(\Inf\{j(x):j\in J\}\big)
            \inf\big( \Inf\{j(y):j\in J\}\big)\\
            &=\Inf J(x)\inf J(y)
        \end{align*}
        Por lo tanto, $\Inf J \in NA$
    \end{enumerate}
\end{sol}

\section{Supremos en familias de derivadas}

\subsubsection{Supremos de derivadas \tps{($DA$)}{(DA)}}

\begin{defn}
  Si $J\subseteq DA$ es un conjunto de derivadas,
  definimos el supremo puntual $\pSup J:A\to A$ de $J$ como la
  función dada por
  \[
    (\pSup J)(a) = \Sup\{f(a) \mid f\in J\}
  \]
  para todo $a\in A$.
\end{defn}
Notemos que, si $J=\emptyset\subseteq DA$, entonces
\begin{align*}
  (\pSup\emptyset)(a)
  &= \Sup\{f(a) \mid f\in\emptyset\} \\
  &= \Sup\emptyset \\
  &= 0 \in A
\end{align*}
lo cual ya no es una derivada (a menos que $A$ sea trivial).
Sin embargo, si $J\neq\emptyset$ es una familia de derivadas,
entonces $\pSup J$ es una derivada y es el supremo de $J$ en
$DA$.
Es decir, los supremos (no vacíos) en $DA$ se calculan
puntualmente.

Observemos que, en $DA$, el menor elemento es la identidad
$\id:A\to A$ y el mayor elemento es $\tp:A\to A$ dada como
$\tp(a)=1$.

\subsubsection{Supremos de estables \tps{($SA$)}{(SA)}}

Supongamos que $J$ es un conjunto no vacío de derivadas estables
($\emptyset\neq J\subseteq SA$).
Entonces, para cualesquiera $a,b\in A$, la derivada $\pSup J$ satisface
\begin{align*}
  (\pSup J)(a)\inf b
  &= \Sup\{f(a) \mid f\in J\} \inf b \\
  &= \Sup\{f(a)\inf b \mid f\in J\}
    & \text{por la ley distributiva de marcos} \\
  &\leq \Sup\{f(a\inf b) \mid f\in J \}
    & \text{cada $f\in SA$} \\
  &= (\pSup J)(a\inf b),
\end{align*}
de modo que $\pSup J\in SA$.
Es decir, los supremos (no vacíos) en $SA$ se calculan
puntualmente.

\subsubsection{Supremos de prenúcleos \tps{($PA$)}{(PA)}}

En contraste con lo que sucede con derivadas y estables, el supremo
de una familia de prenúcleos no se calcula puntualmente, en general.

\begin{exa}
  Tomemos el marco $A$ dado como
  \[
    \begin{tikzcd}
      & 1 \\
      & c \ar[u,no head] \\
      a \ar[ur,no head] & & b \ar[ul,no head] \\
      & 0 \ar[ur, no head] \ar[ul, no head]
    \end{tikzcd}
  \]
  Después de algunos cálculos obtenemos la siguiente tabla de
  los valores de los núcleos abiertos, cerrados y regulares.
  Nótese que, como todo núcleo $j$ fija al $1$ y preserva ínfimos,
  es suficiente conocer los valores de $j$ en $a,c$ y $b$
  (ya que $0=a\inf b$).
    \[ 
        \begin{array}{|c|c|c|c|}
            \hline
            j & j(a) & j(c) & j(b) \\
            \hline
            \vnuc a = \wnuc b & 1 & 1 & b \\
            \vnuc b = \wnuc a & a & 1 & 1 \\
            \vnuc c & a & 1 & b \\
            \unuc a & a & c & c \\
            \unuc b & c & c & b \\
            \unuc c = \wnuc c & c & c & c \\
            \hline
        \end{array}
    \]
    En particular, todos estos son prenúcleos.
    Sin embargo, afirmamos que $\vnuc a \psup \vnuc b$ no es prenúcleo.
    En efecto, tenemos
    \begin{align*}
        (\vnuc a \psup \vnuc b)(a\inf b)
        &= (\vnuc a \psup \vnuc b)(0) \\
        &= \vnuc a(0) \sup \vnuc b(0) \\
        &= b \sup a \\
        &= c,
        \\
        (\vnuc a \psup \vnuc b)(a)
        \inf (\vnuc a \psup \vnuc b)(b)
        &= (\vnuc a(a)\sup\vnuc b(a))
            \inf
            (\vnuc a(b)\sup\vnuc b(b)) \\
        &= (1\sup a)\inf(b\sup 1) \\
        &= 1,
    \end{align*}
    así que $\vnuc a\psup\vnuc b$ no es un prenúcleo.
\end{exa}

Sin embargo, hasta cierto punto, esto se puede enmendar:
aunque el supremo puntual de un subconjunto arbitrario $J\subseteq PA$
puede no ser prenúcleo, la situación es mejor cuando $J$ es dirigido.
\begin{lemma}
    Si $J\subseteq PA$ es un conjunto dirigido,
    entonces la derivada estable $\pSup J$ sí es un prenúcleo.
    Recordemos que un conjunto $J$ es dirigido si todo subconjunto
    finito $X\subseteq J$ tiene una cota superior en $J$.
    En particular, $J\neq\emptyset$, pues $\emptyset\subseteq J$ tiene
    una cota superior en $J$.
\end{lemma}
\begin{proof}
    Si $J$ es dirigido, entonces $\pSup J$ es,
    en efecto, una derivada, pues $J\neq\emptyset$.
    Además,
    \begin{align*}
      (\pSup J)(a) \inf (\pSup J)(b)
      &= \Sup\{f(a)\inf g(b) \mid f,g\in J\}
        & \text{ ley distributiva de marcos} \\
      &= \Sup\{h(a) \inf h(b)  \mid h\in J \}
        & \text{ $J$ es dirigido} \\
      &\leq \Sup\{h(a \inf b)  \mid h\in J \}
        & \text{ cada $h\in J$ es prenúcleo} \\
      &= (\pSup J)(a\inf b),
    \end{align*}
    por lo cual $\pSup J\in PA$.
\end{proof}

\subsubsection{Más sobre conjuntos dirigidos}

Tomemos dos derivadas (estables, prenúcleos) $f$ y $g$ en $A$.
Como $g$ infla, para todo $a\in A$ tenemos $a\leq g(a)$ y,
aplicando $f$, se sigue que $f(a)\leq f(g(a))$.
Como $f$ también infla, tenemos $g(a)\leq f(g(a))$.
Luego, $f(a)\sup g(a) \leq f(g(a))$.
Esto es:
\[
  f\psup g \leq fg
.\]
Análogamente, obtenemos $f\psup g\leq gf$, por lo cual
\[
  f\psup g \leq fg\inf gf
.\]

En particular, si un conjunto $J$ de derivadas (estables,
prenúcleos) es cerrado bajo composición, entonces $J$ es dirigido.

\section{El teorema fundamental de la teoría de marcos}
\subsubsection{Las derivadas estables forman un marco}

Ya vimos que, en $DA$ y en $SA$, todos los ínfimos
y los supremos no vacíos se calculan puntualmente,
mientras que $\Sup\emptyset=\id$.


Ahora veremos cómo interactúan los ínfimos finitos
y los supremos en $SA$.
Tomemos un subconjunto $J\subseteq SA$.
Si $J=\emptyset$, entonces la igualdad
\[
  f\inf \Sup J = \Sup\{f\inf g \mid g\in J\}
\]
se cumple, pues ambos lados son $\id:A\to A$.
Por otro lado, si $J\neq\emptyset$, entonces para todo $x\in A$
se tiene
\begin{align*}
  (f\inf \pSup J)(x)
  &= f(x)\inf (\pSup J)(x) \\
  &= f(x)\inf \Sup\{g(x) \mid g\in J\} \\
  &= \Sup\{f(x)\inf g(x) \mid g\in J\}
    & \text{ley distributiva de marcos} \\
  &= \Sup\{(f\inf g)(x) \mid g\in J\} \\
  &= (\pSup\{f\inf g \mid g\in J\})(x).
\end{align*}
de modo que $f\inf\Sup J = \Sup\{f\inf g\mid g\in J\}$.
Esto muestra que $SA$ es un marco.
En particular, $SA$ tiene una implicación.

\subsubsection{Los núcleos forman un marco}

La afirmación es que $NA$ es un subconjunto implicativo de $SA$.
Es decir: $NA$ es cerrado bajo ínfimo (lo cual ya sabemos) y,
para cualquier estable $f\in SA$ y cualquier núcleo $k\in NA$, la
derivada estable $(f\succ k)$ es un núcleo.
Como toda derivada estable e idempotente es núcleo,
basta demostrar que $(f\succ k)$ es idempotente.

Sea $G=\{g\in SA \mid f\inf g\leq k\}$.
Como $G$ no es vacío (por ejemplo, $\id\in G$), tenemos
\[
  (f\succ k) = \pSup\{g\in SA \mid f\inf g\leq k\}
.\]

Primero mostraremos que $G$ es cerrado bajo composiciones.
Observemos que, para cualesquiera $g,h\in G$ y $x\in A$, tenemos
\begin{align*}
  (f\inf gh)(x)
  &= f(x)\inf g(h(x)) \\
  &= f(x)\inf f(x) \inf g(h(x)) \\
  &\leq f(x) \inf g(f(x)\inf h(x))
    & g\in SA \\
  &\leq f(x) \inf g(k(x))
    & f\inf h \leq k, \text{ pues } h\in G \\
  &\leq f(k(x)) \inf g(k(x)) \\
  &\leq k(k(x))
    & f\inf g \leq k, \text{ pues } g\in G \\
  &= k(x).
\end{align*}
Esto prueba que $f\inf gh \leq k$ y, así, $gh\in G$, como se
quería.

Ahora mostraremos que $j=(f\succ k)\in G$.
En efecto, para todo $x\in A$ se tiene
\begin{align*}
  (f\inf j)(x)
  &= f(x) \inf (\pSup G)(x) \\
  &= f(x) \inf \Sup\{g(x) \mid g\in G\} \\
  &= \Sup\{f(x) \inf g(x) \mid g\in G\} \\
  &= k(x),
\end{align*}
como se quería.
Luego, $j^2=jj\in G$, así que $j^2\leq\pSup G = j$.
Concluimos que $j^2=j$.
Como $j$ es una derivada estable e idempotente,
se sigue que es un núcleo.

Esto muestra lo que queríamos probar: que $NA$ es un subconjunto
implicativo de $SA$.
Una consecuencia inmediata es

\begin{thm}[Isbell-Simmons-Johnstone]
  Para cada marco $A$, el ensamble $NA$ es un marco.
\end{thm}
\begin{proof}
    $NA$ es $\Inf$-cerrado y, $(j\in k)\in NA$ para cualesquiera
    $k\in SA$, $k\in NA$.
    Es decir, $NA$ es un subconjunto implicativo del marco $SA$,
    por lo cual es un marco.
\end{proof}
Sin embargo, el hecho de que $NA$ sea un subconjunto
implicativo de $SA$ nos dice aún más.
Todo subconjunto implicativo de un marco es un cociente de éste,
así que $NA$ es de la forma $NA=(SA)_j$ para algún núcleo
$j:SA\to SA$.
La pregunta es, ¿qué núcleo?

\section{Las iteraciones de una derivada}
\subsubsection{Iteraciones en \tps{$DA$}{DA}.}

Denotaremos como $\Ord$ a la clase de ordinales y, para cada
ordinal, definimos
\begin{align*}
  f^0 &= \id \\
  f^{\alpha+1} &= ff^\alpha \\
  f^\lambda &= \Sup\{f^\alpha \mid \alpha<\lambda\}
    & \text{ si $\lambda$ es límite.}
\end{align*}
De este modo, obtenemos una cadena de derivadas
\[
  f^0\leq f^1\leq f^2\leq\dots\leq f^\alpha\leq f^{\alpha+1}
  \leq\dots
.\]
Como $\Ord$ no es cardinable, la cadena
$(f^\alpha \mid \alpha\in \Ord )$ se detiene, por fuerza, en
algún ordinal $\gamma$, es decir: $f^{\gamma+1}=f^\gamma$.
Además, dado que la clase de ordinales es bien ordenada,
existe un primer ordinal
$\infty\in\Ord$ tal que $f^\infty=f^{\infty+1}$.
Dado que $f^{\infty+1}=f^\infty$, se sigue que $f^\alpha
f^\infty=f^\infty$ para todo $\alpha\in\Ord$, por inducción en
$\alpha$. En particular,
$f^\infty$ es idempotente, así que es un operador cerradura en $A$.
Más aún, $f^\infty$ es el menor operador cerradura en $A$
que está por encima de $f$.
Para ver esto, mostraremos que, si $k$ es un operador cerradura
que está por encima de $f$, entonces también está por encima de
toda la cadena de iteraciones de $f$.
Tomemos $k\in CA$ con $f\leq k$ y hagamos inducción.
\begin{itemize}
  \item Para $\alpha=0$, tenemos $f^0=\id\leq k$.
  \item Supongamos que $f^\alpha\leq k$.
  Entonces
  \begin{align*}
    f^{\alpha+1}
    &= ff^\alpha \\
    &\leq kk \\
    &= k.
  \end{align*}
  \item Finalmente, si $\lambda$ es un ordinal límite, supongamos
  que $f^\alpha\leq k$ para todo $\alpha<\lambda$.
  Entonces
  $f^\lambda=\Sup\{f^\alpha\mid \alpha<\lambda\}\leq k$.
\end{itemize}
Como esto es válido para todo ordinal, en particular tenemos
$f^\infty\leq k$.
Luego, $f^\infty$ es el menor operador cerradura que está por
arriba de $f$.

\subsubsection{\tps{$CA$}{CA} como un conjunto fijo}
Luego, tenemos el siguiente resultado:
\begin{thm}
    La construcción $({-})^\infty$ es un operador cerradura en $DA$
    cuyos puntos fijos son los operadores cerradura en $A$:
    \begin{align*}
      ({-})^\infty &\in CDA \\
      (DA)_\infty &= CA.
    \end{align*}
\end{thm}
\begin{proof}
  Claramente, $f\leq f^\infty$, así que $(-)^\infty$ infla.
  Como $f^\infty$ es idempotente, tenemos $(f^\infty)^\alpha=f^\infty$
  para todo ordinal $\alpha$.
  En particular, $(f^\infty)^\infty=f^\infty$, así que
  $(-)^\infty$ es idempotente.
  Finalmente, si $f\leq g$ son derivadas, veamos que
  $f^\alpha\leq g^\alpha$ para todo ordinal $\alpha$.
  \begin{itemize}
    \item El caso $\alpha=0$ es obvio.
    \item Supongamos que $f^\alpha\leq g^\alpha$.
    Entonces
    \begin{align*}
        f^{\alpha+1}
        &= ff^\alpha \\
        &\leq gg^\alpha \\
        &= g^{\alpha+1}
    \end{align*}
    \item Supongamos que $\lambda$ es un ordinal límite y que
    $f^\alpha\leq g^\alpha$ para todo $\alpha<\lambda$.
    Entonces
    \begin{align*}
        f^\lambda(x)
        &= \Sup\{f^\alpha(x)\mid \alpha<\lambda\} \\
        &\leq \Sup\{g^\alpha(x) \mid \alpha<\lambda\} \\
        &= g^\lambda(x)
    \end{align*}
    para todo $x\in A$, así que $f^\lambda\leq g^\alpha$.
  \end{itemize}
  En particular, $f^\infty\leq g^\infty$, así que $(-)^\infty$ es
  monótono.
  Finalmente, la igualdad $(DA)_\infty=CA$ se obtiene al observar
  que todo operador cerradura es un punto fijo de $(-)^\infty$.
\end{proof}

%\section*{(SESIÓN 11: 19 OCT)}

\subsubsection{Iteraciones en \tps{$SA$}{SA}.}

\begin{lemma}
  Si $f$ es un prenúcleo en $A$, entonces cada iteración
  $f^\alpha$ es un prenúcleo, $f^\infty$ es un núcleo y, más aún,
  $f^\infty$ es el menor núcleo por encima de $A$.
\end{lemma}
\begin{proof}
  Sea $f\in PA$.
  Mostraremos, usando inducción, que $f^\alpha\in PA$ para cada
  ordinal $\alpha$.
  \begin{itemize}
    \item Si $\alpha=0$, entonces $f^0=\id\in PA$.
    \item Supongamos que $f^\alpha$ es prenúcleo.
    Como los prenúcleos son cerrados bajo composición,
    $f^{\alpha+1}=ff^\alpha$ es prenúcleo.
    \item Si $\lambda$ es un ordinal límite, supongamos que
    $f^\alpha\in PA$ para cada ordinal $\alpha<\lambda$.
    Recordemos que
    $f^\lambda=\pSup\{f^\alpha\mid\alpha<\lambda\}$.
    Hay que probar que $f^\lambda$ es prenúcleo.
    Para cada $x,y\in A$ tenemos
    \begin{align*}
      &f^\lambda(x) \inf f^\lambda(y) \\
      &\hspace{10mm}
      = \Sup\{f^\alpha(x) \inf f^\beta(y) \mid
        \alpha,\beta<\lambda\}
        && \text{ley distributiva para marcos} \\
      &\hspace{10mm}
      = \Sup\{f^\gamma(x)\inf f^\gamma(y)
        \mid \gamma<\lambda\} \\
      &\hspace{10mm}
      \leq \Sup\{f^\gamma(x\inf y) \mid \gamma<\lambda\}
        && f^\gamma\in PA \text{ por hipótesis}\\
      &\hspace{10mm}
      = f^\lambda(x\inf y).
    \end{align*}
  \end{itemize}
  En particular, $f^\infty$ es un prenúcleo y, como también es
  idempotente, se sigue que $f^\infty$ es un núcleo.

  Finalmente, recordemos que $f^\infty$ es el menor operador
  cerradura por encima de $f$.
  Luego, para cualquier núcleo $j\in NA$ que esté por encima de
  $f$, se tiene $f^\infty\leq j$, así que $f^\infty$ es el menor
  núcleo por encima del prenúcleo $f$.
\end{proof}

Este resultado se puede refinar un poco más:
\begin{lemma}
  Si $f\in SA$ es cualquier estable, entonces $f^\alpha$ es
  estable para todo ordinal $\alpha$.
  Más aún, $f^\lambda$ es un prenúcleo, para cada ordinal límite
  $\lambda$ y, finalmente, $f^\infty$ es el menor núcleo por
  encima de $f$.
\end{lemma}
\begin{proof}
  Sea $f$ una derivada estable.
  Por inducción, probamos que $f^\alpha$ es estable para cada
  ordinal $\alpha$.
  El caso $\alpha=0$ y el paso inductivo de $\alpha$ a $\alpha+1$
  es exactamente igual a la demostración anterior (porque $SA$ es
  cerrado bajo composición).
  Ahora, si $\lambda$ es un ordinal límite, tenemos
  $f^\lambda = \pSup\{f^\alpha \mid \alpha < \lambda\}$, de modo
  que, para cualesquiera $x,y\in A$ se tiene
  \begin{align*}
    f^\lambda(x)\inf y
    &= \Sup\{f^\alpha(x) \mid \alpha <\lambda\} \inf y \\
    &= \Sup\{f^\alpha(x)\inf y\mid \alpha <\lambda\}
      & \text{ley distributiva para marcos} \\
    &\leq \Sup\{f^\alpha(x\inf y)\mid \alpha <\lambda\} \\
    &= f^\lambda(x\inf y),
  \end{align*}
  como se quería.

  Más aún, debemos probar que $f^\lambda$ es prenúcleo, siempre
  que $\lambda$ es un ordinal límite.
  Para cualesquiera $x,y\in A$ tenemos
  \begin{align*}
    &f^\lambda(x)\inf f^\lambda(y) \\
    &\hspace{10mm}
    = \Sup\{f^\alpha(x)\inf f^\beta(y) \mid \alpha,\beta <\lambda\}
      & \text{ley distributiva de marcos} \\
    &\hspace{10mm}
    \leq \Sup\{f^\alpha(x\inf f^\beta(y)) \mid \alpha,\beta <\lambda\}
      & f^\alpha \in SA \\
    &\hspace{10mm}
    \leq \Sup\{f^\alpha(f^\beta(x\inf y)) \mid \alpha,\beta <\lambda\}
      & f^\beta \in SA \\
    &\hspace{10mm}
    \leq \Sup\{f^\gamma(x\inf y) \mid \gamma<\lambda\}
      & (?) \\
    &\hspace{10mm}
    = f^\lambda(x\inf y),
  \end{align*}
  como se quería.
\end{proof}

\subsubsection{\tps{$NA$}{NA} como un conjunto fijo}
Con estas observaciones, tenemos el siguiente teorema.
\begin{thm}
  Para cada marco $A$, el operador cerradura $({-})^\infty:SA\to
  SA$ es un núcleo cuyo conjunto de puntos fijos es el ensamble
  de $A$:
  \[
    (SA)_\infty = NA
  .\]
\end{thm}
\begin{proof}
  Como $({-})^\infty:SA\to SA$ es un operador cerradura,
  solo queda demostrar que $(-)^\infty$ es prenúcleo.
  Es decir, que la desigualdad
  \[
    f^\infty \inf g^\infty \leq (f\inf g)^\infty
  \]
  se cumple para cualesquiera estables $f,g\in SA$.

  Sea $l=(f \inf g)^\infty$.
  Por inducción, mostraremos que $f^\alpha\inf g\leq l$ para todo
  ordinal $\alpha$.
  \begin{itemize}
    \item Para $\alpha=0$, tenemos $f^0\inf g=\id \leq l$.
    \item Supongamos que $f^\alpha\inf g\leq l$.
    Entonces, para todo $x\in A$ tenemos
    \begin{align*}
      (f^{\alpha+1}\inf g)(x)
      &= f(f^\alpha(x)) \inf g(x) \\
      &= f(f^\alpha(x)) \inf g(x) \inf g(x) \\
      &\leq f(f^\alpha(x)) \inf g(f^\alpha(x)) \inf g(x)
        && \text{pues } x\leq f^\alpha(x) \\
      &\leq l(f^\alpha(x)) \inf g(x) && f\inf g\leq (f\inf
      g)^\infty =l \\
      &\leq l(f^\alpha(x) \inf g(x)) && l\in SA \\
      &\leq l(l(x)) && f^\alpha\inf g\leq l \\
      &= l(x) && l\in CA.
    \end{align*}
    \item
    Si $\lambda$ es límite, supongamos que $f^\alpha\inf g\leq l$
    para todo ordinal $\alpha <\lambda$.
    Entonces, para todo $x\in A$, tenemos
    \begin{align*}
      (f^\lambda\inf g)(x)
      &= f^\lambda(x) \inf g(x) \\
      &= \Sup\{f^\alpha(x) \mid \alpha<\lambda\} \inf g(x) \\
      &= \Sup\{f^\alpha(x)\inf g(x) \mid \alpha<\lambda\} \\
      &\leq l(x) && \text{ pues } f^\alpha\inf g\leq l.
    \end{align*}
  \end{itemize}
  Esto muestra que $f^\infty \inf g\leq l$.
  De manera similar, podemos probar que $f^\infty \inf
  g^\alpha\leq l$ para todo $\alpha$.
  Luego, $f^\infty \inf g^\infty \leq l$, como se quería.

  Como $f^\infty$ es un núcleo para cada $f\in SA$, la
  igualdad $(SA)_\infty=NA$ se obtiene de observar que todo
  núcleo $k$ es un punto fijo de $({-})^\infty:SA\to SA$.
\end{proof}

\section{Cálculos en el ensamble}
\subsubsection{Supremos de núcleos \tps{($NA$)}{NA}.}

Ya probamos que los supremos no vacíos en $DA$ y en $SA$ se
calculan puntualmente, y que los supremos dirigidos en $PA$
también.
Sin embargo, aún queda encontrar una descripción para, al menos,
algunos supremos en $NA$.

Sea $J\subseteq NA$ un conjunto no vacío de núcleos.
Como $J\subseteq SA$, entonces $\pSup J$ es el supremo de $J$ en
$SA$.
Por otro lado, consideremos la familia $J^\circ$
de composiciones (finitas) de elementos de $J$:
\[
  J^\circ = \{j_1\cdots j_m \mid j_i\in J \text{ para } 1\leq
  i\leq m\}
.\]
Dado que $J^\circ$ es una familia dirigida de prenúcleos
(pues $J^\circ$ es cerrado bajo composiciones), $\pSup J^\circ$
es el supremo de $J^\circ$ en $PA$.

\begin{lemma}
  Si $J\subseteq NA$ es una familia no vacía de núcleos sobre un
  marco $A$, entonces el núcleo
  \[
    \left(\pSup J\right)^\infty = \left(\pSup J^\circ\right)^\infty
  \]
  es el supremo de $J$ en $NA$.
\end{lemma}
\begin{proof}
  Sean
  \begin{align*}
    j &= \left(\pSup J\right)^\infty &
    k &= \left(\pSup J^\circ\right)^\infty.
  \end{align*}
  Es claro que $j$ y $k$ son núcleos que acotan superiormente a
  $J$.
  Si $l\in NA$ es un núcleo que acota superiormente a $J$, entonces
  también acota superiormente a $J^\circ$, ya que
  \[
    j_1\cdots j_m \leq l^m = l
  \]
  para cualquier $j_1\cdots j_m\in J$.
  Luego, $\pSup J\leq l$ y $\pSup J^\circ \leq l$,
  pues $\pSup J,\pSup J^\circ\in SA$ son los supremos de $J$ y $J^\circ$
  (respectivamente) en $SA$ y $l\in SA$.
  Se sigue que $j,k\leq l$, pues $j$ y $k$ son el menor núcleo por
  encima de $\pSup J$ y $\pSup J^\circ$, respectivamente.
\end{proof}

Uno de los pasos de la demostración anterior
nos permite mostrar un resultado bastante útil.
\begin{cor}
    Sea $J$ una familia de núcleos.
    Si $j\in J^\circ$ es una cota superior de $J$
    y es idempotente, entonces $j=\Sup J$ en $NA$.
    
    En particular, esto sucede cuando $J$ es finito y todos los
    elementos de $J$ aparecen en $j=j_1\cdots j_n\in J^\circ$.
\end{cor}
\begin{proof}
    Nótese que $j=j_1\cdots j_n\in J^\circ$
    es un prenúcleo porque es composición de prenúcleos.
    Como también es idempotente, se sigue que $j\in NA$.
    Ahora, para cualquier núcleo $k\in NA$ que acote a $J$ por arriba,
    tenemos
    \[
        j = j_1\cdots j_n \leq k^n = k
    .\]
    Luego, $j=\Sup J$ en $NA$.
\end{proof}

\subsubsection{Los núcleos abiertos y cerrados son complementarios}

Recordemos que cualquier elemento $a$ de un marco $A$ tiene
asociados los núcleos $\unuc a$ y $\vnuc a$ dados por
\begin{align*}
  \unuc a(x) &= a\sup x
  &
  \vnuc a(x) &= (a\succ x).
\end{align*}
No es difícil ver que
\begin{align*}
  \unuc 1 &= \tp = \vnuc 0 \\
  \unuc 0 &= \id = \vnuc 1.
\end{align*}

Esto se puede generalizar para cualquier elemento $a\in A$.

\begin{lemma}
  Sea $A$ un marco.
  Para cualquier $a\in A$ se tiene
  \begin{align*}
    \vnuc a\inf\unuc a &= \id
    &
    \vnuc a\sup\unuc a &= \tp
  \end{align*}
  en $NA$.
  Es decir, $\unuc a$ y $\vnuc a$ son complementos uno del otro.
\end{lemma}
\begin{proof}
Sabemos que $\unuc a\vee \vnuc a=(\vnuc a\circ \unuc a)^\infty$, pero $(\vnuc a\circ \unuc a)(x)=(a\succ (a\vee x))=1$, para toda $x\in A$. Entonces $\vnuc a\vee \unuc a=(\vnuc a\circ \unuc a)^\infty=\tp$.\\
Además, $(\vnuc a\wedge \unuc a)(x)=\vnuc a(x)\wedge \unuc a(x)=(a\succ x)\wedge (a\vee x)=x\vee (a\wedge x)=x$. Es decir, $\vnuc a\wedge \unuc a=\id$.\\
Por lo tanto $\unuc a$ y $\vnuc a$ son complementados.
\end{proof}

\subsubsection{La representación en núcleos abiertos y cerrados}

\begin{lemma}
    \label{lemma:sup-nuc-cerrado}
  Sea $A$ un marco.
  Para cualesquiera $j\in NA$ y $a\in A$, se tiene
  \begin{enumerate}%[label=(\roman*)]
    \item\label{item:1} $j\unuc a$ es idempotente y, por lo tanto, un núcleo,
    \item\label{item:2} $j\sup\unuc a = j\unuc a$ (el supremo en $NA$),
    \item\label{item:3} $\unuc{j(a)}\inf\vnuc a\leq j$.
  \end{enumerate}
\end{lemma}
\begin{proof}
  \ref{item:1}. Como $PA$ es cerrado bajo composición, al menos tenemos
  $j\unuc a\in PA$.
  Luego, para ver que $j\unuc a$ es un núcleo,
  basta demostrar que es idempotente.
  Para todo $x\in A$, tenemos
  \begin{align*}
    j\unuc a j\unuc a (x)
    &= j\unuc a j(a\sup x) \\
    &= j(a\sup j(a\sup x)) \\
    &= j(j(a\sup x))
      & \text{ pues } a\leq j(a\sup x) \\
    &= j(a\sup x) \\
    &= j\unuc a (x),
  \end{align*}
  como se quería.

  \ref{item:2}. Por \ref{item:1}, $j\unuc a$ es un núcleo que acota
  superiormente a $j$ y a $\unuc a$.
  Por otro lado, si $k$ es cualquier núcleo con $j\leq k$ y
  $\unuc a\leq k$, tenemos $j\unuc a \leq kk=k$.

  \ref{item:3}. Para cualquier $x\in A$, tenemos
  \begin{align*}
    \unuc{j(a)}(x)
    &= j(a)\sup x \\
    &\leq j(a \sup x) \\
    &= j\unuc a(x).
  \end{align*}
  Luego,
  \begin{align*}
    \unuc{j(a)}
    &\leq j\unuc a \\
    &= j\sup \unuc a && \text{ por \ref{item:2}} \\
    &= \neg \vnuc a \sup j
      && \text{ pues } \neg\vnuc a = \unuc a \\
    &= (\vnuc a \succ j)
  \end{align*}
  Luego, $\unuc{j(a)} \inf \vnuc a \leq j$.
\end{proof}

\begin{thm}[La representación en núcleos abiertos y cerrados]
  Sea $A$ un marco y $j$ un núcleo en $A$.
  Entonces
  \[
    j = \Sup\{ \unuc{j(a)}\inf\vnuc a \mid a\in A\}
  \]
  en $NA$.
\end{thm}
\begin{proof}
Sea $k=\bigvee\{\unuc {j(a)}\wedge \vnuc a\mid a\in A\}$.
Por el lema anterior tenemos que $k\leq j$.
Para la otra desigualdad, tomemos $a\in A$.
Entonces $(\unuc {j(a)}\wedge \vnuc a)(a)=j(a)\wedge (a\succ a)=j(a)\wedge 1=j(a)$;
es decir: $j(a)=(\unuc {j(a)}\wedge \vnuc a)(a)\leq k(a)$,
por lo cual $j\leq k$.
\end{proof}

\begin{cor}
  Si $A$ es un marco finito, entonces su ensamble $NA$ es un
  álgebra booleana completa.
\end{cor}
\begin{proof}
Por el teorema anterior cualquier $j\in NA$ tiene complemento, pues $j$ se puede expresar como un supremo finito de elementos complementados. Por lo tanto $NA$ es un álgebra booleana completa.
\end{proof}

\section{El encaje de un marco en su ensamble}

\begin{defn}
  Si $A$ es un marco, la función $\eta_A:A\to NA$ está dada como
  \[
    \eta_A(a) = \unuc a
  .\]
\end{defn}

\begin{thm}
  Si $A$ es un marco, $\eta_A$ es un morfismo de marcos inyectivo
  y es un epimorfismo.
\end{thm}
\begin{proof}
Para ver que $\eta_A$ es un morfismo de marcos primero notemos que $\eta_A(0)=\unuc 0=\id$ y $\eta_A(1)=\unuc 1=\tp$. También es una función monótona pues si consideramos $a\leq b$, con $a,b\in A$, $\eta_A(a)=\unuc a\leq \unuc b=\eta_A(b)$.\\
Además, $\eta_A(a)\wedge\eta_A(b)=\unuc a\wedge \unuc b$, entonces $$(\unuc a\wedge \unuc b)(x)=\unuc a(x)\wedge \unuc b(x)=(a\wedge b)\vee x=\unuc {a\wedge b}(x)$$
Así $\eta_A(a)\wedge \eta_A(b)=\eta_A(a\wedge b).$\\
Ahora consideremos $X\subseteq A$, queremos ver que $\bigvee\eta_A(X)=\eta_A(\bigvee X)$. Como $\eta_A$ es monótona, se cumple que $\eta_A(x)\leq \eta_A(\bigvee X)$ para todo $x\in A$, es decir, $\bigvee\eta_A (X)\leq \eta_A(\bigvee X).$ Resta ver la otra desigualdad.\\
Consideremos $c=\bigvee X$ y $j=\bigvee \eta_A(X)=\bigvee\{ \unuc x\mid x\in X\}.$ Queremos ver que $\unuc c\leq j$. Sea $a\in A$, entonces $$\unuc c(a)=c\vee a=\bigvee X\vee a=\bigvee\{x\vee a\mid x\in X\}=\left(\dot{\bigvee}\{\unuc x\mid x\in X\}\right)(a)\leq j(a)$$
pues el supremo puntual es menor que el supremo en $NA.$\\
Veamos ahora que $\eta_A$ es inyectiva. Consideremos $a,b\in A$ tales que $\eta_A(a)=\eta_A(b)$, entonces $\unuc a=\unuc b$. Evaluando en $x=0$, obtenemos que $a=a\vee 0=b\vee 0=b$. Así, $\eta_A$ es inyectiva.\\
Por último veamos que $\eta_A$ es un epimorfismo. Para ello consideremos dos morfismos 
\[\begin{tikzcd}
	{NA} \ar[r,"f",shift left] \ar[r,"g"',shift right] & {B},
\end{tikzcd}\]
donde $B$ es un marco y $f\circ\eta_A=g\circ\eta_A$,es decir, $f(\unuc a)=g(\unuc a)$ para todo $a\in A$. Por el teorema anterior tenemos que 
\begin{align*}
    f(j)
    &= f\left(\bigvee\{\unuc {j(a)}\wedge \vnuc a\mid a\in A\}\right)
    =\Sup\{f(\unuc {j(a)})\wedge f(\vnuc a)\mid a\in A\} & \mbox{ y } \\
    g(j)
    &= g\left(\bigvee\{\unuc {j(a)}\wedge \vnuc a\mid a\in A\}\right)
    = \Sup\{g(\unuc {j(a)})\wedge g(\vnuc a)\mid a\in A\} & 
\end{align*}
Ahora usemos que $f(\unuc {j(a)})=g(\unuc {j(a)})$ y,
tomando complementos, $f(\vnuc a)=g(\vnuc a)$, pues $\vnuc a$ y $\unuc a$
son complementarios en $A$ y los morfismos de marcos $f,g$
preservan complementos.
Por lo tanto, para todo $j\in NA$, tenemos $f(j)=g(j)$.
Así, $\eta_A$ es un epimorfismo.
\end{proof}

%\section*{(SESIÓN 12: 21 OCT)}

\subsubsection{El ensamble como solución a un problema universal}
\label{ssec:complementacion}

Vimos que, para cada marco $A$, el ensamble $NA$ es un marco y
tenemos un morfismo
\begin{align*}
  \eta_A: A&\to NA \\
  a &\mapsto \unuc a
\end{align*}
el cual es mono y epi, aunque en general no es suprayectivo.
En particular, $\Frm$ no es una categoría balanceada.

\begin{cor}[Adjunción del ensamble]
  Sea $A$ un marco.
  El adjunto derecho del morfismo $\eta_A:A\to NA$ está dado por
  $\bot:NA\to A$, $\bot(j)=j(0)$.
  Es decir, tenemos
  \[
    \unuc a \leq j \ssi a\leq j(0) 
  .\]
\end{cor}
\begin{proof}
  Si $\unuc a \leq j$, entonces
  $a=\unuc a(0) \leq j(0)$.
  Por otro lado, si $a\leq j(0)$, entonces para todo $x\in A$
  tenemos
  \begin{align*}
    \unuc a (x)
    &= a\sup x \\
    &\leq j(0) \sup x \\
    &\leq j(0) \sup j(x) \\
    &= j(x).
  \end{align*}
  Luego, $\unuc a \leq j$.
\end{proof}

\begin{defn}
  Sea $f:A\to B$ un morfismo de marcos.
  Diremos que $f$ resuelve el problema de complementación para
  $A$ si, para todo $a\in A$, $f(a)\in B$ es complementado en
  $B$.
\end{defn}
\begin{exa}
  Para todo $a\in A$, el núcleo $\unuc a$ es complementado en
  $NA$ (su complemento es $\vnuc a$).
  Es decir, $\eta_A:A\to NA$ resuelve el problema de
  complementación para $A$.
\end{exa}

\begin{thm}
  Sea $A$ un marco.
  El morfismo $\eta_A:A\to NA$ resuelve el problema de
  complementación de manera universal.
  Es decir, para cualquier morfismo $f:A\to B$ que resuelve el
  problema de complementación, existe un único morfismo
  $f^\sharp:NA\to B$ tal que el diagrama
  \[
    \begin{tikzcd}
      A \ar[dr,"\eta_A"'] \ar[rr,"f"] && B \\
      & NA \ar[ur,"f^\sharp"',dotted]
    \end{tikzcd}
  \]
  es conmutativo.

  Más aún, si $f_*:B\to A$ es el adjunto derecho de $f=f^*:A\to
  B$, el adjunto derecho $f_\flat:B\to NA$ de $f^\sharp:NA\to B$
  se calcula como
  \[
    f_\flat(b) = f_*\unuc b f^* \in NA
  .\]
\end{thm}
\begin{proof}
  Para empezar, como $\eta_A$ es epi, la factorización de $f$ a
  través de $\eta_A$ es única, en caso de existir.
  Es decir, si $f^\sharp,f^!:NA\to B$ son tales que
  $f^\sharp\eta_A=f=f^!\eta_A$, entonces $f^\sharp=f^!$.
  \[
    \begin{tikzcd}
      A \ar[dr,"\eta_A"'] \ar[rr,"f"] && B \\[5mm]
      & NA \ar[ur,shift right,"f^!"'] \ar[ur,shift left,"f^\sharp"]
    \end{tikzcd}
  \]
  Por lo tanto, basta con mostrar la existencia de $f^\sharp$.
  
  Recordemos que queremos definir $f^\sharp:NA\to B$ tal que el
  diagrama 
  \[
    \begin{tikzcd}
      A \ar[dr,"\eta_A"'] \ar[rr,"f"] && B \\
      & NA \ar[ur,"f^\sharp"',dotted]
    \end{tikzcd}
  \]
  conmute.
  Es decir, tal que $f^\sharp(\unuc a) = f(a)$.
  Recordemos que cada núcleo $j\in NA$ se puede representar como
  \[
    j = \Sup\{\unuc{j(a)} \inf \neg \unuc a \mid a\in A\}
  ,\]
  pues $\neg\unuc a = \vnuc a$.
  Dado que los morfismos de marcos respetan complementos,
  si existiese un morfismo $f^\sharp:NA\to B$ con las propiedades
  deseadas, necesariamente debería cumplirse que
  \begin{align*}
    f^\sharp(j)
    &= f^\sharp\left(
      \Sup\{\unuc{j(a)}\inf\neg\unuc a\mid a\in A\}
      \right) \\
    &= \Sup\{f^\sharp(\unuc{j(a)}\inf\neg\unuc a)\mid a\in A\} \\
    &= \Sup\{f^\sharp(\unuc{j(a)})\inf f^\sharp(\neg\unuc a)
       \mid a\in A\} \\
    &= \Sup\{f^\sharp(\unuc{j(a)})\inf \neg f^\sharp(\unuc a)
       \mid a\in A\} \\
    &= \Sup\{f(j(a))\inf \neg f(a) \mid a\in A\}.
  \end{align*}
  Con esta motivación, definimos $f^\sharp$ como
  \[
    f^\sharp(j) = \Sup\{f(j(a))\inf\neg f(a) \mid a\in A\}
  .\]
  
  Hay que probar que esta definición nos da un morfismo de marcos
  con las propiedades deseadas.
  Verificamos la monotonicidad.
  Si $k\leq j$ son núcleos en $A$, entonces $k(x)\leq j(x)$ para
  todo $x\in A$.
  Aplicando $f$ tenemos $f(k(x))\leq f(j(x))$, y así $f(k(x))\inf
  f(x)\leq f(j(x))\inf f(x)$.
  Esto nos dice que $f^\sharp$ es monótono.

  Por otro lado $f_\flat:B\to NA$ también es monótona,
  pues si $b\leq c\in B$, entonces $\unuc b\leq\unuc c \in NB$.
  Luego, $f_*\unuc bf^*\leq f_*\unuc cf^*$, pero
  esto es $f_\flat(b)\leq f_\flat(c)$.

  Ahora veamos que $f^\sharp \dashv f_\flat$.
  Dados $j\in NA$ y $b\in B$ arbitrarios, debemos mostrar la
  equivalencia
  \[
    f^\sharp(j)\leq b \ssi j\leq f_\flat(b)
  .\]
  Por definición
  $f^\sharp(j)=\Sup\{f^*(j(x))\inf\neg f^*(x) \mid x\in A\}$.
  Luego, tenemos las equivalencias
  \begin{align*}
    f^\sharp(j) \leq b
    &\iff \forall(x\in A)\;f^*(j(x))\inf\neg f^*(x)\leq b \\
    &\iff \forall(x\in A)\;f^*(j(x))\leq (\neg f^*(x)\succ b) \\
    &\iff \forall(x\in A)\;f^*(j(x)) \leq f^*(x)\sup b
      & \text{caballo de batalla} \\
    &\iff \forall(x\in A)\;j(x) \leq f_*(b\sup f^*(x))
      & \text{adjunción } f^*\dashv f_* \\
    &\iff \forall(x\in A)\;j(x) \leq f_*(\unuc b(f^*(x))) \\
    &\iff j\leq f_*\unuc b f^* = f_\flat(b).
  \end{align*}
  Esto muestra que $f^\sharp\dashv f_*$.
  En particular, $f^\sharp$ preserva supremos arbitrarios.
  Ahora hay que ver que $f^\sharp$ preserva ínfimos finitos.
  Como $f^\sharp$ es monótona, tenemos
  $f^\sharp(j\inf k)\leq f^\sharp(j)\inf f^\sharp(k)$,
  así que falta probar la otra comparación.
  Tenemos
  \begin{align*}
    f^\sharp(j)\inf f^\sharp(k)
    &= \Sup\Big\{[f(j(x))\inf\neg f(x)]
        \inf[f(j(y))\inf\neg f(y)] \mid x,y\in A\Big\} \\
    &= \Sup\Big\{f(j(x)\inf k(y))\inf\neg f(x\sup y)
        \mid x,y\in A\Big\} \\
    &\leq \Sup\Big\{f(j(x\sup y)\inf k(x\sup y))
        \inf\neg f(x\sup y) \mid x,y\in A\Big\} \\
    &= \Sup\Big\{f(j(z)\inf k(z))
        \inf\neg f(z) \mid z\in A\Big\} \\
    &= \Sup\Big\{f((j\inf k)(z))
        \inf\neg f(z) \mid z\in A\Big\} \\
    &= f^\sharp(j\inf k).
  \end{align*}
  Finalmente, hay que ver que $f=f^\sharp \eta_A$.
  En efecto: para cualquier $a\in A$, tenemos
  \begin{align*}
    f^\sharp(\eta_A(a))
    &= f^\sharp(\unuc a) \\
    &= \Sup\{f(\unuc a(x))\inf\neg f(x)\mid x\in A\} \\
    &= \Sup\{f(a\sup x)\inf\neg f(x)\mid x\in A\} \\
    &= \Sup\{(f(a)\sup f(x))\inf\neg f(x)\mid x\in A\} \\
    &= \Sup\{f(a)\inf\neg f(x)\mid x\in A\} \\
    &= f(a)\inf\Sup\{\neg f(x)\mid x\in A\} \\
    &= f(a)\inf 1 \\
    &= f(a),
  \end{align*}
  lo cual concluye la prueba.
\end{proof}

\begin{thm}[El ensamble como indicador de booleanidad]
  Sea $A$ un marco.
  Entonces el encaje $\eta_A:A\to NA$ es suprayectivo (y, por lo
  tanto, un isomorfismo) si, y solo
  si, $A$ es un álgebra booleana completa.
\end{thm}
\begin{proof}
    Supongamos que $\eta_A$ es suprayectiva.
    Entonces, para todo elemento $a\in A$,
    $\eta_A(a)=\unuc a$ tiene complemento $\vnuc a$ en $NA$.
    Como $\eta_A$ es isomorfismo, entonces
    $a$ tiene complemento en $A$.
    Luego, $A$ es booleana.
    
    Por otro lado, supongamos que $A$ es booleana.
    Dado $j\in NA$, mostraremos que $\eta_A(a)=j$,
    donde $a=j(0)$.
    Como $a\leq j(0)$, tenemos que $\unuc a\leq j$
    (por la adjunción $\eta_A\dashv\bot$).
    Queda demostrar la comparación $j\leq\unuc a$;
    esto es: $j(x)\leq x\sup a$ para todo $x\in A$.
    Como $A$ es booleana, podemos usar nuestro
    caballo de batalla, que nos dice que esto es equivalente
    a mostrar que $j(x)\inf\neg x\leq a$ para todo $x\in A$.
    En efecto, tenemos
    \begin{align*}
        j(x)\inf\neg x
        &\leq j(x)\inf j(\neg x) \\
        &= j(x\inf\neg x) \\
        &= j(0) \\
        &= a.
    \end{align*}
    Esto muestra que $\unuc a=j$,
    así que $\eta_A$ es suprayectiva y,
    por lo tanto, un isomorfismo.
\end{proof}

\section{Más cálculos en el ensamble}
\label{ssec:calculos}

\begin{lemma}[Tres equivalencias]
  \label{lemma:tres-equivalencias}
  Sea $A$ un marco.
  Entonces
  \begin{align*}
    \unuc a\leq j &\iff a\leq j(0) &
    \vnuc a\leq j &\iff 1=j(a) &
    j\leq \wnuc a &\iff j(a)=a.
  \end{align*}
  Nótese que la primera equivalencia es la adjunción $\eta_A\dashv\bot$
  que demostramos en \ref{ssec:complementacion}.
  También demostramos, en el lema \ref{lemma:nucleos-densos},
  el caso $a=0$ de la tercera equivalencia.
\end{lemma}
\begin{proof}
    Mostraremos la segunda y tercera equivalencias.
    
    Supongamos que $\vnuc a\leq j$.
    Evaluando en $a$, obtenemos $1=(a\succ a)\leq j(a)$,
    así que $1=j(a)$.
    Por otro lado, supongamos que $1=j(a)$.
    Para todo $x\in A$, tenemos $\vnuc a(x)=a\succ x$,
    por lo cual $\vnuc a(x)\inf a\leq x$.
    Aplicando $j$, obtenemos
    \begin{align*}
        j(x)
        &\geq j(\vnuc a(x)\inf a) \\
        &= j(\vnuc a(x))\inf j(a) \\
        &= j(\vnuc a(x))\inf 1 \\
        &\geq j(\vnuc a(x)) \\
        &\geq \vnuc a(x).
    \end{align*}
    Luego, $\vnuc a\leq j$.
    
    Ahora supongamos que $j\leq\wnuc a$.
    Evaluando en $a$, obtenemos $j(a)\leq a$.
    Como $j$ infla, esto es equivalente a $j(a)=a$.
    Por otro lado, supongamos que $j(a)=a$.
    Debemos mostrar que $j\leq\wnuc a$;
    esto es: que $j(x)\leq((x\succ a)\succ a)$
    para todo $x\in A$.
    Recordando que siempre tenemos $x\inf(x\succ a)=x\inf a$,
    se sigue que
    \begin{align*}
        j(x)\inf(x\succ a)
        &\leq j(x)\inf j(x\succ a) \\
        &= j(x\inf (x\succ a)) \\
        &= j(a) \\
        &= a.
    \end{align*}
    Luego, $j(x)\leq((x\succ a)\succ a)=\wnuc a(x)$,
    como se quería mostrar.
\end{proof}

\begin{lemma}
  Sea $A$ un marco y $j,k\in NA$ núcleos.
  Si $jk\leq kj$, entonces $k\sup j = kj$.
\end{lemma}
\begin{proof}
  Supongamos que $jk\leq kj$.
  Sea $g=kj$.
  Entonces
  \begin{align*}
    g^2
    &= kjkj \\
    &\leq kkjj \\
    &= k^2j^2 \\
    &= kj \\
    &= g.
  \end{align*}
  Es decir, $g$ es un prenúcleo idempotente, y así $g\in NA$
  es un núcleo por encima de $k$ y de $j$.

  Ahora, si $h\in NA$ es cualquier núcleo con $j\leq h$ y $k\leq
  h$, entonces $g=kj\leq h^2=h$.
  Se sigue que $g=k\sup j$.
\end{proof}

\begin{lemma}[Supremos con núcleos abiertos y cerrados]
    \label{lem:sup-ab-cerr}
  Sea $A$ un marco.
  Dado cualquier núcleo $j\in NA$ y elementos $a,b\in A$, tenemos
  \[
    \vnuc b \sup j \sup \unuc a = \vnuc b j \unuc a
  .\]
\end{lemma}
\begin{proof}
    Antes probamos que $j\sup\unuc a = j\unuc a$,
    por lo cual basta demostrar $\vnuc b \sup j = \vnuc b j$.
  Por el lema anterior, es suficiente con probar la desigualdad
  $j\vnuc b \leq \vnuc b j$.
  Es decir, hay que probar que $j(b\succ x) \leq (b\succ j(x))$
  para todo $x\in A$.
  Para esto, observemos que
  \begin{align*}
    j\vnuc b (x) \inf b
    &= j(b\succ x) \inf b \\
    &\leq j((b\succ x)\inf b) \\
    &= j(b\inf x) \\
    &\leq j(x).
  \end{align*}
  Usando la definición de la implicación, esto nos da $j\vnuc
  b(x) \leq (b\succ j(x))$, que es lo que queríamos.
\end{proof}

%\section*{(SESIÓN 13: 26 OCT)}

\subsubsection{La representación de un núcleo generado por una derivada}

Ya probamos que todo núcleo $j\in NA$ se puede representar como
\[
    j = \Sup\{\unuc{j(a)}\inf \vnuc a \mid a\in A \}
.\]
Si existe una derivada $f\in DA$ tal que $j=f^\infty$,
esta construcción se puede mejorar.
En esta sección, fijamos una derivada $f\in DA$ y suponemos que
$j=f^\infty\in NA$.
Usando la cadena de iteraciones de $f$, construiremos una cadena
en $A$, y luego una cadena en $NA$.
\begin{itemize}
  \item
  Para cada $a\in A$ y cada ordinal $\alpha$, definimos
  $a(\alpha)=f^\alpha(a)$.
  Esto nos da una cadena en $A$
  \[
    (a(\alpha) \mid \alpha\in\Ord)
  \]
  la cual, por cardinalidad, se estaciona en algún ordinal.
  En particular, por la definición de $a(\alpha)$, se tiene
  $a(\infty+1)=a(\infty)$ (recordemos que la cadena de los
  $f^\alpha$ se estaciona en el ordinal $\infty$).
  \item
  Usando la cadena anterior, construimos una nueva cadena en $NA$.
  \begin{align*}
    j_{a,0}
    &= \id_A \\
    j_{a,\alpha+1}
    &= (\unuc {a(\alpha+1)}\inf \vnuc {a(\alpha)})\sup j_{a,\alpha}
    \\
    j_{a,\lambda}
    &= \Sup\{j_{a,\alpha} \mid \alpha < \lambda\}
      & \text{(si $\lambda$ es límite).}
  \end{align*}
  Dado que los $a(\alpha)$ se estacionan, los $j_{a,\alpha}$ también.
  En efecto, si $a(\alpha) = a(\alpha+1)$, entonces
  \begin{align*}
    j_{a,\alpha+1}
    &= (\unuc {a(\alpha)}\inf \vnuc {a(\alpha)})
      \sup j_{a,\alpha} \\
    &= \id_A\sup j_{a,\alpha} \\
    &= j_{a,\alpha}.
  \end{align*}
  Sea $j_a$ el mayor de los $j_{a,\alpha}$.
  Es decir,
  \begin{align*}
    j_a
    &= \Sup\{j_{a,\alpha} \mid \alpha\in\Ord\} \\
    &= \Sup\{(\unuc {a(\alpha+1)}\inf\vnuc {a(\alpha)})
      \sup j_{a,\alpha} \mid \alpha\in\Ord\} \\
    &= \Sup\{\unuc {a(\alpha+1)}\inf\vnuc {a(\alpha)}
    \mid \alpha\in\Ord\}
  \end{align*}
  En particular, observemos que $j_a=j_{a,\infty}$.
\end{itemize}
El siguiente resultado nos dice que los núcleos
$j_{a,\alpha}$ tienen una descripción más simple.

\begin{lemma}
  Para cada ordinal $\alpha$, el núcleo $j_{a,\alpha}$ se puede
  expresar como
  \[
    j_{a,\alpha} = \unuc{a(\alpha)} \inf \vnuc a
  .\]
  En particular, para $\alpha=\infty$, tenemos
  \[
    j_a = j_{a,\infty} = \unuc{f^\infty(a)}\inf\vnuc a
  .\]
  Una consecuencia inmediata es que
  \[
    f^\infty = \Sup\{j_a \mid a\in A\}
  ,\]
  pues
  $f^\infty = \Sup\{\unuc{f^\infty(a)}\inf\vnuc a \mid a\in A\}$.
\end{lemma}
\begin{proof}
  Probamos la afirmación por inducción
  \begin{itemize}
    \item Para $\alpha=0$, tenemos $j_{a,0}=\id$, mientras que
    $\unuc{a(0)}\inf \vnuc a = \unuc a \inf \vnuc a = \id$.
    \item Supongamos que
    $j_{a,\alpha} = \unuc{a(\alpha)} \inf \vnuc a$.
    Entonces
    \begin{align*}
      j_{a,\alpha+1}
      &= (\unuc {a(\alpha+1)}\inf \vnuc {a(\alpha)})
        \sup j_{a,\alpha} \\
      &= (\unuc {a(\alpha+1)}\inf \vnuc {a(\alpha)})
        \sup (\unuc{a(\alpha)} \inf \vnuc a) \\
      &=
      (\unuc {a(\alpha+1)}\sup (\unuc{a(\alpha)} \inf \vnuc a))
      \inf(\vnuc {a(\alpha)}\sup (\unuc{a(\alpha)} \inf \vnuc a)) \\
      &= \unuc {a(\alpha+1)}
      \inf(\vnuc {a(\alpha)}\sup (\unuc{a(\alpha)} \inf \vnuc a)) \\
      &= \unuc {a(\alpha+1)}
      \inf(\vnuc {a(\alpha)}\sup\unuc{a(\alpha)})
      \inf(\vnuc {a(\alpha)}\sup \vnuc a) \\
      &= \unuc {a(\alpha+1)} \inf \tp \inf \vnuc a \\
      &= \unuc {a(\alpha+1)}\inf \vnuc a,
    \end{align*}
    como se quería.
    \item Si $\lambda$ es un ordinal límite, supongamos que 
    $j_{a,\alpha} = \unuc{a(\alpha)} \inf \vnuc a$ para todo
    ordinal $\alpha <\lambda$.
    Entonces
    \begin{align*}
      j_{a,\lambda}
      &= \Sup\{j_{a,\alpha} \mid \alpha<\lambda\} \\
      &= \Sup\{\unuc{a(\alpha)}\inf\vnuc a \mid\alpha<\lambda\}
      \\
      &= \Sup\{\unuc{a(\alpha)}\mid\alpha<\lambda\}\inf\vnuc a \\
      &= \unuc{\Sup\{a(\alpha)\mid\alpha<\lambda\}}\inf\vnuc a \\
      &= \unuc{a(\lambda)}\inf\vnuc a,
    \end{align*}
    como se deseaba.
  \end{itemize}
\end{proof}


Con este resultado, podemos probar que el núcleo $j=f^\infty$
tiene una descripción más simple que la canónica.

\begin{lemma}
  Si $f\in DA$ es una derivada tal que $f^\infty$ es un núcleo,
  entonces
  \[
    f^\infty = \Sup\{\unuc{f(a)}\inf\vnuc a \mid a\in A\}
  .\]
\end{lemma}
\begin{proof}
  Dado que
  \[
    \unuc{f(a)}\inf\vnuc a\leq \unuc{f^\infty(a)}\inf\vnuc a
  \]
  para todo $a\in A$, se sigue que
  \[
    \Sup\{\unuc{f(a)}\inf\vnuc a \mid a\in A\}
    \leq
    \Sup\{\unuc{f^\infty(a)}\inf\vnuc a \mid a\in A\}
    = f^\infty
  .\]

  Por otro lado, para cada $a\in A$ y cada ordinal $\alpha$, tenemos
  \[
    a(\alpha+1)=f^{\alpha+1}(a)=f(f^\alpha(a))=f(\alpha(a))
  ,\]
  por lo cual
  \[
     \unuc{a(\alpha+1)}\inf\vnuc{a(\alpha)}
     \in
     \{\unuc{f(b)}\inf\vnuc b \mid b\in A\}
  \]
  (poniendo $b=a(\alpha)$).
  Se sigue que
  \[
     \unuc{a(\alpha+1)}\inf\vnuc{a(\alpha)}
     \leq
     \Sup\{\unuc{f(b)}\inf\vnuc b \mid b\in A\}
  .\]
  Como esto es válido para todos los ordinales, tenemos
  \[
     j_a\leq \Sup\{\unuc{f(b)}\inf\vnuc b \mid b\in A\}
  ,\]
  pues $j_a = \Sup\{\unuc{a(\alpha+1)}\inf\vnuc{a(\alpha)}
   \mid\alpha\in\Ord\}$.
  De nuevo, como esto es válido para cualquier $a\in A$, concluimos
  que
  \[
     f^\infty\leq \Sup\{\unuc{f(b)}\inf\vnuc b \mid b\in A\}
  ,\]
  pues $f^\infty = \Sup\{j_a\mid a\in A\}$ (aquí es donde usamos
  el lema anterior).
\end{proof}

\subsubsection{Los núcleos regulares}

Recordemos que los núcleos regulares de un marco
(esto es, los de la forma $\wnuc a$) son exactamente
los que corresponden a sus cocientes booleanos.
Ahora probaremos que todo núcleo se puede descomponer en
núcleos regulares.

\begin{thm}[La representación en núcleos regulares]
    Sea $A$ un marco y $j:A\to A$ un núcleo.
    Entonces
    \[
        j
        = \Inf\{\wnuc{j(a)}\mid a\in A\}
        = \Inf\{\wnuc a\mid a\in A_j\}
    .\]
    Como $\{\wnuc{j(a)}\mid a\in A\}=\{\wnuc a \mid a\in A_j\}$,
    basta probar la primera igualdad.
\end{thm}
\begin{proof}
    Sea $l = \Inf\{\wnuc{j(a)}\mid a\in A\}$.
    En \ref{ssec:calculos} probamos que
    \[
        j\leq\wnuc a \ssi j(a) = a
    ,\]
    por lo cual $j\leq \wnuc{j(a)}$ para todo $a\in A$.
    Luego, $j\leq l$.
    Para la otra desigualdad,
    observemos que siempre tenemos $l\leq\wnuc{j(a)}$.
    Luego,
    \begin{align*}
        l(a)
        &\leq l(j(a)) \\
        &\leq \wnuc{j(a)}(j(a)) \\
        &= j(a),
    \end{align*}
    así que $l\leq j$, como se quería.
\end{proof}

\begin{lemma}
    Sean $A$ un marco y $a\in A$.
    Si $j\in NA$ es un núcleo tal que $\wnuc a\leq j$, entonces
    \[
        j = \wnuc a \sup\unuc b = \wnuc b
    ,\]
    donde $b=j(0)$.
    Nótese que, en particular, este resultado implica que los núcleos
    regulares forman una sección superior.
    
    También nótese que, por el lema
    [\nameref{lem:sup-ab-cerr}],
    el miembro central es igual a $\wnuc a\unuc b$.
\end{lemma}
\begin{proof}
    Como $b\leq j(0)$, tenemos
    $\unuc b\leq j$ (por la adjunción $\eta_A\dashv \bot$).
    Luego,
    \[
        \wnuc a\sup\unuc b\leq j
    .\]
    Notemos, además, que $j(b)=j(j(0))=j(0)=b$,
    lo cual sucede si, y solo si,
    \[
        j \leq \wnuc b
    .\]
    Finalmente, resta probar que
    $\wnuc b\leq \wnuc a\sup\unuc b$.
    Por nuestro caballo de batalla,
    esto es equivalente a $\wnuc b\inf\vnuc b\leq \wnuc a$
    lo cual sucede si, y solo si, $(\wnuc b\inf\vnuc b)(a)=a$.
    Dado que $\wnuc a\leq j$, tenemos
    $a=\wnuc a(0)\leq j(0)=b$.
    Luego, $(a\succ b)=1$, por lo cual
    \begin{align*}
        (\wnuc b\inf\vnuc b)(a)
        &= \wnuc b(a)\inf \vnuc b(a) \\
        &= ((a\succ b)\succ b)\inf (b\succ a) \\
        &= (1\succ b) \inf (b\succ a)\\
        &= b \inf (b\succ a) \\
        &= b\inf a \\
        &= a.
    \end{align*}
    Esto es lo que se quería mostrar.
\end{proof}

%\section*{SESIÓN 14: 28 OCT}

\begin{lemma}
    Sean $d\in DA$ y $j\in CA$.
    Entonces
    \[
        dj=j \ssi d\leq j \ssi jd=j
    .\]
\end{lemma}
\begin{proof}
    Probaremos la primera equivalencia,
    pues la segunda es completamente análoga.
    Supongamos que $jd=j$.
    Como $x\leq j(x)$, tenemos que $d(x)\leq d(j(x))=j(x)$.
    Luego, $d\leq j$.
    Por otro lado, si $d\leq j$, entonces tenemos
    $dj\leq jj=j$.
    La otra desigualdad ($j\leq dj$) se sigue porque $d$ infla.
\end{proof}

\begin{thm}
    Sean $A$ un marco y $a\in A$.
    Si $k\in NA$ es un núcleo tal que $\unuc a\leq k\leq\wnuc a$,
    entonces, para todo $j\in NA$ se tiene
    \[
        \wnuc a\sup j = \wnuc a j k = \wnuc b
    ,\]
    donde $b=\wnuc a(j(a))$.
\end{thm}
\begin{proof}
    Para la primera igualdad,
    basta ver que el prenúcleo $h=\wnuc ajk$ es idempotente
    y, por lo tanto, un núcleo.
    (En efecto, una vez probado esto, tendremos que
    cualquier núcleo $l$ que esté sobre $\wnuc a$ y $j$ queda
    por debajo de $h$, pues
    $h=\wnuc ajk\leq\wnuc aj\wnuc a\leq l^3=l$).
    
    Ahora, para probar la idempotencia de $h$,
    basta ver que $jh=h$ pues,
    por el resultado anterior,
    $k\leq\wnuc a$ implica $k\wnuc a=\wnuc a$,
    lo cual nos da
    \begin{align*}
        h^2
        &= \wnuc ajk\wnuc ajk \\ 
        &= \wnuc aj\wnuc ajk
            && (k\wnuc a=\wnuc a) \\ 
        &= \wnuc a\wnuc ajk
            && (jh=h)\\ 
        &= \wnuc ajk \\ 
        &= h.
    \end{align*}
    
    Probemos, pues, que $jh=h$.
    Sea $x\in A$ y definamos $y=jk(x)$,
    de modo que $jh(x)=h(x)$ es lo mismo que
    $j(\wnuc a(y))=\wnuc a(y)$.
    Una desigualdad es porque $j$ infla,
    así que queda probar la otra desigualdad:
    $j(\wnuc a(y))\leq\wnuc a(y)$,
    la cual equivale a $j(\wnuc a(y))\inf(y\succ a)\leq a$.
    Como $\unuc a\leq k$, tenemos $a\leq k(0)$, así que
    \begin{align*}
        j\wnuc a(y)\inf(y\succ a)
        &\leq j\wnuc a(y)\inf j(y\succ a) \\
        &= j(\wnuc a(y)\inf (y\succ a)) \\
        &= j(((y\succ a)\succ a)\inf (y\succ a)) \\
        &= j((y\succ a)\inf a) \\
        &= j(a) \\
        &\leq j(k(0)) && (a\leq k(0)) \\
        &\leq j(k(x)) \\
        &= y.
    \end{align*}
    Haciendo ínfimo con $(y\succ a)$, obtenemos
    \begin{align*}
        j(\wnuc a(y))\inf (y\succ a)
        &\leq y\inf (y\succ a) \\
        &= y\inf a \\
        &\leq a,
    \end{align*}
    que es lo que queríamos.
    
    Ahora veamos la otra igualdad.
    Evaluando las desigualdades $\wnuc a\leq k\leq \wnuc a$
    en $0$, obtenemos $k(0)=a$.
    Como $h=\wnuc ajk$ está por encima de $\wnuc a$,
    hace dos lemas vimos que $h=\wnuc b$, donde
    \begin{align*}
        b
        &= h(0) \\
        &= \wnuc ajk(0) \\
        &= \wnuc aj(a),
    \end{align*}
    como se quería.
\end{proof}

\begin{cor}
    Tomando $a=0$ en el resultado anterior,
    vemos que todo $j\in NA$ satisface
    \[
        j\sup\wnuc 0 = \wnuc{\neg\neg j(0)}
    .\]
\end{cor}

\begin{exe}%[Alfredo $\checkmark$]
    Obtén los núcleos regulares del marco
    \[
        9 \hspace{10mm} = \hspace{10mm} 
        \begin{tikzcd}[row sep=3mm, column sep=3mm]
            & & 1 \\
            & p \ar[ur,no head] && q \ar[ul,no head] \\
            l \ar[ur,no head]
                && m \ar[ul,no head] \ar[ur,no head]
                && r \ar[ul,no head] \\
            & a \ar[ul,no head]\ar[ur,no head]
                && b\ar[ul,no head]\ar[ur,no head] \\
            & & 0 \ar[ul,no head]\ar[ur,no head]
        \end{tikzcd}
    \]
\end{exe}
\begin{sol}
    Como los núcleos preservan los ínfimos y el $1$,
    basta conocer su acción sobre $\{l,p,q,r\}$.
    \[ 
        \begin{array}{|c|c|c|c|c|}
            \hline
        j = \wnuc x & j(l) & j(p) & j(q) & j(r) \\
            \hline
      \wnuc 1 = \tp &  1   & 1    & 1    & 1 \\
            \wnuc p &  p   & p    & 1    & 1 \\
            \wnuc q &  1   & 1    & q    & q \\
            \wnuc l &  l   & 1    & 1    & 1 \\
            \wnuc m &  p   & p    & q    & q \\
            \wnuc r &  1   & 1    & 1    & r \\
            \wnuc a &  l   & 1    & q    & q \\
            \wnuc b &  p   & p    & 1    & r \\
            \wnuc 0 &  l   & 1    & 1    & r \\
            \hline
        \end{array}
    \]
%    \[ 
%        \begin{array}{|c|c|c|c|c|}
%            \hline
%                 j=\vnuc x 
%                 & j(l)=(x\succ l) & j(p)=(x\succ p)
%                 & j(q)=(x\succ q) & j(r)=(x\succ r) \\
%            \hline
%      \vnuc 1 = \id &  l   & p    & q    & r \\
%            \vnuc p &  l   & 1    & q    & r \\
%            \vnuc q &  l   & p    & 1    & r \\
%            \vnuc l &  1   & 1    & q    & r \\
%            \vnuc m &  l   & 1    & 1    & r \\
%            \vnuc r &  l   & p    & 1    & 1 \\
%            \vnuc a &  1   & 1    & 1    & 1 \\
%            \vnuc b &  1   & 1    & 1    & 1 \\
%      \vnuc 0 = \tp &  1   & 1    & 1    & 1 \\
%            \hline
%        \end{array}
%    \]
\end{sol}

\part{El espacio de puntos}
\label{part:espacio-de-puntos}

