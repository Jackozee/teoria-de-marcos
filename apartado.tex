\section{Apartado técnico}

En esta sección veremos que los conceptos de negación e implicación se
pueden trabajar en contextos más generales.

\subsection{Negaciones}\label{ss:negaciones}

En \ref{ss:heyting} vimos que toda álgebra booleana tiene una
implicación dada por $(x\succ y)=\neg x\succ y$. Este es nuestro
caballo de batalla. También vimos que el complemento de un elemento se
puede recuperar de la implicación como $\neg x = (x\succ 0)$.
De hecho esta última operación se puede hacer en cualquier álgebra de
Heyting $A$. En este caso, $b=(x\succ 0)$ no es, necesariamente, un
complemento de $A$, pero sí tiene la propiedad de ser el mayor
$b$ que cumple $x\inf b=0$.

La siguiente definición nos permite trabajar con esta noción en una
retícula que puede ni siquiera ser álgebra de Heyting.

\begin{definition}[Negaciones]
  Si $A$ es una retícula y $a\in A$,
  entonces una negación de $a$ es un elemento $b\in A$ tal que
  \[
      x\leq a \ssi x\inf b=0
  \]
  para todo $x\in A$.
\end{definition}
Así, $b,b'$ son negaciones de $a$,
entonces $b\inf a=b'\inf a=0$ (porque $b\leq b$ y $b'\leq b'$).
Así, $b\leq b'$ y $b'\leq b$, por lo cual $b=b$.
Es decir, las negaciones son únicas en caso de existir, así que,
si un elemento $a\in A$ tiene negación,
ésta se denota como $\neg a$.
Nótese que, en este caso, para demostrar la unicidad
no se requirió que $A$ fuera distributiva.

Sin embargo, si $A$ es distributiva
y $a\in A$ tiene complemento $a'$ (único, por distributividad),
entonces $a'$ también es la negación de $a$, es decir, $\neg a=a'$.
En efecto, es claro que $x\leq a'\implies x\inf a=0$;
mientras que, si $x\inf a=0$, entonces
\begin{align*}
    x
    &= x\inf(a'\sup a) \\
    &= (x\inf a')\sup (x\inf a) \\
    &= x\inf a',
\end{align*}
por lo cual $x\leq a'$.

Juntando estas observaciones, obtenemos el siguiente resultado:
\begin{lemma}
  \label{lemma:complementado-ssi-supneg}
   Si $A$ es una retícula distributiva,
   un elemento $a\in A$ con negación tiene complemento $a'$
   si, y solo si, $\neg a\sup a=1$ y, en este caso, $a'=\neg a$.
\end{lemma}

En particular, en un álgebra booleana $A$, todo elemento $a\in A$
tiene negación dada por el complemento: $\neg a = a'$.

De la definición de la negación, tenemos que
$x\leq \neg\neg a$ si, y solo si, $x\inf\neg a=0$.
Dado que $a\inf\neg a=0$, podemos deducir que $a\leq\neg\neg a$.
Aunque en un álgebra booleana se tiene la otra comparación
(es decir, $\neg\neg a = a$),
esto no es cierto en general:

\begin{example}
  \label{exa:negaciones-en-espacios-top}
  Sea $S$ un espacio topológico y consideremos el marco $\cal OS$.
  Todo abierto $u\in\cal O S$ tiene negación $\neg u$ en $\cal OS$.
  En efecto, para todo abierto $v\in\cal OS$ tenemos
  \begin{align*}
        u\cap v = \emptyset
        &\iff v\subseteq u' \\
        &\iff v\subseteq (u')^\circ \in\cal OS.
  \end{align*}
  Por lo tanto, $u$ tiene negación dada como
  $\neg u = (u')^\circ = {\ol u}'$.
  
  En particular, el abierto $u=(-1,0)\cup(0,1)\in\cal O\mathbb R$
  tiene cerradura $\ol u = [-1,1]$, así que
  \begin{align*}
    \neg\neg u
    &= \neg(\ol u ') \\
    &= \neg([-1,1]') \\
    &= ([-1,1]'')^\circ \\
    &= [-1,1]^\circ \\
    &= (-1,1) \\
    &\neq u.
  \end{align*}
\end{example}

Es decir, aunque $\neg a$ tenga negación, en general
solo tenemos la comparación $a\leq\neg\neg a$.
Sin embargo, si $\neg\neg a$ tiene negación, entonces sí se
cumple que $\neg\neg\neg a=\neg a$.
En efecto, dado que
\[
  a \leq \neg\neg a
,\]
al hacer ínfimo con $\neg\neg a$ obtenemos
\begin{align*}
  \neg\neg\neg a\inf a
  &\leq \neg\neg\neg a\inf \neg\neg a  \\
  &= 0,
\end{align*}
lo cual, por definición de la negación, es equivalente a
$\neg\neg\neg a\leq\neg a$, mientras que la otra comparación
$\neg a\leq\neg\neg\neg a$ ya la teníamos.

En el ejemplo \ref{exa:negaciones-en-espacios-top} encontramos
una retícula distributiva donde todo elemento tiene negación.
Ahora consideraremos ese tipo de retículas.

Por el lema \ref{lemma:complementado-ssi-supneg}, sabemos que
$a$ es tiene complemento $a'$ ssi $a\sup\neg a=1$.
Esto justifica el primer punto de la siguiente definición.
\begin{definition}
  Sea $A$ una retícula distributiva donde todo elemento tiene
  negación y tomemos un elemento $a\in A$.
  Decimos que
  \begin{enumerate}[(i)]
    \item $a$ es \emph{complementado} si $a\sup\neg a=1$,
    \item $a$ es \emph{regular} si $\neg\neg a = a$,
    \item $a$ es \emph{denso} si $\neg a = 0$.
  \end{enumerate}
\end{definition}

Nótese que $\neg a$ es regular para todo $a$, ya que
$\neg\neg\neg a =\neg a$.

\begin{example}
  Consideremos un espacio topológico $S$ y un abierto $u\in\cal
  OS$. Por el ejemplo \ref{exa:negaciones-en-espacios-top},
  sabemos que $u$ tiene negación dada como $\neg u=\ol
  u'=(u')^\circ$.
  Por lo tanto
  \begin{enumerate}
    \item
    $u$ es complementado en el sentido de retículas
    ($u\sup\neg u=1$)
    si, y solo si, su complemento es abierto.
    \item
    el abierto $u$ es regular en el sentido de retículas
    ($\neg\neg u = u$)
    si, y solo si, es regular en el sentido topológico
    ($(\ol u)^\circ=u$).
    \item
    Similarmente, $u$ es denso como elemento de la retículas
    ($\neg u=0$)
    si, y solo si, es denso en el sentido topológico
    ($\ol u=S$).
  \end{enumerate}
\end{example}

\begin{proposition}
  Un elemento $a$ de una retícula $A$ es denso
   si y solo si $\neg\neg a = 1$.
\end{proposition}
\begin{proof}
Tenemos
\begin{align*}
  \neg a = 0
  &\iff \neg a \leq 0 \\
  &\iff 1\inf \neg a \leq 0 \\
  &\iff 1\leq \neg\neg a \\
  &\iff 1=\neg\neg a.
\end{align*}
\end{proof}



\begin{proposition}
  Si $A$ es una retícula distributiva y $a,b\in A$ tienen
  negación, entonces $a\sup b$ también tiene negación y tenemos
  \[
    \neg(a\sup b) = \neg a \inf \neg b
  .\]
\end{proposition}
\begin{proof}
  Para cualquier $x\in A$, tenemos
  \begin{align*}
    x\inf(a\sup b) \leq 0
    &\iff (x\inf a)\sup(x\inf b) \leq 0 \\
    &\iff (x\inf a)\leq 0, \; (x\inf b) \leq 0 \\
    &\iff x\leq\neg a, \; x\leq\neg b \\
    &\iff x\leq(\neg a \inf \neg b).
  \end{align*}
  Así, $\neg a\inf\neg b$ es la negación de $a\sup b$, como se
  quería.
\end{proof}

Ahora, si en una retícula distrubutiva $A$ todos los elementos
tienen negación, entonces
\[
  \neg(a\sup\neg a) = \neg a\inf\neg\neg a = 0
,\]
por lo cual $a\sup\neg a$ siempre es denso.
Además
\begin{align*}
  \neg\neg a\inf(a\sup\neg a)
  &= (\neg\neg a\inf a)\sup(\neg\neg a\inf\neg a) \\
  &= a \inf 0 \\
  &= a,
\end{align*}
así que todo elemento $a\in A$ se puede expresar como el ínfimo
de un elemento denso y un elemento regular.

\begin{lemma}
  Sea $A$ una retícula distributiva donde todos los elementos
  tienen negación.
  Entonces $A$ es un álgebra booleana (todo elemento es
  complementado) si, y solo si, todo elemento es regular.
\end{lemma}
\begin{proof}
     Supongamos que $A$ es booleana, entonces todo
    $a\in A$ cumple $a\vee \neg a=1$.
    Luego $\neg a\vee \neg\neg a=1$, es decir, $\neg\neg a$ es
    complementado.
    Como el complemento de $\neg a$ es único, se tiene que $\neg
    \neg a=a$. \\ 
    Recíprocamente si que todos los elementos de $A$ son
    regulares y sea $a\in A$.
    Como $a$ y $a\sup\neg a$ son regulares, tenemos
    $a\sup\neg a=\neg\neg (a\sup\neg a)=\neg(\neg
    a\wedge \neg\neg a)=\neg 0=1$.
    Por lo tanto, $a\vee \neg a=1$, es decir, $a$ es
    complementado. Con ello $A$ es un álgebra booleana.
\end{proof}

\subsection{Implicaciones}

Una implicación en una semirretícula inferior $A$ es una operación
$(-\succ -)\colon A\rightarrow A$ tal que, para cualesquiera $a,x,y\in A$ se tiene
\[
  x\inf y\leq a \ssi x\leq (y\succ a )
.\]

\begin{example}
  En un álgebra booleana, nuestro caballo de batalla nos dice que
  \[
    x\inf y\leq a \ssi
    x\leq \neg y\sup a
  .\]
  Por lo tanto, toda álgebra booleana tiene implicación dada
  como $(y\succ a)=\neg y\sup a$.
\end{example}

Ahora vienen dos lemas técnicos.
\begin{lemma}\label{imp}
  Sea $A$ una $\inf$-semirretícula con implicación.
  Entonces
  \begin{enumerate}
    \item $(x\succ -)$ infla.
    \item $x\inf(x\succ a) = x\inf a$
    \item $(-\succ a)$ es antítona.
  \end{enumerate}
\end{lemma}
\begin{proof}
    \begin{enumerate}
        \item Como $a\wedge x\leq a$, tenemos
        $a\leq (x\succ a).$
        \item Consideremos $a, x\in A$.
        Para cada $z$ tenemos que 
        \begin{align*}
            z\leq x\wedge (x\succ a)
            & \iff z\leq x \mbox{ y } z\leq (x\succ a)\\
            & \iff z\leq x \mbox{ y } z\wedge x\leq a\\
            & \iff z\leq x \mbox{ y } z\leq a\\
            & \iff z\leq a\wedge x.
        \end{align*}
        Por lo tanto $x\inf(x\succ a) = x\inf a$.
        \item Supongamos que $x\leq y$. Consideremos $z=(y\succ a)$, entonces $x\wedge z\leq y\wedge z$ y por los incisos anteriores de este lema, $y\wedge z\leq a$. Así, $x\wedge z\leq a$ y por la definición de implicación obtenemos que $(y\succ a)\leq (x\succ a)$. 
    \end{enumerate}
\end{proof}

\begin{lemma}
  \label{lemma:w-cerradura}
  Sea $A$ una $\inf$-semirretícula con implicación.
  Para cualquier elemento $a\in A$, la función
  $((-\succ a)\succ a):A\to A$ tiene las siguientes propiedades.
  \begin{enumerate}
    \item $((-\succ a)\succ a)$ infla.
    (Inmediado del punto 2 del lema anterior).
    \item $((-\succ a)\succ a)$ es monótona.
      (Inmediato del punto 3 del lema anterior).
    \item $((-\succ a)\succ a)$ es idempotente.
  \end{enumerate}
\end{lemma}
\begin{proof}
  Solo falta probar el punto 3.
  Por el punto 1, tenemos
  \[
    (x\succ a) \leq (((x\succ a)\succ a)\succ a),
  \]
  así que resta probar
  \[
     (((x\succ a)\succ a)\succ a) \leq (x\succ a),
  \]
  que, por definición de la implicación, equivale a
  \[
     (((x\succ a)\succ a)\succ a) \inf x \leq a.
  \]
  Recordemos que $x\leq ((x\succ a)\succ a)$, porque $((-\succ
  a)\succ a)$ infla, y que
  $(y\succ a)\inf y = y\inf a$ para todo $y\in A$.
  En particular, para $y=((x\succ a)\succ a)$, tenemos
  \begin{align*}
     (((x\succ a)\succ a)\succ a) \inf x
     &\leq (((x\succ a)\succ a)\succ a)
       \inf ((x\succ a)\succ a) \\
     &= (y\succ a) \inf y \\
     &= y \inf a \\
     &\leq a,
  \end{align*}
  como se quería. 
\end{proof}

En el capítulo \ref{ch:cocientes} definiremos los operadores
cerradura como funciones $A\to A$ que cumplen estas tres
propiedades (definición \ref{def:operador-cerradura}).
Allí veremos que estos operadores juegan un papel
importante en el contexto de $\Sup$-semirretículas.

Así, el resultado anterior dice que, si $A$ es una
$\inf$-semirretícula con implicación,
entonces $((-\succ a)\succ a)$ es un operador cerradura.
Además, el operador $((-\succ a)\succ a)$ tiene otra propiedad:
preserva ínfimos.

\begin{lemma}
  \label{lemma:modalidad-w}
  Si $A$ es una $\inf$-semirretícula con implicación, entonces
  \[
    (((x\inf y)\succ a)\succ a)
    =
    ((x\succ a)\succ a) \inf
    ((y\succ a)\succ a)
  .\]
\end{lemma}
\begin{proof}
  Por un lado, como $((-\succ a)\succ a)$ es monótono, tenemos
  \begin{align*}
    (((x\inf y)\succ a)\succ a)
    &\leq ((x\succ a)\succ a)
    \\
    (((x\inf y)\succ a)\succ a)
    &\leq ((y\succ a)\succ a)
  \end{align*}
  así que $(((x\inf y)\succ a)\succ a)$ es cota inferior de
  $((x\succ a)\succ a)$ y $((y\succ a)\succ a)$.

  Ahora, dada cualquier cota inferior $z$ de estos dos elementos,
  queremos probar que $z \leq (((x\inf y)\succ a)\succ a)$,
  lo cual equivale a $z\inf((x\inf y)\succ a)\leq a$.
  Como $z$ es cota inferior, tenemos
  \begin{align*}
    z&\leq ((x\succ a)\succ a)
    &
    z&\leq ((y\succ a)\succ a)
  \end{align*}
  lo cual, por definición de la implicación, es
  \begin{align*}
    z\inf (x\succ a) &\leq a
    &
    z\inf (y\succ a) &\leq a.
  \end{align*}
  Ahora sea $w=((x\inf y)\succ a)$, de modo que
  \begin{align*}
    z\inf w\inf x\inf y
    &= z\inf ((x\inf y)\succ a)\inf(x\inf y) \\
    &\leq z\inf a \\
    &\leq a
  \end{align*}
  por definición de la implicación, esto nos da
  \begin{align*}
    z\inf w \inf x &\leq (y\succ a).
  \end{align*}
  Al hacer ínfimo con $z$ obtenemos
  \begin{align*}
    z\inf w \inf x
    &\leq z\inf (y\succ a) \leq a
  \end{align*}
  aplicando de nuevo la definición de la implicación,
  \begin{align*}
    z\inf w &\leq (x\succ a)
  \end{align*}
  y haciendo ínfimo con $z$,
  \begin{align*}
    z\inf w &\leq z\inf (x\succ a) \leq a
  \end{align*}
  Luego, $z\leq (w\succ a)=(((x\inf y)\succ a)\succ a)$,
  como se quería.
\end{proof}

También en el capítulo \ref{ch:cocientes}, veremos que
los operadores cerradura con esta propiedad, llamados núcleos
(definición \ref{def:nucleo}) tienen gran
relevancia en el contexto de marcos.


\section{Ejercicios}\label{exes}

En esta sección veremos varias afirmaciones en forma de ejercicios, esto no solo para que el lector reafirme el contenido, si no, a su vez, para complementar algunas cosas de la teoría.
\begin{itemize}
  \item[(1)]:
  Demuestre que en efecto, los siguientes datos forman una categoría en el sentido de \ref{def:cats}.
  \begin{enumerate}
    \item $\Pos$.
    \item $\Pos^{\wedge}$.
    \item $\Pos^{\vee}$.
    \item $\Lat$
    \item $\DLat$
    \item $\InfLat$
    \item $\Heyt$
    \item $\cHeyt$
    \item $\Bool$
    \item $\Frm$.
  
  \end{enumerate}
\item[(2)]
Veamos que en efecto toda topología esta dada por un operador cerradura.

   Sea $S$ un conjunto y consideremos una topología en $S$, digamos $\mathcal{O}S$.
  Para cualquier subconjunto $A$ de $S$ sea \[J_{\mathcal{O}S}(A)=\bigcup\{U\subseteq A\mid U\in\mathcal{O}S\}\]
  
\begin{enumerate}
  \item[(i)] Demuestre que $J_{\mathcal{O}S}(\_)\colon\mathcal{P}(S)\rightarrow\mathcal{P}(S)$ es una función que satisface:
  \begin{itemize}
    \item[(i.1)] $J_{\mathcal{O}S}(\_)$ es monótona e idempotente.
    \item[(i.2)] $J_{\mathcal{O}S}(A)\subseteq A$ para todo $A\in\mathcal{P}(S)$.
    \item[(ii.3)] $J_{\mathcal{O}S}(A\cup B)\subseteq J_{\mathcal{O}S}(A)\cup J_{\mathcal{O}S}(B)$ (y así se da la igualdad). 
    \item[(iii.4)] $\mathcal{P}(S)_{J_{\mathcal{O}S}(\_)}=\mathcal{O}(S)$. 
  \end{itemize} 
\end{enumerate}






\end{itemize}
