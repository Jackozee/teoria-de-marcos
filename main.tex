\documentclass[12pt,letterpaper,titlepage]{article}
\usepackage[left=3.5cm,right=3.5cm,top=2.5cm,bottom=2.5cm]{geometry}

\usepackage{amsmath,amsfonts,amsthm,mathrsfs,amssymb}
\usepackage{enumitem,mathtools,thmtools,tikz-cd,multicol,stmaryrd,hyperref}
\usepackage{MnSymbol,wasysym}
\let\emptyset\varnothing
\newtheorem*{defn}{Definición}
\newtheorem{exe}{Ejercicio}
\newtheorem*{exa}{Ejemplo}
\newtheorem*{lemma}{Lema}
\newtheorem*{cor}{Corolario}
\newtheorem*{thm}{Teorema}
\newtheorem*{prop}{Proposición}
\theoremstyle{definition}
\newtheorem*{sol}{Solución}
\renewcommand\sup{\vee}
\newcommand\Sup{\bigvee}
\newcommand\down{{\downarrow}}
\newcommand\ol[1]{\overline{#1}}
\renewcommand\inf{\wedge}
\renewcommand\phi{\varphi}
\newcommand\Inf{\bigwedge}
\newcommand\psup{%
    {\;\stackrel{\raisebox{-3pt}{\scalebox{0.4}{$\bullet$}}}{\vee}\;}
  }%
\newcommand\pSup{%
    {\stackrel{\raisebox{-3pt}{\scalebox{0.5}{$\bullet$}}}{\bigvee}}
  }%
\newcommand\N{\mathbb N}
\newcommand\Z{\mathbb Z}
\newcommand\Q{\mathbb Q}
\newcommand\D{\mathcal D}
\renewcommand\cal[1]{\mathcal{#1}}
\newcommand\scr[1]{\mathscr{#1}}
\newcommand\rar[1]{\xrightarrow{#1}}
\newcommand\lar[1]{\xleftarrow{#1}}
\newcommand\simr{{\sim}}
\newcommand\ssi{\hspace{5mm}\text{ si, y solo si }\hspace{5mm}}
\newcommand\com[2]{\ar[equal,from={#1},to={#2},shorten=5mm]}
\newcommand\unuc[1]{\mathbf u_{#1}}
\newcommand\vnuc[1]{\mathbf v_{#1}}
\newcommand\wnuc[1]{\mathbf w_{#1}}
\newcommand\tps[1]{\texorpdfstring{#1}{}}
\newcommand\fprod{A}
\newcommand\<{\langle}
\renewcommand\>{\rangle}
\newcommand\Idl{\text{-}\mathrm{Idl}}

\renewcommand\proofname{Demostración}
\renewcommand\partname{Parte}
\renewcommand\contentsname{Contenido}
\renewcommand\listtheoremname{Lista de ejercicios}

\DeclareMathOperator{\Frm}{Frm}
\DeclareMathOperator{\Con}{Con}
\DeclareMathOperator{\Top}{Top}
\DeclareMathOperator{\Pos}{Pos}
\DeclareMathOperator{\Vect}{Vect}
\DeclareMathOperator{\Ord}{Ord}
\DeclareMathOperator{\Mod}{Mod}
\DeclareMathOperator{\Ab}{Ab}
\DeclareMathOperator{\Ob}{Ob}
\DeclareMathOperator{\Mor}{Mor}
\DeclareMathOperator{\Fld}{Fld}
\DeclareMathOperator{\id}{id}
\DeclareMathOperator{\tp}{tp}
\DeclareMathOperator{\img}{im}

\title{Introducción a la teoría de marcos}
\author{Curso impartido por: Luis Ángel Zaldívar Corichi \\
\small{Notas: sus alumnos}}
\date{Otoño 2020}

\begin{document}
\maketitle

La interacción sin puntos del álgebra y la topología.
Temario (tentativo).
\begin{enumerate}[label=\Roman*.]
  \item Preliminares de la teoría de marcos. Aspectos básicos.
  \begin{itemize}
      \item Cocientes.
      \item Completaciones ordenadas (si hay tiempo).
  \end{itemize}
  \item El ensamble de un marco.
  \begin{itemize}
      \item Derivadas, prenúcleos, estables y núcleos.
      \item Funtorialidad de $N:\Frm\to\Frm$.
      \item Representaciones: cálculos en $NA$.
  \end{itemize}
  \item El espacio de puntos de un marco.
  \item Lo que gusten y lo que alcance.
\end{enumerate}

En la parte IV podemos ver un poco de el problema de reflexión
booleana, los axiomas de
separación en $\Frm$, espacios espectrales, gavillas en un marco,
etc. según lo que se desee y el tiempo que haya.

\newpage
\tableofcontents

\paragraph{Ejercicios por persona:}
\begin{multicols}{2}
\begin{itemize}
    \item Alfredo: 1,5,9,14,19,24,29,34.
    \item Armando: 2,6,10,15,20,25,30,35.
    \item Dante: 3,7,11,16,21,26,31.
    \item Yareli: 4,8,12,17,22,27,32.
    \item Juan: 13,18,23,28,33.
\end{itemize}
\end{multicols}

\newpage
\listoftheorems[ignoreall,show=exe]

\part{Preliminares}

\section{Aspectos básicos}

\subsection{Categorías}
Una categoría consiste en los siguientes datos:
\begin{itemize}
  \item una colección de objetos $\Ob\cal C$.
  \item para cada par de objetos $A,B\in\Ob\cal C$, una colección
    $\cal C(A,B)$,
  \item para cada objeto $A\in\Ob\cal C$
      un morfismo $\id_A\in\cal C(A,A)$ llamado
      \emph{identidad} de $A$,
  \item para cada tercia de objetos $A,B,C\in\Ob\cal C$, una
    función
    \begin{align*}
        \circ:\cal C(A,B)\times\cal C(B,C)&\to\cal C(C,D) \\
        (f,g)&\mapsto g\circ f\equiv gf
    \end{align*}
    llamada composición,
\end{itemize}
sujetos a las condiciones:
\begin{itemize}
  \item asociatividad: $(fg)h=f(gh)$,
  \item identidad: $\id_B f = f \id_A = f$,
\end{itemize}
siempre que éstas tengan sentido.

Algunos autores usan las notaciones $\text{Hom}(A,B)$,
$\text{Hom}_{\cal C}(A,B)$, etc, para lo que nosotros llamos $\cal
C(A,B)$.

\begin{exa}
\begin{itemize}
  \item Sean $\Ob\cal C=\{\bullet\}$ y $\cal
    C(\bullet,\bullet)=\N$, donde la composición está definida
    como multiplicación.
  \item
  En general, cualquier monoide $M$ se puede pensar como una
  categoría definiendo $\cal C(\bullet,\bullet)=M$
  y declarando que la composición es la operación del monoide.
  \item Sean $\Ob\cal C=\N$ y
    \[
      \cal C(m,n) =
      \begin{cases}
        \{\bullet\}, & m\mid n \\
        \emptyset, & m\nmid n.
      \end{cases}
    \]
  \item
  En general, cualquier conjunto parcialmente ordenado $A$
  se puede pensar como una categoría definiendo $\Ob\cal C=A$
  y
    \[
      \cal C(a,b) =
      \begin{cases}
        \{\bullet\}, & m\leq n \\
        \emptyset, & m\nleq n.
      \end{cases}
    \]
  \item La categoría de conjuntos $\Con$, donde los objetos
  son los conjuntos, las flechas son las funciones de conjuntos
  y la composición es la composición usual.
  \item Dado un campo $k$, la categoría $\Vect_k$ de $k$-espacios
    vectoriales sobre $k$.
  \item La categoría $\Top$ de espacios topológicos.
  \item La categoría de extensiones de campos.
 \end{itemize}
\end{exa}

\begin{exe}[Alfredo $\checkmark$ ]
  Probar que la categoría de extensiones de campos es una categoría.
\end{exe}
\begin{proof}
  Sea $\Fld$ la categoría de campos.
  Definiremos una categoría $\Fld^\to$.
  Los objetos de $\Fld^\to$ son las extensiones de campos, esto es,
  los morfismos de $\Fld$.
  \[
    \Ob(\Fld^\to) = \Mor(\Fld)
  .\]
  Dados $(k\rar f L),(k'\rar gL')\in\Ob(\Fld^\to)$, definimos
  los morfismos de $f$ a $g$ en la categoría $\Fld^\to$
  como morfismos de extensiones; esto es,
  pares de morfismos de campos que son compatibles con
  las extensiones:
  \[
    \Fld^\to(f,g)
    = \left\{
      (u,v)\in \Fld(k,k')\times \Fld(L,L')
      \middle|
      \begin{tikzcd}
        L \ar[r,"v"'] & L' \\
        k \ar[u,"f"'] \ar[r,"u"] & k' \ar[u,"g"]
        \ar[equal,from=1-1,to=2-2,shorten=5mm]
      \end{tikzcd}
    \right\}
  .\]
  Dados $f,g,h\in\Ob(\Fld^\to)$, definimos la composición como
  \begin{align*}
    \Fld^\to(f,g)\times\Fld^\to(g,h) &\to \Fld^\to(f,h) \\
    ((u,v) , (x,y)) &\mapsto (x,y)(u,v)=(xu,yv).
  \end{align*}
  En un diagrama:
  \[
      \begin{tikzcd}
        L \ar[r,"v"'] \ar[rr,bend left,"yv"] & L' \ar[r,"y"'] & L''\\
        k \ar[u,"f"'] \ar[r,"u"] \ar[rr,bend right,"xu"']
        & k' \ar[u,"g"] \ar[r,"x"] & k''\ar[u,"h"]
        \ar[equal,from=1-1,to=2-2,shorten=5mm]
        \ar[equal,from=1-2,to=2-3,shorten=5mm]
      \end{tikzcd}
  .\]
  La conmutatividad del rectángulo exterior se hereda de la
  conmutatividad de los dos cuadrados:
    \[
        hxu = ygu = yvf
    .\]
  La asociatividad de la composición se hereda de la
  asociatividad de la composición de morfismos de campos
  \[
      \begin{tikzcd}
        L \ar[r,"v"'] & L' \ar[r,"y"'] & L'' \ar[r,"b"'] & L'''\\
        k \ar[u,"f"'] \ar[r,"u"]
        & k' \ar[u,"g"] \ar[r,"x"]
        & k''\ar[u,"h"] \ar[r,"a"]
        & k''' \ar[u,"j"]
        \ar[equal,from=1-1,to=2-2,shorten=5mm]
        \ar[equal,from=1-2,to=2-3,shorten=5mm]
        \ar[equal,from=1-3,to=2-4,shorten=5mm]
      \end{tikzcd}
  .\]
  Esto es:
  \begin{align*}
      ((a,b)(x,y))(u,v)
      &= (ax,by)(u,v) \\
      &= (axu,byv) \\
      &= (a,b)(xu,yv) \\
      &= (a,b)((x,y)(u,v))
  \end{align*}
  La identidad de la extensión $f:k\to L$ es
  $\id_f = (\id_k,\id_L):f\to f$.
  En efecto, dadas $(u,v):f\to g$ y $(c,d):h\to f$, tenemos
  \begin{align*}
      (u,v)(\id_k,\id_L) &= (u \id_k,v \id_L) = (u,v) \\
      (\id_k,\id_L)(c,d) &= (\id_k c,\id_L d) = (c,d).
  \end{align*}
\end{proof}

\subsection{Funtores}

1:27:00 (video 1)

Un morfismo de espacios topológicos es compatible con
las topologías, en el sentido de que el morfismo preimagen
manda abiertos en abiertos.
Similarmente, un funtor entre dos categorías $\cal C$ y $\cal D$
consiste de...
\begin{itemize}
  \item Una asignación entre objetos...
  \item Una asignación entre flechas...
\end{itemize}

tales que
\begin{itemize}
  \item Preserva composición..
  \item Preserva la identidad..
\end{itemize}

\begin{exa}
  \item El funtor identidad manda todo objeto a sí mismo y todo
    morfismo a sí mismo.
  \item Para un grupo abeliano $G$, sea $TG$ el conjunto de todos
    los elementos de $G$ con orden finito.
    Si $G=TG$ decimos que $G$ es de torsión, y si $TG=0$ decimos
    que $G$ es libre de torsión.
\end{exa}

\begin{exe}[Armando $\checkmark$]
  Mostrar que la construcción del grupo de torsión
  es un funtor $T:\Ab\to\Ab$.
\end{exe}
\begin{sol}
    Primero, sean $a,b\in T_G$ arbitrarios, entonces, existen $n,m\in\Z^+$ tales que
    \begin{equation*}
        a^n=e,\ b^m=e,\ e\in G \text{ el neutro de } G.
    \end{equation*}
    Si consideramos a $p:=mn$, tenemos que 
    \begin{align*}
        (ab)^p & = a^p b^p \\
               & = (a^n)^m (b^m)^n \\
               & = e^m e^n \\
               & = e \cdot e = e
    \end{align*}
    con $p=mn\in\Z^+$. Ya que $a,b$ son arbitrarios, se sigue que $ab\in T_G$ y en consecuencia $T_G$ es cerrado bajo el producto de $G$.\\
    Ahora, sea $c\in T_G$ arbitrario, entonces existe $k\in\Z^+$ tal que $c^k=e$. Ahora nótese que:
    \begin{align*}
        cc^{-1} & = e \\
        (cc^{-1})^k & = e^k = e \\
        \underbrace{c^k}_{=e}(c^{-1})^k & = e \\
        (c^{-1})^k & = e,\quad k\in\Z^+.
    \end{align*}
    Así, como $c$ es arbitrario, obtenemos que $c^{-1}\in T_G$ y por ende $T_G$ contiene a los inversos.\\
    Con lo anterior, concluimos que $T_G$ es un subgrupo de $G$ y por lo tanto, ya que $G$ es arbitrario, el functor $T$ manda objetos de $Ab$ en objetos de $Ab$.
    Continuando, para una función $(A\overset{f}{\to}B)\in Ab$, nótese que $(T_A\overset{Tf}{\to}T_B)$ se define cómo $Tf=f\mid_{T_A}$, la restricción de $f$ en $T_A$.
    Ahora, para $a\in T_A$ arbitrario, se cumple que $a^n=e$ para algún $n\in\Z^+$. Luego:
    \begin{align*}
        f(a^n) = f(e) & = e \\
        f\left(\underbrace{a\cdot a \cdots a}_{n \text{ veces}}\right) & = e \\
        \text{Como $f\in Ab$ ,} \underbrace{f(a)f(a)\cdots f(a)}_{n\text{ veces}} & = e \\
        \left(f(a)\right)^n & = e\\
    \end{align*}
    Así, $(Tf)(a)=f(a)\in T_B$, y cómo $a$ es arbitrario, se tiene que $Tf:T_A\to T_B$ está bien definido.\\
    Luego, nótese que $(Tf)(e)=f(e)=e$, y se stiene qude $e^1 = e$, por tanto, $Tf$ preserva al neutro.\\
    Con esto, $Tf$ es un morfismo de Grupos Abelianos.\\
    Finalmente, es claro que $T1_A=1_{T_A}$, por lo tanto $T$ preserva morfismos, y en conclusión, es un endofunctor de $Ab$. 
\end{sol}

(sesión 2)
Un funtor es contravariante si cambia la dirección de las
flechas.

Una función $f:A\to B$ entre conjuntos parcialmente ordenados
(COPOs) es un morfismo de COPOs si es un funtor, cuando vemos a
$A$ y a $B$ como conjuntos parcialmente ordenados.

\subsection{[Semi]retículas, distributividad y álgebras booleanas}

Una retícula es un COPO tal que el supremo y el ínfimo de dos
elementos siempre existe.

Una retícula es acotada si existen el ínfimo y el supremo vacíos.

Notemos que no se está pidiendo que existan el ínfimo y el supremo de
familias arbitrarias.

Un retícula acotada es distributiva si la siguiente ley es
válida:
\[
  a\sup(b\inf c) = (a\sup b)\inf (a\sup c)
,\]
o bien
\[
  a\inf(b\sup c) = (a\inf b)\sup(a\inf c)
.\]
\begin{exe}[Dante $\checkmark$ ]
  Mostrar que las dos leyes distributivas son equivalentes.
\end{exe}
\begin{sol}
    1. Supóngase que $a\sup(b\inf c)=(a\sup b)\inf(a\sup c)$. Así, se cumple que
    \begin{align*}
        a\inf(b\sup c)&\leq (a\sup b)\inf(b\sup c)=b\sup(a\inf c)\\
        &\leq a\inf(a\inf(b\sup c))\\
        &\leq a\inf(b\sup(a\inf c))\\
        \Rightarrow a\inf(b\sup c)&\leq (a\inf b)\sup(a\inf c)
    \end{align*}
    Por otro lado, 
    \begin{align*}
        (a\inf b)\sup (a\inf c)&=((a\inf b)\sup a)\inf((a\inf b)\sup c)\\
        &=a\inf((a\inf b)\sup c)\\
        \leq a\inf(b\sup c)
    \end{align*}
Por lo que $a\inf(b\sup c)=(a\inf b)\sup(a\inf$ c).\vspace{3mm}

2. Ahora, supóngase que $a\inf(b\sup c)=(a\inf b)\sup(a\inf c) $. Así, se tiene que
\begin{align*}
    a\sup(n\inf c)&\geq(a\inf b)\sup(b\inf c)=b\inf(a\sup c)\\
    \Rightarrow (a\sup a) \sup (b\inf c) &\geq (a\sup b)\inf(a\cup c)\\
    a\sup(b\inf c)&\geq (a\sup b)\inf(a\sup c)
\end{align*}
Por otro lado, 
\begin{align*}
    (a\sup b)\inf(a\sup c)&=((a\sup b)\inf a)\sup((a\sup b)\inf c)\\
    &=a\sup((a\sup b)\inf c)\\
    &\geq a\sup(b\inf c)
\end{align*}
Por lo que $a\sup(b\inf c)=(a\sup b)\inf(a\sup c)$\vspace{2mm}

\hfill $\blacksquare$
\end{sol}
Una retícula distributiva es un álgebra boolena si para todo
$a\in A$ existe un elemento $b\in A$ tal que
\[
  a\inf b = 0 \hspace{10mm} a\sup b = 1
.\]

\begin{exe}[Yareli $\checkmark$ ]
  Probar que este elemento es único.
\end{exe}
\begin{proof}
  Sean $A$ un álgebra booleana y $a,b,b'\in A$ tales que
\begin{align*}
&a\wedge b=0 &a\vee b=1\\
&a\wedge b'=0 &a\vee b'=1.
\end{align*}
Observemos que 
\[b'=b'\vee 0=b'\vee (a \wedge b)=(b'\vee a)\wedge (b'\vee b)=1\wedge (b'\vee b)=b'\vee b.\]
Además,
\[b=b\vee 0=b\vee (a\wedge b')=(b\vee a)\wedge (b\vee b')=1\wedge (b\vee b')=b\vee b'.\]
Así, $b'=b$.
\end{proof}



Definimos los morfismos de la categoría de retículas
distributivas como morfismos que preservan ínfimos y supremos
finitos.

Supongamos que $A$ es una COPO donde todo subconjunto finito
tiene un supremo.
A esto se le llama $\sup$-semiretícula.
Entonces $A$ es un monoide conmutativo donde todo elemento es
idempotente.

De igual forma, se puede hacer esto con los ínfimos, y en ese
caso hablamos de $\inf$-semiretículas.

\begin{exe}[Alfredo $\checkmark$ ]
  Si $A$ es un monoide conmutativo en el cual todo elemento es
  idempotente, entonces existe un orden parcial en $A$ tal que el
  supremo es la operación con la que se empezó.
\end{exe}
\begin{sol}
    Sea $A$ un monoide conmutativo donde todo elemento es idempotente.
    Dados $a,b$, decimos que $a\leq b$ si, y solo si, $a\sup b=b$.
    En efecto, esto es un orden parcial:
    \begin{itemize}
        \item (Refl). Como $a$ es idempotente, tenemos $a\sup a=a$.
        Luego, $a\leq a$.
        \item (Antisim). Supongamos que $a\leq b$ y $b\leq a$.
        Es decir, $a\sup b=b$ y $b\sup a=a$.
        Como $\sup$ es conmutativo, tenemos
        \[
            b = a\sup b = b\sup a = a
        .\]
        \item (Trans). Supongamos que $a\leq b$ y que $b\leq c$.
        Es decir, $a\sup b = b$ y $b\sup c = c$.
        Como $\sup$ es asociativo, tenemos
        \[
            a\sup c = a\sup(b\sup c) = (a\sup b)\sup c = b\sup c = c
        .\]
        Esto es, $a\leq c$.
    \end{itemize}
    Los tres puntos anteriores muestran que la relación $\leq$ es
    un orden parcial.
    Ahora mostraremos que $\sup$ es el supremo de este orden.
    Sean $a,b\in A$, y supongamos que $c\in A$ es tal que $a,b\leq c$.
    Esto es, $a\sup c = c$ y $b\sup c = c$.
    Luego,
    \[
        (a\sup b)\sup c = (a\sup c)\sup(b\sup c) = c\sup c = c
    .\]
    Por lo tanto, $a\sup b\leq c$.
    Notemos que se usaron todas las propiedades del monoide conmutativo.
\end{sol}

Si $A$ es una $\inf$-semiretícula, nos gustaría encontrar una
retícula distributiva $\hat A$ que, en cierto sentido, sea la
completación de $A$.

\subsection{Toda semiretícula completa es una retícula completa.}
Podemos definir $\Sup$-semiretículas ($\Inf$-semiretículas) como
semiretículas donde todo subconjunto tiene supremo (ínfimo).

Si $A$ es una $\Sup$-semiretícula, cualquier subconjunto
$X\subseteq A$ tiene ínfimo dado como
\[
  \Inf X = \Sup\{c\in A \mid \forall x\in X, c\leq x\}
.\]
Es decir, $A$ es completa.

Dualmente, cualquier $\Inf$-semiretícula es completa.

\subsection{La completación de secciones inferiores}
Si $A$ es una $\inf$-semiretícula, nos gustaría encontrar una
retícula distributiva $\hat A$ que "complete" a $A$, en el
sentido de que le agregue los supremos.

¿Qué tal el conjunto potencia $\cal PA$?
Es un álgebra booleana completa, tiene leyes distributivas
fuertes; quizá demasiado fuertes.
Además, la retícula que buscamos debería tener a $A$ como una
subretícula.
La función obvia $A\to\cal PA$ dada por $a\mapsto\{a\}$
no preserva el orden, así que no es una inclusión de
retículas.
Vamos a refinar esta situación.

Si $A$ es un conjunto ordenado, una sección inferior de $A$
es un subconjunto $L\subseteq A$ tal que, "absorbe hacia abajo".
Es decir, si $a\leq b\in L$, entonces $a\in L$.

Denotemos como $\cal LA$ al conjunto de todas las secciones
inferiores en $A$.
Observemos que $\cal LA$ es una subretícula
de $\cal PA$.
Ahora observemos que podemos definir una función
\[
  \down:\cal PA\to\cal PA
\]
que manda un subconjunto $F\subseteq A$ a la sección inferior
generada por $F$:
\[
  \down F = \{a\in A \mid \exists c\in F , a\leq c\}
.\]

No es difícil ver que $\down$ es idempotente, monótona, infla, y su
conjunto fijo es $\cal LA$.
¿Qué pasa con la igualdad $\down(F\cap G)=\down F\cap\down G$?
\begin{exe}[Armando $\checkmark$ ]
  Probar que, para $F\inf G := \{x\inf y \mid x\in F, y\in G\}$ se cumple la igualdad $\down(F\inf G)= \down F\cap\down G $.
  \begin{sol}
      Primero, nótese que $a\in \downarrow(F\wedge G)$ si y sólo si existe $c\in (F\wedge G)$ tal que $a\leq c$ para un elemento arbitrario $a$. Pero, ya que $c=x\wedge y$, para algún $x\in X$, $y\in Y$, lo anterior es equivalente a $$a\leq x \wedge y$$ lo cuál pasa si y sólo si $a\leq x$ y $a\leq y$, luego, esto se cumple si y sólo si $a\in\downarrow F$ y $a\in\downarrow G$, que finalmente sucede si y sólo si $a\in \downarrow F \cap \downarrow G$.
      Así, como $a$ era arbitrario, concluimos que se cumple la igualdad $\downarrow(F\wedge G) = \downarrow F \cap \downarrow G$.
  \end{sol}
\end{exe}

De este último ejercicio, se sigue que la función
\begin{align*}
    \down : A&\to \cal LA \\
    a&\mapsto \down a
\end{align*}
es un morfismo de $\inf$-semiretículas.
Como veremos más adelante, $\cal LA$ es,
en cierto sentido, la mejor manera de completar a $A$.

\subsection*{(SESIÓN 3: 14 SEP)}
\subsection{Marcos}
Definición de marcos

Además, ya tenemos ejemplos de marcos $PA$ y $\cal LA$, donde $A$
es un COPO.

Otro ejemplo de marco es la topología de cualquier espacio
topológico.

Un morfismo de marcos es...

\begin{exe}[Dante $\checkmark$ ]
  Probar que los marcos con sus morfismos forman una categoría.
\end{exe}
\begin{sol}
    Sean $A,B,C\in Frm$ y $f:A\to B, g:B\to C$ morfismos de marcos, y sea $h=gf$.
    \begin{enumerate}
        \item Sean $a,b\in A$ con $a\leq b$. Así, $f(a)\leq f(b)$, por lo que $g(f(a))\leq g(f(b)) \iff h(a)\leq h(b)$. 
        \item Sean, $a,b\in A$ cualesquiera. 
        \begin{align*}
        h(c\inf d)&=g(f(c\inf d))\\
        &=g(f(c)\inf f(d))\\
        &=g(f(c))\inf g(f(d))\\
        &=h(a)\inf h(b)
        \end{align*}
        \item Sea $x\subset A$. 
        \begin{align*}
            h(\Sup X)&=g(f(\Sup X))\\
            &=g(\Sup f(X))\\
            &=\Sup g(f(X))\\
            &=\Sup h(X)
        \end{align*}
        \item Considérese el morfismo identidad $Id_A:a\to A$, y sea $j:C\to A$ morfismo de marcos.. Nótese que como $Id_A(a)=a$, entonces $Id_A\circ j(c)=Id_Aj(c)=j(c)$ por lo que $Id_A\circ j=j$. También, $f\circ Id_A(x)=f(x)$ por lo que $f\circ Id_A=f$.
    \end{enumerate}
    Por lo tanto, $Frm$ es una categoría.\vspace{3mm}
    
    \hfill $\blacksquare$
\end{sol}\vspace{3mm}

Observemos que no es necesariamente cierto que los morfismos de
marcos preserven ínfimos arbitrarios, ni es cierto, en general,
que los supremos finitos distribuyen sobre ínfimos arbitrarios.

Entonces tenemos la categoría de marcos.

Notemos que tenemos un funtor contravariante $\scr O:\Top\to\Frm$.
Más adelante veremos que este funtor es parte de una adjunción.

De la definición, también tenemos que un marco es una retícula
distributiva.

Si $A$ es una retícula, entonces una
negación para $a\in A$ es un elemento $b\in A$ tal que
$x\leq b$ si, y solo si, $x\inf a=0$.
Si existe, la negación es única y se le denota como $\neg a$, de
modo que tenemos 
$x\leq \neg a$ si, y solo si, $x\inf a=0$.
En particular, observemos que $\neg a\inf a = 0$.

\begin{exa}
  Si $K\in PS$, entonces su negación es...
\end{exa}

Observemos que $x\leq \neg\neg a$ si, y solo si, $x\inf\neg a=
0$.
Se sigue que $a\leq\neg\neg a$ pero, en general, $\neg\neg a \neq a$.

\begin{exa}
  Sea $S$ un espacio topológico.
  Si $U\in\cal O S$, entonces su negación...
\end{exa}

\subsection{Álgebras booleanas como marcos}
Un álgebra booleana completa es un álgebra booleana que es
completa como retícula.

Si $\neg a$ es el complemento de $A$, entonces $\neg\neg a=a$.

Algunas curiosidades.
\begin{lemma}[Caballo de batalla]
  \[
    a\inf x \leq y  \ssi x\leq \neg a\sup y
  .\]
\end{lemma}
\begin{lemma}
  Cada álgebra boolena es un marco.
\end{lemma}
Observemos que, si $A$ es un álgebra booleana completa, entonces
la negación de todo elemento es, justamente, su complemento.
\begin{lemma}
  Sea $A$ un álgebra booleana completa y $X\subseteq A$.
  Entonces
  \[
    \neg \Inf X = \Sup \{\neg x \mid x\in X\}
  .\]
  Notemos que esto es como las leyes de DeMorgan.
\end{lemma}

Tenemos dos categorías cuyos objetos son álgebras booleanass
completas: una con morfismos de retículas y la otra con morfismos
completos.

Supongamos que $f:A\to B$, donde $A$ es un álgebra booleana
completa y $X\subseteq A$.
Mostraremos que, de hecho $f$ también preserva ínfimos
arbitrarios.

Es decir, la categoría de álgebras booleanas completas con
morfismos completos es una subcategoría plena de la categoría de
marcos.

\subsection{VIDEO 1: categorías y adjunciones (21 SEP)}
\begin{itemize}
  \item Funtores covariantes y contravariantes.
  \item Funtores plenos y fieles.
  \item Transformaciones naturales.
  \item Composición de transformaciones naturales.
  \item La categoría de funtores entre dos categorías*.
  \item Isomorfismo natural.
  \item Ejemplo: $V\simeq V^{**}$ es natural en $V$.
  \item Equivalencia de categorías**.
  \item Adjunciones como isomorfismos naturales de homs
    (axioma de naturalidad).
\end{itemize}
\begin{exe}[Yareli $\checkmark$ ]
    (*) Probar que la categoría de funtores es una categoría.
\end{exe}
\begin{proof}
  Consideramos la clase de los funtores $F\colon C\to D$. Para las categorías $C,D$ fijas.\\
Consideramos las transformaciones naturales como morfismos.\\
Sean $F,G,H,I:C\to D$ funtores y $\alpha\colon F\to G$, $\beta\colon G\to H$ y $\gamma\colon H\to I$ transformaciones naturales. Tenemos que el siguiente diagrama

\[
\begin{tikzcd}
F(A) \arrow[d, "F(f)"'] \arrow[r, "\alpha_A"] & G(A) \arrow[d, "G(f)" description] \arrow[r, "\beta_A"] & H(A) \arrow[d, "H(f)" description] \arrow[r, "\gamma_A"] & I(A) \arrow[d, "I(f)"] \\
F(A') \arrow[r, "\alpha_{A'}"']               & G(A') \arrow[r, "\beta_{A'}"']                          & H(A') \arrow[r, "\gamma_{A'}"']                          & I(A')                 
\end{tikzcd}
\]

conmuta. Es decir, para cada $A\in C$,
\[\alpha_{A'}\circ F(f)=G(f)\circ \alpha_A \qquad \beta_{A'}\circ G(f)=H(f)\circ \beta_A \qquad \gamma_{A'}\circ H(f)=I(f)\circ \gamma_A.\]
\begin{itemize}
\item[\blacksmiley{}] \textit{La composición es una transformación natural.}\\ Definimos $\beta\alpha\colon F\to H$ a la familia $(F(A)\xrightarrow{\beta_A\circ \alpha_A} H(A))$ de $D$.\\
Sea $(A\xrightarrow{f}A')\in C$. Tenemos que
\begin{align*}
\beta_{A'}\circ(\alpha_{A'}\circ F(f))&=\beta_{A'}\circ(G(f)\circ \alpha_A)\\
&=(\beta_{A'}\circ G(f))\circ \alpha_A\\
&=(H(f)\circ\beta_A)\circ\alpha_A.
\end{align*}
Por lo que $(\beta_{A'}\circ\alpha_{A'})\circ F(f)=H(f)\circ(\beta_A\circ\alpha_A)$, es decir, el siguiente diagrama
\[
\begin{tikzcd}
F(A) \arrow[d, "F(f)"'] \arrow[r, "\beta_A\circ \alpha_A"] & H(A) \arrow[d, "H(f)"] \\
F(A') \arrow[r, "\beta_{A'}\circ\alpha_{A'}"']             & H(A')                 
\end{tikzcd}
\]
conmuta.
\item[\blacksmiley{}] \textit{La composición es asociativa.}\\
Sabemos que $\gamma\circ (\beta\circ\alpha)=(\gamma\circ\beta)\circ\alpha$ ya que $\gamma_A\circ(\beta_A\circ\alpha_A)=(\gamma_A\circ\beta_A)\circ\alpha_A$ para cada $(A\xrightarrow{f} A')\in C$.
\item[\blacksmiley{}] \textit{$I_F$ es una transformación natural.}\\
Sea $F\colon C\to D$ un funtor. Definimos $I_F\colon F\to F$ como la familia $(F(A)\xrightarrow{I_A}F(A))\in D$, donde $I_A$ es el morfismo identidad de $F(A)$, para cada $A\in C$.\\
Sea $(A\xrightarrow{f} A')\in D$. Sabemos que $I_{A'}\circ F(f)=F(f)=F(f)\circ I_A$. Por lo que el diagrama
\[
\begin{tikzcd}
F(A) \arrow[d, "F(f)"'] \arrow[r, "I_A"] & F(A) \arrow[d, "F(f)"] \\
F(A') \arrow[r, "I_{A'}"']               & F(A')                 
\end{tikzcd}
\]
conmuta. Es decir, $I_F$ es una transformación natural.
\item[\blacksmiley{}] \textit{Identidad.}\\
Sean $\alpha\colon F\to G$ y $\beta\colon H\to F$ transformaciones lineales, sabemos que $\alpha\circ I_F=\alpha$ y $I_F\circ \beta=\beta$, ya que $\alpha_A\circ I_A=\alpha_A$ y $I_A\circ\beta_A=\beta_A$, para cada $A\in C$.
\end{itemize}
\end{proof}
\begin{exe}[Alfredo $\checkmark$ ]
    (**) Probar que un funtor es una equivalencia ssi es
      esencialmente suprayectivo y fielmente pleno.
\end{exe}
\begin{sol}
    Primero probaremos el siguiente lema.
    \begin{lemma}
        Si $\alpha:F\to G$ es un isomorfismo natural entre dos funtores
        $F,G:\cal C\to\cal D$, cada componente $\alpha_A:FA\to GA$ de
        $\alpha$ es un isomorfismo en $\cal D$.
    \end{lemma}
    \begin{proof}
        Como $\alpha:F\to G$ es un isomorfismo, existe una trasformación
        natual $\alpha^{-1}:G\to F$ tal que
        \begin{align*}
            \alpha\alpha^{-1} &= \id_G
                & \alpha\alpha^{-1} &= \id_F
        \end{align*}
        Esto significa que, al fijarnos en
        las componentes en cualquier objeto $A$ de $\cal C$, tenemos
        \begin{align*}
                \alpha_A(\alpha^{-1})_A
                = (\alpha\alpha^{-1})_A
                &= (\id_G)_A
                = \id_{GA} \\
                (\alpha^{-1})_A\alpha_A
                =(\alpha^{-1}\alpha)_A
                &= (\id_{F})_A
                = \id_{FA}
        \end{align*}
        Luego, $\alpha_A$ es un isomorfismo con inverso
        $(\alpha_A)^{-1}=(\alpha^{-1})_A$.
        En particular, la notación $\alpha^{-1}_A$ no es ambigua.
    \end{proof}
    Ahora sí, continuamos con el ejercicio.
    Sea $F:\cal C \to \cal D$ una equivalencia.
    Entonces hay otro funtor $G:\cal D\to \cal C$ e isomorfismos
    \begin{align*}
        \epsilon : FG &\simeq \id_{\cal D} & \eta : \id_{\cal C} &\simeq GF.
    \end{align*}
    \begin{itemize}
        \item 
        Primero probaremos que $F$ es fiel.
        Supongamos que $f,g:A\to B$ son morfismos de $\cal C$ tales que
        $Ff = Fg$.
        Aplicando $G$, obtenemos $GFf=GFg$.
        Luego, como $\eta$ es transformación natural, tenemos los siguientes
        diagramas conmutativos
        \[
        \begin{tikzcd}
            GFA \ar[d,"GFf"'] & A \ar[l,"\eta_A"',"\sim"] \ar[d,"f"] \\
            GFB & B \ar[l,"\eta_B","\sim"']
            \com{1-1}{2-2}
        \end{tikzcd}
        \hspace{20mm}
        \begin{tikzcd}
            GFA \ar[d,"GFg"'] & A \ar[l,"\eta_A"',"\sim"] \ar[d,"g"] \\
            GFB & B \ar[l,"\eta_B","\sim"']
            \com{1-1}{2-2}
        \end{tikzcd}
        ,\]
        donde $\eta_A$ y $\eta_B$ son isomorfismos, por el lema que probamos.
        Recordando que $GFf=Gfg$, tenemos
        \begin{align*}
            f
            &= (\eta_B^{-1})(GFf)(\eta_A) \\
            &= (\eta_B^{-1})(GFg)(\eta_A) \\
            &= g,
        \end{align*}
        como se quería.
        
        \item
        Ahora veremos que $F$ es pleno.
        Sea $g:FA\to FB$ un morfismo en $\cal D$.
        Queremos construir un morfismo $f:A\to B$ tal que $Ff=g$.
        Como $G$ es el inverso de $F$ (salvo iso),
        el candidato natural sería $Gg$.
        El problema es que este es un
        morfismo de $GFA$ en $GFB$.
        Podemos intentar arrerglar esto recordando que los componentes de
        la transformación $\eta:\id_{\cal C} \to GF$ son isomorfimos.
        Consideramos la composición $h=(\eta_B^{-1})(Gg)(\eta_A)$:
        \[
        \begin{tikzcd}
            GFA \ar[d,"Gg"'] & A \ar[l,"\eta_A"',"\sim"] 
            \ar[d,"h",dotted]\\
            GFB & B \ar[l,"\eta_B","\sim"']
        \end{tikzcd}
        .\]
        Aplicando $F$, tenemos
        \[
        \begin{tikzcd}
            FA \ar[d,"g"]
                & \ar[l,"\epsilon_{FA}"',"\sim"] FGFA \ar[d,"FGg"']
                & FA \ar[l,"F\eta_A"',"\sim"]  \ar[d,"Fh"]\\
            FB  & FGFB \ar[l,"\epsilon_{FB}","\sim"']
                & FB \ar[l,"F\eta_B","\sim"']
        \end{tikzcd}
        ,\]
        lo cual nos dice que $Fh$ y $g$ difieren por un isomorfismo.
        Tendríamos $Fh=g$ si fuera el caso que
        $(\epsilon_{FA})(F\eta_A)=\id_{FA}$
        y que $(\epsilon_{FB})(F\eta_B)=\id_{FB}$.
        Sin embargo, esto no es cierto, en general.
        
        Para remediar esto, en lugar de tomar $h$, tomamos la composición
        $f$ como
        \[
        \begin{tikzcd}
            GFA \ar[d,"Gg"']
            & GFGFA \ar[l,"GF\eta_A^{-1}"']
            & GFA \ar[l,"G\epsilon_{FA}^{-1}"']
            & A \ar[l,"\eta_A"']
            \ar[d,"f",dotted]\\
            GFB
            & GFGFB \ar[l,"GF\eta_B^{-1}"']
            & GFB \ar[l,"G\epsilon_{FB}^{-1}"']
            & B \ar[l,"\eta_B"']
        \end{tikzcd}
        ,\]
        de modo que, al aplicar $F$, tenemos
        \[
        \begin{tikzcd}
            FA \ar[d,"g"]
            & FGFA \ar[d,"FGg"'] \ar[l,"\epsilon_{FA}"']
            & FGFGFA \ar[l,"FGF\eta_A^{-1}"']
            & FGFA \ar[l,"FG\epsilon_{FA}^{-1}"']
            & FA \ar[l,"F\eta_A"']
            \ar[d,"Ff",dotted]\\
            FB
            & FGFB \ar[l,"\epsilon_{FB}"']
            & FGFGFB \ar[l,"FGF\eta_B^{-1}"']
            & FGFB \ar[l,"FG\epsilon_{FB}^{-1}"']
            & FB \ar[l,"F\eta_B"']
        \end{tikzcd}
        .\]
        La situación puede parecer peor, pero la naturalidad nos salva.
        Agregando arriba y abajo los cuadrados conmutativos que nos
        da la condición de naturalidad, tenemos
        \[
        \begin{tikzcd}
            & FA \ar[equal,dl]
            & FGFA \ar[l,"F\eta_A^{-1}"']
            & FA \ar[l,"\epsilon_{FA}^{-1}"']
            \\
            FA \ar[d,"g"]
            & FGFA \ar[d,"FGg"'] \ar[u,"\epsilon_{FA}"]\ar[l,"\epsilon_{FA}"']
            & FGFGFA \ar[l,"FGF\eta_A^{-1}"'] \ar[u,"\epsilon_{FGFA}"]
            & FGFA \ar[l,"FG\epsilon_{FA}^{-1}"'] \ar[u,"\epsilon_{FA}"]
            & FA \ar[l,"F\eta_A"']
            \ar[d,"Ff",dotted]
            \\
            FB
            & FGFB \ar[l,"\epsilon_{FB}"']\ar[d,"\epsilon_{FB}"']
            & FGFGFB \ar[l,"FGF\eta_B^{-1}"'] \ar[d,"\epsilon_{FGFB}"]
            & FGFB \ar[l,"FG\epsilon_{FB}^{-1}"'] \ar[d,"\epsilon_{FB}"]
            & FB \ar[l,"F\eta_B"']
            \\
            & FB \ar[equal,ul]
            & FGFB \ar[l,"F\eta_B^{-1}"]
            & FB \ar[l,"\epsilon_{FB}^{-1}"]
        \end{tikzcd}
        .\]
        Siguiendo el camino exterior, obtenemos que
        \[
            (F\eta_B^{-1})(\epsilon_{FB}^{-1})(\epsilon_{FB})(F\eta_B)Ff
            = g(F\eta_A^{-1})(\epsilon_{FA}^{-1})(\epsilon_{FA})(F\eta_A)
        .\]
        Es decir, $Ff=g$.
        
        \item
        Con el lema que probamos, es fácil ver que $F$ es
        esencialmente suprayectivo.
        En efecto, 
        para cualquier objeto $B$ de $\cal D$, el componente
        $\epsilon_B:FGB\to B$ de $\epsilon:FG\to\id_{\cal D}$ es un
        isomorfismo, así que $B$ es isomorfo a un objeto en la
        imagen de $F$.
    \end{itemize}
    Ahora la otra implicación.
    Supongamos que $F:\cal C\to\cal D$ es fielmente pleno
    y esencialmente suprayectivo.
    Queremos definir un funtor $G:\cal C\to \cal D$ que haga de
    inverso de $F$ (salvo iso).
    Como $F$ es esencialmente suprayectivo, el axioma de
    elección nos permite elegir, para cada objeto $B$ de $\cal D$,
    un objeto $GB$ de $\cal C$ y un isomorfismo
    $\epsilon_B:FGB\xrightarrow{\sim} B$.
    
    Falta definir la acción de $G$ en morfismos.
    Dado $g:B_1\to B_2$ en $\cal D$, definimos
    $f=(\epsilon_{B_2})^{-1}g(\epsilon_{B_1}):FGB_1\to FGB_2$, de tal
    modo que
    \[
        \begin{tikzcd}
            B_1 \ar[d,"g"']
            & FGB_1 \ar[l,"\epsilon_{B_1}"'] \ar[d,"f",dotted] \\
            B_2 & FGB_2 \ar[l,"\epsilon_{B_2}"]
            \com{1-1}{2-2}
        \end{tikzcd}
    .\]
    Como $F$ es fielmente pleno, podemos definir a $Gg:GB_1\to GB_2$
    como el único morfismo que satisface $FGg=f:FGB_1\to FGB_2$.
    
    Veremos que $G$ es un funtor.
    Si tomamos morfismos
    \[
        \begin{tikzcd}
            B_1 \ar[d,"g"'] \\
            B_2 \ar[d,"h"'] \\
            B_3
        \end{tikzcd}
    \]
    entonces, por definición, $G(hg)$, $Gh$ y $Gg$
    son los únicos morfismos tales que los diagramas
    \[
        \begin{tikzcd}
            B_1 \ar[d,"g"'] & FGB_1 \ar[l,"\epsilon_{B_1}"'] \ar[dd,"FG(hg)"] \\
            B_2 \ar[d,"h"'] \\
            B_3 & FGB_2 \ar[l,"\epsilon_{B_3}"']
            \com{1-1}{3-2}
        \end{tikzcd}
        \hspace{10mm}
        \begin{tikzcd}
            B_1 \ar[d,"g"'] & FGB_1 \ar[l,"\epsilon_{B_1}"'] \ar[d,"FGg"] \\
            B_2 \ar[d,"h"'] & FGB_2 \ar[l,"\epsilon_{B_2}"'] \ar[d,"FGh"] \\
            B_3 & FGB_2 \ar[l,"\epsilon_{B_3}"']
            \com{1-1}{2-2} \com{2-1}{3-2}
        \end{tikzcd}
    .\]
    Se sigue que $FG(hg)=(FGh)(FGg)$.
    Por funtorialidad de $F$, esto es $FG(hg)=F((Gh)(Gg))$.
    Luego, como $F$ es fiel, tenemos $G(hg)=(Gh)(Gg)$.
    
    Por otro lado, tomando el morfismo identidad
    \[
        \begin{tikzcd}
            B \ar[d,"\id_B"'] \\
            B
        \end{tikzcd}
    \]
    tenemos que $G\id_B$ es el único morfismo $GB\to GB$ que hace conmutar
    el diagrama
    \[
        \begin{tikzcd}
            B \ar[d,"\id_B"'] & FGB \ar[l,"\epsilon_B"'] \ar[d,"FG\id_B"] \\
            B & FGB \ar[l,"\epsilon_B"']
            \com{1-1}{2-2}
        \end{tikzcd}
    \]
    Por lo tanto, $FG\id_B=\id_{FGB}=F\id_{GB}$.
    Como $F$ es fiel, esto implica que $G\id_B =\id_{GB}$.
    
    Por definición de la acción de $G$ en morfismos, para cualquier morfismo
    $g:B_1\to B_2$ en $\cal D$ el diagrama
    \[
        \begin{tikzcd}
            B_1 \ar[d,"g"']
            & FGB_1 \ar[l,"\epsilon_{B_1}"'] \ar[d,"FGg"] \\
            B_2 & FGB_2 \ar[l,"\epsilon_{B_2}"]
            \com{1-1}{2-2}
        \end{tikzcd}
    \]
    es conmutativo.
    Esto significa que la familia de morfismos $(\epsilon_B:FGB\to B)_{B\in\Ob\cal D}$
    es una transformación natural
    \[
        \epsilon : FG\to \id_{\cal D}
    .\]
    Como cada $\epsilon_B$ es un isomorfismo y el diagrama anterior
    es conmutativo, se sigue que $\epsilon$ es un
    isomorfismo natural, cuya inversa
    $\epsilon:\id_{\cal D}\to FG$ tiene componentes dadas por
    \[
        (\epsilon^{-1})_B = (\epsilon_B)^{-1}:B\to FGB
    \]
    para cada objeto $B$ de $\cal D$.
    
    Resta construir un isomorfismo natural $\eta:\id_{\cal C}\to GF$.
    Sea $A$ un objeto de $\cal C$.
    Como $F$ es fielmente pleno y $\epsilon:FG\to\id_{\cal D}$ es un
    isomorfismo, podemos definir $\eta_A$ como el único morfismo
    $\eta_A:A\to GFA$ tal que $F\eta_A=\epsilon_{FA}^{-1}:FA\to FGFA$.
    
    Dado que $F\eta_A$ es un isomorfismo (con inverso $\epsilon_{FA}$),
    se sigue que cada $\eta_A$ es un isomorfismo, cuyo inverso $\eta_A^{-1}$ es el
    único morfismo $\eta_A:A\to GFA$ tal que $F\eta_A^{-1}=\epsilon_{FA}:FGFA\to FA$.
    En efecto, si $f:GFA\to A$ es tal que $Ff=\epsilon$, entonces
    \begin{align*}
        F(\eta_Af)
        &=(F\eta_A)(Ff)=\epsilon_{FA}^{-1}\epsilon_{FA}=\id_{FGFA}=F\id_{GFA} \\
        F(f\eta_A)
        &=(Ff)(F\eta_A)=\epsilon_{FA}\epsilon_{FA}^{-1}=\id_{FA}=F\id_A
    \end{align*}
    así que $\eta_Af=\id_{GFA}$ y $f\eta_A=\id_A$, pues $F$ es fiel,
    por lo cual $f=\eta_A^{-1}$.
    
    Finalmente, observemos que $\eta$ es una transformación natural.
    En efecto, para cualquier morfismo $f:A_1\to A_2$ en $\cal C$,
    $Ff:FA_1\to FA_2$ es un morfismo en $\cal D$, por lo cual el diagrama
    \[
        \begin{tikzcd}
            FA_1 \ar[d,"Ff"']
            & FGFA_1 \ar[l,"\epsilon_{FA_1}"'] \ar[d,"FGFf"] \\
            FA_2 & FGFA_2 \ar[l,"\epsilon_{FA_2}"]
            \com{1-1}{2-2}
        \end{tikzcd}
    \]
    es conmutativo.
    Como observamos antes, $F\eta_A^{-1}=\epsilon_A$ para cualquier $A$, así
    que esto es
    \[
        \begin{tikzcd}
            FA_1 \ar[d,"Ff"']
            & FGFA_1 \ar[l,"F\eta_{A_1}^{-1}"'] \ar[d,"FGFf"] \\
            FA_2 & FGFA_2 \ar[l,"F\eta_{A_2}^{-1}"]
            \com{1-1}{2-2}
        \end{tikzcd}
    .\]
    Es decir,
    \[
        F(f\eta_{A_1}^{-1})
        =(Ff)(F\eta_{A_1}^{-1})
        =(F\eta_{A_2}^{-1})(FGFf)
        =F(\eta_{A_2}^{-1}GFf)
    ,\]
    de modo que $f\eta_{A_1}^{-1}=\eta_{A_2}^{-1}GFf$, pues $F$ es fiel.
    Luego, $\eta_{A_2}f=(GFf)\eta_{A_1}$.
    Es decir, el diagrama
    \[
        \begin{tikzcd}
            A_1 \ar[d,"f"'] \ar[r,"\eta_{A_1}"]
            & GFA_1 \ar[d,"GFf"] \\
            A_2 \ar[r,"\eta_{A_2}"'] & GFA_2
            \com{1-2}{2-1}
        \end{tikzcd}
    \]
    es conmutativo.
    Esta es la condición de naturalidad.
\end{sol}

\subsection*{(SESIÓN 4: 21 SEP)}

\subsection{Implicaciones}

Una implicación en una $\inf$-retícula es una operación
$(-\succ -):A\to A$ tal que
\[
  x\inf y\leq a \ssi x\leq (y\succ a )
.\]

Ahora vienen dos lemas técnicos acerca de $\inf$-semiretículas.

\begin{lemma}
  Sea $A$ una $\inf$-retícula con implicación.
  Entonces
  \begin{enumerate}
    \item $(x\succ -)$ infla.
    \item $x\inf(x\succ a) = x\inf a$
    \item $(-\succ a)$ es antítona.
  \end{enumerate}
\end{lemma}
prueba: video 21 de sep, minuto 6:15

\begin{lemma}
  Sea $A$ una $\inf$-semiretícula con implicación.
  Observemos que para $a\in A$, tenemos
  una función $(-\succ a):A\to A$.
  \begin{enumerate}
    \item $((-\succ a)\succ a)$ infla.
    \item $((-\succ a)\succ a)$ es monótona.
      (Inmediato del punto 3 del lema anterior).
    \item $((-\succ a)\succ a)$ es idempotente.
      (Por 1, basta ver que
      $(((x\succ a)\succ a)\succ a)\leq(x\succ a)$).
  \end{enumerate}
\end{lemma}

Una función $f:A\to A$ (donde $A$ es una $\inf$-semiretícula)
es un operador cerradura si es monótona, infla y es idempotente.

Entonces lo que acabamos de probar es que, si $A$ es una
$\inf$-semiretícula, entonces $((-\succ a)\succ a)$ es un
operador cerradura, para cualquier $a\in A$.

\subsection{Álgebras de Heyting}

Si una retícula tiene implicación, diremos que es un álgebra de
Heyting.
Observemos que toda álgebra de Heyting tiene una
negación dada por $\neg a = (a\succ 0)$.

\begin{lemma}
  Toda álgebra de Heyting es distributiva.
\end{lemma}
prueba: video 21 de septiembre, minuto 32:00

\begin{thm}
  Una retícula completa $A$ es un marco si, y solo si, $A$ tiene
  implicación.
\end{thm}
Prueba: video 21 de septiembre, minuto 39:00

\begin{exa}
  En un espacio topológico, tenemos
  \begin{align*}
    w\leq (v\succ u)
    &\iff w\cap v\leq u \\
    &\iff w\leq v'\cup u \\
    &\iff w\leq (v'\cup u)^\circ
  \end{align*}
  Luego, $v\succ u = (u\cup v')^\circ$.
\end{exa}

\begin{exa}
  En un álgebra booleana completa, tenemos
  \begin{align*}
    w\leq (v\succ u)
    &\iff w\inf v\leq u \\
    &\iff w\leq \neg v\sup u
      && \text{caballo de batalla} 
  \end{align*}
\end{exa}

\begin{lemma}
  Sean $A$ un marco, $a\in A$ y $X\subseteq A$.
  Entonces
  \[
    (\Sup X)\succ a = \Inf\{(x\succ a) \mid x\in X\}
  .\]
\end{lemma}
en particular, tomando $X=\{x,y\}$, tenemos
\[
  (x\sup y)\succ a = (x\succ a)\inf(y\succ a)
.\]

Más consecuencias:

Si $A$ es un marco y $a\in A$ entonces $\neg a = (a\succ 0)$.

En una retícula general, decimos que $b\in A$ es
complemento de $a\in A$ si
\[
  a\inf b = 0 \hspace{10mm} a\sup b = 1
.\]

\begin{lemma}
  Si $A$ es un marco, $a\in A$ es complementado ssi $\neg a \sup
a = 1$, y de hecho su complemento es $\neg a$.
\end{lemma}
prueba: video 21 de septiembre, minuto 1:13

\begin{exe}[Armando]
  ¿Quién será la negación y la implicación en $\cal L S$?
  \begin{sol}
      
  \end{sol}
\end{exe}

\begin{defn}
  Sea $A$ un marco y $a\in A$.
  \begin{itemize}
    \item $a$ es complementado si $a\sup\neg a = 1$
    \item $a$ es regular si $\neg\neg a = a$
    \item $a$ es denso si $\neg a = 0$.
  \end{itemize}
\end{defn}

\begin{exe}[Dante $\checkmark$ ]
  Probar que un elemento $a$ de un marco es denso
  si, y solo si $\neg\neg a = 1$.
\end{exe}
\begin{sol}
$\Rightarrow$\\
Sea $a\in A$ denso. Aaí, $\neg a=0$, y nótese que $x\inf\neg a=0 \ \forall \ x\in A$. Por lo tanto, $x\leq \neg\neg a \ \forall \ x\in A \Rightarrow \neg\neg a=1$.\\
$\Leftarrow$\\
Supóngase que $\neg\neg a=1$. Así, $x\leq \neg\neg a \ \forall \ x\in A\Rightarrow x\inf \neg a=0 \ \forall \ x \in A$. Por lo anterior, $1\inf\neg a=\neg a=0$ por lo que $a$ es denso.\vspace{3mm}

\hfill $\blacksquare$
\end{sol}\vspace{3mm}

    
Se puede observar que $\neg\neg a$ es regular para todo $a$.


\subsection{VIDEO 2: cats 2 (23 SEP)}
\begin{itemize}
  \item El axioma de naturalidad.
  \item Las unidades y counidades de una adjunción inducidas por
    las transposiciones.
  \item Ejemplo: espacios vectoriales y conjuntos: la unidad y la
    counidad.
  \item Las identidades triangulares de la unidad y counidad.
  \item Adjunciones como pares de transformaciones naturales con
    una condición de coherencia.
  \item La transposición inducida por la unidad y la counidad*.
\end{itemize}
\begin{exe}[Yareli $\checkmark$ ]
  * Verificar que las transposiciones inducidas por la
  unidad y counidad coinciden con las transposiciones
  originales.
\end{exe}
\begin{proof}
    Recordemos que, por compatibilidad:
    \begin{align*}
        \left(\ol{F(A)\rar g B \rar q B'}\right)
        &=
        \left(A\rar{\ol g} G(B) \rar{G(q)}G(B')\right) \\
        \left(\ol{A'\rar p A \rar f G(B)}\right)
        &=
        \left(F(A')\rar{F(p)} F(A) \rar{\ol f}B\right),
    \end{align*}
    es decir, $\ol{qg}=G(q)\ol g$ y $\ol{fp}=\ol f F(p)$.\\
    Debemos demostrar que $\overline{g}=G(g)\eta_A$ y $\overline{f}=\epsilon_BF(f)$ para $g\colon F(A)\to B$ y $f\colon A\to G(B)$.\\
Tenemos que
\begin{align*}
G(g)\eta_A&=G(g)\overline{id}_{F(A)}\\
&=\overline{g\circ id}_{F(A)}\\
&=\overline{g}.
\end{align*}
Además, 
\begin{align*}
\epsilon_B F(f)&=\overline{id}_{G(B)}F(f)\\
&=\overline{id_{F(A)}\circ f}\\
&=\overline{f}.
\end{align*}
\end{proof}

\subsection{VIDEO 3: adjunciones (23 SEP)}
\begin{itemize}
  \item Repaso de adjunciones.
  \item (forma equivalente de la naturalidad??)
  \item Ejemplo: Espacios vectoriales y conjuntos*.
  \item Ejemplo: Grupos y conjuntos. Grupos libres.
  \item Ejemplo: Grupos abelianos y grupos. Abelianización.
  \item Ejemplo: La adjunción del producto cartesiano y la
    exponenciación en la categoría de conjuntos.
  \item Ejemplo: Las secciones inferiores de una
  $\inf$-semiretícula**.
\end{itemize}
\begin{exe}[Juan $\checkmark$ ]
  * Probar que el funtor del espacio vectorial libre sí es
  adjunto izquierdo del de olvidar.
\end{exe}
\begin{sol}
    Tenemos el funtor olvidadizo $k$-\textbf{\textit{Vect}}$\to$ \textbf{\textit{Set}} que para cada espacio vectorial $V$ “olvida” su estructura y le asocia el conjunto subyacente. A cada conjunto $X$ se puede asociar el espacio vectorial $k\langle X\rangle$ generado por $X$; es decir, el espacio cuya base corresponde a los elementos de $X$. En este caso toda aplicación lineal $f:k\langle X\rangle\to V$ se define de manera única por los imágenes de los elementos de la base.
    \[\begin{tikzcd}
	{X} & {k\langle X\rangle} \\
	{V}
	\arrow["{f}"', from=1-1, to=2-1]
	\arrow[from=1-1, to=1-2, hook]
	\arrow["{\exists !}"', from=2-1, to=1-2, dotted]
\end{tikzcd}\]
Esto nos da una biyección natural $$Hom_{\mbox{k-\textbf{\textit{Vect}}}}(k\langle X\rangle, V)\cong Hom_{\mbox{\textbf{\textit{Set}}}}(X,V)$$
Luego, el funtor $k\langle -\rangle$ es adjunto izquierdo del funtor olvidar $k-$\textbf{\textit{Vect}}$\to$\textbf{\textit{Set}}.
\end{sol}
\begin{exe}[Alfredo $\checkmark$ ]
  ** Probar que la construcción del marco de secciones
  inferiores es el marco libre sobre una $\inf$-semiretícula.
\end{exe}
\begin{sol}
    La subretícula $\cal LA$ de $\cal PA$ es cerrada bajo
    supremos arbitrarios.
    Como las operaciones se heredan de $\cal PA$, también cumple
    la ley distributiva de marcos.
    
    Por lo tanto, la función $\down:A\to\cal LA$ es un morfismo
    de $\inf$-semiretículas que aterriza en un marco.
    Precomponer con $\down$ nos da una flecha
    \[
        \Frm(\cal LA,-)\to \Pos^\inf(A,U-)
    ,\]
    (donde $U:\Frm\to\Pos^\inf$ es el funtor de olvido).
    
    Resta ver que esta flecha es una biyección.
    Es decir, dado un morfismo $f:A\to B$
    de $\inf$-semiretículas,
    debemos probar que existe un único morfismo de marcos
    $f^\sharp:\cal LA\to B$ que factoriza a $f$ a través
    de $\down$:
    \[
        f^\sharp\down = f
    .\]
    Es decir, $f^\sharp(\down a)=f(a)$ para todo $a\in A$.
    Esta condición determina completamente a $f^\sharp$.
    En efecto, para toda sección inferior $F\in\cal LA$ tenemos
    $F=\bigcup\{\down a \mid a\in F\}$ y, como también queremos
    que $f^\sharp$ respete supremos, se debe cumplir
    \begin{align*}
        f^\sharp(F)
        &= \Sup\{f^\sharp(\down a) \mid a\in F\} \\
        &= \Sup\{f(a) \mid a\in F\}.
    \end{align*}
    Tomando esta ecuación como la definición de $f^\sharp$, es
    claro que $f^\sharp(\down a)=f(a)$.
    Por lo tanto, si $f^\sharp:\cal LA\to B$ es un
    morfismo de marcos, es el único con esta propiedad.
    Verificamos las propiedades directamente.
    \begin{itemize}
        \item En efecto, si $F\subseteq G\cal\in LA$, entonces 
        \[
            \{f(a) \mid a\in F\} \subseteq \{f(a) \mid a\in G\}
        .\]
        Tomando supremos, obtenemos
        $f^\sharp(F)\leq f^\sharp(G)$, así que $f^\sharp$ es
        monótona.
        \item
        Dadas $F,G\in\cal LA$, hay que mostrar
        que $f^\sharp(F\cap G)=f^\sharp(F)\inf f^\sharp(G)$.
        La comparación $\leq$ se sigue de la monotonía de
        $f^\sharp$.
        Por otro lado, observemos que
        \[
            \{ a\inf b \mid a\in F, b\in G\}
            \subseteq F\cap G,
            \hspace{10mm} (*)
        \]
        pues $F$ y $G$ son secciones inferiores.
        Luego,
        \begin{align*}
            f^\sharp(F)\inf f^\sharp(G)
            &= \Sup\{f(a)\inf f(b) \mid a\in F, b\in G\}
                && \text{ ley dist. de marcos } \\
            &= \Sup\{f(a\inf b) \mid a\in F, b\in G\} \\
            &\leq \Sup\{f(c) \mid c\in F\cap G\}
                && \text{ por $(*)$ } \\
            &= f^\sharp(F\cap G),
        \end{align*}
        como se quería.
        \item
        Dado $X\subseteq \cal LA$, hay que mostrar que
        $f^\sharp(\bigcup X)=\Sup\{f^\sharp(F) \mid F\in X\}$.
        Como $f^\sharp$ es monótona,
        $f^\sharp(\bigcup X)$ es cota superior de
        $\{f^\sharp(F) \mid F\in X\}$.
        Para ver que es la mínima, sea $b\in B$ tal que
        $f^\sharp(F)\leq b$ para todo $F\in X$.
        Por definición de $f^\sharp$, esto significa que
        $f(a)\leq b$ para cualesquiera $a\in F, F\in X$.
        Luego,
        \begin{align*}
            f^\sharp(\bigcup X)
            &= f^\sharp (
            \{a\in A \mid a\in F\text{ para algún }F\in X\}
            ) \\
            &=
            \Sup\{f(a)\in A \mid a\in F\text{ para algún }F\in X\}) \\
            &\leq b,
        \end{align*}
        como se deseaba.
    \end{itemize}
    Por lo tanto, $f^\sharp$ es morfismo de marcos.
    Así, tenemos un isomorfismo
    \[
        \Frm(\cal LA,-)\rar{-\circ \down} \Pos^\inf(A,U-)
    \]
    y, así, $\cal LA$ es el marco libre en $A$.
    
    Observemos que esto es válido para cualquier
    semiretícula $A$.
    Más aún, dado un morfismo de $\inf$-semiretículas
    $g:A\to A'$, la composición
    \[
        \cal LA' \lar \down A' \lar g A
    \]
    es un morfismo de $\inf$-semiretículas y
    $\cal LA'$ es un marco, así que
    existe un único morfismo de marcos
    $g^\sharp:\cal LA\to \cal LA'$ que factoriza a
    $\down g$ a través de $\down:A\to \cal LA$.
    Si definimos $\cal Lg=g^\sharp:\cal LA\to\cal LA'$,
    obtenemos una función
    \[
        \Pos^\inf(A,A') \to \Frm(\cal LA,\cal LA')
    .\]
    Más aún, las propiedades de unicidad de
    $\cal Lg=g^\sharp$ aseguran que $\cal L$ es un funtor
    $\cal L:\Pos^\inf\to\Frm$.
    Luego, tenemos una adjunción $\cal L\dashv U$.
\end{sol}

\subsection*{(SESIÓN 5: 23 SEP)}

\subsection{Elementos regulares, densos y complementados}

Recordemos que $\neg(a\sup b)=\neg a\inf \neg b$.
Luego,
\[
  \neg(a\sup\neg a) = \neg a\inf\neg\neg a = 0
.\]
Por lo tanto, $a\sup\neq a$ siempre es denso.
Además
\begin{align*}
  \neg\neg a\inf(a\sup\neg a)
  &= (\neg\neg a\inf a)\sup(\neg\neg a\inf\neg a) \\
  &= a \inf 0 \\
  &= a
\end{align*}
es decir, todo elemento $a\in A$ se puede expresar como el ínfimo
de un elemento denso y un elemento regular.

\begin{lemma}
  Un marco $A$ es booleano si, y solo si, cada elemento es
  regular.
  Esto es, todo elemento es complementado si, y solo si, todo
  elemento es regular.
\end{lemma}
prueba: video 23 de septiembre, minuto 17:00

\subsection{Núcleos}

\begin{defn}
  Para un marco $A$ y un elemento $a\in A$, definimos $w_a:A\to
  A$ como
  \[
    w_a(x)=((x\succ a)\succ a)
  .\]
\end{defn}
Ya probamos que, para cada $a$, el operador $w_a$ es un operador
de cerradura.
Ademas, cumple $(w_a(x)\succ a) = w_a(x\succ a)=(x\succ a)$ y
$w_a(x\inf y) = w_a(x)\inf w_a(y)$.
Prueba: video 23 de septiembre, minuto 23:00

\subsection{Los morfismos de marcos como funtores adjuntos}

Cualquier morfismo de COPOs $f:A\to B$ tiene un
adjunto derecho dado como
\[
 f_*(b) = \Sup\{a\in A \mid f(a)\leq b\} 
.\]
minuto 53:00

\begin{exe}[Armando]
  Mostrar que $f_*$ es monótona y preserva ínfimos arbitrarios.
\end{exe}

\begin{exa}
  La implicación es el adjunto derecho del ínfimo.
\end{exa}

\begin{exe}[Dante $\checkmark$ ]
  Si $\phi:S\to T$ es una función continua entre espacios
  topológicos, describir el adjunto derecho de la función
  correspondiente entre los marcos de abiertos.
\end{exe}
\begin{sol}
    Sea $f:\mathscr{O}S\to\mathscr{O}T$ el morfismo de marcos definido como $f(U)=\phi(U)$. Así, $f_*:\mathscr{O}T\to\mathscr{O}S$ definida como 
    \begin{align*}
        f_*(U)&=\Sup\{v\in\mathscr{O}S:f(v)\leq U\}\\
        &=\bigcup\{v\in\mathscr{O}S:\phi(v)\subset U\}\\
        &=\left[\phi^{-1}(U)\right]^{\prime}
    \end{align*}
    Es su adjunto derecho.
\end{sol}
\subsection{Monomorfismos y epimorfismos}

Todo morfismo inyectivo es monomorfismo.

Todo morfismo suprayectivo es epimorfismo.

Todo monomorfismo es inyectivo.

Bimorfismos.

\begin{exa}
  Si $S$ es un espacio topológico $T_1$, entonces la inclusión
  $\cal OS\to PS$ es un epimorfismo.
\end{exa}

\begin{defn}
  Un cociente de un marco $A$ es un morfismo suprayectivo $f:A\to B$.
  Equivalentemente, $B$ es un cociente de $A$ si existe un
  morfismo suprayectivo $f:A\to B$.
\end{defn}

\section{Cocientes}

\subsection*{(SESIÓN 6: 28 SEP)}

Dado un morfismo de grupos $\phi:G\to G'$, existe un subgrupo normal
$\ker \phi\leq G$ tal que $\phi$ se factoriza a través del cociente
$G/\ker \phi$.

\[
    \begin{tikzcd}
        G \ar[dr] \ar[rr,"\phi"] && G' \\
        &G/\ker \phi \ar[ur,"\tilde\phi"']
    \end{tikzcd}
.\]
Más aún, este cociente es universal en el sentido de que, para cualquier 
morfismo $\psi:G\to G''$ con $\ker\phi \leq \ker\psi$, existe un único
morfismo $\phi^\sharp:G/\ker\phi\to G''$ que factoriza a $\psi$ a través
del cociente:
\[
    \begin{tikzcd}
        G \ar[dr] \ar[rr,"\psi"] && G'' \\
        &G/\ker \phi \ar[ur,"\phi^\sharp"']
    \end{tikzcd}
.\]
Vale la pena notar que la relación de equivalencia
\[
    g\sim h \ssi \phi(g) = \phi(h)
\]
está codificada en el subgrupo $\ker \phi$ de $G$.
Esta situación se repite en las categorías de anillos,
espacios vectoriales, módulos, álgebras, etc.
Queremos llevar esta situación a un contexto más general, donde no tenemos
estructura de grupo.

\subsection{Cocientes de conjuntos}
Consideremos una función de conjuntos $\phi:A\to B$.
Queremos definir algo que haga el papel del núcleo de $\phi$.
Aquí no tenemos estructura de grupo, pero sí podemos definir una relación
de equivalencia $\sim$ en $A$:
\[
    a\sim b \ssi \phi(a) = \phi(b)
.\]
Como es usual, denotando al conjunto de bloques de la relación como
$A/\simr$, tenemos la función $\eta: A\to A/\simr$ que manda un elemento
$a\in A$ a su bloque $[a]$ correspondiente.
Es decir:
\begin{align*}
    \eta:A&\to A/\simr \\
    a&\mapsto [a] = \{x\in A \mid x\sim a\}.
\end{align*}
En particular, se tiene que
\[
    \eta(x) = \eta(y) \ssi x\sim y
.\]
Una consecuencia es que tenemos una función $\phi^\sharp:A/\simr \to B$ que
cierra el triángulo:
\[
\begin{tikzcd}
    A\ar[rr,"\phi"] \ar[dr,"\eta"'] && B \\
    & A/\simr \ar[ur,"\phi^\sharp"']
\end{tikzcd}
\]
dada como $\phi^\sharp([a]) = \phi(a)$.
El hecho de que $\phi^\sharp$ esté bien definida como función se debe,
precisamente, a la definición de la relación de equivalencia:
si $[a]=[b]$, entonces $a\sim b$, por lo cual $\phi(a)=\phi(b)$, i.e.,
$\phi^\sharp([a])=\phi^\sharp([b])$.
Más aún, también en este caso tenemos $A/\simr\simeq \img \phi$ como
conjuntos.
En particular, si $\phi$ es suprayectiva, entonces $A/\simr \simeq B$.
También aquí sucede que $\phi^\sharp$ es universal con respecto a esta
propiedad: para cualquier función $\psi:A\to B'$ tal que $\psi(a)=\psi(b)$
siempre que $\phi(a)=\phi(b)$, existe una función $\psi^\sharp:A/\simr\to B$
que factoriza a $\psi$ a través de $\eta$.
Luego, $\eta$ es la única función universal con respecto a esta propiedad.
Nos gustaría decir que el ``núcleo'' de $\phi$ es la relación de
equivalencia $\sim$ inducida por $\phi$.

\subsection{Congruencias y cocientes como límites y colímites}
La relación de equivalencia asociada a una función
de conjuntos $\phi:A\to B$ se puede ver, en un contexto categórico, como
el límite del siguiente diagrama:
\[
    \begin{tikzcd}
        & A\ar[d,"\phi"] \\
        A \ar[r,"\phi"'] & B
    \end{tikzcd}
\]
Es decir, un conjunto $\sim$ con dos funciones $p,q:\simr\to A$
que hacen conmutar al cuadrado
\[
    \begin{tikzcd}
        \simr \ar[d,"p"'] \ar[r,"q"] & A\ar[d,"\phi"] \\
        A \ar[r,"\phi"'] & B
    \end{tikzcd}
\]
y que son universales con respecto a esta propiedad.
Como es usual, el conjunto $\sim$ se puede ver como un subconjunto de
$A\times A$, ya que la propiedad del límite anterior asegura que
la flecha vertical
\[
    \begin{tikzcd}
        & \sim \ar[dl,"p"'] \ar[d] \ar[dr,"q"] \\
        A & A\times A \ar[l] \ar[r] & A
    \end{tikzcd}
\]
(inducida por las funciones $p,q:\simr\to A$) es un monomorfismo.
Además, la función $\eta:A\to A/\simr$ se puede ver como el colímite
del diagrama
\[
    \begin{tikzcd}
        \sim \ar[r,"p",shift left] \ar[r,"q"',shift right] & A 
    \end{tikzcd}
.\]
Es decir, las composiciones del diagrama
\[
    \begin{tikzcd}
        \sim \ar[r,"p",shift left] \ar[r,"q"',shift right]
        & A \ar[r,"\eta"]
        &A/\simr 
    \end{tikzcd}
\]
coinciden ($\eta p = \eta q$) y la función $\eta:A\to A/\simr$ es
universal con respecto a esta propiedad.

La misma formulación en términos de límites y colímites describe
las congruencias y cocientes en las categorías de grupos, anillos,
módulos y, como veremos más adelante, $\Sup$-semiretículas,
$\inf$-semiretículas, marcos, etc.

\subsection{\tps{$\Sup$}{}-congruencias y morfismos}
Vamos a fijarnos en la categoría cuyos objetos son las retículas
completas bajo supremos y cuyos morfismos preservan los supremos.

Queremos definir una relación de equivalencia $\simr\subseteq A\times A$
que respete la estructura de $A$ en el siguiente sentido:

Supongamos que tenemos dos subconjuntos $X,Y\subseteq A$ indicados por un
conjunto de índices $I$.
Es decir
\begin{align*}
    X &= \{x_i \mid i\in I\} \\
    Y &= \{y_i \mid i\in I\}
\end{align*}
tales que, para cada índice $i\in I$ se tiene $x_i\sim y_i$.
En esta situación, usaremos la notación $X\sim Y$.

Decimos que una relación de equivalencia $\simr\subseteq A\times A$ es una
$\Sup$-congruencia si, para todo par de subconjuntos $X,Y\subseteq A$ con
$X\sim Y$, se tiene $\Sup X\sim \Sup Y$.

En la categoría de conjuntos, ya vimos que todo morfismo $\phi:A\to B$
induce una relación de equivalencia $\simr$ en $A$  y, recíprocamente,
toda relación de equivalencia $\simr$ en $A$ está inducida por
un morfismo (porque podemos formar el cociente $A/\simr$).

Ahora veremos que esto también sucede en la categoría de retículas
superiormente completas.
El primer resultado está dado por
\begin{lemma}
    La relación de equivalencia $\simr$ en $A$ inducida por un morfismo
    de $\Sup$-semiretículas $\phi:A\to B$ es una $\Sup$-congruencia.
\end{lemma}
\begin{proof}
    Tomemos un morfismo $\phi:A\to B$  de retículas superiormente completas.
    Recordemos que la relación $\simr$ en $A$ está dado como
    \[
        a\sim b \ssi \phi(a) = \phi(b)
    .\]
    Ahora veremos que $\simr$ es una $\Sup$-congruencia.
    Tomemos subconjuntos $X,Y\subseteq A$ tales que $X\sim Y$,
    entonces... (...)
\end{proof}
Ahora nos falta ver que toda $\Sup$-congruencia $\simr$ es la relación de
equivalencia inducida por un morfismo de $\Sup$-semiretículas.
Para esto, basta ver que la función $\eta:A\to A/\simr$ es un
morfismo de $\Sup$-semiretículas.
\begin{lemma}
    Una relación de equivalencia $\simr$ en $A$ es una $\Sup$-congruencia
    si, y solo si, $A/\simr$ es una $\Sup$-semiretícula y
    $\eta:A\to A/\simr$ es un morfismo de $\Sup$-semiretículas.
\end{lemma}
Esta situación es análoga a la que se presenta en las categorías de grupos,
anillos, etc., ya que, en aquellos casos, una relación de equivalencia es
una congruencia de la estructura si, y solo si, la función que la induce es
un morfismo en la categoría.

Para la prueba de este lema, observemos que la parte de ``si'' se sigue
del lema anterior.
La parte de ``solo si'' se deja como ejercicio.
\begin{exe}[Yareli]
    Dada una retícula superiormente completa $A$ y una $\Sup$-congruencia
    $\simr$ sobre $A$, defina un orden en $A/\simr$ y pruebe que
    $\eta:A\to A/\simr$ es un morfismo de retículas superiormente completas.
\end{exe}

De todos modos, la demostración se puede hacer de una manera más limpia con los
lemas adecuados, pasando por otras construcciones equivalentes.

\subsection{\tps{$\Sup$}-congruencias y operadores cerradura}

La idea general es que, dado que estamos en una retícula superiormente completa,
podemos reemplazar los bloques de una $\Sup$-congruencia con sus supremos 
respectivos.
Para esto, lo primero que verificaremos es que el supremo de cada bloque sigue
estando en el bloque:
\begin{lemma}
    Si $\sim$ es una $\Sup$-congruencia en $A$, entonces cada bloque de $\simr$

   tiene un mayor elemento.
    Es decir, para cada $[a]\in A/\simr$, tenemos $\Sup[a]\in [a]$.
\end{lemma}
\begin{proof}
    Tomemos $[a]\in A/\simr$ y definamos nuestro conjunto de índices como $I=[a]$.
    Sean
    \begin{align*}
        X &= \{x_i \mid i\in I\} \\
        Y &= \{y_i \mid i\in I\}
    \end{align*}
    donde $x_i=i$ y $y_i=a$ para cada $i\in I=[a]$.
    Por construcción, tenemos $x_i\sim y_i$ para cada $i\in I$: esto es, $X\sim Y$.
    Luego, como $\sim$ es una $\Sup$-congruencia, se sigue que $\Sup X\sim\Sup Y$.
    Esto es,
    \[
        \Sup[a] = \Sup X \sim \Sup Y = a
    .\]
    Por lo tanto, $\Sup [a]\in [a]$.
\end{proof}
Este resultado dice que, si tomamos un bloque $[a]\in A/\simr$, nos fijamos en su
supremo y luego volvemos a bajar con $\eta:A\to A/\simr$, caemos en
el bloque en el que empezamos.
Es decir:
\[
    \eta(\Sup [a]) = [a]
.\]
Ahora consideramos la otra composición:
\begin{defn}
    Sea $\simr$ una $\Sup$-congruencia en $A$.
    El \emph{selector} de $\simr$ es la función $j:A\to A$ dada por
    \[
        j(a) = \Sup\eta(a) = \Sup[a] = \Sup\{x\in A \mid x\sim a\}
    .\]
\end{defn}
El selector tiene algunas propiedades interesantes.
\begin{itemize}
    \item 
    Si $x\sim a$, entonces $x\leq j(a)$.
    En particular, $a\leq j(a)$, ya que $a\sim a$.
    Decimos que $j$ infla.
    \item
    Supongamos que $a\leq b$.
    Dado que $a\sim j(a)$ y $b\sim j(b)$, entonces
    \[
        b = a\sup b \sim j(a) \sup j(b)
    \]
    porque $\simr$ es $\Sup$-congruencia.
    Luego, $j(a) \sup j(b) \leq j(b)$.
    Entonces
    \[
        j(a)\leq j(a)\sup j(b) \leq j(b)
    .\]
    Es decir, $j$ es monótona.
    \item
    Ahora tomemos $a\in A$.
    Sabemos que $a\sim j(a)$ y que $j(a)\sim j(j(a))$,
    por lo cual $a\sim j(j(a))$.
    Luego, $j(j(a))\leq j(a)$.
    Por otro lado, también tenemos $j(a)\leq j(j(a))$ (porque $j$ infla) así que
    $j(a)=j(j(a))$.
    En otras palabras, $j$ es idempotente.
\end{itemize}
Lo que acabamos de mostrar es que $j$ es un operador cerradura.
En resumen, mostramos que toda $\Sup$-congruencia $\simr$ induce
un operador cerradura $j$.
Hay que mostrar que, dado un operador cerradura $j$ en $A$, podemos obtener una
$\Sup$-congruencia de manera natural y, de hecho, estas construcciones son
inversas una de la otra.
\begin{exe}[Juan $\checkmark$ ]
    Sea $A$ una $\Sup-$retícula. Los selectores de $A$ son precisamente los operadores cerradura. Además cada operador cerradura es selector para una única $\Sup-$congruencia. Es decir, hay una relación biyectiva entre $\Sup-$congruencias y operadores cerradura en $A$.
\end{exe}

\begin{sol}
    Si $j:A\to A$ es un selector para una $\Sup-$congruencia $\simr$ en $A$, veamos que $j$ es un operador cerradura. Esta parte se muestra a raiz de las propiedades que tiene el selector.
    
    Ahora tomemos $j$ un operador cerradura en $A$. Buscamos definir una $\Sup$-congruencia para la cual sea selector. Definamos la relación $x\simr y\leftrightarrows j(x)=j(y)$. La relación $\simr$ es una relación de equivalencia. Veamos que es una $\Sup-$congruencia. Supongamos que $x,Y\subseteq A$ y que $X\simr Y$ y tomemos $x\in X$, $y\in Y$ tales que $x\simr y$. Como $j$ es un operador cerradura tenemos que $y\leq j(y)=j(x)\leq j(\Sup X)$ para cualquier $y\in Y$. Entonces $\Sup Y\leq j(\Sup X)\Rightarrow j(\Sup Y)\leq j(\Sup X)$. Faltaría ver la otra igualdad, pero se realiza de forma análoga cambiando $x$ por $y$. De esta forma la relación definida es una $\Sup-$congruencia. Notemos que es un selector. Supongamos que $f:Ato A$, $f(a)=\Sup\{x\in A|x\simr a\}$ es el selector. Queremos ver que $f=j$. Sea $a\in A$ como $f$ es selector, $a\simr f(a)$, esto es equivalente a que $j(a)=j(f(a))$, pero $j(j(a))=j(f(a))$. Entonces $j(a)\leq f(a)$. Por lo tanto $j\leq f$.
    
    Por otro lado para cualquier $x\simr a$, $x\leq j(x)=j(a)$. Así $f(a)\leq j(a)$.. Por lo tanto $f\leq j$.
    
    Finalmente para ver que es una relación biyectiva la que hay entre selectores y operadores cerradura veremos que un operador cerradura nos define una única $\Sup-$congruencia. Supongamos que $j$, un operador cerradura, es un selector para una $\Sup-$congruencia en $A$. Basta probar que la congruencia está definida como $x\simr y\leftrightarrows j(x)=j(y)$ para cualesquiera $x,y\in A$. Tomamos $x,y\in A$ arbitrarios de tal forma que $x\simr y$. Como $j$ es selector de la congruencia $x\leq j(y)$ y $y\leq j(x)$. De aquí que $j(x)\leq j(y)$ y $j(y)\leq j(x)$, es decir, $x\simr y\Rightarrow j(x)=j(y)$. Ahora bien, si $j(x)=j(y)$ para cualesquiera $x, y\in A$, como $j$ es selector, $x\simr j(x)=j(y)\simr y$ y por lo tanto $x\simr y$.
    
    Así queda probado lo que queríamos.
    
\end{sol}

\subsection{Operadores cerradura y conjuntos \tps{$\Inf$}-cerrados}

Ahora veremos que los operadores cerradura en una $\Sup$-semiretícula $A$
también están en correspondencia con los subconjuntos de $A$ que son cerrados
bajo ínfimos arbitrarios.

\begin{defn}
    Un subconjunto $F\subseteq A$ se dice que es $\Inf$-cerrado si,
    para todo subconjunto $X\subseteq F$, se tiene $\Sup X\in F$.
\end{defn}

La correspondencia entre operadores cerradura y conjuntos $\Inf$-cerrados
está dado por los siguientes dos resultados.

\begin{lemma}
Dado un operador cerradura $j$ en $A$, el conjunto de sus puntos fijos
\[
    A_j = \{x\in A \mid j(x) = x\}
\]
es $\Inf$-cerrado.
\end{lemma}
\begin{proof}
En efecto, tomemos un subconjunto $X\subseteq A_j$.
Para cualquier $x\in X$ tenemos $\Inf X\leq x$, por lo cual
\[
    j(\Inf X) \leq j(x) = x
\]
ya que $j$ es monótona y $x\in A_j$.
Esto significa que $j(\Inf X)$ es cota inferior de $X$, por lo cual
\[
    j(\Inf X)\leq \Inf X
\]
y, además, la otra desigualdad $\Inf X\leq j(\Inf X)$ se da porque $j$ infla.
Luego, $j(\Inf X) = \Inf X$, pero esto es $\Inf X \in A_j$.
\end{proof}

\begin{lemma}
Dado un conjunto $\Inf$-cerrado $F\subseteq A$, podemos construir un
operador cerradura $j_F:A\to A$ asociado como sigue:
\[
    j_F(a) = \Inf\{x\in F \mid a\leq x\}
.\]
\end{lemma}
\begin{proof}
En efecto, dado que cualquier $a\in A$ es cota inferior de
$\{x\in F \mid a\leq x\}$, se sigue que $a\leq j_F(a)$.

Ahora, dados $a\leq b\in A$, tenemos
\[
    \{x\in F \mid a\leq x\} \supseteq \{x\in F \mid b\leq x\}
\]
lo cual, tomando ínfimos, nos da $j_F(a) \leq j_F(b)$.

Finalmente, como $F$ es $\Inf$-cerrado, para cualquier $a\in A$
tenemos $j_F(a)\in F$ (por definición de $j_F$),
de modo que $j_F(a) \in\{x\in F \mid j_F(a) \leq x\}$.
Una vez más, por la definición de $j_F$ se sigue que $j_F(j_F(a)) \leq j_F(a)$
y la otra desigualdad $j_f(a)\leq j_F(j_F(a))$ es porque $j_F$ infla.
\end{proof}
\begin{exe}[Alfredo $\checkmark$ ]
    Verifique que las construcciones $F\mapsto j_F$
    y $j\mapsto A_j$ son inversas una de la otra y, por lo tanto,
    establecen una biyección entre operadores
    cerradura en $A$ y conjuntos $\Inf$-cerrados de $A$.
\end{exe}
\begin{sol}
    Sea $F\subseteq A$ un conjunto $\Inf$-cerrado.
    Entonces
    \begin{align*}
        j_F(a) &= \Inf\{x\in F \mid a\leq x\} \\
        A_{j_F} &= \{a\in A \mid j_F(a)=a\}.
    \end{align*}
    Observemos que $F\subseteq A_{j_F}$, ya que
    $a=\Inf\{x\in F\mid a\leq x\}$ para todo $a\in F$.
    Por otro lado,
    como $F$ es $\Inf$-cerrado, tenemos $j_F[A]\subseteq F$.
    Luego, $A_{j_F}=j_F[A_{j_F}] \subseteq F$ y, así, $F=A_{j_F}$.
    
    Recíprocamente, sea $j:A\to A$ un operador cerradura.
    Entonces
    \begin{align*}
        A_j &= \{a\in A \mid j(a)=a \} \\
        j_{A_j}(a) &= \Inf\{x\in A_j \mid a\leq x\}.
    \end{align*}
    Como $j$ infla y es idempotente, todo $a\in A$ satisface
    \[
        j(a) \in \{x\in A_j \mid a\leq x\}
    .\]
    Además, $j(a)$ es una cota inferior del conjunto,
    pues para todo
    $x\in A_j$ con $a\leq x$ tenemos $j(a)\leq j(x)=x$.
    Por lo tanto,
    \[
        j(a) = \Inf\{x\in A_j \mid a\leq x\} = j_{A_j}(a)
    .\]
    Esto muestra que $j=j_{A_j}$.
    Luego, las construcciones $F\mapsto j_F$ y $j\mapsto A_j$
    son inversas, lo cual establece la biyección deseada.
\end{sol}

\subsection{De conjuntos \tps{$\Inf$}-cerrados a morfismos}

Ahora queremos completar el círculo, volviendo a las $\Sup$-congruencias
inducidas por los morfismos.
Usando lo que probamos acerca de conjuntos $\Inf$-cerrados,
mostraremos que los operadores cerradura de $A$ son exactamente
los núcleos de los morfismos $\phi:A\to B$.
En secciones anteriores vimos que u

Ya vimos que, dado un operador cerradura $j$, el conjunto de puntos fijos $A_j$
de $j$ es $\Inf$ cerrado y, por lo tanto, es una retícula completa.
Es decir $A_j$ tiene supremos arbitrarios aunque, en general,
éstos no coincidan con lo supremos en $A$.
Veremos que la función $j^*:A\to A_j$ dada como $j^*(a) = j(a)$ es $j$
es, de hecho, un morfismo de $\Sup$-retículas que tiene núcleo $j$.

La monotonía de $j^*$ es inmediata, pues $j$ es monótona y $A_j$ hereda el
orden de $A$.
Luego, solo es necesario verificar que $j^*$ preserva los supremos.
Es decir, que para cualquier subconjunto $Y\subseteq A$ se tiene
\[
    j^*(\Sup Y) = \Sup_j j^*[Y]
,\]
donde $\Sup_j$ denota al supremo en el orden de $A_j$.
Para esto, veremos primero cómo se calculan los supremos en $A_j$.
\begin{lemma}
    Para cualquier subconjunto $Y\subseteq A_j$, se tiene
    \[
        \Sup_j Y = j(\Sup Y)
    .\]
\end{lemma}
\begin{proof}
    Como $j$ infla, es claro que $j(\Sup Y)$ es cota superior de $X$.
    Ahora supongamos que $a\in A_j$ es una cota superior de $Y\subseteq A_j$.
    Considerado el supremo de $Y$ en $A$, tenemos
    \[
        \Sup Y\leq a
    .\]
    Luego, aplicando $j$, obtenemos
    \[
        j(\Sup Y) \leq j(a) = a
    .\]
    y así $j(\Sup Y)$ es la mínima cota superior de $Y$ en $A_j$.
\end{proof}
Con este lema, es fácil demostrar que $j^*:A\to A_j$ es un morfismo de
$\Sup$-semiretículas.
\begin{exe}[Armando]
    Muestre que $j^*:A\to A_j$ es un morfismo suprayectivo de $\Sup$-semiretículas.
\end{exe}

\subsection{De morfismos a operadores cerradura}

En la sección anterior vimos que un morfismo de
$\Sup$-semimretículas $f:A\to B$
induce una $\Sup$-congruencia $\simr$ en $A$ dada como
\[
    x\sim y \ssi f(x) = f(y)
.\]
Tambień probamos que las $\Sup$-congruencias están en correspondencia
con los operadores cerradura.
Por lo tanto hacemos la definición
\begin{defn}
    Para cada $\Sup$-morfismo $f:A\to B$, el núcleo de $\phi$
    es el único operador cerradura $k:A\to A$ que le corresponde a
    la $\Sup$-congruencia inducida por $f$.
    Es decir $k$ está determinado por la condición
    \[
        k(x) = k(y) \ssi f(x) = f(y)
    .\]
\end{defn}

Ahora veremos que el núcleo $k:A\to A$ de $f:A\to B$ se puede
describir explícitamente.
Como $f:A\to B$ preserva supremos, sabemos que tiene un adjunto derecho
$f_*:B\to A$ determinado por la condición
\[
    a\leq f_*(b) \ssi f(a) \leq b
\]
o, explícitamente
\[
    f_*(b) = \Sup\{x\in A \mid f(x) \leq b\}
.\]
Queremos ver que, de hecho, el núcleo de $f$ está dado por $f_*f$.
Primero veremos que $k=f_*f$ es un operador cerradura.
\begin{itemize}
    \item 
    Por definición
    \[
        k(a) = f_*f(a) = \Sup\{x\in A \mid f(x) \leq f(a)\}
    .\]
    De esta ecuación podemos ver que $a\leq k(a)$, pues $f(a)\leq f(a)$.
    Es decir, $k$ infla.
    \item
    Para ver la idempotencia de $k$, es suficiente calcular $ff_*f:A\to B$.
    \begin{align*}
        ff_*f(a)
        &= f(k(a)) \\
        &= f(\Sup\{x\in A \mid f(x) \leq f(a) \}) \\
        &= \Sup\{f(x) \mid f(x) \leq f(b) \} \\
        &= f(a)
    \end{align*}
    Por lo tanto, $ff_*f=f$, de modo que $k^2 = f_*ff_*f=f_*f=k$, como
    se quería
    \item
    Finalmente $k=f_*f$ es monótona porque $f$ y $f_*$ lo son.
    Alternativamente, observemos que, si $a\leq b\in A$, entonces $f(a)\leq f(b)$,
    por lo cual
    \[
        \{x\in A \mid f(x) \leq f(a)\} \subseteq \{x\in A \mid f(x) \leq f(b) \}
    .\]
    Tomando supremos, obtenemos $k(a) \leq k(b)$.
\end{itemize}

Recordemos que queríamos probar que $k$ es el núcleo de $f$.
Tomemos $x,y\in A$ tales que $k(x)=k(y)$.
Esto es, $f_*f(x) = f_*f(y)$, lo cual sucede si, y solo si,
$f(x)=ff_*f(x)=f(y)$ (por definición de la adjunción), así que ya está.
De hecho, podemos probar más:

\begin{cor}
    Dado un morfismo $f:A\to B$ de $\Sup$-semiretículas, el núcleo de $f$
    es el operador cerradura $k:A\to A$ determinado por la condición
    \[
        x\leq k(a) \ssi f(x) \leq f(a)
    .\]
\end{cor}
\begin{proof}
    Recordemos que la adjunción nos da la equivalencia
    \[
        x\leq f_*(y) \ssi f(x) \leq y
    .\]
    Tomando $y=f(a)$, tenemos
    \[
        x\leq k(a) \ssi f(x) \leq f(a)
    .\]
\end{proof}

Con toda la información que tenemos, podemos probar el teorema del cociente
en la categoría de $\Sup$-semiretículas.
\begin{thm}
    Sea $A$ una $\Sup$-semiretícula y $k$ un operador cerradura en $A$.
    Si $f:A\to B$ es un morfismo cuyo núcleo $k$ satisface $j\leq k$,
    entonces existe un único morfismo $f^\sharp:A_j\to B$ tal que
    el siguiente triángulo conmuta.
    \[
        \begin{tikzcd}
            A \ar[rr,"f"] \ar[dr,"j^*"'] &&  B \\
            &A_j \ar[ur,"f^\sharp"']
        \end{tikzcd}
    .\]
\end{thm}
    
\subsection*{(SESIÓN 7: 30 SEP)}

\subsection{Núcleos, conjuntos implicativos y cocientes}

\begin{defn}[Núcleo de marcos]
Un núcleo en un marco $A$ es un operador cerradura $j:A \to A$ que cumple con $j(a\wedge b)=j(a)\wedge j(b) \ \forall \ a,b \ \in \ A$.Si $j:A\to A$ es un núcleo, se dice que $j\in NA$
Nótese que para un morfismo $f: A\to A$ el operador cerradura $k$ definido como el núcleo de conjunto parcialmente ordenado también es un núcleo de marcos, ya que
\begin{align*}
x\leq k(a\wedge b)&\iff f(x)\leq f(a\wedge b)=f(a)\wedge f(b)\\
&\iff f(x)\leq f(a) , f(x)\leq f(b) \\
&\iff x\leq k(a), x\leq k(b)\\
& \iff x\leq k(a)\wedge k(b)
\end{align*}
\end{defn}
\begin{defn}[Conjunto implicativo]
Un Conjunto $F\subset A$ es implicativo si es $\Inf-$ cerrado y $a\in F \iff (x\succ a)\in F \ \forall \ a,x \ \in \ A$
\end{defn}
\begin{lemma}
Sea $k\in CA$. $k$ es un núcleo si  y sólo si $A_k$ es implicativo.
\end{lemma}
\begin{proof}
$\Leftarrow$\\
Sean $x,y \in A$ y sea $a=k(x\inf y)$, y nótese que $a\in A_k$. Así, se cumple que 
\begin{align*}
    x\inf y\leq a & \iff y\leq (x\succ a)\in A_k\\
    \Rightarrow & k(y)\leq (x\succ a)\\
    \iff & k(y)\inf x\leq a
\end{align*}
Similarmente, $k(x)\inf y\leq a$, por lo que 
$$ k(x)\inf k(y)\leq a$$
También, $a\leq k(x)\inf k(a)$, ya que $k$ infla. Por lo tanto, $k(x\inf y)=k(x)\inf k(y)$ y $k\in NA$.\vspace{5mm}

$\Rightarrow$\\
Sea $y=(x\succ a)$, donde $a\in A_k, x\in A$.Así, $x\inf y\leq a$, y se cumple que 
$$x\inf k(y)\leq k(x)\inf k(y) =k(x\inf y)\leq k(a)$$
por lo que $x\inf k(y)\leq k(a)=a$, lo que implica que $k(y)\leq (x\succ a)=y$. Por lo tanto, $k(y)=y$ y $y\in A_k$.
\end{proof}
Nótese que si $j\in CA$, entonces $A_k$ es una $\Sup$-semiretícula. También, por el lema anterior, si $A_j$ es implicativo, entonces $A_j$ tiene una implicación, lo que implica que $A_j$ es un marco. Por ello, se cumple que 
$$j(\Sup X)=\Sup^jX$$
para cualesquiera $j\in NA, X\subset A_j$.
También, si $j \in NA$, la función $j^*:A\to A_j$ definida como $j^*(a)=j(a)$ es un morfismo de marcos suprayectivo.
\begin{exe}[Dante $\checkmark$ ]
Para cualquier marco $A$, existe una biyección entre el conjunto de cocientes de $A$ y $NA$.
\end{exe}
\begin{sol}
    Defínase $\mathcal{C}=\{\text{Cocientes de } \ A\}$. Sea $f:A\to B\in \mathcal{C}$. Así, $F=f_*\circ f\in NA$, y si $g:A\to B\in\mathcal{C}$ es tal que $G=g_*\circ g=f_*\circ f=F$, entonces $A_F=A_G$, y $F^*=G^*$, donde $F^*:A\to A_F$ se define como $F^*(a)=F(a)$. Sabemos entonces que existe un único morfismo $f^\sharp:A_F\to B$ tal que el diagrama
    \[
        \begin{tikzcd}
            A \ar[rr,"f"] \ar[dr,"F^*"'] &&  B \\
            &A_j \ar[ur,"f^\sharp"']
        \end{tikzcd}
    \]
    conmuta, así como $g^\sharp$ que cumple lo mismo para $g$ y $A_G$. Sin embargo, como $G=F$, $A_F=A_G$ y $F^*=G^*$, entonces $f=F^\sharp\circ F^*=G^\sharp\circ G^*=G$, y por lo tanto la función $\phi:\mathcal{C}\to NA$ definida como $\phi(f)=F=f_*\circ f$ es inyectiva. También, si $j\in NA$, claramente $j^*:A\to A_j$ es un cociente de $A$, con $j_*\circ j =j$. Por lo tanto, $\phi$ es una biyección entre $\mathcal{C} y NA$.\vspace{3mm}
    
    \hfill $\blacksquare$
\end{sol}\vspace{3mm}

\begin{exe}[Yareli]
Sean $A\in Frm$ y $j\in NA$, y considérese $f:A\to B$ morfismo de marcos con núcleo $k\leq j$.Entonces, existe un único morfismo $f^{\sharp}:A_j \to B$ tal que el diagrama
\[
        \begin{tikzcd}
            A \ar[rr,"f"] \ar[dr,"j^*"'] &&  B \\
            &A_j \ar[ur,"f^\sharp"']
        \end{tikzcd}
    .\]
conmuta.
\end{exe}
El ejercicio anterior implica que todo núcleo $j\in NA$ es el núcleo de un morfismo; específicamente, del morfismo $j^*:A\to A_j$.
\begin{defn}[Núcleos cerrado, abierto y regular]
Sea $a\in A$ fijo, y considérense los operadores siguientes.
\begin{itemize}
\item $u_a:A\to A$, definido como $u_a(x)= a\sup x$.
\item $v_a:A\to A$, definido como $v_a(x)=(a\succ x)$
\item $w_a:A\to A$, definido como $w_a(x)=((x\succ a)\succ a)$
\end{itemize}

\subsection{Núcleos en una topología}

Claramente, estos opradores son núcleos sobre $A$. $u_a$ es llamado núcleo cerrado, $v_a$ es llamado núcleo abierto, y $w_a$, núcleo regular.
\end{defn}
\begin{exe}[Juan $\checkmark$ ]
Considérese el marco $\cal O S$, donde $S\in Top$. ¿Cómo se definen los núcleos abierto, cerrado y regular?
\end{exe}

\begin{sol}
    Sea $\mathcal{O}S\in $\textbf{\textit{Frm}}, donde $\mathcal{O}S$ es el conjunto de abiertos de $S$. Consideremos $A\in \mathcal{O}S$
    
    \begin{itemize}
        \item $u_A(X)=A\cup X$
        \item $v_A(X)=(A\succ X)=(A'\cup X)^\circ$
        \item $w_A(X)=((X\succ A)\succ A)=((X\succ A)'\cup A)^\circ=(((X'\cup A)^\circ)'\cup A)^\circ=(X'-A')^-$
    \end{itemize}
\end{sol}

Sean $S\in Top$, y $E\subset S$ fijo. se define el operador $[E]:\cal O S\to \cal O S$ como $[E](U)=\big(E\cup U\big)^{\circ}$.
Nótese que, si $E \ \in \cal O S$, entonces $[E]=\unuc E$, y que $V \ \in \cal O S_{\unuc E} \iff E\cup V=V \iff E\subset V$.
Luego,
\[
    (\cal OS)_{\unuc E} = \{V\in \cal OS \mid E\subset V\}
.\]
Ahora considérese, para un $U\in \cal O S$ fijo, el núcleo abierto $\vnuc U$ calculado de manera $\vnuc U(V)=(U^{\prime}\cup V)^{\circ}$, y nótese que $V \in(\cal OS)_{\vnuc U} \iff (U^{\prime}\cup V)^{\circ}=V \iff V\subset U$, por lo que $\cal O S_{\vnuc U}=\cal O U$.

[El último $\iff$ no es cierto.
Esto se ve en ejercicio siguiente -Alfredo].
\begin{exe}[Alfredo $\checkmark$ ]
    Demostrar que
    \[
        \cal OS_{\vnuc U} \simeq
        \{V\in \cal OS : V\subseteq U\} = \cal OU
    \]
    como conjuntos parcialmente ordenados.
\end{exe}
\begin{sol}
    Consideremos las funciones
    $\cal OU\leftrightarrows(\cal OS)_{\vnuc U}$
    dadas por
    \begin{align*}
         V &\mapsto (U\succ V) \\
         W\cap U &\mapsfrom W.
    \end{align*}
    Para cualesquiera $V\in \cal OU$
    y $W\in (\cal OS)_{\vnuc U}$, tenemos
    \begin{align*}
        (U\succ V)\cap U
        &= U\cap V \\
        &= V \\
        (U\succ (W\cap U))
        &= (U\succ W)\cap(U\succ U) \\
        &= U\succ W \\
        &= W
    \end{align*}
    Más aún, la intersección es monótona
    y la implicación es monótona en la segunda entrada, así que
    estas funciones son un isomorfismo en $\Pos$.
\end{sol}

Considerando $w_{\emptyset}:\cal O S\to \cal O S$, definido como 
$$w_{\emptyset}(U)=((U\succ\emptyset)\succ\emptyset)=(\overline{U})^{\circ}$$
se tiene que los puntos fijos son 
$$\cal O S_{W_{\emptyset}}=\{U\in \cal O S : (\overline{U})^{\circ}=U\}=\{\text{Abiertos regulares de }S\}$$.
\begin{exe}[Armando]
Demostrar que el conjunto de abiertos regulares de un espacio topológico es un álgebra Booleana completa.
\end{exe}
\begin{lemma}
Si $a\in \ A \ \in \ Frm$, entonces $A_{w_a} \ \in \ CBA$.
\end{lemma}
\begin{proof}
$$w_a(0)=((0\succ a)\succ a)=(1\succ a)=a$$
por lo tanto, $a \ \in \ A_{w_a}$ y $a$ es el menor elemento de $A_{w_a}$. Ahora bien, sea $x \ \in A_{w_a}$, y considérese $y=(x\succ a)$. Por un lado, 
\begin{align*}
    w_a(y) & =w_a(x\succ a)\\
    &=(w_a(x)\succ a)\\
    &=(x\succ a)\\
    &=y
\end{align*}
por lo que $y \ \in \ A_{w_a}$. Ahora bien, 
\begin{align*}
    x\inf y&=x\inf(x\succ a)\\
    &=x\inf a\\
    &=a
\end{align*}
Y también, 
\begin{align*}
    w_a(x\sup y)&=((x\sup y)\succ a)\succ a)\\
    &=(((x\succ a)\inf(y\succ a))\succ a)\\
    &=((y\inf(y\succ a))\succ a)\\
    &=(a\succ a)\\
    &=1
\end{align*}
Por lo tanto, todo elemento de $A_{w_a}$ tiene un complemento, y $A_{w_a} \ \in \ CBA$.
\end{proof}
Un caso particular del resultado anterior es $A_{\neg \neg}=A_{w_0} \ \in \ CBA$.
\subsection{VIDEO 4: ordinales: (4 OCT)}
\begin{exe}[Dante $\checkmark$ ]
  Sea $A$ un conjunto linealmente ordenado. $A$ es bien ordenado si y sólo si $A^{*}=A\cup \{\star\}$, donde $\star$ es un elemento genérico, también lo es.
\end{exe}
\begin{sol}
    $\Rightarrow$\\
    Sea $(A,\leq^{\prime})$ bien ordenado y considérese la relación $\leq\subset A^*\times A^*$ definida como $\leq^{\prime} \cup \{(\star,a):a \ \in A\}\cup \{(\star,\star)\}$. 
    \begin{enumerate}
        \item Claramente, $\leq$ es una relación de orden.
        \item Sean $x,y\in A^*$. Si $x,y \neq \star$, como $A$ es totalmente ordenado, $x\leq y$ o $y\leq y$. Si $x\neq \star, y=\star$, entonces $y\leq x$,  si $x,y=\star$, entonces $x\leq y $ y $y\leq x$.
        \item Sea $B\subset A$. Si $\star\in B, \star\leq b \ \forall \ b\in B$, y por como está definida $\leq$, no existe ningún $b\in B$ tal que $b\leq \star$, por lo que $\star$ es el menor elemento de $B$.
        Si $\star\not\in B$, entonces $B\subset A$, y como $\leq$ es equivalente a $\leq^{\prime}$ sobre $A$, entonces $B$ tiene un menor elemento en $B\subset A \subset A^*$.
    \end{enumerate}
    Por lo anterior, $A^*$ es bien ordenado.\vspace{3mm}
    
    $\Leftarrow$\vspace{3mm}
    
    Si $(A^*,\leq)$ es bien ordenado, defínase el orden $\leq^{\prime}\subset A\times A$ como
    \[
        \leq^{\prime}=\leq\setminus \left( \{(\star,a):a\in A\} \cup \{(a,\star):a\in A\}\right)
    .\]
    \begin{enumerate}
        \item Claramente, $\leq^{\prime} $ es una relación de orden total.
        \item Sea $B\subset A$. Como $A\subset A^*$, entonces $B\subset A^*$, por lo que $\exists b\in B$ que es el menor elemento de $B$.
    \end{enumerate}
    Por lo anterior, $A$ es bien ordenado.\vspace{3mm}
    
    \hfill $\blacksquare$
\end{sol}\vspace{3mm}

\begin{exe}[Yareli $\checkmark$ ]
  Sean $A, B$ conjuntos bien ordenados, y $f,g:A\to B$ dos encajes. Entonces $f=g$.
\end{exe}
\begin{proof}
Sea $a_0\in A$. Como $B$ es bien ordenado, los elementos $f(a_0)$ y $g(a_0)$ son comparables. Sin perder generalidad, supongamos que $f(a_0)\leq g(a_0)$.\\
Entonces $f(a_0)=g(a_1)\leq g(a_0)$ para algún $a_1\in A$ ya que $g[A]$ es sección inferior. Además $a_1\leq a_0$ ya que, en caso contrario, se tiene que $g(a_1)>g(a_0)$, lo cual es una contradiccion.\\
Notemos que $f(a_1)\in g[A]$ ya que $f(a_1)\leq f(a_0)$, es decir, existe un $a_2\in A$ tal que $f(a_1)=g(a_2)$. Repitiendo este proceso obtenemos las siguientes cadenas:
\[
\begin{tikzcd}[row sep=3mm]
                  &  &                                & g(a_0)               \\
a_0               &  & f(a_0) \arrow[r,equal]               & g(a_1) \arrow[u,no head]     \\
a_1 \arrow[u,no head]     &  & f(a_1) \arrow[u,no head] \arrow[r,equal]     & g(a_2) \arrow[u,no head]     \\
a_2 \arrow[u,no head]     &  & f(a_2) \arrow[u,no head] \arrow[r,equal]     & g(a_3) \arrow[u,no head]     \\
\vdots \arrow[u,no head]  &  & \vdots \arrow[u,no head]               & \vdots \arrow[u,no head]     \\
a_{n-2} \arrow[u,no head] &  & f(a_{n-2}) \arrow[u,no head] \arrow[r,equal] & g(a_{n-1}) \arrow[u,no head] \\
a_{n-1} \arrow[u,no head] &  & f(a_{n-1}) \arrow[u,no head] \arrow[r,equal] & g(a_n) \arrow[u,no head]     \\
a_{n} \arrow[u,no head]   &  & f(a_n) \arrow[u,no head] \arrow[r,equal]     & g(a_{n+1}) \arrow[u,no head] \\
a_{n+1} \arrow[u,no head] &  & f(a_{n+1}) \arrow[u,no head] \arrow[r,equal] & g(a_{n+2}) \arrow[u,no head] \\
\vdots \arrow[u,no head]  &  & \vdots \arrow[u,no head]               & \vdots \arrow[u,no head]    
\end{tikzcd}
\]
Como $A$ es bien ordenado, la cadena de la izquierda tiene un mínimo. Si $a_n$ es el mínimo de esta cadena, entonces $a_n=a_{n+1}$.
Tenemos que
\begin{align*}
a_{n+1}=a_n&\Rightarrow f(a_n)=g(a_{n+1})=g(a_n)=f(a_{n-1})\\
\Rightarrow a_n=a_{n-1}, \textit{ por inyectividad de f}&\Rightarrow f(a_{n-1})=g(a_n)=g(a_{n-1})=f(a_{n-2})\\
\Rightarrow a_{n-1}=a_{n-2},\textit{ por inyectividad de f}&\Rightarrow f(a_{n-2})=g(a_{n-1})=g(a_{n-2})=f(a_{n-3})\\
&\vdots\\
\Rightarrow a_2=a_1,\textit{ por inyectividad de f}&\Rightarrow f(a_1)=g(a_2)=g(a_1)=f(a_0)\\
\Rightarrow a_1=a_0&\Rightarrow g(a_0)=g(a_1)=f(a_0)
\end{align*}
Como el elemento $a_0\in A$ es arbitrario, concluimos que $f$ y $g$ tienen la misma regla de correspondencia, es decir, $f=g$.
\end{proof}

\subsection*{(SESIÓN 8: 5 OCT)}

\subsection{Núcleos espacialmente inducidos en una topología}

\begin{exe}[Juan $\checkmark$ ]
  Muestra que, dado cualquier núcleo $j\in NA$ se tiene
  \[
    j(0)=0 \ssi j\leq \wnuc 0
  .\]
\end{exe}

\begin{sol}
    Supongamos que $j\leq w_0$.\\
    Notemos que $0\leq j(0)\leq w_0(0)=((0\succ 0)\succ 0)=0$. Así, $j(0)=0$.\\
    \noindent
    Supongamos ahora que $j(0)=0$. Como $w_0$ es núcleo, $j(x)\leq w_0(j(x))=((j(x)\succ 0)\succ 0)=((j(x)\succ J(0))\succ 0)$. Notemos que $(j(x)\succ j(0))=(x\succ j(0))$. Así, $((j(x)\succ j(0))\succ j(a))=((x\succ j(0))\succ j(0))=((x\succ 0)\succ 0)=w_0(x)$. Por lo tanto $j(x)\leq w_0(x).$
\end{sol}

Por el teorema de factorización que ya probamos, siempre que
$j\leq k\in NA$, existe un único morfismo $A_j\to A_k$ tal que el
diagrama
\[
    \begin{tikzcd}
      A \ar[r,"k^*"] \ar[d,"j^*"'] & A_k \\
      A_j \ar[ur]
    \end{tikzcd}
\]
es conmutativo.

\begin{thm}
  Sea $\phi:T\to S$ un morfismo de espacios topológicos y
  consideremos el morfismo de marcos
  $\phi^*:\cal OS\to\cal OT$.
  Entonces el adjunto derecho $\phi_*:\cal OT \to \cal OS$
  de $\phi^*$, dado por
  \[
    \phi_*(W)
    = \bigcup\{U\in\cal OS \mid \phi^*(U) \leq W\}
  ,\]
  tiene la descripción
  \[
    \phi_*(W) = \ol{\phi[W']}' = (\phi[W']')^\circ
  .\]
\end{thm}

\begin{thm}
  Su $\phi:T\to S$ es un morfismo de espacios topológicos,
  el núcleo $k:\cal OS\to\cal OS$ del morfismo de marcos
  $\phi^*:\cal OS\to\cal OT$ tiene la descripción
  \[
    \phi(U) = (U\cup (S-\phi[T]))^\circ
  .\]
\end{thm}

\begin{defn}
  Un núcleo $j\in N\cal O S$ es \emph{espacialmente
  inducido} si es de la forma $j=[E]$ para algún
  $E\subseteq S$.
\end{defn}

\begin{thm}
  Supongamos que $f:A\to B$ es un morfismo suprayectivo,
  donde $B$ es un álgebra booleana completa.
  Si $k$ es el núcleo de $f$,
  Entonces $k=\wnuc a$, donde $a=k(0)$.
  
  En otras palabras, todo cociente booleano de un marco
  es un $\wnuc a$.
\end{thm}

\begin{thm}
  Sea $A$ un marco.
  Si $A$ es un álgebra booleana completa y $j\in NA$,
  entonces $j=\unuc a$, donde $a=j(0)$.
  
  Dado que todo cociente de un álgebra booleana completa
  también lo es, del resultado anterior también se sigue
  que $j=\wnuc a$.
\end{thm}

\part{El ensamble}

\section{Derivadas, estables, prenúcleos y núcleos}

\subsection*{(SESIÓN 9: 7 OCT)}

\subsection{Motivación no topológica de por qué usar núcleos}

El morfismo de localización $\phi:\Z\to\Q$ induce una adjunción
entre $\Mod_\Z$ y $\Mod_\Q$.

El sistema dirigido
\begin{align*}
     \Lambda(\Z)^* &\to \Mod_\Z \\
     I &\mapsto \Mod_\Z(I,\Z)
\end{align*}
tiene un co-cono con vértice $\Q$,
\[
    \begin{tikzcd}[column sep=2mm]
      \Mod_\Z(\<n\>,\Z) \ar[rr] \ar[dr]
      && \Mod_\Z(\<m\>,\Z) \ar[dl] \\[7mm]
      & \Q
    \end{tikzcd}
\]
donde los morfismos $\Mod_\Z(\<n\>,\Z)\to\Q$ están dados por
$f\mapsto f(n)/n$.

\begin{exe}[Alfredo $\checkmark$ ]
  Mostrar que el morfismo inducido en el colímite
  \begin{align*}
      \nu : \varinjlim_{n\neq 0}\Mod_\Z(\<n\>,\Z)&\to \Q \\
      [f] &\mapsto \frac{f(n)}{n}
  \end{align*}
  es un isomorfismo.
\end{exe}
\begin{sol}
    Primero veamos que $\nu$ está bien definida.
    Tomemos $[f]\in [g]$.
    Es decir, $f:\<m\>\to\Z$ y $g:\<n\>\to\Z$ satisfacen
    \[
        f|_{\<k\>} = g|_{\<k\>}
    \]
    para alguna $k\in\Z$ con $m\mid k$, $n\mid k$.
    Es decir, $k=rm$, $k=sn$ para algunos $r,s\in\Z$.
    Luego, tenemos
    \begin{align*}
        \nu([g])
        &= \frac {g(n)} n
        = \frac {sg(n)}{sn} 
        = \frac {g(sn)}{sn} 
        = \frac {g(k)}{k} \\
        &= \frac {f(k)}{k} 
        = \frac {f(rm)}{rm} 
        = \frac {rf(m)}{rm} 
        = \frac {f(m)}{m} 
        = \nu([f]).
    \end{align*}
    Ahora, dado $r/m\in\Q$, consideremos la función
    $f:\<m\>\to\Z$ dada por $am\mapsto ar$.
    En efecto, $f$ respeta sumas y productos por enteros,
    así que podemos considerar $[f]$.
    
    Queremos ver que la asignación $r/m\mapsto [f]$ no
    depende de $r$ y de $m$, sino solo del cociente $r/m$.
    
    Para esto, tomamos $s/n=r/m$ y
    debemos mostrar que $g:\<n\>\to\Z$
    dada como $an\mapsto as$ cumple $[f]=[g]$.
    Como $s/n=r/m$, entonces $sm=rn$, así que $mn$ cumple
    \begin{align*}
        f|_{\<mn\>}(amn)
        &= f(amn)
        = arn
        = asm
        = g(anm)
        = g|_{\<mn\>}(amn),
    \end{align*}
    por lo cual $f|_{\<mn\>}=g|_{\<mn\>}$ y $[f]=[g]$.
    
    Luego, la asignación $r/m\mapsto [(am\mapsto ar)]$
    es una función
    \[
        \mu:\Q\to\varinjlim_{n\neq 0}\Mod_\Z(\<n\>,\Z)
    .\]
    bien definida.
    También es $\Z$-lineal, ya que
    \begin{align*}
        \mu\left(a\frac r m + \frac s n\right)
        &= \mu\left(\frac{arn + ms}{mn}\right) \\
        &= \left( xmn \mapsto x(arn+ms) \right) \\
        &= (xmn \mapsto xarn+xms) \\
        &= (xmn \mapsto xarn) + (xmn \mapsto xms) \\
        &= a(xmn \mapsto xrn) + (xmn \mapsto xms) \\
        &= a\mu\left(\frac{rn}{mn}\right)
            + \mu\left(\frac{ms}{mn}\right) \\
        &= a\mu\left(\frac{r}{m}\right)
            + \mu\left(\frac{s}{n}\right).
    \end{align*}
    Finalmente, observemos que
    \begin{align*}
        \mu(\nu(r/m))
        &= \mu(am\mapsto ar) \\
        &= \frac{(am\mapsto ar)(m)}{m} \\
        &= \frac{r}{m} \\
        \nu(\mu([f]))
        &= \nu\left(\frac{f(n)} n \right) \\
        &= [an\mapsto af(n)] \\
        &= [an\mapsto f(an)] \\
        &= [f]
    .\end{align*}
    Esto muestra que $\mu$ es la inversa
    de $\nu$, así que
    \[
        \begin{tikzcd}
        \mathbb Q \ar[r,shift left=2,"\nu"] &
        \displaystyle
        \varinjlim_{n\neq 0}\Mod_\Z(\<n\>,\Z),
        \ar[l,shift left=2,"\mu","\simeq"']
       \end{tikzcd} 
    \]
    como se quería mostrar.
\end{sol}

Consideremos el copo $\Lambda(\Z)^*$.
Notemos que $\Lambda(\Z)^*$ es un filtro de
$\Lambda(\Z)$ (absorbe hacia arriba
y es cerrado bajo ínfimos finitos).

Más aún, dados $I\in\Lambda(\Z)^*$ y $K\in\Lambda(\Z)$, tenemos
\[
    (\forall k\in K \;\; (k:I)\in\Lambda(\Z)^*) \implies I\in\Lambda(\Z)^*
.\]

\subsection*{(SESIÓN 10: 14 OCT)}

\subsection{Derivadas, derivadas estables, prenúcleos, operadores cerradura y núcleos}

Sabemos que cualquier morfismo $f=f^*:A\to B$ de marcos tiene
un adjunto derecho $f_*:B\to A$, y que $f$ induce un núcleo
$k:A\to A$ dado como $k=f_*f^*$.
Más aún, todo núcleo en $A$ es el núcleo de un morfismo de
marcos, y los cocientes de $A$ están en correspondencia con
los núcleos en $A$ y con los subconjuntos implicativos de $A$.

También consideramos otro tipo de operadores monótonos:
los operadores cerradura sobre una $\Sup$-semiretícula.
Como todo marco es, en particular, una $\Sup$-semiretícula, también
podemos considerar los operadores cerradura sobre un marco.

Ahora estudiaremos otras familias de funciones monótonas $A\to A$.

\begin{defn}
  Una inflación o derivada en $A$ es una función $f:A\to A$ que
  es monótona e infla.
  Es decir:
  \begin{itemize}
    \item $a\leq b$ implica $f(a) \leq f(b)$.
    \item $a\leq f(a)$.
  \end{itemize}
  Usaremos la notacióN $DA$ para el conjunto de todas las
  derivadas en $A$.
\end{defn}
Notemos que un operador cerradura es una derivada idempotente, y
recordemos que usamos la notación $CA$ para el conjunto de los 
operadores cerradura en $A$.

\begin{defn}
  Una derivada $f\in DA$ es \emph{estable} si
  \[
    f(x)\inf y \leq f(x\inf y)
  \]
  para cualesquiera $x,y\in A$.
  Denotamos como $SA$ al conjunto de derivadas estables en $A$.
\end{defn}

\begin{defn}
  Un prenúcleo sobre $A$ es una derivada $f\in DA$ tal que
  \[
    f(x)\inf f(y) \leq f(x\inf y)
  .\]
  Notemos que la otra desigualdad se cumple para cualquier
  función monótona, de modo que los prenúcleos separan ínfimos
  binarios.
  Usaremos la notación $PA$ para referirnos al conjunto de
  prenúcleos de $A$.
  Además, cualquier prenúcleo es estable, ya que
  \begin{align*}
    f(x) \inf y
    &\leq f(x) \inf f(y) & \text{ pues $f\in DA$} \\
    &\leq f(x\inf y) & \text{ pues $f\in PA$}
  \end{align*}
\end{defn}

En resumen tenemos las contenciones
\begin{align*}
  NA \subseteq PA \subseteq SA \subseteq DA \\
  NA \subseteq CA \subseteq DA.
\end{align*}
También es claro que $NA=PA\cap CA$.
De hecho, tenemos $NA=SA\cap CA$.
Estos conjuntos son, en sí mismos, conjuntos parcialmente
ordenados, donde el orden está dado puntualmente.
Esto es, dadas dos funciones $f,g:A\to A$, decimos que $f\leq g$
si, y solo si,
\[
  \forall(x\in A) f(x)\leq g(x)
.\]
\begin{exe}[Armando]
  Muestra que $DA$ es un conjunto parcialmente ordenado con el
  orden puntual y que las contenciones
  \begin{align*}
    NA \subseteq PA \subseteq SA \subseteq DA \\
    NA \subseteq CA \subseteq DA
  \end{align*}
  son inclusiones de subCOPOs.
\end{exe}

Notemos que las derivadas $DA$, las estables $SA$ y los
prenúcleos $PA$ son cerrados bajo composición.
Es decir:
\begin{itemize}
  \item si $f,g\in DA$, entonces $fg,gf\in DA$,
  \item si $f,g\in SA$, entonces $fg,gf\in SA$,
  \item si $f,g\in PA$, entonces $fg,gf\in PA$.
\end{itemize}
Por otro lado, las cerraduras $CA$ y los núcleos $NA$
no lo son, en general.

Veremos que cada uno de estos conjuntos tiene más
estructura que la de conjunto parcialmente ordenado.

\subsection{Ínfimos de derivadas \tps{($DA$)}}

Sea $J\subseteq DA$.
Definimos la función $\Inf J:A\to A$ como
\[
  (\Inf J)(a) = \Inf\{f(a) \mid f\in J\}
.\]
Afirmamos que $\Inf J$ es una derivada y, de hecho, es el ínfimo
de $J$ en $DA$:
\begin{itemize}
  \item
  Si $a\leq b\in A$, entonces $f(a)\leq f(b)$ para cada $f\in J$.
  Luego, cada elemento de $\{f(b)\mid f\in J\}$ está acotado
  inferiormente por un elemento de $\{f(a)\mid f\in J\}$, por lo
  cual
  \[
    \Inf\{f(a)\mid f\in J\} \leq \Inf\{f(b)\mid f\in J\}
  .\]
  \item
  Similarmente $\Inf J$ infla.
  \item
  Si $h\in DA$ es una derivada que está por debajo de cada
  elemento $f\in J$.
  Esto es, para cualesquiera $a\in A$ y $f\in J$ se tiene
  $h(a)\leq f(a)$.
  Luego, $h(a)$ es cota inferior de $\{f(a) \mid f\in J\}$, por
  lo cual $h(a)\leq (\Inf J)(a)$.
\end{itemize}

De hecho, se puede probar lo siguiente:
\begin{exe}[Dante $\checkmark$ ]
  \begin{enumerate}
    \item si $J\subseteq SA$, entonces $\Inf J\in SA$,
    \item si $J\subseteq PA$, entonces $\Inf J\in PA$,
    \item si $J\subseteq CA$, entonces $\Inf J\in CA$,
    \item si $J\subseteq NA$, entonces $\Inf J\in NA$.
  \end{enumerate}
\end{exe}
\begin{sol}
    \begin{enumerate}
        \item Sean $x,y \ \in A$.  Así, 
        \begin{align*}
            \Inf J(x)\inf y&=\Inf\{j(x):j\in J\}\inf y\\
            &\leq j(x)\inf y\\
            &\leq j(x\inf y) \ \forall j \ \in J
        \end{align*}
        Por lo que $\Inf J(x)\inf y$ es una cota inferior del conjunto $\{j(x\inf y) : j \ \in J\}$. Así, como $\Inf J(x\inf y)$ es el ínfimo de $\{j(x\inf y) : j \ \in J\}$, se cumple que $\Inf J(x)\inf y\leq \Inf J(x\inf y)$, y $\inf J \ \in SA$.
        \item Sean $x, y \ \in A$. 
        \begin{align*}
            \Inf J(x)\inf \Inf J(y)&=\Inf\{j(x):j \ \in J\} \inf \Inf \{j(y) : j \ \in J\}\\
            &\leq j(x) \ \inf j(y) \\
            &\leq j(x\inf y) \ \forall j \ \in J
        \end{align*}
        por lo que $J(x)\inf \Inf J(y)$ es una cota inferior del conjunto $\{j(x\inf y): j \in J\}$. Así, se cumple que $\Inf J(x) \inf \Inf J(y)\leq \Inf J(x\inf y)$, y entonces $\Inf J \ \in PA$.
        \item Nótese que si $j\in J$, $j$ infla, y se cumple que 
        $$f(x)\leq j(f(x)) \ \forall j,f \in J$$
        Por lo tanto, 
        $$\Inf J(x)\leq f(\Inf J(x)) \ \forall x\in A, f\in J$$
        y $\Inf J$ es cota inferior del conjunto $\{f(\Inf J(x)):f\in J\}$.
        \\
        Ahora bien, sea $f\in DA$ tal que $f(x) \leq j(\Inf J(x)) \ \forall x\in A, j\in J$. Entonces, como $j(j(x))=j(x) \ \forall j\in J$, ocurre que 
        $$f(x)\leq j(x) \ \forall j\in J$$
        $$\Rightarrow f(x)\leq \Inf J(x) ' \forall x\in A$$
        Por lo anterior, $\Inf J(\Inf J(x))=\Inf J(x)$, y $\Inf J \in CA$.
        \item sean $x,y\in A$. Así, 
        \begin{align*}
            \Inf J(x\inf y)&=\Inf\{j(x\inf y):j\in J\}\\
            &=\Inf\{j(x)\inf j(y):j\in J\}\\
            &=\large(\Inf\{j(x):j\in J\}\large)\inf\large( \Inf\{j(y):j\in J\}\large)\\
            &=\Inf J(x)\inf J(y)
        \end{align*}
        Por lo tanto, $\Inf J \in NA$
    \end{enumerate}
\end{sol}
\subsection{Supremos de derivadas \tps{($DA$)}}

\begin{defn}
  Si $J\subseteq DA$ es un conjunto de derivadas,
  definimos el supremo puntual $\pSup J:A\to A$ de $J$ como la
  función dada por
  \[
    (\pSup J)(a) = \Sup\{f(a) \mid f\in J\}
  \]
  para todo $a\in A$.
\end{defn}
Notemos que, si $J=\emptyset\subseteq DA$, entonces
\begin{align*}
  (\pSup\emptyset)(a)
  &= \Sup\{f(a) \mid f\in\emptyset\} \\
  &= \Sup\emptyset \\
  &= 0 \in A
\end{align*}
lo cual ya no es una derivada (a menos que $A$ sea trivial).
Sin embargo, si $J\neq\emptyset$ es una familia de derivadas,
entonces $\pSup J$ es una derivada y es el supremo de $J$ en
$DA$.
Es decir, los supremos (no vacíos) en $DA$ se calculan
puntualmente.

Observemos que, en $DA$, el menor elemento es la identidad
$\id:A\to A$ y el mayor elemento es $\tp:A\to A$ dada como
$\tp(a)=1$.

\subsection{Supremos de estables \tps{($SA$)}}

Supongamos que $J$ es un conjunto de derivadas estables
($J\subseteq SA$).
Entonces, para cualesquiera $a,b\in A$ tenemos
\begin{align*}
  (\pSup J)(a)\inf b
  &= \Sup\{f(a) \mid f\in J\} \inf b \\
  &= \Sup\{f(a)\inf b \mid f\in J\}
    & \text{por la ley distributiva de marcos} \\
  &\leq \Sup\{f(a\inf b) \mid f\in J \}
    & \text{cada $f\in SA$} \\
  &= (\pSup J)(a\inf b).
\end{align*}
Es decir, los supremos (no vacíos) en $SA$ se calculan
puntualmente.

\subsection{Supremos de prenúcleos \tps{($PA$)} y conjuntos dirigidos}

En contraste a lo que sucede con derivadas y estables, el supremo
de prenúcleos no se calcula puntualmente, en general.

\begin{exa}
  Tomemos el marco $A$ dado como
  \[
    \begin{tikzcd}
      & 1 \\
      & c \ar[u,no head] \\
      a \ar[ur,no head] & & b \ar[ul,no head] \\
      & 0 \ar[ur, no head] \ar[ul, no head]
    \end{tikzcd}
  \]
  minuto 37:00
\end{exa}

Sin embargo, si $J\subseteq PA$ es un conjunto dirigido y no
vacío, entonces $\pSup J$ es un prenúcleo.

(Un conjunto $J$ es dirigido si, para cualesquiera $f,g\in J$
existe un $h\in J$ tal que $f,g\leq h$).

En efecto, si $J$ es dirigido, entonces
\begin{align*}
  (\pSup J)(a) \inf (\pSup J)(b)
  &= \Sup\{f(a)\inf g(b) \mid f,g\in J\}
    & \text{ ley distributiva de marcos} \\
  &= \Sup\{h(a) \inf h(b)  \mid h\in J \}
    & \text{ $J$ es dirigido} \\
  &\leq \Sup\{h(a \inf b)  \mid h\in J \}
    & \text{ cada $h\in J$ es prenúcleo} \\
  &= (\pSup J)(a\inf b).
\end{align*}

Tomemos dos derivadas (estables, prenúcleos) $f$ y $g$ en $A$.
Como $g$ infla, tenemos $a\leq g(a)$ y,
aplicando $f$, se sigue que $f(a)\leq f(g(a))$.
Como $f$ también infla, tenemos $g(a)\leq f(g(a))$.
Luego, $f(a)\sup g(a) \leq f(g(a))$, pero el lado izquierdo es
$(f\psup g)(a)$ y el lado dercho es $(fg)(a)$, así que
\[
  f\psup g \leq fg
,\]
pues lo anterior es válido para todo $a\in A$.
Similarmente, $f\psup g\leq gf$.
Se sigue que $f\psup g$ es cota inferior de $fg$ y $gf$.
Esto nos da
\[
  f\psup g \leq fg\inf gf
.\]

En particular, si un conjunto $J$ de derivadas (estables,
prenúcleos) es cerrado bajo composición, entonces $J$ es dirigido.

\subsection{Las derivadas estables forman un marco}

Ya vimos que, para cualquier conjunto $J\subseteq SA$ de
derivadas estables, el ínfimo de $J$ en $SA$ se calcula
puntualmente y, si $J\neq\emptyset$, entonces el supremo de $J$
en $SA$ también se calcula puntualmente, mientras que
$\Sup\emptyset=\id$ en $SA$.


Veremos la interacción de los ínfimos finitos con los supremos de
$SA$.
Tomemos un subconjunto $J\subseteq SA$.
Si $J=\emptyset$, entonces las dos derivadas estables
\begin{align*}
  f\inf \Sup J
  &= f\inf\id \\
  &= \id \\
  &= \Sup\{f\inf g \mid g\in J\}.
\end{align*}
Por otro lado, si $J\neq\emptyset$, entonces para todo $x\in A$
se tiene
\begin{align*}
  (f\inf \pSup J)(x)
  &= f(x)\inf (\pSup J)(x) \\
  &= f(x)\inf \Sup\{g(x) \mid g\in J\} \\
  &= \Sup\{f(x)\inf g(x) \mid g\in J\}
    & \text{ley distributiva de marcos} \\
  &= \Sup\{(f\inf g)(x) \mid g\in J\} \\
  &= (\pSup\{f\inf g \mid g\in J\})(x).
\end{align*}
Es decir, $SA$ es un marco.
En particular, $SA$ tiene una implicación.

\subsection{Los núcleos forman un marco}

La afirmación es que $NA$ es un subconjunto implicativo de $SA$.
Es decir: $NA$ es cerrado bajo ínfimo (lo cual ya sabemos) y,
para cualquier estable $f\in SA$ y cualquier núcleo $k\in NA$, la
derivada estable $(f\succ k)$ es un núcleo.
Para esto, basta demostrar que es idempotente, ya que toda
derivada estable e idempotente es núcleo.

Como $G$ no es vacío (por ejemplo, $\id\in G$), tenemos
\[
  (f\succ k) = \pSup\{g\in SA \mid f\inf g\leq k\}
.\]
Sea $G=\{g\in SA \mid f\inf g\leq k\}$.

Primero mostraremos que $G$ es cerrado bajo composiciones.
Observemos que, para cualesquiera $g,h\in G$ y $x\in A$, tenemos
\begin{align*}
  (f\inf gh)(x)
  &= f(x)\inf g(h(x)) \\
  &= f(x)\inf f(x) \inf g(h(x)) \\
  &\leq f(x) \inf g(f(x)\inf h(x))
    & g\in SA \\
  &\leq f(x) \inf g(k(x))
    & f\inf h \leq k, \text{ pues } h\in G \\
  &\leq f(k(x)) \inf g(k(x)) \\
  &\leq k(k(x))
    & f\inf g \leq k, \text{ pues } g\in G \\
  &= k(x).
\end{align*}
Esto prueba que $f\inf gh \leq k$ y, así, $gh\in G$, como se
quería.

Ahora mostraremos que $j=(f\succ k)\in G$.
En efecto, para todo $x\in A$ se tiene
\begin{align*}
  (f\inf j)(x)
  &= f(x) \inf (\pSup G)(x) \\
  &= f(x) \inf \Sup\{g(x) \mid g\in G\} \\
  &= \Sup\{f(x) \inf g(x) \mid g\in G\} \\
  &= k(x),
\end{align*}
como se quería.
Luego, $j^2=jj\in G$, así que $j^2\leq\pSup G = j$.
Concluimos que $j^2=j$.
Como $j$ es una derivada estable e idempotente,
se sigue que es un núcleo.

Esto muestra lo que queríamos probar: que $NA$ es un subconjunto
implicativo de $SA$.
Una consecuencia inmediata es

\begin{thm}[Isbell-Simmons-Johnstone]
  Para cada marco $A$, el ensamble $NA$ es un marco.
\end{thm}
\begin{proof}
  Sabemos que $NA$ es una retícula completa, por lo cual
  basta ver que también tiene una implicación.
  En efecto, si $j,k\in NA$, entonces la derivada estable
  $j\succ k$ cumple
  \[
    g\inf j\leq k
    \ssi
    g\leq (j\succ k)
  \]
  para toda derivada estable $g\in SA$.
  En particular, lo mismo es cierto para toda $g\in NA$, así que
  $(j\succ k)$ también es la implicación en $NA$.
\end{proof}
Sin embargo, el hecho de que $NA$ sea un subconjunto
implicativo de $SA$ nos dice aún más.
Todo subconjunto implicativo de un marco es un cociente de éste,
así que $NA$ es de la forma $NA=(SA)_j$ para algún núcleo
$j:SA\to SA$.
La pregunta es, ¿qué núcleo?

\subsection{Las iteraciones de una derivada y el operador \tps{$(-)^\infty:DA\to DA$}.}

Denotaremos como $\Ord$ a la clase de ordinales y, para cada
ordinal, definimos
\begin{align*}
  f^0 &= \id \\
  f^{\alpha+1} &= ff^\alpha \\
  f^\lambda &= \Sup\{f^\alpha \mid \alpha<\lambda\}
    & \text{ si $\lambda$ es límite.}
\end{align*}
De este modo, obtenemos una cadena de derivadas
\[
  f^0\leq f^1\leq f^2\leq\dots\leq f^\alpha\leq f^{\alpha+1}
  \leq\dots
.\]
Como $\Ord$ no es cardinable, la cadena
$(f^\alpha \mid \alpha\in \Ord )$ se detiene, por fuerza, en
algún ordinal $\gamma$, es decir: $f^{\gamma+1}=f^\gamma$.
Como la clase de ordinales es bien ordenada, existe un primer
ordinal $\infty\in\Ord$ tal que $f^\infty=f^{\infty+1}$.
Se sigue que la derivada $f^\infty$ es idempotente.
Esto es, $f^\infty$ es un operador cerradura en $A$.
Más aún, $f^\infty$ es el menor operador cerradura en $A$
que está por encima de $f$.
Para ver esto, mostraremos que, si $k$ es un operador cerradura
que está por encima de $f$, entonces toda la cadena de iteraciones
de $f$ está por debajo de $k$.
En efecto, tomemos $k\in CA$ y $f\leq k$ y hagamos inducción.
\begin{itemize}
  \item Para $\alpha=0$, tenemos $f^0=\id\leq k$.
  \item Supongamos que $f^\alpha\leq k$.
  Entonces
  \begin{align*}
    f^{\alpha+1}
    &= ff^\alpha \\
    &= kk \\
    &= k.
  \end{align*}
  \item Finalmente, si $\lambda$ es un ordinal límite, supongamos
  que $f^\alpha\leq k$ para todo $\alpha<\lambda$.
  Entonces
  $f^\lambda=\Sup\{f^\alpha\mid \alpha<\lambda\}\leq k$.
\end{itemize}
Como esto es válido para todo ordinal, en particular tenemos
$f^\infty\leq k$.
Luego, $f^\infty$ es el menor operador cerradura que está por
arriba de $f$.

Más aún, la construcción $({-})^\infty:DA\to CA$ se comporta bien
con el orden:
es claro que, si $f\leq g$, entonces $f^\infty\leq g^\infty$.
Más aún, como $({-})^\infty$ es idempotente:
$(f^\infty)^\infty=f^\infty$.
Esto muestra que $({-})^\infty$ es un operador cerradura en $DA$
y, de hecho, sus puntos fijos son los
operadores cerradura en $A$:
\begin{align*}
  ({-})^\infty &\in CDA \\
  (DA)_\infty &= CA.
\end{align*}

\subsection{VIDEO 5: El espacio de puntos, parte 1. (18 OCT)}
\begin{defn}
  Sea $A\in Frm$. Un punto o elemento $\inf-$irreducible de A es un elemento $p\in A$ con $p\neq 1$ tal que si $x\inf y\leq p$, entonces $x\leq p$ ó $y\leq p$. Denotamos por ptA el conjunto de todos los puntos de A.
\end{defn}
\begin{lemma}
  Sea $A\in Frm$.
  \begin{itemize}
      \item Cada máximo de A es $\inf-$irreducible.
      \item Si A es booleano, entonces todo elemento $\inf-$irreducible de A es máximo.
      \item Si A es una cadena, entonces cada elemento propio de A es $\inf-$irreducible.
  \end{itemize}
\end{lemma}
\begin{proof}\quad
  \begin{itemize}
      \item Sea $p\in A$ máximo, entonces $p<1$. Si $x\inf y\leq p$ y suponiendo que $x\not\leq p$, entonces $p<x\sup p$ y, por la maximalidad de p, tenemos que $p\sup x=1$. Similarmente, $y\not\leq p$ implica $p\sup y=1$. Si $x\not\leq p$ y $y\not\leq p$, se tiene que 
      \[p=p\sup (x\inf y)=(p\sup x)\inf(p\sup y)=1.\]
      Esto es una contradicción ya que $p<1$.
      \item Supongamos que A es booleano. Sean $p\in ptA$ y $x,y\in A$ con $p<x$ y $y=\neg x$. Tenemos que $x\inf y=0\leq p$, entonces $x\leq p$ ó $y\leq p$ ya que p es $\inf-$irreducible. Además $y\leq p<x$ puesto que $p<x$. En consecuencia, $x\sup y=1=x$, así, p es máximo.
      \item Supongamos que A es una cadena. Para cualesquiera $x,y\in A$, tenemos que $x\leq y$ ó $y\leq x$, es decir, $x\inf y\leq x$ ó $x\inf y\leq y$. Sea $p\in A$ con $p<1$. Si $x\inf y\leq p$, entonces $x\leq p$ ó $y\leq p$.
  \end{itemize}
\end{proof}
Sean $S\in Top$, $s\in S$ y $u\in OS$ con $u\subseteq \overline{s}'$. Tenemos que
\[u\subseteq \overline{s}' \iff \overline{s}\subseteq u' \iff s\in u' \iff s\notin u.\]
Observemos que $\overline{s}'\neq S$ ya que, si esto sucede, entonces $\overline{s}=\emptyset$.\par 
Sean $u,v\in OS$ con $u\cap v\subseteq \overline{s}'$, entonces $s\notin u\cap v$. Esto implica que $s\notin u$ ó $s\notin v$, es decir, $u\subseteq \overline{s}'$ ó $v\subseteq \overline{s}'$. Por lo que $\overline{s}'\in pt(OS)$.\par \vspace{5mm}
Sean $A\in Frm$ y $a\in A$. Decimos que un punto $p\in ptA$ está en $U_A(a)$ si, y sólo si $a\not\leq p$.
\begin{exe}[Yareli $\checkmark$ ]
Demostrar el siguiente lema:
  \begin{lemma}
    Sean $A\in Frm$ y $a,b\in A$.
    \begin{itemize}
        \item $U_A(1)=ptA$.
        \item $U_A(0)=\emptyset$.
        \item $U_A(a\inf b)=U_A(a)\cap U_A(b)$.
        \item $U_A(\Sup X)=\bigcup \{U_A(x)|x\in X\}$, $\forall X\subseteq A$.
    \end{itemize}
  \end{lemma}
\end{exe}
\begin{proof}
  Sean $A\in Frm$ y $a,b\in A$.
\begin{itemize}
\item Por definición $U_A(1)\subseteq ptA$. Sea $p\in ptA$, entonces $p\neq 1$. Además $1\not\leq p$, por lo que $p\in U_A(1)$. Así, $U_A(1)=ptA$.
\item Supongamos que $U_A(0)\neq \emptyset$. Sea $p\in U_A(0)$. Por definición, $0\not\leq p$ pero $0\leq a, \forall a\in A$. Por lo tanto, $U_A(0)=\emptyset$.
\item Sea $p\in ptA$. Tenemos que
\begin{align*}
p\in U_A(a\wedge b)&\iff a\wedge b\not\leq p\\
&\iff a\not\leq p\quad y\quad b\not\leq p\\
&\iff p\in U_A(a)\quad y\quad p\in U_A(b)\\
&\iff p\in U_A(a)\cap U_A(b).
\end{align*}
Por lo que $U_A(a\wedge b)=U_A(a)\cap U_A(b)$.
\item Sea $X\subseteq A$ y notemos que si $X=\emptyset$, entonces ocurre el segundo punto. En caso contrario,
\begin{align*}
p\in U_A(\bigvee X)&\Rightarrow \bigvee X\not\leq p\\
&\Rightarrow \textit{existe }x\in X\textit{ tal que }x\not\leq p\\
&\Rightarrow p\in U_A(x)\\
&\Rightarrow p\in \bigcup \{U_A(x)\mid x\in X\}.
\end{align*}
Además,
\begin{align*}
p\in \bigcup\{U_A(x)\mid x\in X\}&\Rightarrow p\in U_A(x)\textit{ para algún }x\in X\\
&\Rightarrow x\not \leq p\\
&\Rightarrow \bigvee X\not\leq p\\
&\Rightarrow p\in U_A(\bigvee X).
\end{align*}
Por lo tanto, $U_A(\sup x)=\bigcup \{U_A(x)|x\in X\}$, $\forall X\subseteq A$.
\end{itemize}
\end{proof}
Por el lema anterior, $U_A$ es una topología en ptA.\\
Llamamos al par $\left(ptA, \left(U_A(a)|a\in A\right)\right)$ el \textit{espacio de puntos} de A.\\
Además $U_A\colon A\to OptA$ es un morfismo de marcos suprayectivo al que llamamos la \textit{reflexión espacial} de A. Más aún, existe un núcleo $S\in NA$ tal que $A_S\cong OptA$.\par 
Queremos ver que si varía A, obtenemos una transformación natural
\[U_\bullet\colon Id_{Frm}\to Opt(\_).\]
Sabemos que $S$ está caracterizado como $x\leq S(a) \iff U(x)\subseteq U(a)$. Probaremos que $S(a)=\Inf \{p\in ptA|a\leq p\}$.
\begin{align*}
    x\leq S(a) &\iff U(x)\subseteq U(a)\\
    &\iff (\forall p\in ptA)[x\not\leq p\Rightarrow a\not\leq p]\\
    &\iff (\forall p\in ptA)[a\leq p \Rightarrow x\leq p]\\
    &\iff x\leq \Inf \{p\in ptA|a\leq p\}
\end{align*}
\begin{exe}[Juan]
  Sean $S\in Top$ y $p,q\in S$. Probar que la relación
  \[q\sqsubseteq p\iff \overline{q}\subseteq \overline{p}\]
  es un preorden. Si el espacio es $T_0$, entonces es un orden parcial.
\end{exe}
\noindent A la relación del ejercicio anterior le llamamos el \textit{orden de especialización}.\par 
Sean $p,q\in ptA$. Notemos que
\begin{align*}
    q\sqsubseteq p&\iff \overline{q}\subseteq \overline{p}\\
    &\iff (\forall x\in A)[q\in U(x)\Rightarrow p\in U(x)]\\
    &\iff (\forall x\in A)[x\leq p\Rightarrow x\leq q]\\
    &\iff p\leq q.
\end{align*}
Es decir, el orden de especialización del espacio de puntos es el orden opuesto del marco original. En particular, esto prueba que el espacio de puntos, ptA, es $T_0$.\par 
Notemos que, para un morfismo de marcos $f\colon A\to B$ y un punto $p\in ptB$, $z\leq f_\ast (p)\iff f(z)\leq p$, donde $f_\ast$ es adjunto derecho de $f$. En particular, si $1\leq f_\ast(p)\iff f(1)=1\leq p$. Esto es imposible ya que $p\in ptB$. Por lo que $f_\ast(p)\neq 1$.\\
Sean $x,y\in A$ tales que $x\inf y\leq f_\ast(b)$. Esto pasa si, y sólo si $f(x)\inf f(y)\leq p$, en consecuencia, $f(x)\leq p$ ó $f(y)\leq p$, i.e., $x\leq f_\ast (p)$ ó $y\leq f_\ast (p)$. Por lo que $f_\ast (p)\in ptA$.\par 
En resumen, dado un morfismo de marcos $f\colon A\to B$, obtenemos una función $f_\ast \colon ptB\to ptA$ (restringida).\par
Observemos que
\begin{align*}
    p\in f^{-1}_\ast \left(U_A(a)\right)&\iff f_\ast (p)\in U_A(a)\\
    &\iff a\not\leq f_\ast (p)\\
    &\iff f(a)\not\leq p\\
    &\iff p\in U_B\left(f(a)\right).
\end{align*}
Por lo tanto $f_\ast\colon ptB\to ptA$ es continua.
\subsection*{(SESIÓN 11: 19 OCT)}

\subsection{Las iteraciones de una derivada estable y el operador \tps{$(-)^\infty:SA\to SA$}.}

La última vez demostramos el teorema de Isbell-Simmons-Johnstone:

\begin{thm}[Teorema fundamental de la teoría de marcos]
  Si $A$ es un marco, entonces $NA$ es un marco.
\end{thm}

En particular, $N^2A$ es un marco.
... torre de ensambles.

\begin{lemma}
  Si $f$ es un prenúcleo en $A$, entonces cada iteración
  $f^\alpha$ es un prenúcleo, $f^\infty$ es un núcleo y, más aún,
  $f^\infty$ es el menor núcleo por encima de $A$.
\end{lemma}
\begin{proof}
  Sea $f\in PA$.
  Mostraremos, usando inducción, que $f^\alpha\in PA$ para cada
  ordinal $\alpha$.
  \begin{itemize}
    \item Si $\alpha=0$, entonces $f^0=\id\in PA$.
    \item Supongamos que $f^\alpha$ es prenúcleo.
    Como los prenúcleos son cerrados bajo composición,
    $f^{\alpha+1}=ff^\alpha$ es prenúcleo.
    \item Si $\lambda$ es un ordinal límite, supongamos que
    $f^\alpha\in PA$ para cada ordinal $\alpha<\lambda$.
    Recordemos que
    $f^\lambda=\pSup\{f^\alpha\mid\alpha<\lambda\}$.
    Hay que probar que $f^\lambda$ es prenúcleo.
    Para cada $x,y\in A$ tenemos
    \begin{align*}
      &f^\lambda(x) \inf f^\lambda(y) \\
      &\hspace{10mm}
      = \Sup\{f^\alpha(x) \inf f^\beta(y) \mid
        \alpha,\beta<\lambda\}
        && \text{ley distributiva para marcos} \\
      &\hspace{10mm}
      = \Sup\{f^\gamma(x)\inf f^\gamma(y)
        \mid \gamma<\lambda\} \\
      &\hspace{10mm}
      \leq \Sup\{f^\gamma(x\inf y) \mid \gamma<\lambda\}
        && f^\gamma\in PA \text{ por hipótesis}\\
      &\hspace{10mm}
      = f^\lambda(x\inf y).
    \end{align*}
  \end{itemize}
  En particular, $f^\infty$ es un prenúcleo y, como también es
  idempotente, se sigue que $f^\infty$ es un núcleo.

  Finalmente, recordemos que $f^\infty$ es el menor operador
  cerradura por encima de $f$.
  Luego, para cualquier núcleo $j\in NA$ que esté por encima de
  $f$, se tiene $f^\infty\leq j$, así que $f^\infty$ es el menor
  núcleo por encima del prenúcleo $f$.
\end{proof}

Este resultado se puede refinar un poco más:
\begin{lemma}
  Si $f\in SA$ es cualquier estable, entonces $f^\alpha$ es
  estable para todo ordinal $\alpha$.
  Más aún, $f^\lambda$ es un prenúcleo, para cada ordinal límite
  $\lambda$ y, finalmente, $f^\infty$ es el menor núcleo por
  encima de $f$.
\end{lemma}
\begin{proof}
  Sea $f$ una derivada estable.
  Por inducción, probamos que $f^\alpha$ es estable para cada
  ordinal $\alpha$.
  El caso $\alpha=0$ y el paso inductivo de $\alpha$ a $\alpha+1$
  es exactamente igual a la demostración anterior (porque $SA$ es
  cerrado bajo composición).
  Ahora, si $\lambda$ es un ordinal límite, tenemos
  $f^\lambda = \pSup\{f^\alpha \mid \alpha < \lambda\}$, de modo
  que, para cualesquiera $x,y\in A$ se tiene
  \begin{align*}
    f^\lambda(x)\inf y
    &= \Sup\{f^\alpha(x) \mid \alpha <\lambda\} \inf y \\
    &= \Sup\{f^\alpha(x)\inf y\mid \alpha <\lambda\}
      & \text{ley distributiva para marcos} \\
    &\leq \Sup\{f^\alpha(x\inf y)\mid \alpha <\lambda\} \\
    &= f^\lambda(x\inf y),
  \end{align*}
  como se quería.

  Más aún, debemos probar que $f^\lambda$ es prenúcleo, siempre
  que $\lambda$ es un ordinal límite.
  Para cualesquiera $x,y\in A$ tenemos
  \begin{align*}
    &f^\lambda(x)\inf f^\lambda(y) \\
    &\hspace{10mm}
    = \Sup\{f^\alpha(x)\inf f^\beta(y) \mid \alpha,\beta <\lambda\}
      & \text{ley distributiva para marcos} \\
    &\hspace{10mm}
    \leq \Sup\{f^\alpha(x\inf f^\beta(y)) \mid \alpha,\beta <\lambda\}
      & f^\alpha \in SA \\
    &\hspace{10mm}
    \leq \Sup\{f^\alpha(f^\beta(x\inf y)) \mid \alpha,\beta <\lambda\}
      & f^\beta \in SA \\
    &\hspace{10mm}
    \leq \Sup\{f^\gamma(x\inf y) \mid \gamma<\lambda\}
      & (?) \\
    &\hspace{10mm}
    = f^\lambda(x\inf y),
  \end{align*}
  como se quería.
\end{proof}

Con estas observaciones, tenemos el siguiente
\begin{thm}
  Para cada marco $A$, el operador cerradura $({-})^\infty:SA\to
  SA$ es un núcleo cuyo conjunto de puntos fijos es el ensamble
  de $A$:
  \[
    (SA)_\infty = NA
  .\]
\end{thm}
\begin{proof}
  Como $({-})^\infty:SA\to SA$ es un operador cerradura,
  solo queda demostrar que la desigualdad
  \[
    f^\infty \inf g^\infty \leq (f\inf g)^\infty
  \]
  se cumple para cualesquiera estables $f,g\in SA$.
  (La otra desigualdad ya se tiene, pues $({-})^\infty$ es monótono).

  Sea $l=(f \inf g)^\infty$.
  Por inducción, mostraremos que $f^\alpha\inf g\leq l$ para todo
  ordinal $\alpha$.
  \begin{itemize}
    \item Para $\alpha=0$, tenemos $f^0\inf g=\id \leq l$.
    \item Supongamos que $f^\alpha\inf g\leq l$.
    Entonces, para todo $x\in A$ tenemos
    \begin{align*}
      (f^{\alpha+1}\inf g)(x)
      &= f(f^\alpha(x)) \inf g(x) \\
      &= f(f^\alpha(x)) \inf g(x) \inf g(x) \\
      &\leq f(f^\alpha(x)) \inf g(f^\alpha(x)) \inf g(x)
        && \text{pues } x\leq f^\alpha(x) \\
      &\leq l(f^\alpha(x)) \inf g(x) && f\inf g\leq (f\inf
      g)^\infty =l \\
      &\leq l(f^\alpha(x) \inf g(x)) && l\in SA \\
      &\leq l(l(x)) && f^\alpha\inf g\leq l \\
      &= l(x) && l\in CA.
    \end{align*}
    \item
    Si $\lambda$ es límite, supongamos que $f^\alpha\inf g\leq l$
    para todo ordinal $\alpha <\lambda$.
    Entonces, para todo $x\in A$, tenemos
    \begin{align*}
      (f^\lambda\inf g)(x)
      &= f^\lambda(x) \inf g(x) \\
      &= \Sup\{f^\alpha(x) \mid \alpha<\lambda\} \inf g(x) \\
      &= \Sup\{f^\alpha(x)\inf g(x) \mid \alpha<\lambda\} \\
      &\leq l(x) && \text{ pues } f^\alpha\inf g\leq l.
    \end{align*}
  \end{itemize}
  Esto muestra que $f^\infty \inf g\leq l$.
  De manera similar, podemos probar que $f^\infty \inf
  g^\alpha\leq l$ para todo $\alpha$.
  Luego, $f^\infty \inf g^\infty \leq l$, como se quería.

  Como $f^\infty$ es un núcleo para cada $f\in SA$, la
  igualdad $(SA)_\infty=NA$ se obtiene de observar que todo
  núcleo $k$ es un punto fijo de $({-})^\infty:SA\to SA$.
\end{proof}

\begin{exe}[Alfredo $\checkmark$ ]
  Describir el ensamble del marco
  \[
    A\hspace{10mm} = \hspace{10mm}
    \begin{tikzcd}
      & 1 \\
      a \ar[ur,no head] && b\ar[ul,no head] \\
      & 0 \ar[ur,no head] \ar[ul,no head]
    \end{tikzcd}
  .\]
\end{exe}
\begin{sol}
    Consideremos primero los núcleos asociados a $a$.
    Por supuesto, todos los núcleos mandan el $1$ al $1$.
    Además, todos los núcleos preservan ínfimos y $0=a\inf b$,
    solo necesitamos sus valores en $a$ y en $b$.
   \[ 
        \begin{array}{|c|}
            \hline
            \unuc a \\
            a\mapsto a\sup a = a \\
            b\mapsto a\sup b = 1 \\ \hline
            \unuc b \\
            a\mapsto b\sup a = 1 \\
            b\mapsto b\sup b = b \\
            \hline
        \end{array}
        \begin{array}{c|}
            \hline
            \vnuc a \\
            a\mapsto (a\succ a) = 1 \\
            b\mapsto (a\succ b) = b \\ \hline
            \vnuc b \\
            a\mapsto (b\succ a) = a \\
            b\mapsto (b\succ b) = 1 \\
            \hline
        \end{array}
        \begin{array}{c|}
            \hline
            \wnuc a \\
            a\mapsto ((a\succ a)\succ a) = (1\succ a) = a \\
            b\mapsto ((b\succ a)\succ a) = (a\succ a) = 1 \\
            \hline
            \wnuc b \\
            a\mapsto ((a\succ b)\succ b) = (b\succ b) = 1 \\
            b\mapsto ((b\succ b)\succ b) = (1\succ b) = b \\
            \hline
        \end{array}
    \]
    Más aún, cada núcleo solo puede mandar $a$ a sí misma
    o al $1$, y $b$ a sí misma o al $1$.
    Por lo tanto hay, a lo más, $4$ núcleos.
    En efecto, como se ve en la tabla anterior, tenemos
    \begin{align*}
        \unuc a = \vnuc b = \wnuc a \\
        \unuc b = \vnuc a = \wnuc b
    \end{align*}
    y los otros dos núcleos son $\id$ y $\tp$.
    En particular, observemos que $\unuc a$ y $\unuc b$ no son
    comparables.
    Luego, 
  \[
    NA\hspace{10mm} = \hspace{10mm}
    \begin{tikzcd}
      & \tp \\
      \unuc a \ar[ur,no head] && \unuc b \ar[ul,no head] \\
      & \id \ar[ur,no head] \ar[ul,no head]
    \end{tikzcd}
    \hspace{10mm} \simeq \hspace{10mm} A
  .\]
\end{sol}


\subsection{Supremos de núcleos \tps{($NA$)}.}

Ya probamos que los supremos no vacíos en $DA$ y en $SA$ se
calculan puntualmente, y que los supremos dirigidos en $PA$
también.
Sin embargo, aún queda encontrar una descripción para, al menos,
algunos supremos en $NA$.

Sea $J\subseteq NA$ un conjunto no vacío de núcleos.
Como $J\subseteq SA$, entonces $\pSup J$ es el supremo de $J$ en
$SA$.
Por otro lado, consideremos la familia $J^\circ$
de composiciones (finitas) de elementos de $J$:
\[
  J^\circ = \{j_1\cdots j_m \mid j_i\in J \text{ para } 1\leq
  i\leq m\}
.\]
Dado que $J^\circ$ es una familia dirigida de prenúcleos
(pues $J^\circ$ es cerrado bajo composiciones), $\pSup J^\circ$
es el supremo de $J^\circ$ en $PA$.

\begin{lemma}
  Si $J\subseteq NA$ es una familia no vacía de núcleos sobre un
  marco $A$, entonces el núcleo
  \[
    \left(\pSup J\right)^\infty = \left(\pSup J^\circ\right)^\infty
  \]
  es el supremo de $J$ en $NA$.
\end{lemma}
\begin{proof}
  Sean
  \begin{align*}
    j &= \left(\pSup J\right)^\infty &
    k &= \left(\pSup J^\circ\right)^\infty.
  \end{align*}
  Es claro que $j$ y $k$ son núcleos que acotan superiormente a
  $J$.
  Ahora sea $l\in NA$ un núcleo que acota superiormente a $J$.

  Por un lado, tenemos $\pSup J\leq l$, ya que $\pSup J$
  es el supremo de $J$ en $SA$ y $l\in SA$.
  Luego, $j\leq l$, pues $j$ es el menor núcleo por encima de
  $\pSup J$.

  Por otro lado, como $l\in NA$ acota superiormente a $J$,
  tenemos que
  \[
    j_1\cdots j_m \leq l^m = l
  \]
  para cualquier composición finita $j_1\cdots j_m$ de núcleos en
  $J$.
  Luego, $l$ es un núcleo que acota superiormente a $J^\circ$.
  Se sigue que
  \[
    k \leq l
  ,\]
  pues $k$ es el menor núcleo por encima de
  $\pSup J^\circ$.
\end{proof}

\subsection{Complementos, algunos supremos y representaciones canónicas en \tps{$NA$}.}

Recordemos que cualquier elemento $a$ de un marco $A$ tiene
asociados los núcleos $\unuc a$ y $\vnuc a$ dados por
\begin{align*}
  \unuc a(x) &= a\sup x
  &
  \vnuc a(x) &= (a\succ x).
\end{align*}
No es difícil ver que
\begin{align*}
  \unuc 1 &= \tp = \vnuc 0 \\
  \unuc 0 &= \id = \vnuc 1.
\end{align*}

Esto se puede generalizar para cualquier elemento $a\in A$.

\begin{lemma}
  Sea $A$ un marco.
  Para cualquier $a\in A$ se tiene
  \begin{align*}
    \vnuc a\inf\unuc a &= \id
    &
    \vnuc a\sup\unuc a &= \tp
  \end{align*}
  en $NA$.
  Es decir, $\unuc a$ y $\vnuc a$ son complementos uno del otro.
\end{lemma}
\begin{proof}
  minuto 1:03:00
\end{proof}

\begin{lemma}
  Sea $A$ un marco.
  Para cualesquiera $j\in NA$ y $a\in A$, se tiene
  \begin{enumerate}[label=(\roman*)]
    \item $j\unuc a$ es idempotente y, por lo tanto, un núcleo,
    \item $j\sup\unuc a = j\unuc a$ (el supremo en $NA$),
    \item $\unuc{j(a)}\inf\vnuc a\leq j$.
  \end{enumerate}
\end{lemma}
\begin{proof}
  (i). Como $PA$ es cerrado bajo composición, al menos tenemos
  $j\unuc a\in PA$.
  Luego, para ver que $j\unuc a$ es un núcleo,
  basta demostrar que es idempotente.
  Para todo $x\in A$, tenemos
  \begin{align*}
    j\unuc a j\unuc a (x)
    &= j\unuc a j(a\sup x) \\
    &= j(a\sup j(a\sup x)) \\
    &= j(j(a\sup x))
      & \text{ pues } a\leq j(a\sup x) \\
    &= j(a\sup x) \\
    &= j\unuc a (x),
  \end{align*}
  como se quería.

  (ii). Por (i), $j\unuc a$ es un núcleo que acota
  superiormente a $j$ y a $\unuc a$.
  Por otro lado, si $k$ es cualquier núcleo con $j\leq k$ y
  $\unuc a\leq k$, tenemos $j\unuc a \leq kk=k$.

  (iii). Para cualquier $x\in A$, tenemos
  \begin{align*}
    \unuc{j(a)}(x)
    &= j(a)\sup x \\
    &\leq j(a \sup x) \\
    &= j\unuc a(x).
  \end{align*}
  Luego,
  \begin{align*}
    \unuc{j(a)}
    &\leq j\unuc a \\
    &= j\sup \unuc a && \text{ por (ii)} \\
    &= \neg \vnuc a \sup j
      && \text{ pues } \neg\vnuc a = \unuc a \\
    &= (\vnuc a \succ j)
  \end{align*}
  Luego, $\unuc{j(a)} \inf \vnuc a \leq j$.
\end{proof}

\begin{thm}
  Sea $A$ un marco y $j$ un núcleo en $A$.
  Entonces
  \[
    j = \Sup\{ \unuc{j(a)}\inf\vnuc a \mid a\in A\}
  \]
  en $NA$.
\end{thm}
\begin{proof}
  minuto 1:20:00
\end{proof}

\begin{cor}
  Si $A$ es un marco finito, entonces su ensamble $NA$ es un
  álgebra booleana completa.
\end{cor}
\begin{proof}
  minuto 1:25:00
\end{proof}

\begin{defn}
  Si $A$ es un marco, la función $\eta_A:A\to NA$ está dada como
  \[
    \eta_A(a) = \unuc a
  .\]
\end{defn}

\begin{thm}
  Si $A$ es un marco, $\eta_A$ es un morfismo de marcos inyectivo
  y es un epimorfismo.
\end{thm}
\begin{proof}
  minuto 1:30:00 hasta el final (está larga).
\end{proof}

\subsection*{(SESIÓN 12: 21 OCT)}

\subsection{El ensamble como solución a un problema universal}

Vimos que, para cada marco $A$, el ensamble $NA$ es un marco y
tenemos un morfismo
\begin{align*}
  \eta_A: A&\to NA \\
  a &\mapsto \unuc a
\end{align*}
el cual es mono y epi, aunque en general no es suprayectivo.
En particular, $\Frm$ no es una categoría balanceada.

\begin{cor}
  Sea $A$ un marco.
  El adjunto derecho del morfismo $\eta_A:A\to NA$ está dado por
  $\bot:NA\to A$, $\bot(j)=j(0)$.
\end{cor}
\begin{proof}
  Hay que probar que
  \[
    a\leq j(a) \ssi \unuc a \leq j
  .\]
  Observemos que, si $\unuc a \leq j$, entonces
  $\unuc a(0) \leq j(0)$.
  Esto es, $a\leq j(0)$.
  Por otro lado, si $a\leq j(0)$, entonces para todo $x\in A$
  tenemos
  \begin{align*}
    \unuc a (x)
    &= a\sup x \\
    &\leq j(0) \sup x \\
    &\leq j(0) \sup j(x) \\
    &= j(x).
  \end{align*}
  Luego, $\unuc a \leq j$.
\end{proof}

\begin{defn}
  Sea $f:A\to B$ un morfismo de marcos.
  Diremos que $f$ resuelve el problema de complementación para
  $A$ si, para todo $a\in A$, $f(a)\in B$ es complementado en
  $B$.
\end{defn}
\begin{exa}
  Para todo $a\in A$, el núcleo $\unuc a$ es complementado en
  $NA$ (su complemento es $\vnuc a$).
  Es decir, $\eta_A:A\to NA$ resuelve el problema de
  complementación para $A$.
\end{exa}

\begin{thm}
  Sea $A$ un marco.
  El morfismo $\eta_A:A\to NA$ resuelve el problema de
  complementación de manera universal.
  Es decir, para cualquier morfismo $f:A\to B$ que resuelve el
  problema de complementación, existe un único morfismo
  $f^\sharp:NA\to B$ tal que el diagrama
  \[
    \begin{tikzcd}
      A \ar[dr,"\eta_A"'] \ar[rr,"f"] && B \\
      & NA \ar[ur,"f^\sharp"',dotted]
    \end{tikzcd}
  \]
  es conmutativo.

  Más aún, si $f_*:B\to A$ es el adjunto derecho de $f=f^*:A\to
  B$, el adjunto derecho $f_\flat:B\to NA$ de $f^\sharp:NA\to B$
  se calcula como
  \[
    f_\flat(b) = f_*\unuc b f^* \in NA
  .\]
\end{thm}
\begin{proof}
  Para empezar, como $\eta_A$ es epi, la factorización de $f$ a
  través de $\eta_A$ es única, en caso de existir.
  Es decir, si $f^\sharp,f^!:NA\to B$ son tales que
  $f^\sharp\eta_A=f=f^!\eta_A$, entonces $f^\sharp=f^!$.
  \[
    \begin{tikzcd}
      A \ar[dr,"\eta_A"'] \ar[rr,"f"] && B \\[5mm]
      & NA \ar[ur,shift right,"f^!"'] \ar[ur,shift left,"f^\sharp"]
    \end{tikzcd}
  \]
  Por lo tanto, basta con mostrar la existencia de $f^\sharp$.
  
  Recordemos que queremos definir $f^\sharp:NA\to B$ tal que el
  diagrama 
  \[
    \begin{tikzcd}
      A \ar[dr,"\eta_A"'] \ar[rr,"f"] && B \\
      & NA \ar[ur,"f^\sharp"',dotted]
    \end{tikzcd}
  \]
  conmute.
  Es decir, tal que $f^\sharp(\unuc a) = f(a)$.
  Recordemos que cada núcleo $j\in NA$ se puede representar como
  \[
    j = \Sup\{\unuc{j(a)} \inf \neg \unuc a \mid a\in A\}
  ,\]
  pues $\neg\unuc a = \vnuc a$.
  Dado que los morfismos de marcos respetan complementos,
  si existiese un morfismo $f^\sharp:NA\to B$ con las propiedades
  deseadas, necesariamente debería cumplirse que
  \begin{align*}
    f^\sharp(j)
    &= f^\sharp\left(
      \Sup\{\unuc{j(a)}\inf\neg\unuc a\mid a\in A\}
      \right) \\
    &= \Sup\{f^\sharp(\unuc{j(a)}\inf\neg\unuc a)\mid a\in A\} \\
    &= \Sup\{f^\sharp(\unuc{j(a)})\inf f^\sharp(\neg\unuc a)
       \mid a\in A\} \\
    &= \Sup\{f^\sharp(\unuc{j(a)})\inf \neg f^\sharp(\unuc a)
       \mid a\in A\} \\
    &= \Sup\{f(j(a))\inf \neg f(a) \mid a\in A\}.
  \end{align*}
  Con esta motivación, definimos $f^\sharp$ como
  \[
    f^\sharp(j) = \Sup\{f(j(a))\inf\neg f(a) \mid a\in A\}
  .\]
  
  Hay que probar que esta definición nos da un morfismo de marcos
  con las propiedades deseadas.
  Verificamos la monotonicidad.
  Si $k\leq j$ son núcleos en $A$, entonces $k(x)\leq j(x)$ para
  todo $x\in A$.
  Aplicando $f$ tenemos $f(k(x))\leq f(j(x))$, y así $f(k(x))\inf
  f(x)\leq f(j(x))\inf f(x)$.
  Esto nos dice que $f^\sharp$ es monótono.

  Por otro lado $f_\flat:B\to NA$ también es monótona,
  pues si $b\leq c\in B$, entonces $\unuc b\leq\unuc c \in NB$.
  Luego, $f_*\unuc bf^*\leq f_*\unuc cf^*$, pero
  esto es $f_\flat(b)\leq f_\flat(c)$.

  Ahora veamos que $f^\sharp \dashv f_\flat$.
  Dados $j\in NA$ y $b\in B$ arbitrarios, debemos mostrar la
  equivalencia
  \[
    f^\sharp(j)\leq b \ssi j\leq f_\flat(b)
  .\]
  Por definición
  $f^\sharp(j)=\Sup\{f^*(j(x))\inf\neg f^*(x) \mid x\in A\}$.
  Luego, tenemos las equivalencias
  \begin{align*}
    f^\sharp(j) \leq b
    &\iff \forall(x\in A)\;f^*(j(x))\inf\neg f^*(x)\leq b \\
    &\iff \forall(x\in A)\;f^*(j(x))\leq (\neg f^*(x)\succ b) \\
    &\iff \forall(x\in A)\;f^*(j(x)) \leq f^*(x)\sup b
      & \text{caballo de batalla} \\
    &\iff \forall(x\in A)\;j(x) \leq f_*(b\sup f^*(x))
      & \text{adjunción } f^*\dashv f_* \\
    &\iff \forall(x\in A)\;j(x) \leq f_*(\unuc b(f^*(x))) \\
    &\iff j\leq f_*\unuc b f^* = f_\flat(b).
  \end{align*}
  Esto muestra que $f^\sharp\dashv f_*$.
  En particular, $f^\sharp$ preserva supremos arbitrarios.
  Ahora hay que ver que $f^\sharp$ preserva ínfimos finitos.
  [Minuto 38:00]
  Finalmente, hay que ver que $f=f^\sharp \eta_A$.
  [minuto 50:00 hasta 55:20]
\end{proof}

\begin{lemma}
  Sea $A$ un marco.
  Entonces el encaje $\eta_A:A\to NA$ es suprayectivo (y, por lo
  tanto, un isomorfismo) si, y solo
  si, $A$ es un álgebra booleana completa.
\end{lemma}
\begin{proof}
  [minuto 56:00]
\end{proof}

\subsection{Algunos cálculos en el ensamble}

\begin{lemma}
  Sea $A$ un marco.
  Entonces
  \begin{align*}
    \unuc a\leq j &\iff a\leq j(0) &
    \vnuc a\leq j &\iff 1=j(a) &
    j\leq \wnuc a &\iff j(a)=a.
  \end{align*}
\end{lemma}
\begin{proof}
  [minuto 1:06:20]
\end{proof}

\begin{lemma}
  Sea $A$ un marco y $j,k\in NA$ núcleos.
  Si $jk\leq kj$, entonces $k\sup j = kj$.
\end{lemma}
\begin{proof}
  Supongamos que $jk\leq kj$.
  Sea $g=kj$.
  Entonces
  \begin{align*}
    g^2
    &= kjkj \\
    &\leq kkjj \\
    &= k^2j^2 \\
    &= kj \\
    &= g.
  \end{align*}
  Es decir, $g$ es un prenúcleo idempotente, y así $g\in NA$
  es un núcleo por encima de $k$ y de $j$.

  Ahora, si $h\in NA$ es cualquier núcleo con $j\leq h$ y $k\leq
  h$, entonces $g=kj\leq h^2=h$.
  Se sigue que $g=k\sup j$.
\end{proof}

\begin{lemma}
  Sea $A$ un marco.
  Dado cualquier núcleo $j\in NA$, tenemos
  \[
    \vnuc b \sup j \sup \unuc a = \vnuc b j \unuc a
  .\]
\end{lemma}
\begin{proof}
  Como $(\vnuc b \sup j)\sup \unuc a = (\vnuc b\sup j) \unuc a$, basta demostrar que $\vnuc b \sup j = \vnuc b j$.
  Por el lema anterior, es suficiente con probar la desigualdad
  $j\vnuc b \leq \vnuc b j$.
  Es decir, hay que probar que $j(b\succ x) \leq (b\succ j(x))$
  para todo $x\in A$.
  Para esto, observemos que
  \begin{align*}
    j\vnuc b (x) \inf b
    &= j(b\succ x) \inf b \\
    &\leq j((b\succ x)\inf b) \\
    &= j(b\inf x) \\
    &\leq j(x).
  \end{align*}
  Usando la definición de la implicación, esto nos da $j\vnuc
  b(x) \leq (b\succ j(x))$, que es lo que queríamos.
\end{proof}

\subsection{VIDEO 6: El espacio de puntos, 2da parte (25 OCT)}
\begin{exe}[Armando]
  [16:06]
\end{exe}
\subsection*{(SESIÓN 13: 26 OCT)}

\subsection{La representación de un núcleo generado por una derivada}

Ya probamos que todo núcleo $j\in NA$ se puede representar como
\[
    j = \Sup\{\unuc{j(a)}\inf \vnuc a \mid a\in A \}
.\]
Si existe una derivada $f\in DA$ tal que $j=f^\infty$,
esta construcción se puede mejorar.
En esta sección, fijamos una derivada $f\in DA$ y suponemos que
$j=f^\infty\in NA$.
Usando la cadena de iteraciones de $f$, construiremos una cadena
en $A$, y luego una cadena en $NA$.
\begin{itemize}
  \item
  Para cada $a\in A$ y cada ordinal $\alpha$, definimos
  $a(\alpha)=f^\alpha(a)$.
  Esto nos da una cadena en $A$
  \[
    (a(\alpha) \mid \alpha\in\Ord)
  \]
  la cual, por cardinalidad, se estaciona en algún ordinal.
  En particular, por la definición de $a(\alpha)$, se tiene
  $a(\infty+1)=a(\infty)$ (recordemos que la cadena de los
  $f^\alpha$ se estaciona en el ordinal $\infty$).
  \item
  Usando la cadena anterior, construimos una nueva cadena en $NA$.
  \begin{align*}
    j_{a,0}
    &= \id_A \\
    j_{a,\alpha+1}
    &= (\unuc {a(\alpha+1)}\inf \vnuc {a(\alpha)})\sup j_{a,\alpha}
    \\
    j_{a,\lambda}
    &= \Sup\{j_{a,\alpha} \mid \alpha < \lambda\}
      & \text{(si $\lambda$ es límite).}
  \end{align*}
  Dado que los $a(\alpha)$ se estacionan, los $j_{a,\alpha}$ también.
  En efecto, si $a(\alpha) = a(\alpha+1)$, entonces
  \begin{align*}
    j_{a,\alpha+1}
    &= (\unuc {a(\alpha)}\inf \vnuc {a(\alpha)})
      \sup j_{a,\alpha} \\
    &= \id_A\sup j_{a,\alpha} \\
    &= j_{a,\alpha}.
  \end{align*}
  Sea $j_a$ el mayor de los $j_{a,\alpha}$.
  Es decir,
  \begin{align*}
    j_a
    &= \Sup\{j_{a,\alpha} \mid \alpha\in\Ord\} \\
    &= \Sup\{(\unuc {a(\alpha+1)}\inf\vnuc {a(\alpha)})
      \sup j_{a,\alpha} \mid \alpha\in\Ord\} \\
    &= \Sup\{\unuc {a(\alpha+1)}\inf\vnuc {a(\alpha)}
    \mid \alpha\in\Ord\}
  \end{align*}
  En particular, observemos que $j_a=j_{a,\infty}$.
\end{itemize}
El siguiente resultado nos dice que los núcleos
$j_{a,\alpha}$ tienen una descripción más simple.

\begin{lemma}
  Para cada ordinal $\alpha$, el núcleo $j_{a,\alpha}$ se puede
  expresar como
  \[
    j_{a,\alpha} = \unuc{a(\alpha)} \inf \vnuc a
  .\]
  En particular, para $\alpha=\infty$, tenemos
  \[
    j_a = j_{a,\infty} = \unuc{f^\infty(a)}\inf\vnuc a
  .\]
  Una consecuencia inmediata es que
  \[
    f^\infty = \Sup\{j_a \mid a\in A\}
  ,\]
  pues
  $f^\infty = \Sup\{\unuc{f^\infty(a)}\inf\vnuc a \mid a\in A\}$.
\end{lemma}
\begin{proof}
  Probamos la afirmación por inducción
  \begin{itemize}
    \item Para $\alpha=0$, tenemos $j_{a,0}=\id$, mientras que
    $\unuc{a(0)}\inf \vnuc a = \unuc a \inf \vnuc a = \id$.
    \item Supongamos que
    $j_{a,\alpha} = \unuc{a(\alpha)} \inf \vnuc a$.
    Entonces
    \begin{align*}
      j_{a,\alpha+1}
      &= (\unuc {a(\alpha+1)}\inf \vnuc {a(\alpha)})
        \sup j_{a,\alpha} \\
      &= (\unuc {a(\alpha+1)}\inf \vnuc {a(\alpha)})
        \sup (\unuc{a(\alpha)} \inf \vnuc a) \\
      &=
      (\unuc {a(\alpha+1)}\sup (\unuc{a(\alpha)} \inf \vnuc a))
      \inf(\vnuc {a(\alpha)}\sup (\unuc{a(\alpha)} \inf \vnuc a)) \\
      &= \unuc {a(\alpha+1)}
      \inf(\vnuc {a(\alpha)}\sup (\unuc{a(\alpha)} \inf \vnuc a)) \\
      &= \unuc {a(\alpha+1)}
      \inf(\vnuc {a(\alpha)}\sup\unuc{a(\alpha)})
      \inf(\vnuc {a(\alpha)}\sup \vnuc a) \\
      &= \unuc {a(\alpha+1)} \inf \tp \inf \vnuc a \\
      &= \unuc {a(\alpha+1)}\inf \vnuc a,
    \end{align*}
    como se quería.
    \item Si $\lambda$ es un ordinal límite, supongamos que 
    $j_{a,\alpha} = \unuc{a(\alpha)} \inf \vnuc a$ para todo
    ordinal $\alpha <\lambda$.
    Entonces
    \begin{align*}
      j_{a,\lambda}
      &= \Sup\{j_{a,\alpha} \mid \alpha<\lambda\} \\
      &= \Sup\{\unuc{a(\alpha)}\inf\vnuc a \mid\alpha<\lambda\}
      \\
      &= \Sup\{\unuc{a(\alpha)}\mid\alpha<\lambda\}\inf\vnuc a \\
      &= \unuc{\Sup\{a(\alpha)\mid\alpha<\lambda\}}\inf\vnuc a \\
      &= \unuc{a(\lambda)}\inf\vnuc a,
    \end{align*}
    como se deseaba.
  \end{itemize}
\end{proof}


Con este resultado, podemos probar que el núcleo $j=f^\infty$
tiene una descripción más simple que la canónica.

\begin{lemma}
  Si $f\in DA$ es una derivada tal que $f^\infty$ es un núcleo,
  entonces
  \[
    f^\infty = \Sup\{\unuc{f(a)}\inf\vnuc a \mid a\in A\}
  .\]
\end{lemma}
\begin{proof}
  Dado que
  \[
    \unuc{f(a)}\inf\vnuc a\leq \unuc{f^\infty(a)}\inf\vnuc a
  \]
  para todo $a\in A$, se sigue que
  \[
    \Sup\{\unuc{f(a)}\inf\vnuc a \mid a\in A\}
    \leq
    \Sup\{\unuc{f^\infty(a)}\inf\vnuc a \mid a\in A\}
    = f^\infty
  .\]

  Por otro lado, para cada $a\in A$ y cada ordinal $\alpha$, tenemos
  \[
    a(\alpha+1)=f^{\alpha+1}(a)=f(f^\alpha(a))=f(\alpha(a))
  ,\]
  por lo cual
  \[
     \unuc{a(\alpha+1)}\inf\vnuc{a(\alpha)}
     \in
     \{\unuc{f(b)}\inf\vnuc b \mid b\in A\}
  \]
  (poniendo $b=a(\alpha)$).
  Se sigue que
  \[
     \unuc{a(\alpha+1)}\inf\vnuc{a(\alpha)}
     \leq
     \Sup\{\unuc{f(b)}\inf\vnuc b \mid b\in A\}
  .\]
  Como esto es válido para todos los ordinales, tenemos
  \[
     j_a\leq \Sup\{\unuc{f(b)}\inf\vnuc b \mid b\in A\}
  ,\]
  pues $j_a = \Sup\{\unuc{a(\alpha+1)}\inf\vnuc{a(\alpha)}
   \mid\alpha\in\Ord\}$.
  De nuevo, como esto es válido para cualquier $a\in A$, concluimos
  que
  \[
     f^\infty\leq \Sup\{\unuc{f(b)}\inf\vnuc b \mid b\in A\}
  ,\]
  pues $f^\infty = \Sup\{j_a\mid a\in A\}$ (aquí es donde usamos
  el lema anterior).
\end{proof}


\subsection*{SESIÓN 14: 28 OCT}
\subsection*{SESIÓN 15: 4 NOV}
\subsection*{SESIÓN 15: 9 NOV}
\subsection*{SESIÓN 15: 11 NOV (Dualidad de Stone)}
\subsection*{SESIÓN 15: 16 NOV (Expo Juan)}
\subsection*{SESIÓN 15: 18 NOV (Expo Juan 2)}
\subsection*{SESIÓN 15: 23 NOV (Expo Armando 1)}
\subsection*{SESIÓN 15: 25 NOV (Expo Armando 2, Yareli, Dante, Alfredo)}
\subsection*{SESIÓN 15: 30 NOV (Expo Alfredo, dudas tarea)}
\subsection*{SESIÓN 15: 2 DIC (Producto en Top y coproducto en Frm)}

\part{Productos y coproductos de marcos}
\section{Productos y coproductos de marcos}
\subsection{Productos de marcos}

\begin{thm}[Producto de marcos]
    Sea $\{A_\lambda\}_{\lambda\in\mathscr{I}}$ de marcos arbitraria
    y sea $L$ el producto cartesiano de los $A_\lambda$
    (vistos como conjuntos).
    Entonces $L$, dotado con los operadores puntuales $\inf,\Sup$,
    es un marco y, equipado con las proyecciones canónicas
    \begin{align*}
        p_\lambda: L&\to A_\lambda \\
        a &\mapsto p_\lambda(a)=a_\lambda
    \end{align*}
    satisface la propiedad universal del producto de los $A_\lambda$.
\end{thm}

\begin{lemma}
    $(L,\leq,\wedge, 0, \bigvee, 1)$, con
    $0:=(0_\lambda)_{\lambda\in\scr{I}}$ y
    $1:=(1_\lambda)_{\lambda\in\scr{I}}$ es un marco.
\end{lemma}
\begin{proof}
    $\leq$ es orden parcial ya que:
    \begin{align*}
        x \leq x
            & \iff x_\lambda \leq x_\lambda \forall \lambda\in\scr I. \\
        x \leq y, y \leq x
            &\implies x_\lambda \leq y_\lambda,
            y_\lambda \leq x_\lambda, \forall  \lambda\in\scr{I}\\
            &\implies x_\lambda = y_\lambda
            \forall\lambda\in\scr{I}\\
            &\implies x = y.\\
        x \leq y, y \leq z
            &\implies x_\lambda \leq y_\lambda,
            y_\lambda \leq z_\lambda
            \forall\lambda\in\scr{I}\\
            &\implies x_\lambda  \leq z_\lambda
            \forall  \lambda\in\scr{I}\\
            &\implies x \leq z.
    \end{align*}
\end{proof}
Para $x,y\in A$, ya que $x_\lambda\wedge_\lambda y_\lambda\in A_\lambda$ para todo $\lambda\in\mathscr{I}$, entonces $x\wedge_\mathscr{I}y\in A$.\\
Luego, ya que $0_\lambda\leq_\lambda x_\lambda$ para todo $x_\lambda\in A_\lambda$ y para cada $\lambda\in\mathscr{I}$, entonces $0_\mathscr{I}\leq_\mathscr{I}x$ para todo $x\in A$.\\
Similarmente, para $X\subseteq A$ arbitrario, consideramos los subconjuntos $X_\lambda\subseteq A_\lambda$ formados por los $x_\lambda$ componentes, y tomamos $\bigvee X_\lambda=:s_\lambda\in A_\lambda$,  para cada $\lambda\in\mathscr{I}$.\\
Entonces $\bigvee_\mathscr{I}X=(s_\lambda)_{\lambda\in\mathscr{I}}\in A$.\\
Además, como $1_\lambda\geq_\lambda x_\lambda$ para todo $x_\lambda\in A_\lambda$, para cada $\lambda\in\mathscr{I}$, entonces $1_\mathscr{I}\geq_\mathscr{I} x\ \forall x\in A$.
Por útlimo, se cumple que:
\begin{align*}
    a\wedge_\mathscr{I}\left(\bigvee_\mathscr{I} X\right) & = (a_\lambda\wedge_\lambda s_\lambda)_{\lambda\in\mathscr{I}}\\
    & = \left(\bigvee_\lambda\{a_\lambda\wedge_\lambda x_\lambda\mid x_\lambda\in X_\lambda\}\right)_{\lambda\in\mathscr{I}}\\
    & = \bigvee_\mathscr{I} \{a\wedge_\mathscr{I}x\mid x\in X\}
\end{align*}
Y con esto, se cumple que $A$ es un marco.
    Para cada $\lambda\in\mathscr{I}$, se define $p^*_\lambda:A\to A_\lambda$ la proyección de $A$ sobre el respectivo componente $A_\lambda$, nótese el siguiente resultado:
    \begin{lemma}
        $p^*_\lambda$ es morfismo de marcos, para cada $\lambda\in\mathscr{I}$.\\
    \end{lemma}
        \begin{proof}
            \begin{align*}
                x\leq y & \Longleftrightarrow x_\lambda\leq y_\lambda \ \forall\lambda\in\mathscr{I}\\
                & \Rightarrow p^*_\lambda(x)\leq p^*_\lambda(y)\ \forall\lambda\in\mathscr{I} 
            \end{align*}
            Además se cumple que $p^*_\lambda(1_\mathscr{I})=1_\lambda$ y $p^*_\lambda(0_\mathscr{I})$.\\
            Finalmente:
            \begin{align*}
                p^*_\lambda(x\wedge y) & = p^*_\lambda[(x_\mu\wedge y_\mu)_{\mu\in\mathscr{I}}]\\
                & = x_\lambda\wedge y_\lambda\\
                & = p^*_\lambda(x)\wedge p^*_\lambda(y).
            \end{align*}
        \end{proof}
\begin{align*}
    p^*_\lambda(\bigvee X) & = p^*_\lambda[(s_\mu)_{\mu\in\mathscr{I}}] \\
                           & = s_\lambda \\
                           & = \bigvee X_\lambda \\
                           & = \bigvee p^*_\lambda[X].
\end{align*}
\begin{proof}
    Sea $B\in\Frm$, y considere una familia arbitraria de morfismos:
    \begin{equation*}
        \{r^*_\lambda:B\to A_\lambda\mid\lambda\in\mathscr{I}\}.
    \end{equation*}
    Definimos $p^*:B\to A$, de manera que, para $b\in B$:
    \begin{equation*}
        p^*(b) = (r^*_\lambda(b))_{\lambda\in\mathscr{I}}.
    \end{equation*}
\end{proof}
Se sigue que $p^*$ es morfismo de marcos, ya que la familia de $r^*_\lambda$ son morfismos de marcos, y por la construcción de los operadores de $A$.\\
Luego, para todo $b\in B$ y todo $\lambda\in\mathscr{I}$, se cumple que:
\begin{equation*}
    (p^*_\lambda\circ p^*)(b) = p^*_\lambda[(r_\mu(b))_{\mu\in\mathscr{I}}] = r_\lambda(b),
\end{equation*}
y nótese que $p^*$ es único por construcción. Así, se concluye que $A$ satisface la propiedad universal del producto en $\Frm$.

\subsection{Sitios}

Ya probamos que la categoría de marcos tiene productos.
Para probar que también tiene coproductos,
usaremos una técnica distinta, para la cual necesitamos
algunos resultados concernientes a sitios.

\begin{defn}
    Sea $A$ una semíretícula inferior.
    Una función $C:A\to\mathcal{P}[\mathcal{P(A)}]$
    es una \textit{cobertura} o \textit{función de cubiertas}
    sobre $A$ si,
    para cualesquiera $a,b\in A$, se cumple que:
    \begin{enumerate}
        \item $S\in C(a) \implies S\subseteq\down(a)$
        \item $S\in C(a), b\leq a \implies \{b\wedge s\mid s\in S\}\in C(b)$.
    \end{enumerate}
    Al par $(A,C)$ se le llama sitio.
\end{defn}
\begin{defn}[$C$-ideales]
    Dado un sitio $(A,C)$, un subconjunto
    $I\subset A$ es un $C$-ideal de $A$ si
    \begin{enumerate}
        \item $I\in\mathcal{L}(A)$
        \item Siempre que $a\in A$ y $S\in C(a)$, entonces
        $S\subseteq I\Rightarrow a\in I$.
    \end{enumerate}
    Al conjunto de todos los $C$-ideales del sitio $(A,C)$
    se le denota por $C\Idl$.
\end{defn}
\begin{exa}[Ejemplos de sitios]
    Sea $A$ una $\inf$-retícula.
    Consideremos las funciones cubrientes 
    $C_\emptyset$ y $C_T$ definidas como:
    \begin{align*}
        C_\emptyset(a) &= \emptyset \\
        C_T(a) &= \{\emptyset\}
    \end{align*}
    para todo $a\in A$.
    \begin{itemize}
        \item
        En el primer caso,
        ninguna famila cubre a ningún elemento.
        Entonces, por vacuidad,
        toda sección inferior $F\subseteq A$ es
        un $C_\emptyset$-ideal.
        Es decir, $C_\emptyset\Idl(A)=\cal L(A)$.
        \item
        En el segundo caso,
        la familia vacía cubre a todos los elementos,
        así que el único $C_T$-ideal $F\subseteq A$
        es la sección total $F=A$.
        Es decir, $C_T\Idl(A)=\{A\}$.
    \end{itemize}
\end{exa}

\begin{exa}[Más ejemplos de sitios]
    Sea $A$ un marco y consideremos las funciones cubrientes
    $C_\sup$ y $C_{\Sup}$ definidas como
    \begin{align*}
        X\in C_\sup(a)
        &\ssi
        \text{$X$ es finito y } a=\Sup X \\
        X\in C_{\Sup}(a)
        &\ssi
        a=\Sup X 
    \end{align*}
    para todo $a\in A$ y todo $X\subseteq A$.
    Entonces
    \begin{itemize}
        \item
        una sección inferior $F\subseteq A$ es un
        $C_\sup$-ideal si, y solo si es cerrada bajo supremos
        finitos.
        \item
        una sección inferior $F\subseteq A$ es un
        $C_{\Sup}$-ideal si, y solo si es cerrada bajo supremos
        arbitrarios.
    \end{itemize}
\end{exa}

\begin{defn}[Marcos generado por un sitio]
    Sea $(A,C)$ un sitio.
    Decimos que un marco $B$ es generado por $(A,C)$
    si existe un morfismo de semiretículas inferiores
    $f:A\to B$ tal que, para todo $a\in A$, se cumple
    que $f$ manda $C$-cubiertas en $A$ a $C_{\Sup}$-cubiertas
    de $B$:
    \begin{equation*}
        S\in C(a) \implies f(a) = \Sup\{f(s)\mid s\in S\}
    \end{equation*}
    y $f$ es universal con respecto a esta propiedad.
    Explícitamente, la universalidad significa que,
    si $B'$ es un marco y $f':A\to B'$ es un morfismo
    de $\inf$-semiretículas
    que convierte cubiertas en supremos,
    entonces existe un único morfismo de marcos
    $\ol{f}:B\to B'$ tal que el siguiente diagrama conmuta:
    \[
        \begin{tikzcd}[ampersand replacement=\&]
            \& B \arrow[dd, "\exists!\overline{f}", dashed] \\
            A \arrow[ru, "f"] \arrow[rd, "f'"'] \& \\
            \& B'                                          
        \end{tikzcd}
    \]
    Por el argumento usual, entre cualesquiera dos
    marcos generados por $(A,C)$ existe un único isomorfismo,
    de modo que podemos hablar de \emph{el} marco generado
    por $(A,C)$.
\end{defn}

\begin{thm}[$C\Idl$ es un marco]
    Sea $(A,C)$ un cubriente, entonces, su conjunto de $C$-ideales $C\Idl$ es un marco con la contención, y es generado por $(A,C)$.
\end{thm}

Recordemos que $\mathcal{L}A$ es un marco bajo la contención
y que, si $B$ es un marco y $\nu\in NA$ es cualquier núcleo,
entonces $\nu[A]=A_\nu$ es marco con el orden heredado de $A$.

\begin{lemma}[Previo]
    La intersección arbitraria de $C-$ideales es un $C-$ideal.
\end{lemma}
\begin{proof}
    Sean $\{I_\alpha\}_{\alpha\in\Gamma}$ una familia arbitraria de $C-$ideales, y $I=\bigcap_{\alpha\in\Gamma}I_\alpha$ su intersección. Primero, sean $x\leq y\ \in A$ arbitrarios, con $y\in I$. Entonces $y\in I_\alpha\forall\alpha\in\Gamma$, y así, como cada $I_\alpha$ es sección inferior, entonces $x\in I_\alpha\forall\alpha\in\Gamma$, y esto implica que $x\in I$. Por tanto, cómo $x,y$ son arbitrarios, se cumple que $I$ es sección inferior de $A$.\\
    Ahora, sean $a\in A, S\in C(a)$ arbitrarios, y supongáse que $S\subseteq I$, entonces, se cumple que $S\subset I_\alpha$ para todo $\alpha\in\Gamma$. Por tanto, ya que cada $I_\alpha$ es $C-$ideal, se sigue que $a\in I_\alpha\ \forall\alpha\in\Gamma$, y se sigue que $a\in I$.\\
    Con esto, concluimos que $I$ es un $C-$ideal.
\end{proof}

\begin{lemma}[Parte 1]
    Existe un núcleo $j:\cal LA\to\cal LA$ tal que
    $C\Idl=(\cal LA)_j$, de modo que $C\Idl$ es un cociente
    de $\cal LA$.
    En particular, $C\Idl$ es un marco.
\end{lemma}
\begin{proof}
    Considere el marco de secciones inferiores de $A$, $\mathcal{L}A$, y considere la función $j:\mathcal{L}A\to\mathcal{L}A$ tal que
    \begin{equation*}
        j(S)=\bigcap\{I\in C\Idl\mid S\subseteq I\}\quad S\in\mathcal{L}A.
    \end{equation*}
    Primero, para cualquier $S\in\mathcal{L}A$, por la definición de $j$, se tiene que $S\subseteq j(S)$, por tanto $j$ infla. Luego, por el lema anterior, se tiene que $j(S)$ es $C-$ideal, y cómo $j(S)\subseteq j(S)$ trivialmente, y $j(j(S))$ debe ser el menor $C-$ideal que contiene a $j(S)$, obtenemos que $j(j(S))=j(S)$, i.e. $j$ es idempotente.
    
    Ahora, sean $R,T\in\mathcal{L}A, I:=j(R\cap T)$. Por definición de $j$, se tiene que $R\subseteq j(R)$ y $T\subseteq j(T)$, entonces $R\cap T\subseteq j(R)\cap j(T)$, y cómo $j$ es idempotente, se sigue que $j(R\cap T)\subseteq j(R)\cap j(T)$.\\
    Defináse ahora el conjunto
    \begin{equation*}
        R' := \{d\in A\mid \forall t\in T,d\wedge t\in I\}.
    \end{equation*}
    Por la definición de $I$ y ya que $j$ infla, se cumple que $R\subseteq R'$, $T\cap R'\subset I$. Por otro lado, cómo $I\in\mathcal{L}(A)$, obtenemos que $R'\in\mathcal{L}(A)$. Luego, consideremos $U\in C(A)$ tal que $U\subseteq R'$, entonces, para todo $t\in T$, se cumple que $\{u\wedge t\mid u\in U\}\subseteq I$, y además, utilizando la propiedad $(ii)$ de $C$, se cumple que $\{u\wedge t\mid u\in U\}\in C(a\wedge t)$ , y en consecuencia $a\wedge t\in I$, ya que $I$ es $C-$ideal. Luego, cómo $t\in T$ es arbitrario, tenemos que $a\in R'$ y por tanto $R'$ es un $C-$ideal.
    
    Análogamente, se construye el $C-$ideal
    \begin{equation*}
         T' := \{e\in A\mid \forall r\in R',e\wedge r\in I\},
    \end{equation*}
    y se cumple que $T\subseteq T'$, $T'\cap R\subseteq I$ y $T'$ es $C-$ideal. Así, se sigue que $j(R)\subseteq R',j(T)\subseteq T'$ y por tanto
    \begin{equation*}
        j(R)\cap j(T)\subseteq R'\cap T' \subseteq  I = j(R\cap T)
    \end{equation*}
    y en conclusion, $j$ es núcleo cuyo conjunto de puntos fijos
    es $C\Idl$.
\end{proof}

\begin{lemma}[Parte 2]
    La función $f:A\to C\Idl$ definida como
    \begin{equation*}
        f(a)=j(\down(a))
    \end{equation*}
    es un morfismo de $\inf$-semiretículas que
    manda $C$-cubiertas en supremos de $C\Idl$.
    Más aún, $f$ es universal con respecto a esta propiedad,
    de modo que $C\Idl$ es el marco generado por $(A,C)$.
\end{lemma}

\begin{proof}[$C\Idl$ es generado por $(A,C)$]
    Cómo $j$ es núcleo y $\down$ es morfismo
    de $\inf$-semiretículas,
    se cumple que $f$ es morfismo de $\inf-$retículas.
    Sean $a\in A,S\in C(a)$ arbitrarios. Entonces, cómo $j(\bigcup\{j(\down(s))\mid s\in S\})=:\mathcal{J}$ es un $C-$ideal que contiene a $S$, se sigue que $a\in\mathcal{J}$, luego 
    \begin{equation*}
        j(\down(a))\subseteq \mathcal{J}.
    \end{equation*}
    Por otro lado, se tiene $a\geq s$ para todo $s\in S$, por lo cuál $j(\down(a))\supseteq j(\down(s))$. Luego, se sigue que
    \begin{equation*}
        j(\down(a))\supseteq \bigcup\{j(\down(s))\mid s\in S\}
    \end{equation*}
    y con esto, cómo $j$ es mónotona e idempotente, obtenemos que
    \begin{equation*}
        f(a)\supseteq\mathcal{J}.
    \end{equation*}

    Por otra parte, sean $B\in\Frm$, $g:A\to B$ un morfismo de $\wedge-$semiretículas que convierte cubiertas de $C$ en supremos. Entonces, la función 
    \begin{align*}
        \overline{g}: &\mathcal{L}A\to B \\
                      & S\to \bigvee_B\{g(s)\mid s\in S\}
    \end{align*}
    es el único morfismo de marcos que factoriza a $g$ a tráves del marco libre de $A$, $\mathcal{L}A$, y además existe su adjunto derecho $g_*:B\to\mathcal{L}A$, que, por definición del adjunto derecho, para $b\in B:$

    \begin{align*}
        g_*(b) & = \bigcup\{L\in\mathcal{L}A\mid\overline{g}(L)\leq b\} \\
               & = \bigcup\left\{L\in\mathcal{L}A\mid \bigvee_B\{g(a)\mid a\in L\}\leq b\right\} \\
               & = \bigcup\{L\in\mathcal{L}A\mid g(a)\leq b (\forall a \in L)\} \\
               & = \{a\in A\mid g(a)\leq b\} 
    \end{align*}

    Y nótese que, si $S\in C(a)$ y $S\subseteq g_*(b)$, entonces se cumple que
    \begin{equation*}
        g(a)= \bigvee_B\{g(s)\mid s\in S\}\leq b
    \end{equation*}
    y esto implica que $a\in g_*(b)$, por tanto $g_*(b)$ es un $C-$ideal. 

    Sea $a\in A$ arbitrario. Cómo $\overline{g}$ y $g_*$ son adjuntos, se tiene que
    \begin{equation*}
        (\overline{g}\circ g_*)(g(a))\leq g(a).
    \end{equation*}
    Además, cómo $g_*(g(a))$ es un $C-$ideal que contiene a $\down(a)$, tenemos que $$f(a)=j(\down(a))\subseteq g_*(g(a))$$, y en consecuencia, se cumple la cadena de desigualdades:
    \begin{equation*}
        g(a)\leq\overline{g}(f(a))\leq(\overline{g}\circ g_*)(g(a))\leq g(a)
    \end{equation*}

    Es decir, que $(\overline{g}\circ f)(a)=g(a)$
    con $a$ arbitrario,
    por tanto $\overline{g}\circ f = g$,
    con $\overline{g}$ único, así que $f$ es universal.
\end{proof}

\subsection{Coproductos de marcos}

Ya probamos que la categoría de marcos tiene productos.
Ahora queremos demostrar el siguiente teorema.

\begin{thm}
Para una familia de marcos $\{A_\lambda\}_{\lambda\in\scr I}$,
existe el coproducto
\[\coprod_{\lambda\in\scr I} A_\lambda.\]
\end{thm}

Construiremos el coproducto como sigue:
Tomaremos el coproducto $A$ de la familia
en la categoría $\Pos^\inf$ de $\inf$-semiretículas.
Le daremos a $A$ una estructura de sitio, equipándole una
cobertura $C$.
Finalmente, veremos que el marco $C\Idl$ generado por $(A,C)$
viene equipado con morfismos que lo convierten en el coproducto
de nuestra familia en la categoría de marcos.

\begin{lemma}
    Consideremos el conjunto
    \[
    A=
    \left\{ a\in\prod_{\lambda\in\scr I} A_\lambda
    \mid a_\lambda\neq 1_{A_\lambda}
    \text{ para una cantidad finita de índices} \right\}.
    \]
    Notemos que $A$ es una subsemiretícula inferior del producto,
    ya que $1\in A$ y, si $a,b\in A$, entonces
    $c=a\inf b\in \prod_{\lambda\in\scr I} A_\lambda$
    satisface
    \[c_\lambda=
    \begin{cases}
        1_\lambda, & \textit{si }a_\lambda=b_\lambda=1_\lambda\\
        a_\lambda\wedge b_\lambda, & \textit{ en otro caso}
    \end{cases}
    \]
    Como hay una cantidad finita de
    $a_\lambda\neq 1_\lambda$ y $b_\lambda\neq 1_\lambda$,
    entonces los $a_\lambda\wedge b_\lambda$ son finitos,
    por lo que $c\in A$.
    
    Definimos las funciones
    $q_\lambda\colon A_\lambda\to A$,
    para cada $\lambda\in \scr I$, dadas por 
    \[q_\lambda(x)=a,\]
    donde $a_\lambda=x$ y $a_\mu =1_{\mu}$
    para $\mu\in \scr I\setminus \{\lambda\}$.
    
    Claramente, $q_\lambda$ es monótona y $q_\lambda(1_\lambda)=1$.
    Además, si $x,y\in A_\lambda$, entonces
    $q_\lambda(x\wedge y)=q_\lambda(x)\wedge q_\lambda(y)$,
    ya que las operaciones son puntuales.
    Así, cada $q_\lambda$ es un morfismo de semiretículas
    inferiores.
    
    Afirmamos que $A$, junto con sus funciones
    \[
        q_\lambda:A_\lambda\to A
    \]
    es el coproducto de la familia
    $\{A_\lambda\}_{\lambda\in\scr I}$
    en la categoría de $\inf$-semiretículas.
\end{lemma}
\begin{proof}
    Para este resultado, los $\{A_\lambda\}_{\lambda\in\scr I}$
    solo necesitan ser $\inf$-semiretículas.
    Sean $B$ una $\inf$-semiretícula y
    $r_\lambda:A_\lambda\to B$ una familia de morfismos
    indicada por $\lambda\in\scr I$.
    Definimos $R:A\to B$ como
    \[
        R(a) = \Inf\{r_\lambda(a_\lambda) \mid \lambda \in \scr I\}
    .\]
    Este ínfimo existe en $B$, ya que todo $a\in A$ tiene soporte
    finito.
    Claramente, $R$ preserva ínfimos y hace conmutar el diagrama
    \[
        \begin{tikzcd}
            A_\lambda \ar[r,"r_\lambda"] \ar[d,"q_\lambda"']
            & B \\
            A \ar[ur,"R"']
        \end{tikzcd}
    \]
    para todo $\lambda\in \scr I$.
    Finalmente, si $R':A\to B$ es cualquier morfismo
    que hace conmutar el diagrama, tenemos
    \begin{align*}
        R'(a)
        &= R'(\Inf\{q_\lambda(a_\lambda)\mid a\in \scr I\}) \\
        &= (\Inf\{R'(q_\lambda(a_\lambda))\mid a\in \scr I\}) \\
        &= (\Inf\{r_\lambda(a_\lambda)\mid a\in \scr I\}) \\
        &= R(a).
    \end{align*}
    De este modo, la semiretícula $A$, equipada con $R$,
    es el coproducto de las $A_\lambda$ en $\Pos^\inf$.
\end{proof}

\begin{defn}
Sean $a=(a_\lambda)\in A$, $\mu\in\scr I$ y $S\subseteq A_\mu$, definimos el reemplazo de $a$ por $S$ en la entrada $\mu-$ésima como
\[S(a,\mu)=\{(b_\lambda)\in A\mid b_\lambda=a_\lambda\textit{ para }\lambda\in\scr I\setminus \{\mu\}\textit{ y }b_\mu\in S\}\]
\end{defn}
\begin{exa}
Si $a=(a_1,a_2,a_3,a_4)\in\prod_{\lambda=1}^4A_\lambda$, $\mu=2$ y $S=\{x,y,z\}\subseteq A_2$, entonces
    \[
      S(a,\mu) = \left\{
      \begin{array}{c}
        (a_1,x,a_3,a_4), \\
        (a_1,y,a_3,a_4), \\
        (a_1,z,a_3,a_4)
      \end{array}
      \right\}
    .\]
\end{exa}


\begin{lemma}[La estructura de sitio en $A$]
    La función $C:A\to\cal P(\cal P(A))$
    definida como
    \[
        C(a)
        =\left\{ S(a,\mu)
        \mid \mu\in\scr I, S\subseteq A_\mu
        \text{ tal que }\Sup S=a_\mu\right \}.
    \]
    para cada $a\in A$, es una cobertura en $A$.
\end{lemma}
\begin{proof}
    Sean $a=(a_\lambda),b=(b_\lambda)\in A$ con $b\leq a$ y $W=S(a,\mu)\in C(a)$.
    \begin{itemize}
    \item Sea $c=(c_\lambda)\in W$. Por definición, $c_\lambda=a_\lambda$ para $\lambda\in \scr I\setminus\{\mu\}$ y $c_\mu \in S$, entonces $c_\mu\leq a_\mu$. Esto implica que $c\leq a$, es decir, $c\in \down(a)$.
    \end{itemize}
    
    \begin{itemize}
    \item Probaremos que $\{b\inf w\mid w\in W\}=\{b_\mu \inf w_\mu\mid w_\mu\in S\}(b,\mu)$.\par 
    \begin{itemize}
    \item[$\subseteq)$] Sea $b\inf w\in\{b\inf w\mid w\in W\}$. Observamos que 
    \begin{itemize}
    \item $b\inf w\in A$.
    \item $b_\lambda\inf w_\lambda=b_\lambda\inf a_\lambda=b_\lambda$ para $\lambda\in\scr I\setminus \{\mu\}$.
    \item $b_\mu\inf w_\mu\in\{b_\mu\in w_\mu\mid w_\mu\in S\}$.
    \end{itemize}
    Por lo que $b\inf w\in \{b_\mu \inf w_\mu\mid w_\mu\in S\}(b,\mu)$.
    \item[$\supseteq)$] Sea $c=(c_\lambda) \in \{b_\mu \inf w_\mu\mid w_\mu\in S\}(b,\mu)$. Entonces $c_\lambda=b_\lambda$ para $\lambda\in\scr I\setminus\{\mu\}$ y $c_\mu\in \{b_\mu \inf w_\mu\mid w_\mu\in S\}$.\par 
    Notemos que $c=b\inf m$, donde $m_\lambda=a_\lambda$ para $\lambda\in\scr I\setminus \{\mu\}$ y $m_\mu=w_\mu$. Así, $m\in W$, es decir, $c\in\{b\in w\mid w\in W\}$.
    \end{itemize}
    Además, $\{b_\mu \inf w_\mu\mid w_\mu\in S\}\subseteq A_\mu$ y
    \[\Sup\{b_\mu \inf w_\mu\mid w_\mu\in S\}=b_\mu\inf \left(\Sup S\right)=b_\mu\inf a_\mu=b_\mu\]
    Por lo tanto $\{b\inf w\mid w\in W\}\in C(b)$.
    \end{itemize}
\end{proof}

La estructura de los $C$-ideales en $A$ no es tan sencilla
de entender a primera vista.
Incluimos un ejemplo de un cálculo en $C\Idl(A)$.

\begin{exa}[El menor elemento de $C\Idl(A)$.]
    Supongamos que $a\in A$ tiene
    alguna entrada cero.
    Es decir, $a_\lambda=0\in A_\lambda$ para algún $\lambda$.
    
    Al fijarnos en la entrada $a_\lambda=0\in A_\lambda$,
    tenemos $\Sup\emptyset 0\in A_\lambda$.
    Como $\emptyset(a,\lambda)=\emptyset\subseteq A$,
    de la definición de las $C$-cubiertas
    se sigue que $\emptyset\in C(a)$;
    es decir: la familia vacía cubre a $a$.
    
    En particular, para cualquier $C$-ideal $F\subseteq A_\scr I$,
    tenemos
    \[
        \emptyset \in C(a)
        \hspace{10mm} \text{y} \hspace{10mm}
        \emptyset\subseteq F
    ,\]
    y, por la definición de $C$-ideal, se sigue que $a\in F$.
    
    Esto nos dice que el conjunto $G$ de los elementos que tienen
    alguna entrada nula
    \[
        G=\{a\in A \mid a_\lambda = 0\in A_\lambda
        \text{ para algún } \lambda\in \scr I\}
    \]
    está contenido en todos los $C$-ideales.
    De hecho, probaremos que $G$ es un $C$-ideal.
    Luego, es el menor elemento en el marco de $C$-ideales.

    Claramente, $G$ es sección inferior, por lo cual resta
    probar que es cerrado bajo $C$-cubiertas.
    En efecto, tomemos una cubierta
    \[
        Y(a,\lambda) \in C(a)
        \hspace{10mm} \text{con} \hspace{10mm}
        Y(a,\lambda) \subseteq G
    .\]
    Es decir, $a$ tiene una entrada $a_\lambda$ tal que
    $Y\subseteq A_\lambda$ y $\Sup Y=a_\lambda$.
    
    Si $\Sup Y=0$, entonces $0=\Sup Y=a_\lambda$, así que $a\in G$.
    De otro modo, existe un $y\in Y$, $y\neq 0$.
    Sea $b\in Y(a,\lambda)$ el único elemento de
    $\{y\}(a,\lambda)$.
    Entonces $b$ tiene $\lambda$-ésima entrada
    $b_\lambda=y\neq 0$, pero como $b\in Y(a,\lambda)\subseteq G$,
    existe un índice $\mu\neq\lambda$ tal que $b_\mu = 0$.
    Como $b$ es igual a $a$ en todas las entradas
    que no son $\lambda$ (pues $b\in Y(a,\lambda)$),
    en particular tenemos $b_\mu=a_\mu=0$, así que $a\in G$.
    
    Se sigue que $G$ es un $C$-ideal y, como está contenido
    en todos los $C$-ideales, concluimos que $G$ es el
    menor elemento de $C\Idl(A)$.
\end{exa}


\begin{lemma}[Parte 2]
    Consideremos $C\Idl$ y los morfismos
    $Q_\lambda\colon A_\lambda\to C\Idl$
    definidos como
    \[
        Q_\lambda=j(\down(q_\lambda(\_)))
    .\]
    Entonces $C\Idl$, equipado con los morfismos $Q_\lambda$,
    es el coproducto de la famila de los $A_\lambda$
    en la categoría de marcos.
\end{lemma}
\begin{proof}
Para cualquier marco $Y$ y morfismos
$f_\lambda\colon A_\lambda\to Y$ indicados por $\scr I$,
queremos encontrar un único morfismo
$f\colon C\Idl\to Y$ tal que el siguiente diagrama
\[
    \begin{tikzcd}
        A_\lambda
            \ar[d,"Q_\lambda"']
            \ar[r,"f_\lambda"]
        & Y \\
        C\Idl \ar[ru,"f"']
    \end{tikzcd}
\]
conmuta.

Sean $X$ un marco y una familia de morfismos
\[\{r_\lambda\colon A_\lambda\to X\mid \lambda\in \scr I\}.\]
Dado que $A$, junto con los $q_\lambda:A_\lambda\to A$,
es el coproducto de los $A_\lambda$ como $\inf$-semiretículas,
obtenemos el morfismo de $\inf$-semiretículas
$R\colon A\to X$ dado por
\begin{align*}
    R(a)
    &=\Inf_X \{r_\lambda(a_\lambda)\mid \lambda\in\scr I\} \\
    &= 1_X\inf r_{\lambda_1}(a_{\lambda_1})\inf\cdots\inf r_{\lambda_n}(a_{\lambda_n})
\end{align*}
donde $\{\lambda_1,\dots,\lambda_n\}$ es el conjunto
de las coordenadas de $a$ distintas de $1$.
Este es el único morfismo de $\inf$-semiretículas que
factoriza a todos los $r_\lambda$ a través de $A$.
Probaremos que $R$ convierte cubiertas de $C$ en supremos.

Consideremos $\mu\in\scr I$, $S\subseteq A_\mu$ y $a=(a_\lambda)\in A$ tal que $a_\mu=\Sup_{A_\mu} S$.
Hay que probar que $\Sup R(S(a,\mu)) = R(a)$.
Si $a_\mu= 1_{A_\mu}$, tenemos
\begin{align*}
    \Sup R(S(a,\mu))
    &= \Sup\{R(b) \mid b\in S(a,\mu)\} \\
    &= \Sup\{ 1_X\inf r_{\lambda_1}(a_{\lambda_1})\inf
        \cdots\inf r_{\lambda_n}(a_{\lambda_n}) \inf r_\mu(s) \mid s\in S\} \\
    &= 1_X\inf r_{\lambda_1}(a_{\lambda_1})\inf
        \cdots\inf r_{\lambda_n}(a_{\lambda_n}) \inf
        \Sup\{ r_\mu(s) \mid s\in S\} \\
    &= 1_X\inf r_{\lambda_1}(a_{\lambda_1})\inf
        \cdots\inf r_{\lambda_n}(a_{\lambda_n}) \inf
        r_\mu(a_\mu) \\
    &= R(a).
\end{align*}

Por otro lado, si $a_\mu\neq 1_{A_\mu}$, tenemos $\mu=\lambda_i$.
Entonces
\begin{align*}
    &\hspace{-10mm}\Sup R(S(a,\mu)) \\
    &= \Sup R(S(a,\lambda_i)) \\
    &= \Sup\{R(b) \mid b\in S(a,\lambda_i)\} \\
    &= \Sup\{ 1_X\inf r_{\lambda_1}(a_{\lambda_1})\inf
        \cdots \inf r_{\lambda_i}(s)\inf\cdots\inf r_{\lambda_n}(a_{\lambda_n}) \mid s\in S\} \\
    &= 1_X\inf r_{\lambda_1}(a_{\lambda_1})\inf \cdots
        \inf\Sup\{ r_{\lambda_i}(s) \mid s\in S\} \inf
        \cdots\inf r_{\lambda_n}(a_{\lambda_n}) \\
    &= 1_X\inf r_{\lambda_1}(a_{\lambda_1})\inf \cdots
        \inf r_{\lambda_i}(a_{\lambda_i}) \inf
        \cdots\inf r_{\lambda_n}(a_{\lambda_n})
        \\
    &= R(a).
\end{align*}

Como $R:A\to X$ manda $C$-cubiertas a supremos de $X$,
la propiedad universal del marco $C\Idl$ generado por $(A,C)$
asegura que existe un único morfismo de marcos
$g\colon C\Idl\to X$ tal que el diagrama
\[
    \begin{tikzcd}[ampersand replacement=\&]
        \& X  \\
        C\Idl \ar[ru,"g"] \& A \ar[u,"R"'] \ar[l,"{j(\down({\_}))}"]
    \end{tikzcd}
\]
conmuta.

Notemos que
\[r_\lambda=R\circ q_\lambda=g\circ j(\down(q_\lambda(\_)))=g\circ Q_\lambda.\]
Por lo que el diagrama
\[
    \begin{tikzcd}[ampersand replacement=\&]
        A_\lambda \arrow[d, "Q_\lambda"'] \arrow[r, "r_\lambda"]
        \& X \\
        C\Idl \arrow[ru, "g"']
        \& A \arrow[l, "j(\down(\_))"] \arrow[u, "R"']
    \end{tikzcd}
\]
conmuta.
\end{proof}

\section{El teorema de Tychonoff en marcos}

Un espacio topológico es compacto si cualquier cubierta
abierta del espacio admite una subcubierta finita.
En la categoría de espacios topológicos tenemos el
teorema de Tychonoff, que dice que un producto de espacios
topológicos es compacto, si, y solo si, cada uno de los
factores es un compacto.

Del mismo modo, cualquier marco tiene su noción de compacidad.
\begin{defn}[Compacidad de marcos]
Un marco $A$ es compacto si, para cualquier $S\in A$
se satisface lo siguiente:
\[
    \Sup S= 1_A
    \implies
    \textit{ existe } T\subseteq S\text{ finito, tal que }
    \Sup T= 1_A
.\]
\end{defn}
El objetivo de esta sección es probar el teorema análogo
al de Tychonoff, pero en la categoría de marcos:
\begin{thm}
    Sea $\{A_\lambda\}_{\lambda\in\scr I}$ una familia de marcos.
    El coproducto
    \[
        \coprod_{\lambda\in\scr I}A_\lambda
    \]
    es compacto si, y solo si, cada $A_\lambda$ es compacto.
\end{thm}

\subsection{Idea de la demostración}

En la sección anterior probamos que
$\coprod_{\lambda\in\scr I}A_\lambda$
es el marco de $C$-ideales de $A$, donde $A$ es el coproducto
de los $A_\lambda$ como $\inf$-semiretículas y $C$ es cierta
cobertura en $A$.
Como $C\Idl$ es un cociente de $\cal LA$,
los supremos en $C\Idl$ se calculan como
\[
    \Sup S = j(\bigcup S)
,\]
donde $j:\cal LA\to\cal LA$ es el núcleo asociado a $C\Idl$.
Por lo tanto, es de esperarse que la demostración de
nuestra versión del teorema de Tychonoff involucre al núcleo
$j$ de alguna manera.
De hecho, toda la demostración consiste
en dar una construcción de $j$ que facilite convertir
ciertos supremos arbitrarios en supremos finitos.
Para llevar a cabo esta idea, primero definiremos una
cierta subcobertura $C_f\subseteq C$, de tal modo que
factorizaremos la proyección $\cal LA\to C\Idl$ a través de
$C_f\Idl$:
\[
    \begin{tikzcd}
        \cal LA \ar[d] \ar[r] & C\Idl \\
        C_f\Idl \ar[ur]
    \end{tikzcd}
\]

Más específicamente, el plan es el siguiente:
\begin{itemize}
    \item
    Definimos una subcobertura $C_f\subseteq C$ en $A$ y
    construimos la proyección $\cal LA\to C_f\Idl$.
    Es decir, dado $S\in\cal L(A)$,
    construimos el $C_f$-ideal $FS$ generado por $S$.
    \item 
    Vemos que el conjunto $\cal DS$ de supremos dirigidos
    de $S$ está contenido en el $C$-ideal generado por $S$.
    \item
    Probamos que, si $S$ es un $C_f$-ideal, entonces
    $\cal DS$ también lo es.
    Por lo tanto, $\cal D(FS)$ es un $C_f$ ideal que
    contiene a $S$ y está contenido en el $C$-ideal $j(S)$
    generado por $S$.
    \[
        S\subseteq \cal D(FS) \subseteq j(S)
    .\]
    Sin embargo, la última contención puede ser propia.
    \item
    Para saltar de $\cal D(FS)$ a $j(S)$,
    iteramos $\cal D$ para construir una
    cadena de $C_f$-ideales que contienen a $D(FS)$
    \[
        FS\subseteq \cal D(FS) \subseteq \cal D^2(FS)
        \subseteq \dots
    \]
    y probamos que, eventualmente, se alcanza $j(S)$.
    \item
    Con nuestra construcción de $j(S)$ veremos
    que, dada una familia $P\subseteq C\Idl(A)$,
    la igualdad
    $j(\bigcup P)=A$ implica $F(\bigcup P)=A$.
    En otras palabras,
    $\Sup P = 1_{C\Idl(A)}$ implica $1\in F(\bigcup P)$.
    \item
    Con esta última herramienta, podremos demostrar
    la implicación $\impliedby$ del teorema.
    \item
    La implicación $\implies$ es directa.
\end{itemize}

\subsection{La cobertura de cubiertas finitas}

\begin{lemma}
La función $C_f:A\to\cal P(\cal P(A))$
definida para todo $a\in A$ como
\[C_f(a)=\{S\subseteq A\mid S\in C(a), S\textit{ finito}\}\]
es una cobertura sobre $A$.
\end{lemma}

Nuestra primera tarea será obtener una construcción para
el núcleo asociado al cociente $\cal LA\to C_f\Idl$.
Para esto, introduciremos algo de notación.

\begin{defn}
Sean $\lambda_1,\dots,\lambda_n\in \scr I$ índices distintos y
$x_1,\dots,x_n$ elementos tales que
$x_i\in A_{\lambda_i}$ para todo $i$.
En otras palabras, $x=(x_1,\dots,x_n)$ es un elemento del producto
$A_{\lambda_1}\times\dots\times A_{\lambda_n}$.

Para cualquier elemento $a\in A$ del coproducto de los $A_\lambda$
en $\Pos^\inf$, denotaremos como
$a(x)=a(x_1,\dots,x_n)$ al elemento de $A$ que
es igual a $a$ pero con las entradas $\lambda_1,\dots,\lambda_n$
reemplazadas por $x_1,\dots,x_n$.
Es decir,
\[
    p_\lambda(a(x_1,\dots,x_n))
    =
    a(x_1,\dots,x_n)_\lambda
    =
    \begin{cases}
        x_i, & \lambda = \lambda_i \\
        a_\lambda & \lambda\not\in\{\lambda_1,\dots\lambda_n\}
    \end{cases}
.\]
En particular, si $x_i\in A_{\lambda_i}$, se cumple que
\begin{align*}
  1(x_i)
    &= q_{\lambda_i}(x_i) \\
  1(x_1,\dots,x_n)
    &= \Inf\{q_{\lambda_i}(x_i) \mid i=1,\dots,n\}.
\end{align*}
%
%Estrictamente hablando,
%habría que denotar no solo los elementos $x_1,\dots,x_n$
%que van a reemplazar las
%entradas de $a$, sino también exactamente en qué entradas se
%está haciendo la sustitución
%(ya que podría darse el caso que $A_\lambda=A_\mu$ aún cuando
%$\lambda\neq\mu$).
%Sin embargo, no lo haremos así, pues todas las
%sustituciones se harán en un contexto que permita saber
%el conjunto exacto de índices.
\end{defn}

\begin{defn}
  Dado un conjunto $S\subseteq A$, defino el conjunto $FS$
  especifivando que $a\in FS$ si, y solo si,
  existe un conjunto finito no vacío de índices
  $\Gamma=\{\lambda_1,\dots,\lambda_n\}$ que contiene al
  soporte de $a$ y conjuntos $S_1,\dots,S_n$ con
  $S_i\subseteq A_{\lambda_i}$ tales que $\Sup S_i =
  a_{\lambda_i}$ y, para toda tupla
  $(s_1,\dots,s_n)\in S_1\times\cdots\times S_n$, el elemento
  \[
    a(s_1,\dots,s_n)
  \]
  está en $S$.
  En tal caso, decimos que $a$ es $\Gamma$-generado por $S$ y
  que $(\Gamma,S_1,\dots,S_n)$ es el testigo de $a\in FS$,
  ya que $(\Gamma,S_1,\dots,S_n)$ atestigua (o prueba) que
  $a$ pertenece a $FS$.
\end{defn}

\subsection{Si \tps{$S$} es sección inferior, \tps{$FS$} también.}

Sea $S$ una sección inferior, $a\in FS$ y $b\leq a$.

Sea $a\in FS$ y sea $(\Gamma,S_1,\dots,S_n)$ su testigo.
Sin perder generalidad, podemos suponer que $\Gamma$ también
contiene al soporte de $b$.
En efecto, si $b_\gamma\neq 1_\gamma$ y $\gamma\not\in\Gamma$,
podemos agregar $\gamma$ a $\Gamma$ y poner
$S_\gamma = \{1_\gamma\}$,
de modo que $a_\gamma = 1_\gamma = \Sup S_\gamma$
y, para todo
$(s_1,\dots,s_n,1_\gamma)\in\prod_{i=1}^nS_i\times S_\gamma$
se tiene
\[
  1(s_1,\dots,s_n,1_\gamma) = 1(s_1,\dots,s_n) \in S
.\]

Supongamos, pues, que $b$ también tiene soporte contenido en
$\Gamma$.
Entonces haciendo $T_i=\{b_{\lambda_i}\inf s \mid s\in S_i\}$
para todo $i=1,\dots,n$, tenemos que
\begin{align*}
  \Sup T_i
  &= b_{\lambda_i}\inf\Sup S_i \\
  &= b_{\lambda_i}\inf a_{\lambda_i} \\
  &= b_{\lambda_i}.
\end{align*}

Más aún, dado $(b_{\lambda_1}\inf s_1,\dots,b_{\lambda_n}\inf
s_n)\in \prod_{i=1}^n T_\lambda$, tenemos
\begin{align*}
  1(b_{\lambda_1}\inf s_1,\dots,b_{\lambda_n}\inf s_n)
  \leq 1(s_1,\dots,s_n) \in S.
\end{align*}
y, como $S$ es sección inferior, tenemos
\[
  1(b_{\lambda_1}\inf s_1,\dots,b_{\lambda_n}\inf s_n) \in S
.\]
Se sigue que $b$ es finitamente generado por $S$ con testigo
$(\Gamma,T_1,\dots,T_n)$, es decir: $b\in FS$,
así que $FS$ es sección inferior.

\subsection{Si \tps{$S$} es sección inferior, \tps{$FS$} está contenido en
todos los \tps{$C_f$}-ideales que contienen a \tps{$S$}.}

Sea $J$ un $C_f$-ideal que contiene a $S$.
Tomemos un elemento $a\in FS$ con testigo $(\Gamma,S)$, donde
$\Gamma=\{\lambda_1,\dots,\lambda_n\}$.

Probaremos, por inducción sobre $k\leq n$, que, para cada tupla
$(s_k,\dots,s_n)\in S_k\times\cdots\times S_n$, el elemento
$a(s_k,\dots,s_n)$ está en $J$.

Para toda tupla
$(s_1,\dots,s_n)\in S_1\times\cdots\times S_n$, tenemos que
\[
  a(s_1,\dots,s_n)\in S\subseteq J
\]
ya que $a\in FS$.
Esto prueba el caso base ($k=1$).

La hipótesis de inducción ($k$) nos dice que, para cada tupla
$(s_k,\dots,s_n)\in S_k\times\cdots\times S_n$,
el elemento $a(s_k,\dots,s_n)$ está en $J$.

Ahora (el paso de inducción)
sea $(s_{k+1},\dots,s_n)\in S_{k+1}\times\cdots\times S_n$
una tupla arbitraria y consideremos el conjunto
\[
  \{a(s,s_{k+1},\dots,s_n) \mid s\in S_k\}
  =
  S_k(a(s_{k+1},\dots,s_n),\lambda_k)
  \in
  C_f(a(s_{k+1},\dots,s_n))
.\]
Esta es, en efecto, una cubierta, ya que la $\lambda_k$-ésima
coordenada de $a(s_{k+1},\dots,s_n)$ es $a_{\lambda_k}=\Sup S_k$.
Más aún: por hipótesis de inducción, cada elemento
de la cubierta está en $J$.
Como $J$ es un $C_f$-ideal, se sigue que
\[
  a(s_{k+1},\dots,a_n) \in J
.\]
Esto concluye la inducción para $k\leq n$.
En particular, tenemos $a(s_n)\in J$ para cada $s_n\in S_n$,
pero esta es una $C_f$-cubierta de $a$ contenida en $J$,
así que $a\in J$.
Por lo tanto, $FS\subseteq J$.

\subsection{Si \tps{$S$} es sección inferior, \tps{$FS$} es cerrado bajo
cubiertas y, por lo tanto, es un \tps{$C_f$}-ideal.}

Si $a\in A$ tiene una cubierta vacía, entonces tiene una entrada
cero, digamos $a_{\lambda_1}=0$.
Sea $\lambda_2,\dots,\lambda_n$ el resto de su soporte.
Poniendo $S_1=\{\}$ y $S_i=\{a_{\lambda_i}\}$ para $i=2,\dots,n$,
obtenemos $\Sup S_i = a_{\lambda_i}$.
Como $S_1\times S_2\times\cdots\times S_n=\emptyset$,
se cumple que $a\in FS$ por vacuidad.

Luego, basta considerar cubiertas de dos elementos.

Sean $a,b\in FS$ con $a_\gamma=b_\gamma$ para todo
$\gamma\neq\gamma_0$.
Debemos mostrar que $a\sup b\in FS$.

Como antes, podemos suponer que tanto $a$ como $b$ tienen soporte
en el mismo $\Gamma=\{\gamma_0,\gamma_1,\dots,\gamma_n\}$.

Por definición, tenemos $F_\gamma,G_\gamma$ tales que $\Sup
F_\gamma = a_\gamma$, $\Sup G_\gamma = b_\gamma$ para todo
$\gamma\in\Gamma$ y
\begin{align*}
  1(f_0,\dots,f_n)\in S && 1(g_0,\dots,g_n)\in S.
\end{align*}
siempre que $f_i\in F_{\gamma_i}$ y $g_i\in G_{\gamma_i}$.

Definamos $H_{\gamma_0}=F_{\gamma_0}\cup G_{\gamma_0}$ y
$H_\gamma = \{f\inf g\mid f\in F_\gamma,g\in G_\gamma\}$ para
todo $\gamma\in\Gamma,\gamma\neq\gamma_0$.
Entonces
\begin{align*}
  \Sup H_{\gamma_0}
  &= \Sup(F_{\gamma_0}\cup G_{\gamma_0}) \\
  &= \Sup F_{\gamma_0} \sup \Sup G_{\gamma_0} \\
  &= a_{\gamma_0} \sup b_{\gamma_0}
  \\
  \Sup H_\gamma
  &= \Sup\{f\inf g \mid f\in F_\gamma, g\in G_\gamma\} \\
  &= \Sup F_\gamma \inf \Sup G_\gamma \\
  &= a_\gamma \inf b_\gamma \\
  &= a_\gamma\sup b_\gamma
  \text{ para } \gamma\neq\gamma_0,
\end{align*}
lo cual son las entradas de $a\sup b$.

Ahora, las tuplas $x=\prod_{i=0}^n H_{\gamma_i}$ son de
dos formas posibles.
\begin{itemize}
  \item Si $x=(f_0,f_1\inf g_1,\dots,f_n\inf g_n)$,
  entonces
  \begin{align*}
    1(f_0,f_1\inf g_1,\dots,f_n\inf g_n)
    &\leq 1(f_0,f_1,\dots,f_n) \in S.
  \end{align*}
  \item de otro modo, $x=(g_0,f_1\inf g_1,\dots,f_n\inf g_n)$,
  así que
  \begin{align*}
    1(g_0,f_1\inf g_1,\dots,f_n\inf g_n)
    &\leq 1(g_0,g_1,\dots,g_n)\in S.
  \end{align*}
\end{itemize}
Como $S$ es sección inferior, se sigue que $1(x)$ siempre está en
$S$, así que $a\sup b$ está en $FS$.

\subsection{El conjunto de supremos dirigidos \tps{$\D S$} está contenido en todos los \tps{$C$}-ideales que contienen a \tps{$S$}.}

Sea $S$ una sección inferior y $\D S$ su conjunto de supremos
dirigidos.

Sea $J$ un $C$-ideal que contiene a $S$.
Como todo $a\in\D S$ es el supremo de un $D\subseteq S$ dirigido
y $D\subseteq S\subseteq J$, basta ver que $J$ es cerrado bajo
supremos dirigidos.

Tomemos un subconjunto dirigido $D\subseteq J$.
Como $D$ es dirigido, para cualquier $d\in D$ tenemos
\[
  \Sup D = \Sup(D\cap{\uparrow}d)
.\]
($\geq$ es obvia, para $\leq$ basta observar que todo $a\in D$
está por debajo de un $c\in D$ con $\geq a\sup d$).
Luego, podemos suponer que $D$ está contenido en una sección
superior principal.
En particular podemos suponer que todos los elementos de $D$
tienen soporte en $\Gamma=\{\lambda_1,\dots,\lambda_n\}$.

Sea $a=\Sup D$.
Como $D$ es dirigido, entonces sus proyecciones $D_1,\dots,D_n$ a
los marcos $A_{\lambda_1},\dots,A_{\lambda_2}$ son dirigidas
(todos los $x_1,y_1\in D_1$ vienen de elementos $x,y\in D$;
luego, existe $z\in D$ con $x,y\leq z$, por lo cual $x_1,y_1\leq
z_1$).

Por definición del supremo, tenemos
\[
  a = 1(\Sup D_1,\dots,\Sup D_n)
.\]

Por inducción sobre $k\leq n$, probaremos que, para toda tupla
$(d_k,\dots,d_n)\in D_k\times\cdots\times D_n$, el elemento
\[
  a(d_k,\dots,d_n)
  =
  1(\Sup D_1,\dots,\Sup D_{k+1},d_k,\dots,d_n)
\]
está en $J$.

Para cualquier tupla $(d_1,\dots,d_n)\in D_1\times\cdots\times
D_n$, cada $d_i$ es la $\lambda_i$-ésima
proyección de algún $x_i\in D$.
Como $D$ es dirigido, existe $z\in D$ con $x_1,\dots,x_n\leq z$.
Luego,
\[
  1(d_1,\dots,d_n)\leq z\in D \subseteq J
.\]
Como $J$ es sección inferior, tenemos $1(d_1,\dots,d_n)\in J$.
Esto prueba el caso base ($k=1$).

La hipótesis de inducción ($k$) dice que, para cada tupla
$(d_k,\dots,d_n)\in D_k\times\cdots\times D_n$, el elemento
$a(d_k,\dots,d_n)$ está en $J$.

Ahora (paso de inducción) sea
$(d_{k+1},\dots,d_n)\in D_{k+1}\times\cdots\times D_n$ una tupla
arbitraria y consideremos el conjunto
\
\[
  \{a(d,d_{k+1},\dots,d_n) \mid d\in D_k\}
  =
  D_k(a(d_{k+1},\dots,d_n),\lambda_k)
  \in
  C(a(d_{k+1},\dots,d_n))
.\]
Esta es, en efecto, una cubierta, ya que la $\lambda_k$-ésima
coordenada de $a(d_{k+1},\dots,d_n)$ es $\Sup D_k$.
Más aún: por hipótesis de inducción, cada elemento de la cubierta
está en $J$.
Como $J$ es un $C$-ideal, se sigue que
$a(d_{k+1},\dots,d_n)\in J$.
Esto concluye la inducción.
En particular, para $k=n$, mostramos que $a(d_n)\in J$ para todo
$d_n\in D_n$, pero este conjunto es una $C$-cubierta de $a$.
Se sigue que $a\in J$, así que $J$ es cerrado bajo supremos
dirigidos.
Por lo tanto, $\D S\subseteq J$.

\subsection{Si la sección inferior \tps{$S$} es un \tps{$C_f$}-ideal,
entonces \tps{$\D S$} también.}

Dado $a\in S$, $\{a\}\subseteq S$ es dirigido, así que
$a=\Sup\{a\}\in \D S$, por lo que $S\subseteq\D S$.

Primero veamos que $\D S$ es sección inferior.
Sean $a\in \D S$ y $b\leq a$.
Podemos suponer que $a$ y $b$ tienen soporte en $\Gamma$.
Tenemos $a=\Sup D$ con $D\subseteq S$ dirigido.
Es claro que
\[
  b = \Sup\{b\inf d\mid d\in D\}
.\]
Como $S$ es sección inferior, este último conjunto esta en $S$,
así que basta probar que es dirigido.
Sean $b\in d,b\inf d'$ con $d,d'\in D$.
Entonces existe $d''\in D$ con $d\sup d'\leq d''$, por lo
cual $b\inf d''$ cumple
$(b\inf d)\sup(b\inf d')=b\inf(d\sup d')\leq b\inf d''$ y, así,
el conjunto es dirigido.
Luego $b\in\D S$ y $\D S$ es sección inferior.

Ahora veamos que $\D S$ es cerrado bajo $C_f$-cubiertas.
Si $a\in A$ tiene una cubierta vacía, tenemos $a\in S$, así que
$a\in\D S$ (pues $S$ es $C_f$-ideal),
por lo cual basta ver que $\D S$ es cerrado
bajo $C_f$-cubiertas de dos elementos.
Sean $a,b\in\D S$ con todas sus entradas iguales excepto
$\gamma_0$.
Por definición, existen $D,E\subseteq S$ dirigidos con
$a=\Sup D$ y $b=\Sup E$.

Definimos $*:D\times E\to A$ como
\begin{align*}
  (d*e)_{\gamma_0} &= d_{\gamma_0} \sup e_{\gamma_0} \\
  (d*e)_{\gamma} &= d_{\gamma} \inf e_{\gamma} \text{ para }
  \gamma\neq\gamma_0
\end{align*}
y sea $F=\{d*e \mid d\in D, e\in E\}$.

Ahora veamos que $F$ es dirigido.
Si $d*e,d'*e'\in F$, entonces existen $d''\in D$ y $e''\in E$ con
$d\sup d'\leq d''$ y $e\sup e'\leq e''$.
Luego, $(d*e)\sup(d'*e')\leq d''*e''$, pues la comparación es
puntual y $*$ es monótono en cada coordenada; en efecto:
\begin{align*}
  ((d*e)\sup(d'*e'))_{\gamma_0}
  &= (d*e)_{\gamma_0} \sup (d'*e')_{\gamma_0} \\
  &= d_{\gamma_0}\sup e_{\gamma_0}\sup d'_{\gamma_0}\sup
    e'_{\gamma_0} \\
  &= (d_{\gamma_0}\sup d'_{\gamma_0})
    \sup(e_{\gamma_0}\sup e'_{\gamma_0}) \\
  &\leq d''_{\gamma_0} \sup e''_{\gamma_0}
  \\
  ((d*e)\sup(d'*e'))_{\gamma}
  &= (d*e)_{\gamma} \sup (d'*e')_{\gamma} \\
  &= (d_\gamma\inf e_\gamma)\sup(d'_\gamma\inf e'_\gamma) \\
  &\leq d_\gamma \sup e'_\gamma \\
  &\leq d''_\gamma \sup e''_\gamma.
\end{align*}
Se sigue que $F$ es dirigido.
Mas aún, $\Sup F = a\sup b$, pues
\begin{align*}
  (\Sup F)_{\gamma_0}
  &= \Sup\{(d*e)_{\gamma_0} \mid d\in D,e\in E\} \\
  &= \Sup\{d_{\gamma_0}\sup e_{\gamma_0} \mid d\in D,e\in E\} \\
  &= \Sup\{d_{\gamma_0}\mid d\in D\}
    \sup \Sup\{e_{\gamma_0}\mid e\in E\} \\
  &= (\Sup D)_{\gamma_0} \sup (\Sup E)_{\gamma_0} \\
  &= a_{\gamma_0} \sup b_{\gamma_0}
  \\
  (\Sup F)_\gamma
  &= \Sup\{(d*e)_\gamma\mid d\in D,e\in E\} \\
  &= \Sup\{d_\gamma\inf e_\gamma\mid d\in D,e\in E\} \\
  &= \Sup\{d_\gamma\mid d\in D\}
    \inf \Sup\{e_\gamma\mid e\in E\} \\
  &= (\Sup D)_\gamma \inf (\Sup E)_\gamma \\
  &= a_\gamma \inf b_\gamma \\
  &= a_\gamma \sup b_\gamma \text{ para }\gamma\neq\gamma_0,
\end{align*}
lo cual son las entradas de $a\sup b$.

Para mostrar que $a\sup b\in\D S$,
solo queda ver que $F\subseteq S$.
Para esto, usaremos que $S$ es $C_f$-ideal (nótese que no lo
habíamos usado excepto en un caso muy simple).

Sea $d*e\in F$; es decir: $d\in D\subseteq S$ y $e\in E\subseteq S$.
Queremos construir una $C_f$-cubierta de $d*e$ que esté contenida
en $S$.
Consideremos  $r,s\in A$ dados por
\begin{align*}
  r_\gamma = s_\gamma
    &= d_\gamma \inf e_\gamma \text{ para } \gamma\neq\gamma_0 \\
  r_{\gamma_0} &= d_\gamma \\
  s_{\gamma_0} &= e_\gamma.
\end{align*}
Observemos que $r\leq d\in D\subseteq S$ y que $s\leq e\in
E\subseteq S$.
Como $S$ es sección inferior, tenemos $r,s\in S$.
Finalmente, es claro que $r\sup s = d*e$ donde el supremo se
concentra en una entrada.
Es decir, $\{r,s\}\in C_f(d*e)$ y $\{r,s\}\subseteq S$.
Como $S$ es $C_f$-ideal, se sigue que $d*e\in S$.
Así, $F\subseteq S$, por lo cual $a\sup b\in\D S$.
Es decir, $\D S$ es cerrado bajo $C_f$-cubiertas, por lo cual
también es un $C_f$-ideal.

\subsection{Lema: saltar de \tps{$\D(FS)$} a \tps{$j(S)$}.}
    Sea $I$ un $C_f$-ideal de $A$.
    Como ya vimos, $\D I$ es un $C_f$-ideal contenido
    entre $I$ y el $C$-ideal $j(I)$ generado por $I$.
    Aplicando esto a $\D I$, obtenemos
    \[
        I\subseteq \D I
        \subseteq \D(\D I)
        \subseteq j(\D I)=j(I)
    .\]
    Poniendo $I=FS$ (el $C_f$-ideal generado por una sección
    inferior $S\subseteq A$) e iterando $\D$, obtenemos una
    cadena de $C_f$-ideales
    \[
        FS \subseteq \cal D(FS)
        \subseteq D^2(FS)
        \subseteq \dots
    \]
    contenidos en $j(FS)=j(S)$.


    La extensión de esta cadena a todos los ordinales
    \begin{align*}
        I_0 &= FS \\
        I_{\alpha+1} &= \D(I_\alpha) \\
        I_\lambda
        &= \bigcup\{I_\alpha \mid \alpha<\lambda\}
            && \text{ si $\lambda$ es límite }
    \end{align*}
    se estaciona en el $C$-ideal $j(S)$ generado por $S$. \\
    En efecto, sea $\gamma$ el primer ordinal donde la cadena
    se detiene.
    Entonces $I_\gamma=I_{\gamma+1}=\D(I_\alpha)$.
    Es decir, $I_\gamma$ es cerrado bajo supremos dirigidos.
    Como también es cerrado bajo supremos finitos
    (entrada por entrada), se sigue
    que es cerrado bajo supremos arbitrarios
    (entrada por entrada), así que es
    un $C$-ideal.
    Se sigue que $I_\gamma=j(S)$, como se afirmó.


\subsection{Lema: generar a \tps{$A$} como \tps{$C$}-ideal implica generarlo como \tps{$C_f$}-ideal}
    Sea $A_\lambda$ una familia de marcos compactos y $A$
    su coproducto como $\inf$-semiretículas equipado con
    los cubrientes $C$ y $C_f$.
    
    Si $P\subseteq C\Idl(A)$ es tal que $\Sup P=1_{C\Idl(A)}$,
    entonces $1\in F(\bigcup P)$.
    En otras palabras, si
    $j(\bigcup P)=A$, entonces $F(\bigcup P)=A$.
    \\
    %\pause
    \textbf{\emph{Demostración:}} \\
    Supongamos que $A=j(\bigcup P)$.
    De la construcción anterior, tenemos
    $I_\gamma=j(\bigcup P)$ (donde $\gamma$ es el primer
    ordinal donde se detiene la cadena de iteraciones
    $\D^\alpha(\bigcup P)$),
    así que $1\in I_\gamma$. \\
    Afirmamos que $\gamma$ no puede ser ordinal límite.
    En efecto, si lo fuera, tendríamos
    \[
        1 \in I_\gamma = \bigcup\{I_\alpha \mid \alpha<\gamma\}
    ,\]
    por lo cual $1\in I_\alpha$ para algún $\alpha<\gamma$, lo
    cual no puede suceder por la minimalidad de $\gamma$.

    Más aún, $\gamma$ no puede ser sucesor.
    Si fuera el caso que $\gamma=\beta+1$, tendríamos
    $1\in I_\gamma = \D(I_\beta)$;
    es decir: $1=\Sup D$ para algún conjunto dirigido
    $D\subseteq I_\beta$.
    Dado que $D$ es dirigido, podemos tomar cualquier
    $d\in D$ y obtener
    \[
        1
        = \Sup D
        = \Sup\{a\in D\mid d\leq a\}
        = \Sup(D\cap{\uparrow}d)
    ,\]
    lo cual nos dice que,
    reemplazando a $D$ por $D\cap{\uparrow}d$ en
    caso de ser necesario,
    podemos suponer que $D$ está contenido en una
    sección superior principal.
    En particular, podemos suponer que todos los elementos
    de $D$ tienen soporte contenido en
    un conjunto finito de índices $\Gamma$.

    Para cada $\lambda\in\Gamma$, sea $D_\lambda=p_\lambda(D)$.
    Como $1=\Sup D$, tenemos $1_\lambda=\Sup D_\lambda$ y,
    por compacidad de los $A_\lambda$, esto nos da familias finitas
    $E_\lambda\subseteq D_\lambda$ tales que
    $1_\lambda=\Sup E_\lambda$.
    
    Luego, todos los elementos de todos los $E_\lambda$ aparecen
    como entradas de los elementos de un conjunto finito
    $E\subseteq D$.
    Como $D$ es dirigido y $E$ es finito,
    existe $a\in D$ tal que $\Sup E\leq a$,
    pero $\Sup E=1$, así que
    \[
        1 = a \in D\subseteq I_\beta
    ,\]
    lo cual contradice la minimalidad de $\gamma$.
    
    Se sigue que $\gamma=0$.
    Esto es, $A=j(\Sup P) = I_0 = F(\bigcup P)$, como se deseaba.

\subsection{Una implicación (Tychonoff)}
    Con toda esta herramienta, podemos demostrar el teorema
    de Tychonoff.
    Si $A_\lambda$ es una familia de marcos compactos y $A$
    es su coproducto como $\inf$-semiretículas,
    debemos mostrar que $C\Idl(A)$ es compacto.
    Sea $P\subseteq C\Idl(A)$ una familia de $C$-ideales tal que
    $\Sup P = 1_{C\Idl(A)}$.
    En otras palabras: $\bigcup P$ genera a $A$ como $C$-ideal.
    \[
        j(\bigcup P) = A
    .\]
    
    Como acabamos de probar, esto implca que $\bigcup P$ también
    genera a $A$ como $C_f$-ideal:
    \[
        F(\bigcup P)=A
    .\]
    En particular, $1\in F(\bigcup P)$; esto es:
    existe algún conjunto finito no vacío de índices $\Gamma$
    tal que $1\in A$ es $\Gamma$-finitamente generado por
    $\bigcup P$.

    Es decir: para cada $\lambda\in\Gamma$ existe un conjunto
    finito $S_\lambda\subseteq A_\lambda$
    con $\Sup S_\lambda = 1_\lambda$
    y, siempre que se tenga una tupla
    $x=(x_\lambda)_{\lambda\in\Gamma}$
    con $x_\lambda\in A_\lambda$,
    se cumple
    \[
        1(x)
        =\Inf\{q_\lambda(x_\lambda)\mid\lambda\in\Gamma\}
        \in\bigcup P
    .\]
    Ahora, como $\Gamma$ es finito y cada $S_\lambda$
    también, solo se puede formar
    una cantidad finta de tuplas
    $x=(x_\lambda)_{\lambda\in\Gamma}$.
    Así, el conjunto de los $1(x)$ es finito y, por lo tanto,
    está contenido en una cantidad finita de factores
    $P_1,\dots,P_n\in P$ de $\bigcup P$.
    
    Luego, $1\in A$ es $\Gamma$-finitamente generado
    por $\bigcup_{i=1}^nP_i$.
    Esto es $F(\bigcup_{i=1}^n P_i)=A$ y, así,
    \[
        \Sup\{P_1,\dots,P_n\}
        = j(\bigcup_{i=1}^nP_i)
        = A
        = 1_{C\Idl(A)}
    .\]
    Luego, $C\Idl(A)$ es compacto.

\subsection{La otra implicación}

Necesitamos este pequeño resultado.
\begin{lemma}
Dada un elemento $x\in A_\lambda$, entonces
\[
    j(\down q_\lambda(x)) = \down q_\lambda(x)
.\]
\end{lemma}
\begin{proof}
    En efecto, sea $a\in A$ y $S(a,\mu)\subseteq\down q_\lambda(x)$
    una $C$-cubierta de $a$.
    Queremos mostrar que $a\leq q_\lambda(x)$.
    Si $\mu\neq\lambda$, entonces $S(a,\mu)\subseteq \down q_\lambda(x)$
    nos dice que $a_\lambda\leq x$, así que $a\leq q_\lambda(x)$.
    Por otro lado, si $\mu=\lambda$, entonces
    $S(a,\lambda) \subseteq \down q_\lambda(x)$ nos dice que, para todo
    $s\in S$ tenemos $s\leq x$.
    Por lo tanto, $a_\lambda=\Sup S\leq x$, así que $a\leq q_\lambda(x)$.
    Se sigue que $\down q_\lambda(x)$ es un $C$-ideal.
\end{proof}
Ahora sí.

\begin{lemma}
    Si el coproducto $C\Idl(A)$ de una familia de marcos
    $A_\lambda$ es compacto, entonces cada $A_\lambda$
    también es compacto.
\end{lemma}
\begin{proof}
    Supongamos que $C\Idl$ es compacto.
    Si $S\subseteq A_\lambda$ es tal que $\Sup S=1_\lambda$,
    entonces $S(\lambda,1)\in C(1)$.
    
    Para cada $s\in S$, sea $Q_\lambda(s)
    = j(\down q_\lambda(s))$
    el $C$-ideal generado por $s$ bajo $q_\lambda$.
    Luego,
    \[
        \Sup\{Q_\lambda(s) \mid s\in S\} = 1_{C\Idl(A)} = A
    .\]
    Por la compacidad de $C\Idl(A)$,
    existen $s_1,\dots,s_n\in S$ tales que
    \[
        \Sup\{Q_\lambda(s_i) \mid i=1,\dots,n\} = A
    .\]
    Como $Q_\lambda$ es morfismo de marcos, esto es
    \[
        Q_\lambda(\Sup\{s_i \mid i=1,\dots,n\}) = A
    .\]
    
    Ahora, dado que $Q_\lambda(x) = \down q_\lambda(x)$, tenemos
    \[
        1 \in \down q_\lambda(\Sup\{s_i \mid i=1,\dots,n\})
    ,\]
    o bien
    \[
        1 = q_\lambda(\Sup\{s_i \mid i=1,\dots,n\})
    .\]
    Proyectando a la $\lambda$-ésima coordenada, obtenemos
    \[
        1_\lambda = \Sup\{s_i \mid i=1,\dots,n\}
    .\]
    Se sigue que $A_\lambda$ es compacto.
\end{proof}

\end{document}
