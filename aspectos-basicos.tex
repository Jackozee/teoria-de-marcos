\chapter{Aspectos básicos de la teoría de marcos}

\section{Copos}
Sean $A,B$ dos conjuntos parcialmente ordenados
(copos, para abreviar).
Una función $f:A\to B$ es un morfismo de copos si,
siempre que $a\leq b\in A$ se tiene $f(a)\leq f(b)$.
En otras palabras, $f:A\to B$ es un morfismo de copos
si, cuando vemos a $A$ y a $B$ como categorías, $f$ es un funtor.
A los morfismos de copos también se les llama funciones monótonas.

Si $A$ es un copo y $X\subseteq A$ es cualquier subconjunto,
un elemento $a\in A$ se llama cota superior de $X$ (en $A$)
si $x\leq a$ para todo $x\in X$.
Similarmente, si $a\leq x$ para todo $x\in X$, entonces decimos
que $a$ es una cota inferior de $X$. \\
Si $x\in X$ es una cota inferior de $X$,
entonces $a$ es único con esta propiedad y decimos que es el
menor elemento de $X$. Similarmente, $a\in A$ es el menor
elemento de $X$ si $x\in X$ es una cota superior de $X$. \\
Si el conjunto de cotas superiores de $X$ tiene un menor
elemento, este elemento se llama supremo de $X$ y lo denotamos
como $\Sup X$.
Similarmente, si el conjunto de cotas inferiores de $X$ tiene un
mayor elemento, este elemento se llama ínfimo de $X$ y lo
denotamos $\Inf X$.

Dado que los morfismos de copos (funciones monótonas)
son cerrados bajo composición y la función identidad
de cualquier copo es un morfismo,
éstos forman una categoría, a la cual denotamos $\Pos$.

\section{Semiretículas}
\label{ss:semireticulas}
Decimos que un copo es una semiretícula superior
(o $\sup$-semiretícula) si todos sus subconjuntos finito tienen
supremo.
Equivalentemente, un copo $A$ es una $\sup$-semiretícula si
todo par de elementos $a,b$ tiene supremo $a\sup b$
y $A$ tiene un menor elemento $0\in A$ (que es
el supremo del cojunto vacío).

Como $0\leq a$ para todo $a\in A$, entonces $a\sup 0=a$.
Además, $a\sup b=b\sup a$, $(a\sup b)\sup c=a\sup(b\sup c)$
y $a\sup a=a$ para cualesquiera $a,b,c\in A$.
Por lo tanto, el conjunto $A$ equipado con el supremo, pensado
como operación binaria en $A$, es un monoide
conmutativo donde todo elemento es idempotente
(el elemento neutro es el $0\in A$).

Recíprocamente, si $(A,\sup,0)$ es un monoide conmutativo
en el cual todo elemento es idempotente,
entonces la relación definida como
\[
  a\leq b \ssi a\sup b=b
\]
es un orden parcial en $A$ tal que el supremo es $\sup$.
%[Alfredo $\checkmark$ ]
\begin{proof}
    En efecto, esto es un orden parcial:
    \begin{itemize}
        \item (Refl). Como $a$ es idempotente, tenemos $a\sup a=a$.
        Luego, $a\leq a$.
        \item (Antisim). Supongamos que $a\leq b$ y $b\leq a$.
        Es decir, $a\sup b=b$ y $b\sup a=a$.
        Como $\sup$ es conmutativo, tenemos
        \[
            b = a\sup b = b\sup a = a
        .\]
        \item (Trans). Supongamos que $a\leq b$ y que $b\leq c$.
        Es decir, $a\sup b = b$ y $b\sup c = c$.
        Como $\sup$ es asociativo, tenemos
        \[
            a\sup c = a\sup(b\sup c) = (a\sup b)\sup c = b\sup c = c
        .\]
        Esto es, $a\leq c$.
    \end{itemize}
    Ahora mostraremos que $\sup$ es el supremo de este orden.
    Por inducción, basta mostrarlo en el vacío y en
    pares de elementos.
    Como $0\sup a=a$, entonces $0\leq a$ para todo $a\in A$.
    Ahora sean $a,b\in A$, y supongamos que $c\in A$ es tal que
    $a,b\leq c$.
    Esto es, $a\sup c = c$ y $b\sup c = c$.
    Luego,
    \[
        (a\sup b)\sup c = (a\sup c)\sup(b\sup c) = c\sup c = c
    .\]
    Por lo tanto, $a\sup b\leq c$.
\end{proof}

Si $A,B$ son $\sup$-semiretículas,
decimos que una función $f:A\to B$ es
un $\sup$-morfismo (o un morfismo de $\sup$-semiretículas)
si $f(0)=0$ y si $f(a\sup b)=f(a)\sup f(b)$.
De nuevo, las $\sup$-semiretículas con sus morfismos forman una
categoría, a la cual denotamos como $\Pos^\sup$.

Dado que $a\leq b$ si, y solo si, $a\sup b=a$, entonces
cualquier morfismo de retículas es una función monótona,
pues $a\sup b=a$ implica $f(a)\sup f(b)=f(a)$.
Así, tenemos un funtor $\Pos^\sup\to\Pos$. Sin embargo, este
funtor no es pleno, pues existen funciones monótonas que no
preservan el supremo.

Las definiciones y observaciones de esta sección se pueden hacer
de manera análoga usando ínfimos en vez de supremos, obteniendo
la noción de semiretículas inferiores, $\inf$-morfismos, la
categoría $\Pos^{\inf}$ y un funtor $\Pos^{\inf}\to\Pos$.

\section{Retículas}

Decimos que un copo $A$ es una retícula si cualquier
subconjunto finito $X\subseteq A$ tiene supremo e ínfimo.
Equivalentemente, un copo $A$ es una retícula si, y solo si:
\begin{enumerate}
    \item cualquier par de elementos $a,b\in A$ tiene
    supremo $a\sup b=\Sup\{a,b\}\in A$ e ínfimo $a\inf
    b=\Inf\{a,b\}\in A$,
    \item
    $A$ tiene un menor elemento $0$ y un mayor elemento $1$.
    Éstos son el supremo y el ínfimo del subconjunto vacío
    $\emptyset\subseteq A$, respectivamente.
\end{enumerate}
Algunos autores denominan retículas a los copos que
cumplen el punto 1 aunque no cumplan el punto 2.
Con esa convención, lo que nosotros llamamos retícula se llama
retícula acotada.
Nótese también que un subconjunto arbitrario de una retícula $A$
puede no tener supremo o ínfimo.

Sean $a,b,c$ elementos de una retícula $A$.
Por definición del supremo, siempre tenemos
$a\leq a\sup b$ y $a\leq a\sup c$.
Es decir, $a$ es cota inferior de $\{(a\sup b),(a\sup c)\}$.
Así, $a\leq (a\sup b)\inf(a\sup c)$.
Además,
\begin{align*}
  b\inf c &\leq b\leq a\sup b \\
  b\inf c &\leq c\leq a\sup c.
\end{align*}
Por lo tanto, $b\inf c$ también es cota inferior de $\{(a\sup
b),(a\sup c)\}$, por lo cual $b\inf c\leq(a\sup b)\inf(a\sup c)$.
Esto muestra que $(a\sup b)\inf(a\sup c)$ es cota superior de
$\{a,(b\inf c)\}$. Se sigue que
\[
  a\sup(b\inf c)
  \leq (a\sup b)\inf(a\sup c)
.\]
Un argumento similar muestra que
\[
  a\inf(b\sup c)
  \geq (a\inf b)\sup(a\inf c)
.\]
Sin embargo, estas desigualdades no siempre son igualdades.
Un retícula es distributiva si las ecuaciones
\begin{align*}
  a\sup(b\inf c) &= (a\sup b)\inf (a\sup c) \\
  a\inf(b\sup c) &= (a\inf b)\sup(a\inf c)
\end{align*}
son válidas para cualesquiera $a,b,c\in A$.
La primera igualdad es la distributividad del supremo sobre el
ínfimo y la segunda es la distributividad del ínfimo sobre el
supremo.
De hecho, basta pedir una de las dos igualdades:
si en una retícula $A$ se cumple una de las ecuaciones para
cualesquiera $a,b,c\in A$, entonces la otra también se cumple.
\begin{proof}
    Sean $a,b,c\in A$.
    \begin{enumerate}
      \item
      Supongamos que el supremo distribuye sobre el ínfimo,
      entonces tenemos la primera y tercera igualdad en
      \begin{align*}
          (a\inf b)\sup(a\inf c)
          &= ((a\inf b)\sup a)\inf((a\inf b)\sup c) \\
          &= a\inf((a\inf b)\sup c) \\
          &= a\inf(a\sup c)\inf (b\sup c) \\
          &= a\inf (b\sup c).
      \end{align*}

      \item
      Por otro lado, si el ínfimo distribuye sobre el
      supremo, entonces
      \begin{align*}
          (a\sup b)\inf(a\sup c)
          &= ((a\sup b)\inf a)\sup((a\sup b)\inf c) \\
          &= a\sup((a\sup b)\inf c) \\
          &= a\sup (a\inf c)\sup (b\inf c) \\
          &= a\sup (b\inf c).
      \end{align*}
    \end{enumerate}
\end{proof}

Ahora definimos la categoría de retículas $\Lat$.
Sus objetos son las retículas y sus
morfismos son funciones que preservan el $1$, el $0$,
el ínfimo y el supremo: $f(0)=0$, $f(1)=1$,
$f(a\sup b)=f(a)\sup f(b)$ y $f(a\inf b)=f(a)\inf f(b)$.

De este modo, tenemos funtores $\Pos^{\inf}\from\Lat\to\Pos^{\sup}$.
Notemos que estos funtores son fieles, pero no son plenos,
ya que una función puede preservar supremos sin preservar ínfimos
o viceversa.

La categoría $\DLat$ se define como la subcategoría plena de
$\Lat$ cuyos objetos son las retículas distributivas.
Es decir, los morfismos en $\DLat$ son simplemente morfismos de
retículas. Así, tenemos una inclusión $\DLat\rmono\Lat$.

\section{Complementos y álgebras booleanas}
\label{ss:complementos-algebras-booleanas}

\begin{defn}[Complementos]
    Sea $A$ una retícula.
    Decimos que un elemento $a\in A$ tiene complementos
    si existe algún $b\in A$ tal que
    \[
        a\inf b = 0 \hspace{10mm} a\sup b = 1
    .\]
    En este caso, decimos que $b$ es un complemento de $a$.
\end{defn}
Si la retícula $A$ no es distributiva,
entonces un mismo elemento puede
tener más de un complemento.
Sin embargo, si $A$ es distributiva, un elemento $a\in A$ tiene,
a lo más, un complemento.
\begin{proof}%[Yareli $\checkmark$ ]
  En efecto, sean $a,b_1,b_2\in A$ tales que
  \begin{align*}
    a\wedge b_1&=0 & a\vee b_1&=1 \\
    a\wedge b_2&=0 & a\vee b_2&=1.
  \end{align*}
  Si $A$ es distributiva, entonces
  \[
    b_1
    =b_1\vee 0
    =b_1\vee (a\wedge b_2)
    =(b_1\vee a)\wedge (b_1\vee b_2)
    =1\wedge (b_1\vee b_2)
    =b_1\vee b_2
  \]
  por lo cual $b_2\leq b_1$.
  Similarmente,
  \[
    b_2
    =b_2\vee 0
    =b_2\vee (a \wedge b_1)
    =(b_2\vee a)\wedge (b_2\vee b_1)
    =1\wedge (b_2\vee b_1)
    =b_2\vee b_1
  \]
  por lo cual $b_1\leq b_2$.
  Así, $b_1=b_2$.
\end{proof}
Como no hay ambigüedad, en una retícula distributiva denotamos
al complemento de $a$ como $a'$, en caso de que éste exista.

\begin{defn}
  Un álgebra booleana es una retícula distributiva
  donde todo elemento tiene complemento (único, por
  distributividad).
\end{defn}

\begin{exa}
    Dado cualquier conjunto $S$, el conjunto potencia $\cal PS$
    es un álgebra booleana.
    El complemento de un subconjunto $X\subseteq S$ es el
    complemento en el sentido usual:
    \[
      X' = \{s\in S \mid s\not\in X\}
    .\]
\end{exa}

Notemos que cualquier morfismo de retículas si $f:A\to B$
preserva los complementos automáticamente, ya que, si $a\sup b=1$
y $a\inf b=0$, entonces
\begin{align*}
  f(a)\inf f(b)=f(a\inf b)=f(0)=0, \\
  f(a)\sup f(b)=f(a\sup b)=f(1)=1.
\end{align*} 
En particular, si $A$ y $B$ son álgebras booleanas, tenemos
$f(a')=f(a)'$. (En realidad, basta con que $B$ sea distributiva).
Entonces tiene sentido definir los morfismos de álgebras booleanas
simplemente como morfismos de retículas o, equivalentemente,
morfismos de retículas distributivas.
En otras palabras, definimos la categoría $\Bool$ de las álgebras
booleanas como
\[
  \Bool(A,B) = \DLat(A,B) = \Lat(A,B)
.\]

Así, la inclusión $\Bool\rmono\DLat$ exhibe a $\Bool$ como una
subcategoría plena de $\DLat$.
Hasta ahora, tenemos los siguientes funtores plenos entre
nuestras categorías
\[
    \begin{tikzcd}
        & \Pos \\
        \Pos^\wedge \arrow[ru]
            & \Lat \arrow[r] \arrow[l]
            & \Pos^\vee \arrow[lu] \\
        & \DLat \arrow[u,hook] \\
        & \Bool \arrow[u,hook]
    \end{tikzcd}
\]
donde los funtores marcados como $\rmono$ son fielmente plenos.

\section{Negaciones}
\label{ss:negaciones}

En \ref{ss:complementos-algebras-booleanas} dijimos que, dada una
retícula $A$, un complemento de $a\in A$ es otro elemento
$b\in A$ que cumple
\begin{align*}
  a\inf b &= 0 & a\sup b=1.
\end{align*}
Y vimos que, si $A$ es distributiva, entonces cada elemento
$a\in A$ tiene, a lo más, un complemento.
Ahora definiremos una noción relacionada: las negaciones.
\begin{defn}
  Si $A$ es una retícula, entonces una negación
  de $a\in A$ es un elemento $b\in A$ tal que
  \[
      x\leq b \ssi x\inf a=0
  \]
  para todo $x\in A$.
\end{defn}
Así, $b,b'$ son negaciones de $a$,
entonces $b\inf a=b'\inf a=0$ (porque $b\leq b$ y $b'\leq b'$).
Así, $b\leq b'$ y $b'\leq b$, por lo cual $b=b$.
Es decir, las negaciones son únicas en caso de existir, así que,
si un elemento $a\in A$ tiene negación,
ésta se denota como $\neg a$.
Nótese que, en este caso, para demostrar la unicidad
no se requirió que $A$ fuera distributiva.

Sin embargo, si $A$ es distributiva
y $a\in A$ tiene complemento $a'$ (único, por distributividad),
entonces $a'$ también es la negación de $a$, es decir, $\neg a=a'$.
En efecto, es claro que $x\leq a'\implies x\inf a=0$;
mientras que, si $x\inf a=0$, entonces
\begin{align*}
    x
    &= x\inf(a'\sup a) \\
    &= (x\inf a')\sup (x\inf a) \\
    &= x\inf a',
\end{align*}
por lo cual $x\leq a'$.

Juntando estas observaciones, obtenemos el siguiente resultado:
\begin{lemma}
  \label{lemma:complementado-ssi-supneg}
   Si $A$ es una retícula distributiva,
   un elemento $a\in A$ con negación tiene complemento $a'$
   si, y solo si, $\neg a\sup a=1$ y, en este caso, $a'=\neg a$.
\end{lemma}

En particular, en un álgebra booleana $A$, todo elemento $a\in A$
tiene negación dada por el complemento: $\neg a = a'$.

De la definición de la negación, tenemos que
$x\leq \neg\neg a$ si, y solo si, $x\inf\neg a=0$.
Dado que $a\inf\neg a=0$, podemos deducir que $a\leq\neg\neg a$.
Aunque en un álgebra booleana se tiene la otra comparación
(es decir, $\neg\neg a = a$),
esto no es cierto en general:

\begin{exa}
  \label{exa:negaciones-en-espacios-top}
  Sea $S$ un espacio topológico y consideremos el marco $\cal OS$.
  Todo abierto $u\in\cal O S$ tiene negación $\neg u$ en $\cal OS$.
  En efecto, para todo abierto $v\in\cal OS$ tenemos
  \begin{align*}
        u\cap v = \emptyset
        &\iff v\subseteq u' \\
        &\iff v\subseteq (u')^\circ \in\cal OS.
  \end{align*}
  Por lo tanto, $u$ tiene negación dada como
  $\neg u = (u')^\circ = {\ol u}'$.
  
  En particular, el abierto $u=(-1,0)\cup(0,1)\in\cal O\mathbb R$
  tiene cerradura $\ol u = [-1,1]$, así que
  \begin{align*}
    \neg\neg u
    &= \neg(\ol u ') \\
    &= \neg([-1,1]') \\
    &= ([-1,1]'')^\circ \\
    &= [-1,1]^\circ \\
    &= (-1,1) \\
    &\neq u.
  \end{align*}
\end{exa}

Es decir, aunque $\neg a$ tenga negación, en general
solo tenemos la comparación $a\leq\neg\neg a$.
Sin embargo, si $\neg\neg a$ tiene negación, entonces sí se
cumple que $\neg\neg\neg a=\neg a$.
En efecto, dado que
\[
  a \leq \neg\neg a
,\]
al hacer ínfimo con $\neg\neg a$ obtenemos
\begin{align*}
  \neg\neg\neg a\inf a
  &\leq \neg\neg\neg a\inf \neg\neg a  \\
  &= 0,
\end{align*}
lo cual, por definición de la negación, es equivalente a
$\neg\neg\neg a\leq\neg\neg a$, mientras que la otra comparación
$\neg a\leq\neg\neg\neg a$ ya la teníamos.

En el ejemplo \ref{exa:negaciones-en-espacios-top} encontramos
una retícula distributiva donde todo elemento tiene negación.
Ahora consideraremos ese tipo de retículas.

Por el lema \ref{lemma:complementado-ssi-supneg}, sabemos que
$a$ es tiene complemento $a'$ ssi $a\sup\neg a=1$.
Esto justifica el primer punto de la siguiente definición.
\begin{defn}
  Sea $A$ una retícula distributiva donde todo elemento tiene
  negación y tomemos un elemento $a\in A$.
  Decimos que
  \begin{itemize}
    \item $a$ es complementado si $a\sup\neg a=1$,
    \item $a$ es regular si $\neg\neg a = a$,
    \item $a$ es denso si $\neg a = 0$.
  \end{itemize}
\end{defn}

Nótese que $\neg a$ es regular para todo $a$, ya que
$\neg\neg\neg a =\neg a$.

\begin{exa}
  Consideremos un espacio topológico $S$ y un abierto $u\in\cal
  OS$. Por el ejemplo \ref{exa:negaciones-en-espacios-top},
  sabemos que $u$ tiene negación dada como $\neg u=\ol
  u'=(u')^\circ$.
  Por lo tanto
  \begin{enumerate}
    \item
    $u$ es complementado en el sentido de retículas
    ($u\sup\neg u=1$)
    si, y solo si, su complemento es abierto.
    \item
    el abierto $u$ es regular en el sentido de retículas
    ($\neg\neg u = u$)
    si, y solo si, es regular en el sentido topológico
    ($(\ol u)^\circ=u$).
    \item
    Similarmente, $u$ es denso como elemento de la retículas
    ($\neg u=0$)
    si, y solo si, es denso en el sentido topológico
    ($\ol u=S$).
  \end{enumerate}
\end{exa}

\begin{prop}%[Dante $\checkmark$ ]
  Un elemento $a$ de una retícula $A$ es denso
  ($\neg a=0$) si, y solo si $\neg\neg a = 1$.
\end{prop}
\begin{proof}
Tenemos
\begin{align*}
  \neg a = 0
  &\iff \neg a \leq 0 \\
  &\iff 1\inf \neg a \leq 0 \\
  &\iff 1\leq \neg\neg a \\
  &\iff 1=\neg\neg a.
\end{align*}
\end{proof}

%\section*{(SESIÓN 5: 23 SEP)}

\begin{prop}
  Si $A$ es una retícula distributiva y $a,b\in A$ tienen
  negación, entonces $a\sup b$ también tiene negación y tenemos
  \[
    \neg(a\sup b) = \neg a \inf \neg b
  .\]
\end{prop}
\begin{proof}
  Para cualquier $x\in A$, tenemos
  \begin{align*}
    x\inf(a\sup b) \leq 0
    &\iff (x\inf a)\sup(x\inf b) \leq 0 \\
    &\iff (x\inf a)\leq 0, \; (x\inf b) \leq 0 \\
    &\iff x\leq\neg a, \; x\leq\neg b \\
    &\iff x\leq(\neg a \inf \neg b).
  \end{align*}
  Así, $\neg a\inf\neg b$ es la negación de $a\sup b$, como se
  quería.
\end{proof}

Ahora, si en una retícula distrubutiva $A$ todos los elementos
tienen negación, entonces
\[
  \neg(a\sup\neg a) = \neg a\inf\neg\neg a = 0
,\]
por lo cual $a\sup\neg a$ siempre es denso.
Además
\begin{align*}
  \neg\neg a\inf(a\sup\neg a)
  &= (\neg\neg a\inf a)\sup(\neg\neg a\inf\neg a) \\
  &= a \inf 0 \\
  &= a,
\end{align*}
así que todo elemento $a\in A$ se puede expresar como el ínfimo
de un elemento denso y un elemento regular.

\begin{lemma}
  Sea $A$ una retícula distributiva donde todos los elementos
  tienen negación.
  Entonces $A$ es un álgebra booleana (todo elemento es
  complementado) si, y solo si, todo elemento es regular.
\end{lemma}
\begin{proof}
    $\Rightarrow )$ Supongamos que $A$ es booleana, entonces todo
    $a\in A$ cumple $a\vee \neg a=1$.
    Luego $\neg a\vee \neg\neg a=1$, es decir, $\neg\neg a$ es
    complementado.
    Como el complemento de $\neg a$ es único, se tiene que $\neg
    \neg a=a$. \\ 
    $\Leftarrow)$ Supongamos que todos los elementos de $A$ son
    regulares y sea $a\in A$.
    Como $a$ y $a\sup\neg a$ son regulares, tenemos
    $a\sup\neg a=\neg\neg (a\sup\neg a)=\neg(\neg
    a\wedge \neg\neg a)=\neg 0=1$.
    Por lo tanto, $a\vee \neg a=1$, es decir, $a$ es
    complementado. Con ello $A$ es un álgebra booleana.
\end{proof}

\section{Completez}
\label{ss:completez}
Decimos que un copo $A$ es superiormente completo
si cualquier subconjunto $X\subseteq A$ tiene supremo $\Sup X$.
En este caso, también decimos que $A$ es una $\Sup$-semiretícula.
Notemos que una $\Sup$-semiretícula es, en particular,
una $\sup$-semiretícula.
Un morfismo de $\Sup$-semiretículas (o $\Sup$-morfismo)
es una función que preserva supremos; es decir:
$f(\Sup X)=\Sup\{f(x)\mid x\in X\}$
para todo subconjunto $X\subseteq A$.

Las $\Sup$-semiretículas, junto con sus morfismos,
forman una categoría $\SupLat$.
Nótese que un morfismo de $\Sup$-semiretículas es un morfismo de
$\sup$-semiretículas (aunque no al revés),
así que tenemos un funtor $\SupLat\to\Pos^\sup$.
En particular, un $\Sup$-morfismo
también es una función monótona (ver \ref{ss:semireticulas}).

De manera completamente análoga, decimos que un copo es
inferiormente completo (o es una $\Inf$-semiretícula)
si todo $X\subseteq A$ tiene ínfimo $\Inf X$, tenemos
morfismos de $\Inf$-semiretículas y un funtor
$\InfLat\to\Pos^{\inf}$.

Decimos que un copo es completo (o una retícula completa)
si todo subconjunto $X\subseteq A$ tiene ínfimo $\Inf X$ y
supremo $\Sup X$. En particular, una retícula completa es una
$\Sup$-semiretícula y una $\Inf$-semiretícula.
Definimos un morfismo de retículas completas
como una función que es tanto $\Sup$-morfismo como
$\Inf$-morfismo.
En este caso, decimos que el morfismo es completo.
Así, las retículas completas y los morfismos completos forman una
categoría $\CLat$.

Si un copo $A$ es superiormente completo, entonces
dado un subconjunto $X\subseteq A$ podemos considerar el conjunto
$\cotInf X$ de cotas inferiores de $X$.
\[
  \cotInf X = \{c\in A \mid \forall x\in X, c\leq x\}
.\]
Notemos que, todo $x\in X$ es una cota superior de $\cotInf X$,
ya que tenemos $y\leq x$ para todo $y\in\cotInf X$.
Así, $\Sup\cotInf X\leq x$, por lo cual
$\Sup\cotInf X\in\cotInf X$.
Esto significa que $\Sup\cotInf X$ es el ínfimo de $X$:
\[
  \Sup\cotInf X = \Inf X
.\]
Como $X\subseteq A$ era cualquier subconjunto,
esto muestra que, si $A$ es superiormente completo, entonces
también es inferiormente completo.
Análogamente, si $A$ es inferiormente completo, entonces también
es superiormente completo, por lo cual hemos demostrado que

\begin{thm}
  Un copo $A$ es superiormente completo si, y solo si, es
  inferiormente completo.
\end{thm}

En otras palabras, las categorías $\InfLat$, $\SupLat$ y
$\CLat$ tienen exactamente los mismos objetos: las retículas
completas.
Es importante observar que esto no implca que las categorías
sean la misma:
en general, dadas dos retículas completas $A,B$, las contenciones
entre los conjuntos de morfismos
\[
  \InfLat(A,B)\supseteq \CLat(A,B)\subseteq\SupLat(A,B)
\]
son propias.
Es decir, hay $\Sup$-morfismos que no son $\Inf$-morfismos y
viceversa.
En cualquier caso, sí tenemos funtores fieles
$\SupLat \leftarrow \CLat \to \InfLat$.

En resumen, hasta ahora tenemos categorías y funtores fieles
\[
    \begin{tikzcd}
        & & \Pos & & \\
        & \Pos^\wedge \arrow[ru]
        & \Lat \arrow[l] \arrow[r]  
        & \Pos^\vee \arrow[lu] & \\
        & \InfLat \arrow[u]
            & \DLat \arrow[u,hook] &
            \SupLat \arrow[u] & \\
        & & \Bool \arrow[u,hook] & \\
        & & \CLat \arrow[ruu]
        \arrow[luu] & &
    \end{tikzcd}
\]
donde los funtores marcados como $\rmono$ son fielmente plenos.

%\section*{(SESIÓN 3: 14 SEP)}
\section{Marcos}
Hasta ahora, las definiciones y los teoremas han sido
completamente duales.
Ahora introducimos los marcos, que son un cierto tipo de
retículas completas, pero la simetría que hasta ahora hemos
observado deja de mantenerse.

La definición de un marco surge al estudiar las propiedades
algebraicas de las topologías:
si $S$ es un espacio topológico, la topología $\cal OS$ es
un subcopo del conjunto potencia $\cal PS$ que,
por definición, es cerrado bajo ínfimos finitos y
supremos arbitrarios. Es decir, $\cal OS$ es una
sub-$\inf$-semiretícula y una sub-$\Sup$-semiretícula de $\cal
PS$.
Además, como en $\cal PS$ se cumple la ley distributiva
$U\cap\bigcup \cal F=\bigcup\{U\cap V\mid V\in\cal F\}$,
entonces la misma ley se satisface en $\cal OS$.
Estas son las propiedades de una topología que abstrae un marco.

\begin{defn}
  Un marco $A$ es una retícula completa que satisface la siguiente
  ley distributiva:
  \[
      y\inf\Sup X = \Sup\{y\inf x\mid x\in X\}
  .\]
  En particular, un marco es una retícula distributiva.
  Un morfismo de marcos es un morfismo de copos que respeta
  supremos arbitrarios e ínfimos finitos.
  Ya que los $\inf$-morfismos y los $\Sup$-morfismos son cerrados
  bajo composición, también lo son los morfismos de marcos, con
  lo cual obtenemos una categoría $\Frm$.
\end{defn}

\begin{exa}
  \begin{enumerate}
    \item Dado un conjunto $S$ cualquiera, el conjunto potencia
    $\cal PS$ es un marco.
    \item Dados dos conjuntos $S,T$ y una función $f:S\to T$,
    la preimagen $f^{-1}:\cal PT\to\cal PS$ preserva ínfimos
    y supremos arbitrarios, así que, en particular, es un
    morfismo de marcos.
    \item Dado cualquier espacio topológico $S$, la topología
    $\cal OS$ es un marco y la inclusión $\cal OS\to\cal PS$
    es un morfismo de marcos.
    \item Dados dos espacios topológicos $S,T$ y una función
    continua $f:S\to T$, la preimagen $f^{-1}:\cal PT\to\cal
    PS$ manda abiertos en abiertos, así que se restringe a
    una función $\cal OT\to\cal OS$ que denotamos como $\cal
    Of$.
    En la parte \ref{part:espacio-de-puntos} de este texto,
    veremos que esta asignación es, de hecho, un funtor
    contravariante que forma parte de una
    adjunción entre $\Top$ y $\Frm$.
  \end{enumerate}
\end{exa}

Notemos que los morfismos de marcos pueden no preservar
ínfimos arbitrarios.
Además, en general,
los supremos finitos no distribuyen sobre ínfimos arbitrarios.

Nuestro diagrama de categorías ahora se ve así
\[
    \begin{tikzcd}
        &  & \Pos & & \\
        & \Pos^\wedge \arrow[ru]
        & \Lat \arrow[l] \arrow[r]
        & \Pos^\vee  \arrow[lu] & \\
        & & \DLat \arrow[u,hook] & & \\
        \InfLat \arrow[ruu]
        & & \Bool \arrow[u,hook]
        & \Frm \arrow[r] \arrow[lu]
        & \SupLat \arrow[luu] \\
        & & \\
        & & \CLat \arrow[rruu]
        \arrow[lluu] & &
    \end{tikzcd}
\]

\section{La completación de secciones inferiores}
Si $A$ es una $\inf$-semiretícula, nos gustaría encontrar un
marco $\hat A$ que complete a $A$ ``de la mejor manera posible''.

¿Qué tal el conjunto potencia $\cal PA$?
Es un álgebra booleana completa, tiene leyes distributivas
fuertes; quizá demasiado fuertes.
Además, la retícula que buscamos debería tener a $A$ como una
subretícula, mientras que
la función obvia $A\to\cal PA$ dada por $a\mapsto\{a\}$
no preserva el orden, así que no es una inclusión de
retículas.
Vamos a refinar esta situación.

Si $A$ es un copo, una sección inferior de $A$
es un subconjunto $L\subseteq A$ que "absorbe hacia abajo".
Es decir, si $a\leq b\in L$, entonces $a\in L$.

Denotemos como $\cal LA$ al conjunto de todas las secciones
inferiores en $A$.
Nótese que, por vacuidad, el conjunto vacío
$\emptyset\subseteq A$ es una sección inferior de $A$.
Además, la intersección de dos secciones
inferiores vuelve a ser una sección inferior, mientras que la
unión arbitraria de secciones inferiores también lo es.
En otras palabras, $\cal LA$ es un submarco de $\cal PA$.

Así, $\cal LA$ es una topología en $A$, que podríamos llamar la
topología de coespecialización. (La topología de especialización
en un copo $A$ es el conjunto de secciones superiores).
Las topologías de especialización y coespecialización tienen la
propiedad interesante de que una función entre dos copos es
monótona ssi es continua en la topología de
especialización ssi es continua en la topología de
coespecialización.
Sin embargo, ahora nos enfocaremos más en el aspecto reticular
de $\cal LA$ que en sus propiedades como topología de $A$.

Para cada subconjunto $F\subseteq A$, el conjunto
\[
  \down F = \{a\in A \mid \exists c\in F , a\leq c\} \subseteq A
\]
es una sección inferior.
De hecho, es la sección inferior más pequeña que contiene a $F$.
Decimos que $\down F$ es la sección inferior generada por $F$.
La asignación $F\mapsto\down F$ nos da una función
\[
  \down:\cal PA\to\cal PA
\]
cuyo conjunto de sus puntos fijos
(es decir, los $F\in\cal PA$ con $F=\down F$) es $\cal LA$.
Además, $\down$ es una función monótona, idempotente, infla
(es decir, $F\subseteq\down F$) y cumple $\down(F\cup G)=\down
F\cup\down G$.
En general, la igualdad $\down(F\cap G)=\down F\cap\down G$ no se
cumple. Sin embargo, si $A$ es una semiretícula inferior,
entonces tenemos una identidad similar.
Definiendo $F\inf G := \{x\inf y \mid x\in F, y\in G\}$,
tenemos $\down(F\inf G)= \down F\cap\down G$, ya que
\begin{align*}
  a\in\down(F\inf G)
  &\iff \exists(c\in F\inf G).(a\leq c) \\
  &\iff \exists(f\in F,g\in G).(a\leq f\inf g) \\
  &\iff \exists(f\in F,g\in G).(a\leq f, a\leq g) \\
  &\iff a\in \down F, a\in \down G \\
  &\iff a\in \down F\cap\down G.
\end{align*}

En particular, si $A$ es una $\inf$-semiretícula,
entonces la función
\begin{align*}
    \down : A&\to \cal LA \\
    a&\mapsto \down a:=\down\{a\}
\end{align*}
es un morfismo de $\inf$-semiretículas.
Así, $A$ se puede ver como una sub-$\inf$-semiretícula del marco
$\cal LA$.
En este sentido, $\cal LA$ ``le da'' a $A$ los supremos que le
faltan para ser un marco.
Además, es de esperarse que existan otros marcos $B$ y morfismos
de $\inf$-semiretículas $A\to B$. Sin embargo, afirmamos que
$\down:A\to\cal LA$ es el mejor de estos morfismos, en el sentido
de que $\down:A\to LA$ exhibe a $\cal LA$ como el marco libre
sobre la $\inf$-semiretícula $A$.

\begin{sol}
%[Alfredo $\checkmark$]
    Consideremos el funtor de olvido $U:\Frm\to\Pos^{\inf}$.
    Precomponer con $\down$ nos da una flecha
    \[
        \Frm(\cal LA,-)\to \Pos^{\inf}(A,U-)
    ,\]
    
    Resta ver que esta flecha es una biyección.
    Es decir, dado un morfismo $f:A\to B$
    de $\inf$-semiretículas,
    debemos probar que existe un único morfismo de marcos
    $f^\sharp:\cal LA\to B$ que factoriza a $f$ a través
    de $\down$:
    \[
        f^\sharp\down = f
    .\]
    Es decir, $f^\sharp(\down a)=f(a)$ para todo $a\in A$.
    Esta condición determina completamente a $f^\sharp$.
    En efecto, para toda sección inferior $F\in\cal LA$ tenemos
    $F=\bigcup\{\down a \mid a\in F\}$ y, como también queremos
    que $f^\sharp$ respete supremos, se debe cumplir
    \begin{align*}
        f^\sharp(F)
        &= \Sup\{f^\sharp(\down a) \mid a\in F\} \\
        &= \Sup\{f(a) \mid a\in F\}.
    \end{align*}
    Tomando esta ecuación como la definición de $f^\sharp$, es
    claro que $f^\sharp(\down a)=f(a)$.
    Por lo tanto, si $f^\sharp:\cal LA\to B$ es un
    morfismo de marcos, es el único con esta propiedad.
    Verificamos las propiedades directamente.
    \begin{itemize}
        \item En efecto, si $F\subseteq G\cal\in LA$, entonces 
        \[
            \{f(a) \mid a\in F\} \subseteq \{f(a) \mid a\in G\}
        .\]
        Tomando supremos, obtenemos
        $f^\sharp(F)\leq f^\sharp(G)$, así que $f^\sharp$ es
        monótona.
        \item
        Dadas $F,G\in\cal LA$, hay que mostrar
        que $f^\sharp(F\cap G)=f^\sharp(F)\inf f^\sharp(G)$.
        La comparación $\leq$ se sigue de la monotonía de
        $f^\sharp$.
        Por otro lado, observemos que
        \[
            \{ a\inf b \mid a\in F, b\in G\}
            \subseteq F\cap G,
            \hspace{10mm} (*)
        \]
        pues $F$ y $G$ son secciones inferiores.
        Luego,
        \begin{align*}
            f^\sharp(F)\inf f^\sharp(G)
            &= \Sup\{f(a)\inf f(b) \mid a\in F, b\in G\}
                && \text{ ley dist. de marcos } \\
            &= \Sup\{f(a\inf b) \mid a\in F, b\in G\} \\
            &\leq \Sup\{f(c) \mid c\in F\cap G\}
                && \text{ por $(*)$ } \\
            &= f^\sharp(F\cap G),
        \end{align*}
        como se quería.
        \item
        Dado $X\subseteq \cal LA$, hay que mostrar que
        $f^\sharp(\bigcup X)=\Sup\{f^\sharp(F) \mid F\in X\}$.
        Como $f^\sharp$ es monótona,
        $f^\sharp(\bigcup X)$ es cota superior de
        $\{f^\sharp(F) \mid F\in X\}$.
        Para ver que es la mínima, sea $b\in B$ tal que
        $f^\sharp(F)\leq b$ para todo $F\in X$.
        Por definición de $f^\sharp$, esto significa que
        $f(a)\leq b$ para cualesquiera $a\in F, F\in X$.
        Luego,
        \begin{align*}
            f^\sharp(\bigcup X)
            &= f^\sharp (
            \{a\in A \mid a\in F\text{ para algún }F\in X\}
            ) \\
            &=
            \Sup\{f(a)\in A \mid a\in F\text{ para algún }F\in X\}) \\
            &\leq b,
        \end{align*}
        como se deseaba.
    \end{itemize}
    Por lo tanto, $f^\sharp$ es morfismo de marcos.
    Así, tenemos un isomorfismo
    \[
        \Frm(\cal LA,-)\xto{-\circ \down} \Pos^{\inf}(A,U-)
    \]
    y, así, $\cal LA$ es el marco libre en $A$.
    
    Observemos que esto es válido para cualquier
    semiretícula $A$.
    Más aún, dado un morfismo de $\inf$-semiretículas
    $g:A\to A'$, la composición
    \[
        \cal LA' \lar \down A' \lar g A
    \]
    es un morfismo de $\inf$-semiretículas y
    $\cal LA'$ es un marco, así que
    existe un único morfismo de marcos
    $g^\sharp:\cal LA\to \cal LA'$ que factoriza a
    $\down g$ a través de $\down:A\to \cal LA$.
    Si definimos $\cal Lg=g^\sharp:\cal LA\to\cal LA'$,
    obtenemos una función
    \[
        \Pos^{\inf}(A,A') \to \Frm(\cal LA,\cal LA')
    .\]
    Más aún, las propiedades de unicidad de
    $\cal Lg=g^\sharp$ aseguran que $\cal L$ es un funtor
    $\cal L:\Pos^{\inf}\to\Frm$.
    Luego, tenemos una adjunción $\cal L\dashv U$
    (ver \ref{ss:adjunciones}).
\end{sol}


\section{Álgebras booleanas completas}
Recordemos que un álgebra booleana es una retícula distributiva
donde todo elemento tiene un complemento, el cual es único por la
distributividad. (Ver \ref{ss:complementos-algebras-booleanas}).
Un álgebra booleana completa es un álgebra booleana que es
completa como retícula. (Ver \ref{ss:completez}).

También recordemos que, en un álgebra booleana, la negación
coincide con el complemento, así que la función $\neg:A\to A$
que cumple $\neg\neg=\id_A:A\to A$. (Ver \ref{ss:negaciones}).

La categoría $\CBA$ de las álgebras booleanas completas se define
estipulando que los morfismos entre dos álgebras booleanas
completas $A,B$ son los morfismos completos:
\[
  \CBA(A,B)=\CLat(A,B)
.\]
Es decir, funciones que preservan supremos arbitrarios e
ínfimos arbitrarios.
Así, tenemos una inclusión $\CBA\rmono\CLat$.
En particular, los morfismos de álgebras booleanas completas
son morfismos de álgebras booleanas,
ya que preservan supremos e ínfimos finitos, por lo cual tenemos
un funtor fiel $\CBA\to\CLat$. Sin embargo no todo morfismo de
álgebras booleanas entre álgebras booleanas completas preserva
los supremos e ínfimos arbitrarios, así que este funtor no
es pleno.

El primer objetivo de esta sección es probar que el funtor
compuesto $\CBA\to\Bool\rmono\DLat$ se factoriza a través de
$\Frm$, para lo cual es suficiente mostrar que toda álgebra
booleana completa es un marco, pues los morfismos completos son
morfismos de marcos automáticamente.

Para esto, demostraremos un lema que usaremos recurrentemente.

\begin{lemma}[Caballo de batalla] \label{lemma:caballo}
    Sea $A$ una retícula distributiva
    y $a\in A$ un elemento complementado.
    Entonces $\neg a = a'$ satisface
    \[
        a\inf x \leq y  \ssi x\leq \neg a\sup y
    \]
    para cualesquiera $x,y\in A$.
\end{lemma}
\begin{proof}
    Por un lado, supongamos que $a\inf x\leq y$.
    Entonces
    \begin{align*}
        x
        &= x \inf 1 \\
        &= x \inf (a\sup \neg a) \\
        &= (x\inf a)\sup(x\inf\neg a) \\
        &\leq y \sup (x\inf \neg a) \\
        &\leq y \sup \neg a.
    \end{align*}
    Recíprocamente, si $x\leq \neg a\sup y$, entonces tenemos
    \begin{align*}
        a\inf x
        &\leq a\inf(\neg a\sup y) \\
        &= (a\inf\neg a) \sup (a\inf y) \\
        &= 0\sup (a\inf y) \\
        &= a\inf y \\
        &\leq y,
    \end{align*}
    como se quería.
\end{proof}

Con nuestro caballo de batalla, demostraremos el siguiente
resultado.

\begin{lemma}
    Toda álgebra booleana completa $A$ es un marco.
    Es decir, satisface la distributividad de marcos
    \[
        a\inf\Sup X = \Sup\{a\inf x\mid x\in X\}
    \]
    para cualesquiera $a\in A$ y $X\subseteq A$.
\end{lemma}
\begin{proof}
    Por un lado, para todo $x\in X$ tenemos que $x\leq\Sup X$, así que
    $a\inf x\leq a\inf\Sup X$.
    Luego,
    \[
        \Sup\{a\inf x\mid x\in X\} \leq a\inf\Sup X
    ,\]
    así que solo resta demostrar la otra comparación:
    \[
        a\inf\Sup X \leq \Sup\{a\inf x\mid x\in X\}
    .\]
    Aquí es donde usamos nuestro lema.
    Éste nos dice que la comparación anterior es equivalente a
    \[
        \Sup X \leq \neg a\sup \Sup\{a\inf x\mid x\in X\}
    ,\]
    lo cual es fácil: en efecto, tenemos
    \begin{align*}
        \neg a\sup\Sup\{a\inf x\mid x\in X\}
        &= \Sup\{\neg a \sup(a\inf x) \mid x\in X\} \\
        &= \Sup\{(\neg a\sup a)\inf(\neg a\sup x) \mid x\in X\} \\
        &= \Sup\{1\inf(\neg a\sup x) \mid x\in X\} \\
        &= \Sup\{\neg a\sup x \mid x\in X\} \\
        &\geq \Sup\{x \mid x\in X\} \\
        &= \Sup X,
    \end{align*}
    ya que $\neg a\sup x\geq x$ para todo $x\in X$.
    Esto es lo que se quería.
\end{proof}

Tomemos dos álgebras booleanas completas $A,B$.
El resultado anterior nos dice que $A$ y $B$ son marcos.
Además, como los morfismos $\CBA(A,B)$ son morfismos completos,
así que también son morfismos de marcos, ya que preservan supremos
arbitrarios e ínfimos finitos.
Es decir, tenemos una inclusión entre los conjuntos de morfismos
\[
  \CBA(A,B) \subseteq \Frm(A,B)
.\]
Como habíamos anunciado, esto muestra que el funtor
$\CBA\to\Bool\rmono\DLat$ se factoriza a través de $\Frm$, por lo
cual que tenemos un diagrama conmutativo de funtores fieles
\[
  \begin{tikzcd}
    \CBA \ar[r] \ar[d] & \Frm \ar[d] \\
    \Bool \ar[r,hook] & \DLat.
  \end{tikzcd}
\]
(De nuevo, $\rmono$ indica un funtor fielmente pleno).
El segundo objetivo de la sección es mostrar que el lado superior
de este cuadrado también es fielmente pleno.
Para lograr esto, primero demostraremos el siguiente resultado.

\begin{lemma}[Ley de DeMorgan]
  Sea $A$ un álgebra booleana completa y $X\subseteq A$.
  Entonces
  \[
    \neg \Inf X = \Sup \{\neg x \mid x\in X\}
  .\]
\end{lemma}
\begin{proof}
  Sabemos que, como $A$ es booleana, la negación es el complemento.
  Por lo tanto, basta mostrar que
  $\Inf X$ y $\Sup \{\neg x \mid x\in X\}$ son complementarios;
  es decir: que su ínfimo es $0$ y su supremo es $1$.
  Por un lado, cualquier $x\in X$ cumple $\Inf X\leq x$, así que
  \[
      \Inf X\inf\neg x \leq x\inf \neg x = 0.
  \]
  Luego, por la ley distributiva de marcos, tenemos
  \begin{align*}
      \Inf X \inf \Sup\{\neg X\mid x\in X\}
      &= \Sup\{\Inf x \inf \neg x\mid x\in X\} \\
      &= \Sup\{0\} \\
      &= 0.
  \end{align*}
  Por otro lado, para todo $x\in X$ tenemos
  $\neg x\leq \Sup\{\neg x\mid x\in X\}$.
  Usando el caballo de batalla dos veces, vemos que esto equivale a
  $\neg\Sup\{\neg x\mid x\in X\}\leq x$.
  Como $x\in X$ era arbitraria, tenemos que
  \[
      \neg\Sup\{\neg x\mid x\in X\}\leq \Inf X
  .\]
  Usando el caballo de batalla una vez más, obtenemos
  \[
      1\leq \Sup\{\neg x\mid x\in X\} \sup \Inf X
  .\]
  Esto concluye la demostración.
\end{proof}

Usando la ley de DeMorgan que acabamos de demostrar,
podemos probar lo que queríamos. De hecho, podemos probar un poco
más.
\begin{lemma}
    Cualquier morfismo de marcos $f:A\to B$,
    donde $A$ es un álgebra booleana completa,
    también respeta los ínfimos arbitrarios.
\end{lemma}
\begin{proof}
    Por un lado, la desigualdad
    \[
        f(\Inf X)\leq\Inf\{f(x)\mid x\in X\}
    \]
    viene de la monotonía de $f$.
    Por otro lado, como $\Inf X$ y $\Sup\{\neg x\mid x\in X\}$
    son complementarios en $A$ (por DeMorgan)
    y $f$ preserva los complementos
    (pues respeta ínfimos y supremos finitos), entonces $f(\Inf X)$
    y $\Sup\{f(\neg x)\mid x\in X\}$ son complementarios en $B$.
    Luego, la desigualdad faltante
    \[
        \Inf\{f(x)\mid x\in X\} \leq f(\Inf X)
    \]
    es equivalente, por nuestro caballo de batalla, a la desigualdad
    \[
        \Inf\{f(x)\mid x\in X\} \inf \Sup\{f(\neg x)\mid x\in X\} = 0,
    \]
    la cual es sencilla de comprobar: para todo $y\in X$, tenemos
    \begin{align*}
        \Inf\{f(x)\mid x\in X\} \inf f(\neg y)
        &\leq f(y)\inf f(\neg y) \\
        &\leq f(y\inf \neg y) \\
        &= 0,
    \end{align*}
    así que la distributividad de marcos implica
    \begin{align*}
        \Inf\{f(x)\mid x\in X\} \inf \Sup\{f(\neg x)\mid x\in X\}
        &= \Sup\Big\{\Inf\{f(x)\mid x\in X\}\inf f(\neg y)
            \;\;\Big|\;\; y\in X\Big\} \\
        &= \Sup\{0\} \\
        &= 0.
    \end{align*}
\end{proof}

En particular, si $A,B$ son álgebras booleanas completas,
cualquier morfismo de marcos $f:A\to B$ es un morfismo
completo.
Es decir, la inclusión $\CBA(A,B) \subseteq \Frm(A,B)$
es una igualdad:
\[
    \CBA(A,B) = \Frm(A,B).
\]
Por lo tanto, $\CBA$ es una subcategoría plena de $\Frm$.
Nuestro diagrama ahora se ve como sigue:
\[
    \begin{tikzcd}
        &  & \Pos & & \\
        & \Pos^\wedge \arrow[ru]
        & \Lat \arrow[l] \arrow[r]
        & \Pos^\vee  \arrow[lu] & \\
        & & \DLat \arrow[u,hook] & & \\
        \InfLat \arrow[ruu]
        & & \Bool \arrow[u,hook]
        & \Frm \arrow[r] \arrow[lu]
        & \SupLat \arrow[luu] \\
        & & \CBA \arrow[d] \arrow[u] \arrow[ur,hook] 
        & & \\
        & & \CLat \arrow[rruu]
        \arrow[lluu] & &
    \end{tikzcd}
\]

%\section*{(SESIÓN 4: 21 SEP)}

\section{Implicaciones}

Una implicación en una semiretícula inferior $A$ es una operación
$(-\succ -):A\to A$ tal que, para cualesquiera $a,x,y\in A$ se tiene
\[
  x\inf y\leq a \ssi x\leq (y\succ a )
.\]

\begin{exa}
  En un álgebra booleana, nuestro caballo de batalla nos dice que
  \[
    x\inf y\leq a \ssi
    x\leq \neg y\sup a
  .\]
  Por lo tanto, toda álgebra booleana tiene implicación dada
  como $(y\succ a)=\neg y\sup a$.
\end{exa}

Ahora vienen dos lemas técnicos.
\begin{lemma}
  Sea $A$ una $\inf$-semiretícula con implicación.
  Entonces
  \begin{enumerate}
    \item $(x\succ -)$ infla.
    \item $x\inf(x\succ a) = x\inf a$
    \item $(-\succ a)$ es antítona.
  \end{enumerate}
\end{lemma}
\begin{proof}
    \begin{enumerate}
        \item Como $a\wedge x\leq a$, tenemos
        $a\leq (x\succ a).$
        \item Consideremos $a, x\in A$.
        Para cada $z$ tenemos que 
        \begin{align*}
            z\leq x\wedge (x\succ a)
            & \iff z\leq x \mbox{ y } z\leq (x\succ a)\\
            & \iff z\leq x \mbox{ y } z\wedge x\leq a\\
            & \iff z\leq x \mbox{ y } z\leq a\\
            & \iff z\leq a\wedge x.
        \end{align*}
        Por lo tanto $x\inf(x\succ a) = x\inf a$.
        \item Supongamos que $x\leq y$. Consideremos $z=(y\succ a)$, entonces $x\wedge z\leq y\wedge z$ y por los incisos anteriores de este lema, $y\wedge z\leq a$. Así, $x\wedge z\leq a$ y por la definición de implicación obtenemos que $(y\succ a)\leq (x\succ a)$. 
    \end{enumerate}
\end{proof}

\begin{lemma}
  \label{lemma:w-cerradura}
  Sea $A$ una $\inf$-semiretícula con implicación.
  Para cualquier elemento $a\in A$, la función
  $((-\succ a)\succ a):A\to A$ tiene las siguientes propiedades.
  \begin{enumerate}
    \item $((-\succ a)\succ a)$ infla.
    (Inmediado del punto 2 del lema anterior).
    \item $((-\succ a)\succ a)$ es monótona.
      (Inmediato del punto 3 del lema anterior).
    \item $((-\succ a)\succ a)$ es idempotente.
  \end{enumerate}
\end{lemma}
\begin{proof}
  Solo falta probar el punto 3.
  Por el punto 1, tenemos
  \[
    (x\succ a) \leq (((x\succ a)\succ a)\succ a),
  \]
  así que resta probar
  \[
     (((x\succ a)\succ a)\succ a) \leq (x\succ a),
  \]
  que, por definición de la implicación, equivale a
  \[
     (((x\succ a)\succ a)\succ a) \inf x \leq a.
  \]
  Recordemos que $x\leq ((x\succ a)\succ a)$, porque $((-\succ
  a)\succ a)$ infla, y que
  $(y\succ a)\inf y = y\inf a$ para todo $y\in A$.
  En particular, para $y=((x\succ a)\succ a)$, tenemos
  \begin{align*}
     (((x\succ a)\succ a)\succ a) \inf x
     &\leq (((x\succ a)\succ a)\succ a)
       \inf ((x\succ a)\succ a) \\
     &= (y\succ a) \inf y \\
     &= y \inf a \\
     &\leq a,
  \end{align*}
  como se quería. 
\end{proof}

En el capítulo \ref{ch:cocientes} definiremos los operadores
cerradura como funciones $A\to A$ que cumplen estas tres
propiedades (definición \ref{def:operador-cerradura}).
Allí veremos que estos operadores juegan un papel
importante en el contexto de $\Sup$-semirretículas.

Así, el resultado anterior dice que, si $A$ es una
$\inf$-semiretícula con implicación,
entonces $((-\succ a)\succ a)$ es un operador cerradura.
Además, el operador $((-\succ a)\succ a)$ tiene otra propiedad:
preserva ínfimos.

\begin{lemma}
  \label{lemma:modalidad-w}
  Si $A$ es una $\inf$-semirretícula con implicación, entonces
  \[
    (((x\inf y)\succ a)\succ a)
    =
    ((x\succ a)\succ a) \inf
    ((y\succ a)\succ a)
  .\]
\end{lemma}
\begin{proof}
  Por un lado, como $((-\succ a)\succ a)$ es monótono, tenemos
  \begin{align*}
    (((x\inf y)\succ a)\succ a)
    &\leq ((x\succ a)\succ a)
    \\
    (((x\inf y)\succ a)\succ a)
    &\leq ((y\succ a)\succ a)
  \end{align*}
  así que $(((x\inf y)\succ a)\succ a)$ es cota inferior de
  $((x\succ a)\succ a)$ y $((y\succ a)\succ a)$.

  Ahora, dada cualquier cota inferior $z$ de estos dos elementos,
  queremos probar que $z \leq (((x\inf y)\succ a)\succ a)$,
  lo cual equivale a $z\inf((x\inf y)\succ a)\leq a$.
  Como $z$ es cota inferior, tenemos
  \begin{align*}
    z&\leq ((x\succ a)\succ a)
    &
    z&\leq ((y\succ a)\succ a)
  \end{align*}
  lo cual, por definición de la implicación, es
  \begin{align*}
    z\inf (x\succ a) &\leq a
    &
    z\inf (y\succ a) &\leq a.
  \end{align*}
  Ahora sea $w=((x\inf y)\succ a)$, de modo que
  \begin{align*}
    z\inf w\inf x\inf y
    &= z\inf ((x\inf y)\succ a)\inf(x\inf y) \\
    &\leq z\inf a \\
    &\leq a
  \end{align*}
  por definición de la implicación, esto nos da
  \begin{align*}
    z\inf w \inf x &\leq (y\succ a).
  \end{align*}
  Al hacer ínfimo con $z$ obtenemos
  \begin{align*}
    z\inf w \inf x
    &\leq z\inf (y\succ a) \leq a
  \end{align*}
  aplicando de nuevo la definición de la implicación,
  \begin{align*}
    z\inf w &\leq (x\succ a)
  \end{align*}
  y haciendo ínfimo con $z$,
  \begin{align*}
    z\inf w &\leq z\inf (x\succ a) \leq a
  \end{align*}
  Luego, $z\leq (w\succ a)=(((x\inf y)\succ a)\succ a)$,
  como se quería.
\end{proof}

También en el capítulo \ref{ch:cocientes}, veremos que
los operadores cerradura con esta propiedad, llamados núcleos
(definición \ref{def:nucleo}) tienen gran
relevancia en el contexto de marcos.

\section{Álgebras de Heyting}

Si una retícula tiene implicación, diremos que es un álgebra de
Heyting.
Un morfismo de álgebras de Heyting es un morfismo de retículas
que preserva la implicación.
Observemos que toda álgebra de Heyting tiene una
negación dada por $\neg a = (a\succ 0)$.

\begin{lemma}
  Toda álgebra de Heyting es distributiva, por lo que
  tenemos un funtor de inclusión $\Heyt\to\DLat$.
\end{lemma}

\begin{proof}
    Notemos que $a\wedge b \leq (a\wedge b)\vee (a\wedge c)$ y de igual manera $a\wedge c\leq  (a\wedge b)\vee (a\wedge c)$. Por la definición de implicación, estas desigualdades equivalen a las siguientes:
    $$b\leq (a \succ ((a \wedge b) \vee (a \wedge c))) \mbox{ y } c \leq (a\succ ((a \wedge b) \vee (a \wedge c)))$$
    Entonces $b \vee c \leq (a \succ ((a \wedge b) \vee (a \wedge c)))$ y de nuevo por la definición de implicación resulta $a \wedge (b \vee c)\leq (a\wedge b)\vee (a \wedge c)$. Además, para cualquier retícula se cumple que $a \wedge (b \vee c)\geq (a\wedge b)\vee (a \wedge c)$, es decir, $a \wedge (b \vee c)= (a\wedge b)\vee (a \wedge c)$, pero esto pasa si y sólo si $a \vee (b \wedge c)= (a\vee b)\wedge (a \vee c)$. Por lo tanto, toda álgebra de Heyting es distributiva.
\end{proof}

\begin{thm}
  Una retícula completa $A$ es un marco si, y solo si, $A$ tiene
  implicación.
\end{thm}
\begin{proof}
    $\Rightarrow )$ Supongamos que $A$ es un marco y sean $a,b\in A$.
    Consideremos el elemento
    $$y=\Sup\{c\in A\mid a\wedge c\leq b\}$$
    Por un lado, para todo $x\in A$ se tiene
    $a\wedge x\leq b\implies x\leq y$,
    pues es consecuencia directa de la definición de $y$.
    Ahora, si $x\leq y$, tenemos que 
    \begin{align*}
        a\inf x
        & \leq a\inf y \\
        & = a\inf\Sup\{x\in A\mid a\inf x\leq b\}\\
        & = \Sup \{a\inf x \mid a\inf x\leq b\}\leq b
    \end{align*}
    Por lo tanto, $a\wedge x\leq b\Leftrightarrow x\leq y$,
    así que $y=(a\succ b)$. \\
    $\Leftarrow )$ Supongamos que $A$ tiene implicación.
    Sean $a\in A$ y $X\subseteq A$.
    Basta probar que $A$ cumple la LDM.
    Como $a\inf x\leq a\inf\Sup X$, tenemos que
    $\Sup\{x\inf a\mid x\in X\}\leq a\inf\Sup X$.
    Solo falta demostrar la otra desigualdad. Consideremos $y=\bigvee\{x\wedge a\mid x\in X\}$, entonces $a\wedge x\leq y$ para toda $x\in X$, es decir, $x\leq (a\succ y)$ para toda $x\in X$. Entonces $\bigvee X\leq (a\succ y)\Leftrightarrow a\wedge\bigvee X\leq y=\bigvee \{x\wedge a\mid x\in X\}$. Por lo tanto se cumple la LDM.
\end{proof}

En particular, este resultado nos dice que un marco es lo mismo
que un álgebra de Heyting completa.
Sin embargo, los morfismos de marcos no necesariamente respetan
la implicación, ni los morfismos de álgebras de Heyting
necesariamente respetan los supremos arbitrarios.
En particular, la inclusión $\Ob(\Frm)\to\Ob(\Heyt)$
no se extiende a un funtor $\Frm\to\Heyt$.

\begin{exa}
  Sea $S$ un espacio topológico.
  Por el teorema anterior, dados $u,v\in\cal OS$
  existe la implicación $v\succ u\in \cal OS$.
  Para todo $w\in\cal OS$, tenemos
  \begin{align*}
    w\leq (v\succ u)
    &\iff w\cap v\leq u \\
    &\iff w\leq v'\cup u \\
    &\iff w\leq (v'\cup u)^\circ.
  \end{align*}
  Se sigue que $v\succ u = (u\cup v')^\circ$.
\end{exa}

\begin{exa}
  Ya vimos que toda álgebra booleana $A$ tiene implicación dada
  como $v\succ u = \neg v\sup u$, así que $A$ también
  es un álgebra de Heyting.
  Además, como un morfismo de álgebras booleanas preserva supremos
  e ínfimos finitos y negaciones, se sigue que
  también respeta las implicaciones, por lo cual es morfismo
  de álgebras booleanas.
  Por lo tanto, tenemos un funtor de inclusión $\Bool\to\Heyt$.
\end{exa}

Ahora nuestro diagrama de inclusiones se ve así:
\[
    \begin{tikzcd}
        &  & \Pos & & \\
        & \Pos^\wedge \arrow[ru]
        & \Lat \arrow[l] \arrow[r]
        & \Pos^\vee  \arrow[lu] & \\
        & & \DLat \arrow[u,hook] & & \\
        \InfLat \arrow[ruu]
        & \Heyt \arrow[ur]
        & \Bool \arrow[l] \arrow[u,hook]
        & \Frm \arrow[r] \arrow[lu]
        & \SupLat \arrow[luu] \\
        & & \CBA \arrow[d] \arrow[u] \arrow[ur] 
        & & \\
        & & \CLat \arrow[rruu]
        \arrow[lluu] & &
    \end{tikzcd}
\]

\begin{lemma}
    Sean $A$ un marco, $a\in A$ y $X\subseteq A$.
    Entonces
    \[
        (\Sup X)\succ a = \Inf\{(x\succ a) \mid x\in X\}
    .\]
    En particular, tomando $X=\{x,y\}$, tenemos
    \[
        (x\sup y)\succ a = (x\succ a)\inf(y\succ a)
    .\]
\end{lemma}
\begin{proof}
    Para todo $y\in A$, tenemos
    \begin{align*}
        y\leq (\Sup X)\succ a
        &\iff y\inf \Sup X \leq a \\
        &\iff \Sup\{y\inf x\mid x\in X\} \leq a \\
        &\iff (\forall x\in X,\; y\inf x\leq a) \\
        &\iff (\forall x\in X,\; y\leq(x\succ a)) \\
        &\iff y\leq \Sup\{(x\succ a)\mid x\in X\}.
    \end{align*}
\end{proof}

\begin{exe}[Para el lector]%[Armando]
  Dado un copo $A$,
  ¿Quién será la negación y la implicación en $\cal LA$?
\end{exe}

\section{Morfismos adjuntos de copos}
\label{ss:adj-copos}
Recordemos que, cuando vemos a $A$ y a $B$ como categorías,
un morfismo de copos $f:A\to B$ es lo mismo que un funtor,
así que podemos aplicar el concepto de adjunción entre funtores.
En este caso, dos morfismos de copos $f:A\to B$, $g:B\to A$
cumplen $(f\dashv g)$ (es decir, $g$ es adjunto derecho de $f$ y
$f$ es adjunto izquierdo de $g$) si
\[
    B(fa,b) \simeq A(a,gb)
;\]
esto es:
\[
    fa\leq b \ssi a\leq gb
.\]

\begin{exa}
  Sea $A$ un álgebra de Heyting (por ejemplo, un marco).
  Para todo $a\in A$ a implicación $(a\succ -)$ es
  el adjunto derecho del ínfimo $-\inf a$, pues
  \[
    y\inf a\leq b \ssi y\leq (a\succ b)
  .\]
\end{exa}

\begin{exa}
  \label{exa:adjuncion-potencia}
  Sea $\phi:S\to T$ una función entre conjuntos.
  Entonces la imagen directa $\phi_!:\cal PS\to\cal PT$ y
  la imagen inversa $\phi^{-1}:\cal PT\to\cal PS$, definidas para
  cualesquiera $U\subseteq S$ y $V\subseteq T$ como
  \begin{align*}
    \phi_!(U)
    &= \{\phi(x) \mid x\in U\}
    = \{y\in Y\mid \exists x\in U, y=\phi(x) \}
    \\
    \phi^{-1}(V)
    &= \{x\in X \mid \phi(x)\in V\}
  \end{align*}
  son morfismos de copos que satisfacen
  \begin{align*}
    \phi_!(U) \subseteq V
    &\iff \forall x\in U, \phi(x)\in V \\
    &\iff \forall x\in U, x\in \phi^{-1}(V) \\
    &\iff U\subseteq \phi^{-1}(V),
  \end{align*}
  así que $\phi_!\dashv \phi^{-1}$.

  Más aún, $\phi^{-1}$ también tiene adjunto derecho. En efecto,
  tenemos
  \begin{align*}
    \phi^{-1}(V) \subseteq U
    &\iff U'\subseteq \phi^{-1}(V)' \\
    &\iff U'\subseteq \phi^{-1}(V') \\
    &\iff \phi_!(U')\subseteq V' \\
    &\iff V\subseteq \phi_!(U')'.
  \end{align*}
  Así, definiendo $\phi_?:\cal PS\to\cal PT$ por la fórmula
  $\phi_?(U)=\phi_!(U')'$, tenemos $\phi^{-1}\dashv \phi_?$.
  Explícitamente,
  \begin{align*}
    \phi_?(U)
    &= \{y\in T \mid x\not\in \phi_!(U') \} \\
    &= \{y\in T \mid \forall x\in U', y\neq \phi(x) \}.
  \end{align*}
\end{exa}

\begin{exa}%[Dante $\checkmark$ ]
  \label{exa:adjunto-derecho-top}
  Si $\phi:S\to T$ es una función continua entre espacios
  topológicos, el morfismo de marcos
  $\phi^*=\cal O\phi:\cal OT\to\cal OS$,
  definido como $\phi^*(v)=\phi^{-1}(v)$ para cada $v\in\cal OT$,
  tiene adjunto derecho.
  Si $\phi_?:\cal PS\to\cal PT$ es el adjunto derecho de la
  preimagen $\phi^{-1}:\cal PT\to\cal PS$ (ver ejemplo
  \ref{exa:adjuncion-potencia}), tenemos
  \begin{align*}
    \phi^*(v) \leq u
    &\iff \phi^{-1}(v) \leq u \\
    &\iff v \leq \phi_?(u) \\
    &\iff v\leq \phi_?(u)^\circ,
  \end{align*}
  ya que $\phi_?(u)^\circ$ es el abierto más grande contenido en
  $\phi_?(u)$.
  Por lo tanto, el adjunto derecho $\phi_*:\cal OS\to\cal OT$ de
  $\phi^*$ se calcula como
  \begin{align*}
    \phi_*(u) &= \phi_?(u)^\circ = (\phi_!(u')')^\circ = \ol{\phi_!(u')}',
  \end{align*}
  donde $\phi_!:\cal PS\to\cal PT$ es la imagen directa.
\end{exa}

\begin{lemma}
    Sea $f:A\to B$ un morfismo de $\Sup$-semiretículas
    (por ejemplo, un morfismo de marcos).
    Entonces la función $f_*:B\to A$ dada como
    \[
        f_*(b) = \Sup\{x\in A \mid f(x)\leq b\} 
    \]
    es un morfismo de copos y es adjunto derecho de $f$.
    Más aún, $f_*$ es morfismo de $\Inf$-semiretículas.
\end{lemma}
\begin{proof}
    Sea $X=\{x\in A\mid f(x)\leq b\}$.
    Si $a\in A$ es tal que $f(a)\leq b$, entonces $a\in X$,
    por lo cual
    \[
        a\leq \Sup X= f_*(b)
    .\]
    Recíprocamente, si $a\leq f_*(b)=\Sup X$, entonces
    \begin{align*}
        f(a)
        &\leq f(\Sup X) \\
        &= \Sup\{f(x)\mid f(x)\leq b\} \\
        &\leq b,
    \end{align*}
    ya que $f$ preserva supremos.

    La monotonía de $f_*$ se sigue de que preserva ínfimos (lo
    cual probaremos abajo) pero de todos modos es fácil verlo,
    así que lo haremos por separado.
    Si tenemos $b\leq c\in B$, entonces
    \[
        \{x\in X\mid f(x)\leq b\} \subseteq
        \{x\in X\mid f(x)\leq c\}.
    \]
    Aplicando supremos, obtenemos $f_*(b)\leq f_*(c)$.
    
    Finalmente, veamos que $f_*$ preserva ínfimos.
    Si $Y\subseteq B$ es cualquier subconjunto,
    entonces todo $a\in A$ cumple
    \begin{align*}
        a\leq f_*(\Inf Y)
        &\iff f(a)\leq \Inf Y \\
        &\iff (\forall y\in Y,\; f(a)\leq y) \\
        &\iff (\forall y\in Y,\; a\leq f_*(y)) \\
        &\iff a\leq \Inf\{f_*(y) \mid y\in Y\}.
    \end{align*}
    Se sigue que $f_*(\Inf Y)=\Inf\{f_*(y)\mid y\in Y\}$.
\end{proof}

Sin embargo, nótese que, aunque $f:A\to B$ sea un morfismo
de marcos, su adjunto derecho $f_*:B\to A$
puede no preservar supremos finitos.

\section{Monomorfismos y epimorfismos}

Recordemos que, dada una categoría $\cal C$, un morfismo es
\begin{itemize}
    \item un monomorfismo si es cancelable por la izquierda,
    \item un epimorfismo si es cancelable por la derecha y
    \item un bimorfismo si es monomorfismo y epimorfismo.
\end{itemize}

\begin{exe}[Para el lector]%[Armando]
    Muéstrese que, en la categoría $\Top$ de los espacios topológicos,
    un morfismo es
    \begin{itemize}
        \item suprayectivo si, y solo si, es un epimorfismo,
        \item inyectivo si, y solo si, es un monomorfismo.
    \end{itemize}
\end{exe}

Este también es el caso las categorías de conjuntos, grupos y
espacios vectoriales, pero no en la categoría de anillos.
En efecto, aunque un morfismo de anillos es
\begin{align*}
    \text{inyectivo} &\iff \text{monomorfismo y} \\
    \text{suprayectivo} &\implies \text{epimorfismo},
\end{align*}
la inclusión
\[
    i:\mathbb Z\to\mathbb Q
\]
es un ejemplo de un epimorfismo de anillos que no es suprayectivo.

Veremos que, en este aspecto, la categoría de marcos se comporta
parecido a la categoría de anillos, pues aunque en $\Frm$ es cierto que
\begin{align*}
    \text{inyectivo} &\iff \text{monomorfismo y} \\
    \text{suprayectivo} &\implies \text{epimorfismo},
\end{align*}
también existen epimorfismos no suprayectivos.

\begin{lemma}
    Todo monomorfismo de marcos es inyectivo,
    así que, en $\Frm$, un morfismo es
    \[
        \text{inyectivo} \iff \text{monomorfismo.}
    \]
\end{lemma}
\begin{proof}
    Vamos a usar el marco
    \[
        3 \hspace{10mm} = \hspace{10mm}
        \begin{tikzcd}
            1 \\ \star \ar[u,no head] \\ 0. \ar[u,no head]
        \end{tikzcd}
    \]
    Sea $m:A\to B$ un monomorfismo y $a,b\in A$ tales que $m(a)=m(b)$.
    Como las funciones $f_a,f_b:3\to A$ dadas por
    $f_a(\star)=a$ y $f_b(\star)=b$ son morfismos de marcos que cumplen
    $m(f_a(\star))=m(a)=m(b)=m(f_b(\star))$, se sigue que $mf_a=mf_b$.
    Como $m$ es monomorfismo, se sigue que $f_a=f_b$, así que
    $a=f_a(\star)=f_b(\star)=b$.
\end{proof}

Del truco del marco $3$ que usamos en la demostración anterior,
se puede deducir que las asignaciones
\begin{align*}
    B &\leftrightarrows \Frm(3,B) \\
    a &\mapsto f_a \\
    f(\star) &\mapsfrom f
\end{align*}
forman un isomorfismo de marcos,
donde $\Frm(3,B)$ tiene el orden puntual.

Ahora queremos construir un epimorfismo de marcos que no es
suprayectivo.
\begin{lemma}
    Si $S$ es un espacio topológico $T_1$, entonces
    la inclusión $i:\cal OS\to\cal PS$ es un epimorfismo.
\end{lemma}
\begin{proof}
    Sean
    \[
        \begin{tikzcd}
            A
            & \cal PS \ar[l,"f"',shift right] \ar[l,"g",shift left]
            & \cal OS \ar[l,"i"']
        \end{tikzcd}
    \]
    morfismos en $\Frm$ con $fi=gi$.
    
    Dado $p\in S$, consideremos $X_p=\{p\},U_p=\{p\}'\in\cal PS$.
    Como $X_p$ y $U_p$ son complementarios en $\cal PS$,
    sus imágenes en $A$ (bajo $f$ y bajo $g$) son complementarias.
    Ahora, como $S$ es $T_1$, entonces $X_p$ es cerrado, así que
    $U_p$ es abierto.
    Como $fi=gi$, esto implica que
    \begin{align*}
        f(U_p)
        &= f(i(U_p)) \\
        &= g(i(U_p)) \\
        &= g(U_p).
    \end{align*}
    Luego, la unicidad de los complementos nos da
    \begin{align*}
        f(X_p)
        &= \neg f(U_p) \\
        &= \neg g(U_p) \\
        &= g(X_p).
    \end{align*}
    Ahora, para cualquier $E\in\cal PS$ tenemos
    $E=\Sup\{X_p\mid p\in E\}$.
    Como $f$ y $g$ son morfismos de marcos,
    \begin{align*}
        f(E)
        &= \Sup\{f(X_p) \mid p\in E\} \\
        &= \Sup\{g(X_p) \mid p\in E\} \\
        &= g(E),
    \end{align*}
    por lo cual $f=g$.
    Se sigue que $i$ es un epimorfismo.
\end{proof}
En particular, si $S$ es un espacio topológico $T_1$ no discreto
(es decir, $\cal OS\neq\cal PS$), entonces la inclusión
$i:\cal OS\to\cal PS$ es un ejemplo de epimorfismo de marcos que
no es suprayectivo.
