\chapter{Ordinales}

%\section{VIDEO 4: ordinales: (4 OCT)}
\begin{defn}[Conjuntos linealmente ordenados]
\leavevmode
\begin{itemize}
  \item Un conjunto parcialmente ordenado $(A,\leq)$ es linealmente ordenado si para cualesquiera $a,b\in A$ se tiene $a\leq b$ o $b\leq a$. 

\item Para un conjunto linealmente ordenado $(A,\leq)$, su opuesto es el mismo conjunto $A$ junto con el orden $\leq^{op}$ definido como 
$$a\leq^{op}b\iff b\leq a$$
El opuesto de $A$ es llamado simplemente $A^*$.
\item $(A,\leq)$ es bien ordenado si todo subconjunto de $A$ tiene por lo menos un primer elemento. Nótese que cualquier conjunto linealmente ordenado finito es bien ordenado.
\end{itemize} 
\end{defn}
\begin{exe}%[Dante $\checkmark$ ]
    \label{exe:sucesor}
  Sea $A$ un conjunto linealmente ordenado. $A$ es bien ordenado si y sólo si $A^{\star}=A\cup \{\star\}$, donde $\star$ es un elemento genérico, también lo es.
\end{exe}
\begin{sol}
    $\Rightarrow$\\
    Sea $(A,\leq^{\prime})$ bien ordenado y considérese la relación $\leq\subset A^\star\times A^\star$ definida como $\leq^{\prime} \cup \{(\star,a):a \ \in A\}\cup \{(\star,\star)\}$. 
    \begin{enumerate}
        \item Claramente, $\leq$ es una relación de orden.
        \item Sean $x,y\in A^\star$. Si $x,y \neq \star$, como $A$ es totalmente ordenado, $x\leq y$ o $y\leq y$. Si $x\neq \star, y=\star$, entonces $y\leq x$,  si $x,y=\star$, entonces $x\leq y $ y $y\leq x$.
        \item Sea $B\subset A$. Si $\star\in B, \star\leq b \ \forall \ b\in B$, y por como está definida $\leq$, no existe ningún $b\in B$ tal que $b\leq \star$, por lo que $\star$ es el menor elemento de $B$.
        Si $\star\not\in B$, entonces $B\subset A$, y como $\leq$ es equivalente a $\leq^{\prime}$ sobre $A$, entonces $B$ tiene un menor elemento en $B\subset A \subset A^\star$.
    \end{enumerate}
    Por lo anterior, $A^\star$ es bien ordenado.\vspace{3mm}
    
    $\Leftarrow$\vspace{3mm}
    
    Si $(A^\star,\leq)$ es bien ordenado, defínase el orden $\leq^{\prime}\subset A\times A$ como
    \[
        \leq^{\prime}=\leq\setminus \left( \{(\star,a):a\in A\} \cup \{(a,\star):a\in A\}\right)
    .\]
    \begin{enumerate}
        \item Claramente, $\leq^{\prime} $ es una relación de orden total.
        \item Sea $B\subset A$. Como $A\subset A^\star$, entonces $B\subset A^\star$, por lo que $\exists b\in B$ que es el menor elemento de $B$.
    \end{enumerate}
    Por lo anterior, $A$ es bien ordenado.
\end{sol}
\begin{defn}[Isomorfismo de conjuntos linealmente ordenados]
Un morfismo $f:A\to B$ entre conjuntos linealmente ordenados es una biyección monótona.
\end{defn}
\begin{defn}[Tipo de orden]
  Para un conjunto linealmente ordenado $A$, el tipo de orden $\iota(A)$ es la clase de equivalencia de $A$ bajo la relación de equivalencia $\simeq$ definida como
  $$A\simeq B \iff \exists f: A\to B \ \text{tal que f es isomorfismo}$$
  Si $\alpha=\iota(A)$, entonces $\alpha^*=\iota(A^*)$.
\end{defn}
\begin{exa}[Construcción Von Neumann de $\mathbb{N}$]
$\mathbb{N}$ es un conjunto bien ordenado, pensado como la siguiente construcción:
$$0=\emptyset$$
$$1=\{\emptyset\}=\{0\}$$
$$2=\{\{\emptyset\}\}=1\cup\{1\}$$
$$3=\{\{\emptyset\},\{\{\emptyset\}\}\}=2\cup\{2\}$$
$$\vdots$$
$$n+1=n\cup \{n\}$$
$$\vdots$$
Así, $\mathbb{N}=\{0,1,\cdots\}$ es un conjunto bien ordenado con el orden $\leq$ definido como 
$$a\leq b \iff a\subset b$$
Y esta construcción también cumple con los axiomas de Peano.
\end{exa}
\begin{exa}[Algunos tipos de orden]
$\mathbb{N},\mathbb{Q}$ y $\mathbb{R}$ son conjuntos linealmente ordenados, y sus tipos de orden son 
\begin{itemize}
    \item $\omega=\iota(\mathbb{N})$
    \item $\eta=\iota(\mathbb{Q})$
    \item $\zeta=\iota(\mathbb{R})$
\end{itemize}
\end{exa}
\begin{defn}[Sucesor]
  Para $(A,\leq)$ con $\alpha=\iota(A)$, por el ejercicio \ref{exe:sucesor} $A^+=A\cup\{A\}$ es también linealmente ordenado. $A^+$ es llamad sucesor de $A$, y $\alpha^+=\iota(A^+)=\alpha\cup\{\alpha\}$
\end{defn}
Claramente la construcción de Von Neumann de $\mathbb{N}$ es compatible con esta definición de sucesor, y cualquier $n\in\mathbb{N}$ tiene la forma $n=\{0,1,\cdots,n-1\}$. Por lo tanto, se puede pensar en el tipo de orden de $\mathbb{N}$ como 
$$\omega=\bigcup\{n:n\mathbb{N}\}$$
Y a través de la operación sucesor se pueden obtener los ordinales
\begin{align*}
    \omega+1&=\omega^+=\omega\cup\{\omega\}\\
    \omega+2&=(\omega+1)^+=\omega\cup\{\omega\}\cup\{\omega+1\}\\
    &\vdots\\
    \omega\cdot2&=\omega+\omega=\bigcup\{\omega+n : n<\omega\}\\
    &\vdots\\
    \omega^2&=\omega\cdot\omega=\bigcup\{\omega^n:nz\omega\}\\
    &\vdots
\end{align*}
\begin{defn}[Ordinal]
  Un ordinal es el tipo de orden de un conjunto bien ordenado.
\end{defn}
\begin{defn}[Operaciones con conjuntos linealmente ordenados]
  Sean $A,B$ conjuntos linealmente ordenados, con tipos de orden $\alpha$ y $\beta$ respectivamente.
  \begin{itemize}
      \item La unión ajena de $A$ y $B$ es $A\dot\cup B=(B\times\{0\})\cup(A\times\{1\})$.
      \item La suma de $A$ y $B$ es el conjunto linealmente ordenado $B+A=(B\dot\cup A)$ junto con la relación $\leq$ definida como
      $$(x,i)\leq(y,j)\iff\begin{cases}
          i<j\\
          \text{o}\\
          i=j \ \text{y} \ x\leq y
      \end{cases}$$
      Nótese que si $A$ y $B$ son bien ordenados, $B+A$ también lo es. Más aún, si $A\simeq A^\prime$ y $B\simeq B^\prime$, entonces $B+A\simeq B^\prime+A^\prime$. Por lo tanto, se puede definir el ordinal
      $$\alpha+\beta=\iota(B+A)$$
  \end{itemize}
\end{defn}
Ya que se ha definido la suma de ordinales, vale la pena hacer algunas observaciones con los ordinales conocidos.
\begin{exa}[Algunas propiedades de la suma de ordinales]
    \leavevmode
   \begin{itemize}
       \item Si $A$ y $B$ son conjuntos finitos, la suma de los ordinales $\alpha$ y $\beta$ funciona como la suma usual de números naturales.
       \item $1+\omega=\omega$
       \item $\omega+1\neq\omega$
       \item $\eta+1\neq \eta \neq 1+\eta$
       \item $\eta+\eta=\eta$
       \item $\zeta+\zeta\neq\zeta$
   \end{itemize} 
   También, para cualesquiera $\alpha,\beta,\gamma$ ordinales, se cumples que 
   $$(\gamma+\beta)+\alpha=\gamma+\beta+\alpha)$$
\end{exa}
\begin{defn}[Producto y potencias de ordinales]
  Es fácil ver que para dos conjuntos bien ordenados $A$ y $B$con tipos de orden $\alpha$ y $\beta$ respectivamente, el conjunto $B\times A$ es bien ordenado con un orden que compara los elementos de $B\times A$ entrada por entrada. Así, se puede definir el producto de ordinales
  $$\beta\alpha=\iota(B\times A)$$
  Es fácil ver que el producto de ordinales es asociativo, distributivo y además que 
  $$(\beta\alpha)^*=\beta^*\alpha^*$$
  Tomando en cuenta esta definición del producto, se pueden definir las potencias de un ordinal $\beta$ como
  \begin{align*}
      \beta^0&=1\\
      \beta^{a+1}&=\beta^{a}\beta
  \end{align*}
  para cualquier $a\in\mathbb{N}$.
\end{defn}
Nótese que bajo esta definición de potencia, se cumple que 
$$\omega^a+\omega^b=\omega^b$$
para cualesquiera $a,b\in\mathbb{N}$.
La clase de ordinales $\mathbb{O}rd$ es una clase no cardinable, y es bien ordenada.
\begin{defn}[Encaje bien ordenado]
Un encajebien ordenado entre dos conjuntos bien ordenados $A$ y $B$ es una función $f:A\to B$ si es monótona, inyectiva y se $f(A)\simeq A$ es una sección inferior de $B$.
Si existe un encaje bien ordenado $fA\to B$, se dice que $A$ es un sub-orden de $B$, o $A\trianglelefteq B$.
Es fácil ver que si $A\trianglelefteq B$ y también $B\trianglelefteq A$, entonces $A\simeq B$, con un isomorfismo único.
\end{defn}
\begin{exe}%[Yareli $\checkmark$ ]
  Sean $A, B$ conjuntos bien ordenados, y $f,g:A\to B$ dos encajes. Entonces $f=g$.
\end{exe}
\begin{proof}
Sea $a_0\in A$. Como $B$ es bien ordenado, los elementos $f(a_0)$ y $g(a_0)$ son comparables. Sin perder generalidad, supongamos que $f(a_0)\leq g(a_0)$.\\
Entonces $f(a_0)=g(a_1)\leq g(a_0)$ para algún $a_1\in A$ ya que $g[A]$ es sección inferior. Además $a_1\leq a_0$ ya que, en caso contrario, se tiene que $g(a_1)>g(a_0)$, lo cual es una contradiccion.\\
Notemos que $f(a_1)\in g[A]$ ya que $f(a_1)\leq f(a_0)$, es decir, existe un $a_2\in A$ tal que $f(a_1)=g(a_2)$. Repitiendo este proceso obtenemos las siguientes cadenas:
\[
\begin{tikzcd}[row sep=3mm]
                  &  &                                & g(a_0)               \\
a_0               &  & f(a_0) \arrow[r,equal]               & g(a_1) \arrow[u,no head]     \\
a_1 \arrow[u,no head]     &  & f(a_1) \arrow[u,no head] \arrow[r,equal]     & g(a_2) \arrow[u,no head]     \\
a_2 \arrow[u,no head]     &  & f(a_2) \arrow[u,no head] \arrow[r,equal]     & g(a_3) \arrow[u,no head]     \\
\vdots \arrow[u,no head]  &  & \vdots \arrow[u,no head]               & \vdots \arrow[u,no head]     \\
a_{n-2} \arrow[u,no head] &  & f(a_{n-2}) \arrow[u,no head] \arrow[r,equal] & g(a_{n-1}) \arrow[u,no head] \\
a_{n-1} \arrow[u,no head] &  & f(a_{n-1}) \arrow[u,no head] \arrow[r,equal] & g(a_n) \arrow[u,no head]     \\
a_{n} \arrow[u,no head]   &  & f(a_n) \arrow[u,no head] \arrow[r,equal]     & g(a_{n+1}) \arrow[u,no head] \\
a_{n+1} \arrow[u,no head] &  & f(a_{n+1}) \arrow[u,no head] \arrow[r,equal] & g(a_{n+2}) \arrow[u,no head] \\
\vdots \arrow[u,no head]  &  & \vdots \arrow[u,no head]               & \vdots \arrow[u,no head]    
\end{tikzcd}
\]
Como $A$ es bien ordenado, la cadena de la izquierda tiene un mínimo. Si $a_n$ es el mínimo de esta cadena, entonces $a_n=a_{n+1}$.
Tenemos que
\begin{align*}
a_{n+1}=a_n&\Rightarrow f(a_n)=g(a_{n+1})=g(a_n)=f(a_{n-1})\\
\Rightarrow a_n=a_{n-1}, \textit{ por inyectividad de f}&\Rightarrow f(a_{n-1})=g(a_n)=g(a_{n-1})=f(a_{n-2})\\
\Rightarrow a_{n-1}=a_{n-2},\textit{ por inyectividad de f}&\Rightarrow f(a_{n-2})=g(a_{n-1})=g(a_{n-2})=f(a_{n-3})\\
&\vdots\\
\Rightarrow a_2=a_1,\textit{ por inyectividad de f}&\Rightarrow f(a_1)=g(a_2)=g(a_1)=f(a_0)\\
\Rightarrow a_1=a_0&\Rightarrow g(a_0)=g(a_1)=f(a_0)
\end{align*}
Como el elemento $a_0\in A$ es arbitrario, concluimos que $f$ y $g$ tienen la misma regla de correspondencia, es decir, $f=g$.
\end{proof}
