\chapter{Gavillas}

Recordaremos las noción de gavilla en el contexto
de espacios topológicos y veremos cómo generalizar esto
al contexto de marcos.

%\section*{SESIÓN 24: 7 DIC}
\section{Gavillas sobre espacios topológicos}

\begin{definition}
    Sean $E$, $S$ espacios topológicos
    y $f:E\to S$ una función continua.
    Una sección local de $f$ es una función continua $\sigma:U\to E$,
    donde $U$ es un abierto de $S$, tal que $f\sigma=\id_U$.
    Decimos que una sección $\sigma:U\to E$ de $f$ es global si $U=S$.
\end{definition}

Las secciones locales de una función continua $f:E\to S$
nos proporcionan información acerca del espacio $E$
y la manera en la que éste yace sobre $S$.
¿En qué condiciones se puede reconstruir $f:E\to S$ si solo
conocemos sus secciones?

\begin{definition}
    Sean $E$, $S$ espacios topológicos
    y $f:E\to S$ una función continua.
    Decimos que $f$ es étale (o un homeomorfismo local) si,
    para cada punto $e\in E$ existen vecindades
    $U_e\in\cal OE$ de $e$ y $V_{f(e)}\in\cal OE$ de $f(e)$,
    tales que la restricción
    $f_e:U_e\to V_{f(e)}$ de $f$ es un homeomorfismo.
\end{definition}

Veremos que la respuesta a la pregunta anterior
es que una función $f:E\to S$ está completamente determinada
por su secciones exactamente cuando $f$ es étale.
Para mostrar esto, pasaremos por el concepto de gavilla.

\begin{definition}
    Consideremos un espacio topológico $S$.
    Una pregavilla sobre $S$ es un funtor $(\cal OS)^\op\to\Con$.
\end{definition}

Por ejemplo, tomemos un espacio ``objetivo'' $Y$ y,
para cada abierto $U\subseteq S$,
consideremos los conjuntos
\begin{align*}
    CU &= \{f:U\to Y\mid f \text{ es continua }\}, \\
    KU &= \{f:U\to Y\mid f \text{ es constante }\}.
\end{align*}
Dada una inclusión $U\subseteq V$,
tenemos funciones de restricción $CV\to CU$ y $KV\to KU$
dadas por $f\mapsto f|_U$.
Estas asignaciones son funtoriales,
así que $C$ y $K$ son pregavillas sobre $S$.

Una pregavilla en $S$ asigna información local, abierto por abierto.
Queremos imponer una condición de tal modo que nos permita
convertir esa información local en información global.

\begin{definition}
    Sean $S$ un espacio topológico y $F:(\cal OS)^\op\to\Con$
    una pregavilla sobre $S$.
    
    Además, sean $U\in\cal OS$ un abierto y
    $\cal U = (U_\alpha\mid\alpha\in \Gamma)$
    una cubierta abierta de $U$.
    
    Decimos que una familia
    $(f_\alpha\in FU_\alpha\mid\alpha\in\Gamma)$
    subordinada a la cubierta $\cal U$
    es una familia compatible si se cumple
    \[
        f_\alpha|_{U_\alpha\cap U_\beta}
        =
        f_\beta|_{U_\alpha\cap U_\beta}
    \]
    para cualesquiera $\alpha,\beta\in\Gamma$.
    
    En otras palabras: las inclusiones
    $U_\alpha\cap U_\beta\to U_\alpha$
    y $U_\alpha\cap U_\beta\to U_\beta$ inducen
    dos funciones
    \[
        \prod_{\gamma\in\Gamma}FU_\gamma
        \rightrightarrows
        \prod_{\alpha,\beta\in\Gamma}F(U_\alpha\cap U_\beta)
    \]
    que mandan $(f_\gamma\mid\gamma\in\Gamma)$
    a $(f_\alpha|_{U_\alpha\cap U_\beta}\mid\alpha,\beta\in\Gamma)$
    y $(f_\beta|_{U_\alpha\cap U_\beta}\mid\alpha,\beta\in\Gamma)$,
    respectivamente.
    Entonces una familia
    $(f_\gamma\mid\gamma\in\Gamma)\in\prod_{\gamma\in\Gamma}FU_\gamma$
    es compatible si sus imágenes bajo estas dos funciones coinciden.
\end{definition}

\begin{definition}
    Sea $S$ un espacio topológico.
    Decimos que una pregavilla en $S$
    \[
        F:(\cal OS)^\op\to\Con
    \]
    es una gavilla si,
    para cualquier abierto $U\in\cal OS$, cualquier cubierta abierta
    $\cal U=(U_\alpha\mid \alpha\in\Gamma)$
    y cualquier familia compatible
    $(f_\alpha\in FU_\alpha\mid \alpha\in\Gamma)$,
    existe un único $f\in FU$ tal que
    \[
        f|_\alpha = f_\alpha
    \]
    para todo $\alpha\in\Gamma$.
    
    En otras palabras, $F$ es una gavilla si,
    para cualquier abierto $U$ de $S$ y
    cualquier cubierta abierta
    $(U_\alpha\mid\alpha\in\Gamma)$ de $U$,
    el diagrama
    \[
        FU
        \to
        \prod_{\gamma\in\Gamma} FU_\gamma
        \rightrightarrows
        \prod_{\alpha,\beta\in\Gamma}F(U_\alpha\cap U_\beta)
    \]
    es un igualador.
\end{definition}

Con estas herramientas,
podemos demostrar que los morfismos étales
están completamente determinados por sus secciones.

De hecho, demostraremos un resultado más fuerte,
y la afirmación anterior será un corolario.

Por un lado, sabemos que los funtores $(\cal OS)^\op\to\Con$
y las transformaciones naturales entre éstos
forman una categoría, a la cual denotamos como $\Gav(S)$.

\begin{definition}
    Dado un espacio topológico $S$, definimos la categoría
    $\Et(S)$ como sigue:
    \begin{itemize}
        \item
        Los objetos de $\Top/S$ son pares $(E,f)$,
        donde $E$ es un espacio topológico
        y $f:E\to S$ es una función étale.
        \[
            \begin{tikzcd}
                E \ar[d,"f"] \\ S
            \end{tikzcd}
        \]
        \item
        Dados objetos $(E,f)$, $(E',f')$ de $\Et(S)$,
        un morfismo $g:(E,f)\to(E',f')$ es una función
        continua $g:E\to E'$ tal que el siguiente diagrama conmuta
        \[
            \begin{tikzcd}
                E \ar[dr,"f"'] \ar[rr,"g"] && E' \ar[dl,"f'"]
                \\ & S
            \end{tikzcd}
        \]
    \end{itemize}
    Esta es, en efecto, una categoría,
    con la composición y la identidad
    heredadas de $\Top$.
    
    Nótese que, si en la definición de $\Et(S)$
    no pedimos que la función continua $f:E\to S$ sea coherente,
    seguimos obteniendo una categoría,
    a la cual denotamos como $\Top/S$.
\end{definition}

Ahora ¿qué relación hay entre gavillas y morfismos étales?

Notemos que, dado un objeto $(E,f)$ de $\Et(S)$,
tenemos una pregavilla $\Gamma_f$ en $S$ que,
a cada abierto $U\subseteq S$, le asigna las secciones locales
de $f$ con dominio $U$:
\[
    \Gamma_f U = \{\sigma:U\to E\mid f\sigma = \id_U\}
\]
y, a cada contención $U\subseteq V$, le asigna
la función de restricción $\Gamma V\to\Gamma U$.
De hecho, la pregavilla $\Gamma_f$ asociada a $(E,f)$ es una gavilla,
pues las funciones continuas se pueden pegar a lo largo de abiertos.
(Nótese que no usamos que $f$ fuera étale, por lo cual esta
construcción sigue funcionando en $\Top/S$).
Más aún, dado un morfismo $g:(E,f)\to(E',f')$ en $\Et(S)$,
obtenemos una transformación natural $g_*:\Gamma_f\to\Gamma_{f'}$
dada como $g_*(\sigma)=g\sigma$ pues,
dados abiertos $U\subseteq V$ de $S$, el diagrama
\[
    \begin{tikzcd}
        \Gamma_fV \ar[d,"g_*"'] \ar[r] & \Gamma_fU \ar[d,"g_*"] \\
        \Gamma_{f'}V \ar[r] & \Gamma_{f'}U
    \end{tikzcd}
\]
es conmutativo.

Luego, obtenemos un funtor $\Gamma:\Et(S)\to\Gav(S)$.
El resultado principal es el siguiente teorema.

\begin{theorem}
    El funtor $\Gamma:\Et(S)\to\Gav(S)$ es una equivalencia.
\end{theorem}
\begin{proof}[Bosquejo de la demostración]
    La demostración se hace construyendo el funtor
    $\Gav(S)\to\Et(S)$ inverso a $\Gamma$.
    Es decir, dada una gavilla $F$ en $U$,
    debemos construir un espacio topológico $E$ y un morfismo
    étale $f:E\to S$ tal que $\Gamma_f\simeq F$.
    
    Primero construiremos $E$ fibra a fibra.
    Dado $s\in S$, ponemos
    \[
        F_s = \Big(\bigsqcup_{U\in\cal OS,s\in U} FU\Big)/\simr_s
    ,\]
    donde $\sim_s$ es la relación de equivalencia en
    $\bigsqcup_{U\in\cal OS,s\in U}FU$ dada,
    para cualesquiera $\sigma\in FU$, $\sigma'\in F(U')$, como
    $\sigma\sim_s\sigma'$ si, y solo si, existe un vecindad abierta
    $U_s\subseteq U\cap U'$ de $s$
    tal que $\sigma|_{U_s}=\sigma'|_{U_s}$.
    A la clase de equivalencia de $\sigma\in FU$ bajo la relación
    $\sim_s$ la denotamos como $[\sigma]_s$.
    
    Después ponemos $T(F)=\bigsqcup_{s\in S}F_s$
    y definimos la proyección $f_F:T(F)\to S$
    mandando $[\sigma]_s\in T(F)$ a $s\in S$.
    Equipamos a $T(F)$ con la topología generada por los básicos
    \[
        B(\sigma,U) = \{[\sigma]_s\mid s\in U\}
    \]
    siempre que $\sigma\in FU$ y $U\in\cal OS$.
    Finalmente, se muestra que $f_F:T(F)\to S$ es étale
    y que esta construcción es inversa a $\Gamma$.
\end{proof}

\section{Gavillas sobre marcos}
Ahora, ¿cómo hacemos esto sobre un marco?
En realidad, la teoría se puede usar casi sin modificación.

\begin{definition}
    \leavevmode
    \begin{itemize}
        \item
        Una pregavilla sobre un marco $A$
        es un funtor $F:A^\op\to\Con$.
        Si $x\leq y\in A$, denotamos la función
        $Fy\to Fx$ inducida por $F$ como $f\mapsto f|_x$.
        \item
        Una pregavilla $F:A^\op\to\Con$ es separada si,
        para cualesquiera $X\subseteq A$ y $f,g\in F(\Sup X)$,
        se tiene
        \[
            (\forall x\in X,\; f|_x=g|_x)\implies f=g
        .\]
        \item
        Sea $F$ una pregavilla sobre $A$.
        Dado un subconjunto $X\subseteq A$,
        una familia $(f_x\in Fx\mid x\in X)$
        es compatible si $f_x|_{x\inf y}=f_y|_{x\inf y}$
        para cualesquiera $x,y\in X$.
        \item
        Una pregavilla $F:A^\op\to\Con$ es cotejada si,
        para cualquier subconjunto $X\subseteq A$
        y cualquier familia compatible
        $(f_x\in Fx\mid x\in X)$
        existe un $f\in F(\Sup X)$ tal que
        $f|_x=f_x$ para todo $x\in X$.
        \item
        Una pregravilla $F:A^\op\to\Con$ es una gavilla
        si es separada y cotejada.
    \end{itemize}
\end{definition}

\begin{example}
    Sea $\Omega$ un marco.
    Dado $a\in\Omega$, consideremos
    \begin{align*}
        \Omega(a)
            &= \cal L(\down a)
            \subseteq\cal L\Omega, \\
        \Omega\<a\>
            &= \{X\in\cal L(\down a)\mid\Sup X=a\}
            \subseteq\cal L\Omega, \\
        \Omega[a]
            &= \down a
            \subseteq\Omega.
    \end{align*}
    Cada una de estas asignaciones define una pregavilla,
    donde las restricciones están dadas, para $a\leq b$, como
    \begin{align*}
        X &\mapsto \{x\inf b\mid x\in X\}, \\
        X &\mapsto \{x\inf b\mid x\in X\}, \\
        x &\mapsto x\inf b,
    \end{align*}
    respectivamente.
    Se verifica que, en general,
    \begin{itemize}
        \item $\Omega(\cdot)$ es cotejada pero no separada,
        \item $\Omega\<\cdot\>$ es separada pero no cotejada y
        \item $\Omega[\cdot]$ es gavilla.
    \end{itemize}
\end{example}

Dada una pregavilla $F$ en un marco $\Omega$,
podemos obtener una gavilla $F^+$ en $\Omega$ como sigue.

Para cada $a\in\Omega$, ponemos
\begin{itemize}
    \item
    $\<a\> = \{X\subseteq\Omega\mid\Sup X=a\}$.
    \item
    $F\<a\>
    = \{(f_x\in Fx\mid x\in X)\text{ es compatible}\mid X\in\<a\>\}$
    \item
    $F^+(a)=F\<a\>/\simr$,
    donde $(f_x\in Fx\mid x\in X)\sim(g_y\inf Fy\mid y\in Y)$
    si existe $Z\in\<a\>$, $Z\subseteq X\cap Y$ tal que
    $af_z=g_z$ para todo $z\in Z$.
\end{itemize}
Se puede demostrar que $F^+$ es una gavilla y, más aún,
la asignación $F\mapsto F^+$ se extiende a un funtor
que resulta ser adjunto izquierdo
de la inclusión $\Gav(\Omega)\to[\Omega^\op,\Con]$
de las gavillas en las pregavillas.

Sin embargo, estas construcciones son muy largas y complicadas.
Se puede trabajar con las gavillas en un marco a través
de una categoría equivalente a $\Gav(\Omega)$ llamada la categoría
de $\Omega$-conjuntos y denotada como $\Con(\Omega)$,
pero para desarrollar esta teoría se necesitaría más tiempo.

