\chapter{Productos y coproductos}
%\section*{SESIÓN 20: 23 NOV (Expo Armando 1)}
\section{Productos de marcos}

\begin{theorem}[Producto de marcos]
    Sea $\{A_\lambda\}_{\lambda\in\mathscr{I}}$ una familia arbitraria
    de marcos y sea $L$ el producto cartesiano de los $A_\lambda$
    (vistos como conjuntos).
    Entonces $L$, dotado con los operadores puntuales $\inf,\Sup$:
    \begin{align*}
      (a_\lambda)_{\lambda\in\mathscr I}
      \inf
      (b_\lambda)_{\lambda\in\mathscr I}
      &=
      (a_\lambda \inf b_\lambda)_{\lambda\in\mathscr I} \\
      \Sup X
      &= \left(\Sup\{a_\lambda \mid a\in X\}\right)_{\lambda\in\mathscr I}
    \end{align*}
    es un marco y, equipado con las proyecciones canónicas
    \begin{align*}
        p_\lambda: L&\to A_\lambda \\
        a &\mapsto p_\lambda(a)=a_\lambda
    \end{align*}
    satisface la propiedad universal del producto de los $A_\lambda$.
\end{theorem}

\begin{lemma}
    $(L,\leq,\wedge, 0, \bigvee, 1)$, con
    $0:=(0_\lambda)_{\lambda\in\scr{I}}$ y
    $1:=(1_\lambda)_{\lambda\in\scr{I}}$ es un marco.
\end{lemma}
\begin{proof}
    $\leq$ es orden parcial ya que:
    \begin{align*}
        x \leq x
            & \iff x_\lambda \leq x_\lambda \forall \lambda\in\scr I. \\
        x \leq y, y \leq x
            &\implies x_\lambda \leq y_\lambda,
            y_\lambda \leq x_\lambda, \forall  \lambda\in\scr{I}\\
            &\implies x_\lambda = y_\lambda
            \forall\lambda\in\scr{I}\\
            &\implies x = y.\\
        x \leq y, y \leq z
            &\implies x_\lambda \leq y_\lambda,
            y_\lambda \leq z_\lambda
            \forall\lambda\in\scr{I}\\
            &\implies x_\lambda  \leq z_\lambda
            \forall  \lambda\in\scr{I}\\
            &\implies x \leq z.
    \end{align*}
\end{proof}
Para $x,y\in A$, ya que $x_\lambda\wedge_\lambda y_\lambda\in A_\lambda$ para todo $\lambda\in\mathscr{I}$, entonces $x\wedge_\mathscr{I}y\in A$.\\
Luego, ya que $0_\lambda\leq_\lambda x_\lambda$ para todo $x_\lambda\in A_\lambda$ y para cada $\lambda\in\mathscr{I}$, entonces $0_\mathscr{I}\leq_\mathscr{I}x$ para todo $x\in A$.\\
Similarmente, para $X\subseteq A$ arbitrario, consideramos los subconjuntos $X_\lambda\subseteq A_\lambda$ formados por los $x_\lambda$ componentes, y tomamos $\bigvee X_\lambda=:s_\lambda\in A_\lambda$,  para cada $\lambda\in\mathscr{I}$.\\
Entonces $\bigvee_\mathscr{I}X=(s_\lambda)_{\lambda\in\mathscr{I}}\in A$.\\
Además, como $1_\lambda\geq_\lambda x_\lambda$ para todo $x_\lambda\in A_\lambda$, para cada $\lambda\in\mathscr{I}$, entonces $1_\mathscr{I}\geq_\mathscr{I} x\ \forall x\in A$.
Por útlimo, se cumple que:
\begin{align*}
    a\wedge_\mathscr{I}\left(\bigvee_\mathscr{I} X\right) & = (a_\lambda\wedge_\lambda s_\lambda)_{\lambda\in\mathscr{I}}\\
    & = \left(\bigvee_\lambda\{a_\lambda\wedge_\lambda x_\lambda\mid x_\lambda\in X_\lambda\}\right)_{\lambda\in\mathscr{I}}\\
    & = \bigvee_\mathscr{I} \{a\wedge_\mathscr{I}x\mid x\in X\}
\end{align*}
Y con esto, se cumple que $A$ es un marco.
    Para cada $\lambda\in\mathscr{I}$, se define $p^*_\lambda:A\to A_\lambda$ la proyección de $A$ sobre el respectivo componente $A_\lambda$, nótese el siguiente resultado:
    \begin{lemma}
        $p^*_\lambda$ es morfismo de marcos, para cada $\lambda\in\mathscr{I}$.
    \end{lemma}
        \begin{proof}
            \begin{align*}
                x\leq y & \Longleftrightarrow x_\lambda\leq y_\lambda \ \forall\lambda\in\mathscr{I}\\
                & \Rightarrow p^*_\lambda(x)\leq p^*_\lambda(y)\ \forall\lambda\in\mathscr{I} 
            \end{align*}
            Además se cumple que $p^*_\lambda(1_\mathscr{I})=1_\lambda$ y $p^*_\lambda(0_\mathscr{I})$.\\
            Finalmente:
            \begin{align*}
                p^*_\lambda(x\wedge y) & = p^*_\lambda[(x_\mu\wedge y_\mu)_{\mu\in\mathscr{I}}]\\
                & = x_\lambda\wedge y_\lambda\\
                & = p^*_\lambda(x)\wedge p^*_\lambda(y).
            \end{align*}
        \end{proof}
\begin{align*}
    p^*_\lambda(\bigvee X) & = p^*_\lambda[(s_\mu)_{\mu\in\mathscr{I}}] \\
                           & = s_\lambda \\
                           & = \bigvee X_\lambda \\
                           & = \bigvee p^*_\lambda[X].
\end{align*}
\begin{proof}
    Sea $B\in\Frm$, y considere una familia arbitraria de morfismos:
    \begin{equation*}
        \{r^*_\lambda:B\to A_\lambda\mid\lambda\in\mathscr{I}\}.
    \end{equation*}
    Definimos $p^*:B\to A$, de manera que, para $b\in B$:
    \begin{equation*}
        p^*(b) = (r^*_\lambda(b))_{\lambda\in\mathscr{I}}.
    \end{equation*}
\end{proof}
Se sigue que $p^*$ es morfismo de marcos, ya que la familia de $r^*_\lambda$ son morfismos de marcos, y por la construcción de los operadores de $A$.\\
Luego, para todo $b\in B$ y todo $\lambda\in\mathscr{I}$, se cumple que:
\begin{equation*}
    (p^*_\lambda\circ p^*)(b) = p^*_\lambda[(r_\mu(b))_{\mu\in\mathscr{I}}] = r_\lambda(b),
\end{equation*}
y nótese que $p^*$ es único por construcción. Así, se concluye que $A$ satisface la propiedad universal del producto en $\Frm$.

\section{Sitios}

Ya probamos que la categoría de marcos tiene productos.
Para probar que también tiene coproductos,
usaremos una técnica distinta, para la cual necesitamos
algunos resultados concernientes a sitios.

\begin{definition}
    Sea $A$ una semírretícula inferior.
    Una función $C:A\to\mathcal{P}[\mathcal{P(A)}]$
    es una \textit{cobertura} o \textit{función de cubiertas}
    sobre $A$ si,
    para cualesquiera $a,b\in A$, se cumple que:
    \begin{enumerate}
        \item $S\in C(a) \implies S\subseteq\down(a)$
        \item $S\in C(a), b\leq a \implies \{b\wedge s\mid s\in S\}\in C(b)$.
    \end{enumerate}
    Al par $(A,C)$ se le llama sitio.
\end{definition}
\begin{definition}[$C$-ideales]
    Dado un sitio $(A,C)$, un subconjunto
    $I\subset A$ es un $C$-ideal de $A$ si
    \begin{enumerate}
        \item $I\in\mathcal{L}(A)$
        \item Siempre que $a\in A$ y $S\in C(a)$, entonces
        $S\subseteq I\Rightarrow a\in I$.
    \end{enumerate}
    Al conjunto de todos los $C$-ideales del sitio $(A,C)$
    se le denota por $C\Idl$.
\end{definition}
\begin{example}[Ejemplos de sitios]
    Sea $A$ una semirretícula inferior.
    Consideremos las funciones cubrientes 
    $C_\emptyset$ y $C_T$ definidas como:
    \begin{align*}
        C_\emptyset(a) &= \emptyset \\
        C_T(a) &= \{\emptyset\}
    \end{align*}
    para todo $a\in A$.
    \begin{itemize}
        \item
        En el primer caso,
        ninguna famila cubre a ningún elemento.
        Entonces, por vacuidad,
        toda sección inferior $F\subseteq A$ es
        un $C_\emptyset$-ideal.
        Es decir, $C_\emptyset\Idl(A)=\cal L(A)$.
        \item
        En el segundo caso,
        la familia vacía cubre a todos los elementos,
        así que el único $C_T$-ideal $F\subseteq A$
        es la sección total $F=A$.
        Es decir, $C_T\Idl(A)=\{A\}$.
    \end{itemize}
\end{example}

\begin{example}[Más ejemplos de sitios]
    Sea $A$ un marco y consideremos las funciones cubrientes
    $C_\sup$ y $C_{\Sup}$ definidas como
    \begin{align*}
        X\in C_\sup(a)
        &\ssi
        \text{$X$ es finito y } a=\Sup X \\
        X\in C_{\Sup}(a)
        &\ssi
        a=\Sup X 
    \end{align*}
    para todo $a\in A$ y todo $X\subseteq A$.
    Entonces
    \begin{itemize}
        \item
        una sección inferior $F\subseteq A$ es un
        $C_\sup$-ideal si, y solo si es cerrada bajo supremos
        finitos.
        Por lo tanto, los $C_\sup$-ideales de $A$
        son los ideales de $A$ en el sentido de retículas
        distributivas:
        \[
            C_\sup\Idl(A) = \cal I A.
        \]
        \item
        una sección inferior $F\subseteq A$ es un
        $C_{\Sup}$-ideal si, y solo si es cerrada bajo supremos
        arbitrarios.
    \end{itemize}
\end{example}

\begin{definition}[Marco generado por un sitio]
    Sea $(A,C)$ un sitio, $B$ un marco y $f:A\to B$ un morfismo de
    semirretículas inferiores. Decimos que el sitio genera al marco
    $B$ a través de $f$ si $f$ manda $C$-cubiertas en $A$ a
    $C_{\Sup}$-cubiertas de $B$:
    \begin{equation*}
      \forall a\in A,\quad
        S\in C(a) \implies f(a) = \Sup\{f(s)\mid s\in S\}
    \end{equation*}
    y la pareja $(B,f)$ es inicial con respecto a esta propiedad.
    En este caso, ser inicial significa que,
    si $B'$ es un marco y $f':A\to B'$ es un morfismo
    de $\inf$-semiretículas
    que convierte cubiertas en supremos,
    entonces existe un único morfismo de marcos
    $\ol{f}:B\to B'$ tal que el siguiente diagrama conmuta:
    \[
        \begin{tikzcd}[ampersand replacement=\&]
            \& B \arrow[dd, "\exists!\overline{f}", dashed] \\
            A \arrow[ru, "f"] \arrow[rd, "f'"'] \& \\
            \& B'                                          
        \end{tikzcd}
    \]
    Por el argumento usual, entre cualesquiera dos
    marcos $(B,f)$, $(B',f')$ generados por $(A,C)$ existe un único
    isomorfismo que conmuta con $f$ y $f'$,
    de modo que podemos hablar de \emph{el} marco generado
    por $(A,C)$.
\end{definition}

Ahora veremos que, dado un sitio $(A,C)$, los $C$-ideales forman un
marco que está generado por $(A,C)$. Así, obtenemos una construcción
canónica del marco generado.

\begin{theorem}[$C\Idl$ es un marco]
    Sea $(A,C)$ un sitio. Entonces el conjunto $C\Idl$ de los
    $C$-ideales de $A$ es un marco (con el orden de contención) y es
    generado por $(A,C)$.
\end{theorem}

Recordemos que $\mathcal{L}A$ es un marco bajo la contención
y que, si $B$ es un marco y $\nu\in NA$ es cualquier núcleo,
entonces $\nu[A]=A_\nu$ es marco con el orden heredado de $A$.

\begin{lemma}[Previo]
    La intersección arbitraria de $C$-ideales es un $C$-ideal.
\end{lemma}
\begin{proof}
    Sea $\{I_\alpha\}_{\alpha\in\Gamma}$ una familia arbitraria de
    $C$-ideales. La intersección $I=\bigcap_{\alpha\in\Gamma}I_\alpha$
    es sección inferior porque cada $I_\alpha$ lo es.
    Ahora, sea $a\in A$ y tomemos una familia cubriente $S\in C(a)$
    con $S\subseteq I$. Entonces se cumple que $S\subset I_\alpha$
    para todo $\alpha\in\Gamma$. Ya que cada $I_\alpha$ es $C$-ideal,
    se sigue que $a\in I_\alpha\ \forall\alpha\in\Gamma$. Luego, $a\in I$.
    Concluimos que $I$ es un $C$-ideal.
\end{proof}

\begin{lemma}[Marco generado, parte 1]
    Existe un núcleo $j:\cal LA\to\cal LA$ tal que
    $C\Idl=(\cal LA)_j$, de modo que $C\Idl$ es un cociente
    de $\cal LA$.
    En particular, $C\Idl$ es un marco.
\end{lemma}
\begin{proof}
  Por el lema anterior, $C\Idl$ es un conjunto $\Inf$-cerrado de $\cal
  LA$ (ya que el ínfimo en $\cal LA$ es la intersección). Por lo
  tanto (ver lema \ref{lemma:cerraduras-y-conjuntosfijos}), la función
  $j:\mathcal{L}A\to\mathcal{L}A$ definida como
    \begin{equation*}
        j(S)=\bigcap\{I\in C\Idl\mid S\subseteq I\}\quad S\in\mathcal{L}A.
    \end{equation*}
  es un operador cerradura, i.e. es monótono, idempotente e
  inflacionario.
    
    Ahora, sean $R,T\in\mathcal{L}A, I:=j(R\cap T)$. Por definición de $j$, se tiene que $R\subseteq j(R)$ y $T\subseteq j(T)$, entonces $R\cap T\subseteq j(R)\cap j(T)$, y cómo $j$ es idempotente, se sigue que $j(R\cap T)\subseteq j(R)\cap j(T)$.\\
    Defináse ahora el conjunto
    \begin{equation*}
        R' := \{d\in A\mid \forall t\in T,d\wedge t\in I\}.
    \end{equation*}
    Por la definición de $I$ y ya que $j$ infla, se cumple que
    $R\subseteq R'$, $T\cap R'\subset I$. Por otro lado, cómo
    $I\in\mathcal{L}(A)$, obtenemos que $R'\in\mathcal{L}(A)$. Luego,
    consideremos $U\in C(A)$ tal que $U\subseteq R'$, entonces, para
    todo $t\in T$, se cumple que $\{u\wedge t\mid u\in U\}\subseteq
    I$, y además, utilizando la propiedad $(ii)$ de $C$, se cumple que
    $\{u\wedge t\mid u\in U\}\in C(a\wedge t)$ , y en consecuencia
    $a\wedge t\in I$, ya que $I$ es $C$-ideal. Luego, cómo $t\in T$ es
    arbitrario, tenemos que $a\in R'$ y por tanto $R'$ es un
    $C$-ideal.
    
    Análogamente, se construye el $C$-ideal
    \begin{equation*}
         T' := \{e\in A\mid \forall r\in R',e\wedge r\in I\},
    \end{equation*}
    y se cumple que $T\subseteq T'$, $T'\cap R\subseteq I$ y $T'$ es $C$-ideal. Así, se sigue que $j(R)\subseteq R',j(T)\subseteq T'$ y por tanto
    \begin{equation*}
        j(R)\cap j(T)\subseteq R'\cap T' \subseteq  I = j(R\cap T)
    \end{equation*}
    y en conclusion, $j$ es núcleo cuyo conjunto de puntos fijos
    es $C\Idl$.
\end{proof}

%\section*{SESIÓN 21: 25 NOV (Expo Armando 2, Yareli, Dante, Alfredo)}

\begin{lemma}[Marco generado, parte 2]
    La función $f:A\to C\Idl$ definida como
    \begin{equation*}
        f(a)=j(\down(a))
    \end{equation*}
    es un morfismo de $\inf$-semiretículas que
    manda $C$-cubiertas en supremos de $C\Idl$.
    Más aún, $f$ es universal con respecto a esta propiedad,
    de modo que $C\Idl$ es el marco generado por $(A,C)$.
\end{lemma}

\begin{proof}
    Cómo $j$ es núcleo y $\down$ es morfismo
    de $\inf$-semiretículas,
    se cumple que $f$ es morfismo de $\inf-$retículas.
    Sean $a\in A,S\in C(a)$ arbitrarios. Entonces, cómo $j(\bigcup\{j(\down(s))\mid s\in S\})=:\mathcal{J}$ es un $C-$ideal que contiene a $S$, se sigue que $a\in\mathcal{J}$, luego 
    \begin{equation*}
        j(\down(a))\subseteq \mathcal{J}.
    \end{equation*}
    Por otro lado, se tiene $a\geq s$ para todo $s\in S$, por lo cuál $j(\down(a))\supseteq j(\down(s))$. Luego, se sigue que
    \begin{equation*}
        j(\down(a))\supseteq \bigcup\{j(\down(s))\mid s\in S\}
    \end{equation*}
    y con esto, cómo $j$ es mónotona e idempotente, obtenemos que
    \begin{equation*}
        f(a)\supseteq\mathcal{J}.
    \end{equation*}

    Por otra parte, sean $B\in\Frm$, $g:A\to B$ un morfismo de $\wedge-$semiretículas que convierte cubiertas de $C$ en supremos. Entonces, la función 
    \begin{align*}
        \overline{g}: &\mathcal{L}A\to B \\
                      & S\to \bigvee_B\{g(s)\mid s\in S\}
    \end{align*}
    es el único morfismo de marcos que factoriza a $g$ a tráves del marco libre de $A$, $\mathcal{L}A$, y además existe su adjunto derecho $g_*:B\to\mathcal{L}A$, que, por definición del adjunto derecho, para $b\in B:$

    \begin{align*}
        g_*(b) & = \bigcup\{L\in\mathcal{L}A\mid\overline{g}(L)\leq b\} \\
               & = \bigcup\left\{L\in\mathcal{L}A\mid \bigvee_B\{g(a)\mid a\in L\}\leq b\right\} \\
               & = \bigcup\{L\in\mathcal{L}A\mid g(a)\leq b (\forall a \in L)\} \\
               & = \{a\in A\mid g(a)\leq b\} 
    \end{align*}

    Y nótese que, si $S\in C(a)$ y $S\subseteq g_*(b)$, entonces se cumple que
    \begin{equation*}
        g(a)= \bigvee_B\{g(s)\mid s\in S\}\leq b
    \end{equation*}
    y esto implica que $a\in g_*(b)$, por tanto $g_*(b)$ es un $C-$ideal. 

    Sea $a\in A$ arbitrario. Cómo $\overline{g}$ y $g_*$ son adjuntos, se tiene que
    \begin{equation*}
        (\overline{g}\circ g_*)(g(a))\leq g(a).
    \end{equation*}
    Además, cómo $g_*(g(a))$ es un $C-$ideal que contiene a $\down(a)$, tenemos que $$f(a)=j(\down(a))\subseteq g_*(g(a))$$, y en consecuencia, se cumple la cadena de desigualdades:
    \begin{equation*}
        g(a)\leq\overline{g}(f(a))\leq(\overline{g}\circ g_*)(g(a))\leq g(a)
    \end{equation*}

    Es decir, que $(\overline{g}\circ f)(a)=g(a)$
    con $a$ arbitrario,
    por tanto $\overline{g}\circ f = g$,
    con $\overline{g}$ único, así que $f$ es universal.
\end{proof}

\section{Coproductos de marcos}

Antes probamos que la categoría de marcos tiene productos.
Ahora veremos que también tiene coproductos.
La construcción será la siguiente:
Dada una familia de marcos $\{A_\lambda\}_{\lambda\in\mathscr I}$,
tomaremos el coproducto $A$ de los $A_\lambda$
en la categoría $\Pos^{\inf}$ de $\inf$-semiretículas. Luego
equiparemos esta semirretícula con una cobertura $C$, con lo cual
obtendremos un sitio $(A,C)$.
Finalmente, veremos que el marco $C\Idl$ generado por $(A,C)$
viene equipado con morfismos que lo convierten en el coproducto
de nuestra familia en la categoría de marcos.

\begin{lemma}[Coproducto de semirretículas]
  Sea $\{A_\lambda\}_{\lambda\in\mathscr I}$ una familia de
  semirretículas inferiores.
  Entonces la sub-semirretícula inferior del producto
  \[
    A=
    \left\{ a\in\prod_{\lambda\in\scr I} A_\lambda
    \mid a_\lambda\neq 1_{A_\lambda}
    \text{ para una cantidad finita de índices} \right\}
    \subseteq
    \prod_{\lambda\in\mathscr I} A_\lambda
  \]
  equipada con las funciones $q_\lambda\colon A_\lambda\to A$
  dadas, para cada $\lambda\in \scr I$, por
    \[
      q_\lambda(x)=a,
      \quad
      \text{donde}
      \quad
      a_\mu =
       \begin{cases}
         x & \mu=\lambda \\
         1_\mu & \mu\neq\lambda,
      \end{cases}
    \]
  es el coproducto de los $A_\lambda$ en la categoría de
  $\inf$-semirretículas.
\end{lemma}
\begin{proof}
    Notemos que $A$ es una subsemiretícula inferior del producto,
    ya que $1\in A$ y, si $a,b\in A$, entonces
    $c=a\inf b\in \prod_{\lambda\in\mathscr I} A_\lambda$
    satisface
    \[c_\lambda=
    \begin{cases}
        1_\lambda, & \textit{si }a_\lambda=b_\lambda=1_\lambda\\
        a_\lambda\wedge b_\lambda, & \textit{ en otro caso}
    \end{cases}
    \]
    Como hay una cantidad finita de
    $a_\lambda\neq 1_\lambda$ y $b_\lambda\neq 1_\lambda$,
    entonces los $a_\lambda\wedge b_\lambda$ son finitos,
    por lo que $c\in A$.
    
    Ahora, cada $q_\lambda$ es monótona y $q_\lambda(1_\lambda)=1$.
    Además, si $x,y\in A_\lambda$, entonces
    $q_\lambda(x\wedge y)=q_\lambda(x)\wedge q_\lambda(y)$,
    ya que las operaciones son puntuales.
    Así, cada $q_\lambda$ es un morfismo de semiretículas
    inferiores.
    
    Ahora sean $B$ una $\inf$-semiretícula y
    $r_\lambda:A_\lambda\to B$ una familia de morfismos
    indicada por $\lambda\in\scr I$.
    Definimos $R:A\to B$ como
    \[
        R(a) = \Inf\{r_\lambda(a_\lambda) \mid \lambda \in \scr I\}
    .\]
    Este ínfimo existe en $B$, ya que todo $a\in A$ tiene soporte
    finito.
    Claramente, $R$ preserva ínfimos y hace conmutar el diagrama
    \[
        \begin{tikzcd}
            A_\lambda \ar[r,"r_\lambda"] \ar[d,"q_\lambda"']
            & B \\
            A \ar[ur,"R"']
        \end{tikzcd}
    \]
    para todo $\lambda\in \scr I$.
    Finalmente, si $R':A\to B$ es cualquier morfismo
    que hace conmutar el diagrama, tenemos
    \begin{align*}
        R'(a)
        &= R'(\Inf\{q_\lambda(a_\lambda)\mid a\in \scr I\}) \\
        &= (\Inf\{R'(q_\lambda(a_\lambda))\mid a\in \scr I\}) \\
        &= (\Inf\{r_\lambda(a_\lambda)\mid a\in \scr I\}) \\
        &= R(a).
    \end{align*}
    De este modo, la semiretícula $A$, equipada con $R$,
    es el coproducto de las $A_\lambda$ en $\Pos^{\inf}$.
\end{proof}

Ahora, si $A_\lambda$ son marcos, equiparemos al coproducto $A$ (en la
categoría de semirretículas inferiores) con una cobertura $C$.
Necesitaremos la siguiente definición.

\begin{definition}[Sustituciones en el coproducto de semirretículas]
  Tomemos una familia de marcos $A_\lambda, \lambda\in\scr I$ y sea
  $A$ su coproducto como semirretículas inferiores. Fijemos un
  elemento $a\in A$.
  
  Dado un elemento $x\in A_\mu$ para un índice $\mu\in\scr I$ dado,
  denotamos como $a(x)$ al elemento de $A$ que tiene todas las
  entradas iguales a las de  $a$, excepto la entrada $\mu$-ésima,
  donde está $x$. Es decir,
  \[
    p_\mu(a(x)) = (a_x)_\lambda =
    \begin{cases}
      a_\lambda & \lambda\neq\mu \\
      x & \lambda = \mu.
    \end{cases}
  \]

  Dada una familia $S\subseteq A_\mu$ de un índice $\mu\in\scr I$
  dado, definimos el reemplazo de $a$ por $S$ en la entrada
  $\mu$-ésima como
  \[
    S(a,\mu)
    =
    \{a(x) \mid x\in S\}
  \]
\end{definition}
\begin{example}
Si $a=(a_1,a_2,a_3,a_4)\in\prod_{\lambda=1}^4A_\lambda$, $\mu=2$ y $S=\{x,y,z\}\subseteq A_2$, entonces
    \[
      S(a,\mu) = \left\{
      \begin{array}{c}
        (a_1,x,a_3,a_4), \\
        (a_1,y,a_3,a_4), \\
        (a_1,z,a_3,a_4)
      \end{array}
      \right\}
    .\]
\end{example}


\begin{lemma}[La cobertura en $A$ de supremos en una entrada]
  \label{lemma:cobertura-en-A}
    La función $C:A\to\cal P(\cal P(A))$
    definida para cada $a\in A$ como
    \[
        C(a)
        =\left\{ S(a,\mu)
        \mid \mu\in\scr I, S\subseteq A_\mu
        \text{ tal que }\Sup S=a_\mu\right \}
    \]
    es una cobertura en $A$.
\end{lemma}
\begin{proof}
    Sean $a=(a_\lambda),b=(b_\lambda)\in A$ con $b\leq a$ y $W=S(a,\mu)\in C(a)$.
    \begin{itemize}
    \item Sea $c=(c_\lambda)\in W$. Por definición, $c_\lambda=a_\lambda$ para $\lambda\in \scr I\setminus\{\mu\}$ y $c_\mu \in S$, entonces $c_\mu\leq a_\mu$. Esto implica que $c\leq a$, es decir, $c\in \down(a)$.
    \end{itemize}
    
    \begin{itemize}
    \item Probaremos que $\{b\inf w\mid w\in W\}=\{b_\mu \inf w_\mu\mid w_\mu\in S\}(b,\mu)$.\par 
    \begin{itemize}
    \item[$\subseteq)$] Sea $b\inf w\in\{b\inf w\mid w\in W\}$. Observamos que 
    \begin{itemize}
    \item $b\inf w\in A$.
    \item $b_\lambda\inf w_\lambda=b_\lambda\inf a_\lambda=b_\lambda$ para $\lambda\in\scr I\setminus \{\mu\}$.
    \item $b_\mu\inf w_\mu\in\{b_\mu\in w_\mu\mid w_\mu\in S\}$.
    \end{itemize}
    Por lo que $b\inf w\in \{b_\mu \inf w_\mu\mid w_\mu\in S\}(b,\mu)$.
    \item[$\supseteq)$] Sea $c=(c_\lambda) \in \{b_\mu \inf w_\mu\mid w_\mu\in S\}(b,\mu)$. Entonces $c_\lambda=b_\lambda$ para $\lambda\in\scr I\setminus\{\mu\}$ y $c_\mu\in \{b_\mu \inf w_\mu\mid w_\mu\in S\}$.\par 
    Notemos que $c=b\inf m$, donde $m_\lambda=a_\lambda$ para $\lambda\in\scr I\setminus \{\mu\}$ y $m_\mu=w_\mu$. Así, $m\in W$, es decir, $c\in\{b\in w\mid w\in W\}$.
    \end{itemize}
    Además, $\{b_\mu \inf w_\mu\mid w_\mu\in S\}\subseteq A_\mu$ y
    \[\Sup\{b_\mu \inf w_\mu\mid w_\mu\in S\}=b_\mu\inf \left(\Sup S\right)=b_\mu\inf a_\mu=b_\mu\]
    Por lo tanto $\{b\inf w\mid w\in W\}\in C(b)$.
    \end{itemize}
\end{proof}

La estructura de los $C$-ideales en $A$ no es tan sencilla
de entender a primera vista.
Incluimos un ejemplo de un cálculo en $C\Idl(A)$.

\begin{example}[El menor elemento de $C\Idl(A)$.]
    Supongamos que $a\in A$ tiene
    alguna entrada cero.
    Es decir, $a_\lambda=0\in A_\lambda$ para algún $\lambda$.
    
    Al fijarnos en la entrada $a_\lambda=0\in A_\lambda$,
    tenemos $\Sup\emptyset =0\in A_\lambda$.
    Como $\emptyset(a,\lambda)=\emptyset\subseteq A$,
    de la definición de las $C$-cubiertas
    se sigue que $\emptyset\in C(a)$;
    es decir: la familia vacía cubre a $a$.
    
    En particular, para cualquier $C$-ideal $F\subseteq A$,
    tenemos
    \[
        \emptyset \in C(a)
        \hspace{10mm} \text{y} \hspace{10mm}
        \emptyset\subseteq F
    ,\]
    y, por la definición de $C$-ideal, se sigue que $a\in F$.
    
    Esto nos dice que el conjunto $G$ de los elementos que tienen
    alguna entrada nula
    \[
        G=\{a\in A \mid a_\lambda = 0\in A_\lambda
        \text{ para algún } \lambda\in \scr I\}
    \]
    está contenido en todos los $C$-ideales.
    De hecho, probaremos que $G$ es un $C$-ideal.
    Luego, es el menor elemento en el marco de $C$-ideales.

    Claramente, $G$ es sección inferior, por lo cual resta
    probar que es cerrado bajo $C$-cubiertas.
    En efecto, tomemos una cubierta
    \[
        Y(a,\lambda) \in C(a)
        \hspace{10mm} \text{con} \hspace{10mm}
        Y(a,\lambda) \subseteq G
    .\]
    Es decir, $a$ tiene una entrada $a_\lambda$ tal que
    $Y\subseteq A_\lambda$ y $\Sup Y=a_\lambda$.
    
    Si $\Sup Y=0$, entonces $0=\Sup Y=a_\lambda$, así que $a\in G$.
    De otro modo, existe un $y\in Y$, $y\neq 0$.
    Sea $b\in Y(a,\lambda)$ el único elemento de
    $\{y\}(a,\lambda)$.
    Entonces $b$ tiene $\lambda$-ésima entrada
    $b_\lambda=y\neq 0$, pero como $b\in Y(a,\lambda)\subseteq G$,
    existe un índice $\mu\neq\lambda$ tal que $b_\mu = 0$.
    Como $b$ es igual a $a$ en todas las entradas
    que no son $\lambda$ (pues $b\in Y(a,\lambda)$),
    en particular tenemos $b_\mu=a_\mu=0$, así que $a\in G$.
    
    Se sigue que $G$ es un $C$-ideal y, como está contenido
    en todos los $C$-ideales, concluimos que $G$ es el
    menor elemento de $C\Idl(A)$.
\end{example}


\begin{lemma}[Coproducto de marcos]
  Tomemos una familia de marcos $A_\lambda, \lambda\in\scr I$ y sean
  $A$ su coproducto como semirretículas, $(A,C)$ el sitio asociado y
  $C\Idl(A)$ su marco de ideales. Entonces las funciones
  $Q_\lambda\colon A_\lambda\to C\Idl$ definidas como
    \[
        Q_\lambda=j(\down(q_\lambda(\_)))
    \]
  son morfismos de marcos y convierten a $C\Idl$ 
  en el coproducto de la famila de los $A_\lambda$
  en la categoría de marcos.
\end{lemma}
\begin{proof}
\todo{mostrar que son morfismos}
Sean $X$ un marco y
$\{r_\lambda\colon A_\lambda\to X\mid \lambda\in \scr I\}$
una familia de morfismos.
Dado que $A$, junto con los $q_\lambda:A_\lambda\to A$,
es el coproducto de los $A_\lambda$ como $\inf$-semiretículas,
obtenemos el morfismo de $\inf$-semiretículas
$R\colon A\to X$ dado por
\begin{align*}
    R(a)
    &=\Inf_X \{r_\lambda(a_\lambda)\mid \lambda\in\scr I\} \\
    &= 1_X\inf r_{\lambda_1}(a_{\lambda_1})\inf\cdots\inf r_{\lambda_n}(a_{\lambda_n})
\end{align*}
donde $\{\lambda_1,\dots,\lambda_n\}$ es el conjunto
de las coordenadas de $a$ distintas de $1$.
Este es el único morfismo de $\inf$-semiretículas que
factoriza a todos los $r_\lambda$ a través de $A$.
Probaremos que $R$ convierte cubiertas de $C$ en supremos.

Consideremos $\mu\in\scr I$, $S\subseteq A_\mu$ y $a=(a_\lambda)\in A$ tal que $a_\mu=\Sup_{A_\mu} S$.
Hay que probar que $\Sup R(S(a,\mu)) = R(a)$.
Si $a_\mu= 1_{A_\mu}$, tenemos
\begin{align*}
    \Sup R(S(a,\mu))
    &= \Sup\{R(b) \mid b\in S(a,\mu)\} \\
    &= \Sup\{ 1_X\inf r_{\lambda_1}(a_{\lambda_1})\inf
        \cdots\inf r_{\lambda_n}(a_{\lambda_n}) \inf r_\mu(s) \mid s\in S\} \\
    &= 1_X\inf r_{\lambda_1}(a_{\lambda_1})\inf
        \cdots\inf r_{\lambda_n}(a_{\lambda_n}) \inf
        \Sup\{ r_\mu(s) \mid s\in S\} \\
    &= 1_X\inf r_{\lambda_1}(a_{\lambda_1})\inf
        \cdots\inf r_{\lambda_n}(a_{\lambda_n}) \inf
        r_\mu(a_\mu) \\
    &= R(a).
\end{align*}

Por otro lado, si $a_\mu\neq 1_{A_\mu}$, tenemos $\mu=\lambda_i$.
Entonces
\begin{align*}
    &\hspace{-10mm}\Sup R(S(a,\mu)) \\
    &= \Sup R(S(a,\lambda_i)) \\
    &= \Sup\{R(b) \mid b\in S(a,\lambda_i)\} \\
    &= \Sup\{ 1_X\inf r_{\lambda_1}(a_{\lambda_1})\inf
        \cdots \inf r_{\lambda_i}(s)\inf\cdots\inf r_{\lambda_n}(a_{\lambda_n}) \mid s\in S\} \\
    &= 1_X\inf r_{\lambda_1}(a_{\lambda_1})\inf \cdots
        \inf\Sup\{ r_{\lambda_i}(s) \mid s\in S\} \inf
        \cdots\inf r_{\lambda_n}(a_{\lambda_n}) \\
    &= 1_X\inf r_{\lambda_1}(a_{\lambda_1})\inf \cdots
        \inf r_{\lambda_i}(a_{\lambda_i}) \inf
        \cdots\inf r_{\lambda_n}(a_{\lambda_n})
        \\
    &= R(a).
\end{align*}

Como $R:A\to X$ manda $C$-cubiertas a supremos de $X$,
la propiedad universal del marco $C\Idl$ generado por $(A,C)$
asegura que existe un único morfismo de marcos
$g\colon C\Idl\to X$ tal que el diagrama
\[
    \begin{tikzcd}[ampersand replacement=\&]
        \& X  \\
        C\Idl \ar[ru,"g"] \& A \ar[u,"R"'] \ar[l,"{j(\down({\_}))}"]
    \end{tikzcd}
\]
conmuta.

Notemos que
\[r_\lambda=R\circ q_\lambda=g\circ j(\down(q_\lambda(\_)))=g\circ Q_\lambda.\]
Por lo que el diagrama
\[
    \begin{tikzcd}[ampersand replacement=\&]
        A_\lambda \arrow[d, "Q_\lambda"'] \arrow[r, "r_\lambda"]
        \& X \\
        C\Idl \arrow[ru, "g"']
        \& A \arrow[l, "j(\down(\_))"] \arrow[u, "R"']
    \end{tikzcd}
\]
conmuta.
\end{proof}

