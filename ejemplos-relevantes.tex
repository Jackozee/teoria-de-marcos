\chapter{Ejemplos relevantes}
Sea $A$ un marco y $j\in NA$. Sea $p\in pt(A_j)$, visto como un morfismo $p:A_j\to 2$. Así, el morfismo $p_j=p\circ j$ representa un punto en $A$, esto es, que todo punto en $A_j$ es un punto en $A$. Por lo anterior, $ptA_j=\{p\in ptA \mid j(p)=p\}$.
 Para cualquier retícula distributiva, sea spec$A$ el conjunto de ideales primos de un marco $A$, equivalente al conjunto de filtros primos de $A$. Nótese que spec$A$ siempre es no vacío por el lema de Zorn; sin embargo, el conjunto de puntos de un marco $A$ sí puede ser vacío.
 
 \begin{example}
     Sea $S\in \Top$ sobrio y $T_1$, y considérese el marco $\mathcal{O}(S)_{\neg \neg}$ que es el marco de puntos fijos de la doble negación, o el álgebra Booleana completa formada por los abiertos regulares de $S$, $R\mathcal{O}S$.
     Sea $p\in R\mathcal{O}S$; así, $p=\overline{s}^{\prime}$ para algún $s\in S=\{s\}^{\prime}$, ya que $S$ es $T_1$. Nótese que \begin{align*}
         \{s\}^{\prime}\in R\mathcal{O}S&\iff \neg\neg \{s\}^\prime =\{s\}^\prime\\
         &\iff \overline{\{s\}^\prime}^\circ=\{s\}^\prime\\
         &\iff \overline{\{s\}}^\circ=\{s\}
     \end{align*}
Ahora bien, sea $U=\{s\}^\circ$, y nótese que $U\neq\emptyset$ porque $\{s\}=\overline{U}$. Por ello, $s$ es un punto aislado en $S$.  Por lo tanto, $R\mathcal{O}S$ es un álgebra Booleana completa sin átomos, por lo que el espacio de puntos del marco $R\mathcal{O}S$ es vacío.
 \end{example}
 Un área activa de investigación procura caracterizar los núcleos de un marco $A$ para los que $A_j$ es espacial, así como los marcos para los que todo cociente es espacial.
 
 \begin{example}
 Considérense $\{G\in \mathbb{Z}-\text{Mod} \mid t(G)=G\}=\mathscr{T}_t$ la clase de grupos de torsión, y $\{G\in\mathbb{Z}-\text{Mod}\mid t(G)=0\}=\mathscr{F}_t$, la clase de grupos libres de torsión.Nótese que $\mathbb{T}_t$ es cerrada bajo submódulos, cocientes, isomorfismos, extensiones y productos, por lo que es una clase de torsión hereditaria.
 Ahora bien, las clases de torsión en la categoría de $\mathbb{Z}-\text{Mod}$ forman un conjunto, $\mathbb{Z}-$tors, que a la vez es un marco. Esto es cierto también en el caso de $\mathbb{R}$-módulos y $\mathbb{Q}$-módulos.
 Estos marcos no son la topología de ningún espacio.
 \end{example}

Desarrollar bien los aspectos de la teoría de modulos y categorias trianguladas para dar Ejemplos
