\section{El teorema de Tychonoff en marcos}

Un espacio topológico es  \emph{compacto} si cualquier cubierta
abierta del espacio admite una subcubierta finita.
En la categoría de espacios topológicos tenemos el siguiente 
teorema debido a Tychonoff alumno doctoral de Pavel Alexandrov (que como muchos dicen es el teorema fundamental de la topológia sensible a puntos):

\begin{theorem}
  Sea $\{S_{\alpha}\mid\alpha\in\Lambda\}$ una familia de espacios topológicos, entonces
  El producto $\prod\{S_{\alpha}\mid\alpha\in\Lambda\}$ es compacto si y solo si cada factor $S_{\alpha}$ es compacto para toda $\alpha\in\Lambda$.
\end{theorem}

Del mismo modo, cualquier marco tiene su noción de compacidad.
\begin{definition}[Compacidad de marcos]
Un marco $A$ es compacto si, para cualquier $S\in A$
se satisface lo siguiente:
\[
    \Sup S= 1_A
    \implies
    \textit{ existe } T\subseteq S\text{ finito, tal que }
    \Sup T= 1_A
.\]
\end{definition}
El objetivo de esta sección es probar el teorema análogo
al de Tychonoff, pero en la categoría de marcos:
\begin{theorem}
    Sea $\{A_\lambda\}_{\lambda\in\scr I}$ una familia de marcos.
    El coproducto
    \[
        \coprod_{\lambda\in\scr I}A_\lambda
    \]
    es compacto si, y solo si, cada $A_\lambda$ es compacto.
\end{theorem}

\subsection{Idea de la demostración}
La prueba usa el concepto de \emph{Sitio} para un marco y la \emph{técnica cubriente} estas situaciones tiene un contexto mas general, que se explorara en (chap), también en algún punto cuando se tenga a la mano la tecnología de la toería de gavillas podremos probar que general un marco como el cociente de un sitio adecuado es , en efecto, considerar una engavillanización de un funtor sobre el sitio a conjuntos, basicamente veremos una prueba del \cite[Lemma V.1.7]{johnstone1986stone} 
En la sección anterior probamos que
$\coprod_{\lambda\in\scr I}A_\lambda$
es el marco de $C$-ideales de $A$, donde $A$ es el coproducto
de los $A_\lambda$ como $\inf$-semiretículas y $C$ es cierta
cobertura en $A$.
Como $C\Idl$ es un cociente de $\cal LA$,
los supremos en $C\Idl$ se calculan como
\[
    \Sup S = j(\bigcup S)
,\]
donde $j\colon\cal LA\to\cal LA$ es el núcleo asociado a $C\Idl$.
Por lo tanto, es de esperarse que la demostración de
nuestra versión del teorema de Tychonoff involucre al núcleo
$j$ de alguna manera.
De hecho, toda la demostración consiste
en dar una construcción de $j$ que facilite convertir
ciertos supremos arbitrarios en supremos finitos.
Para llevar a cabo esta idea, primero definiremos una
cierta subcobertura $C_f\subseteq C$, de tal modo que
factorizaremos la proyección $\cal LA\to C\Idl$ a través de
$C_f\Idl$:
\[
    \begin{tikzcd}
        \cal LA \ar[d] \ar[r] & C\Idl \\
        C_f\Idl \ar[ur]
    \end{tikzcd}
\]

Más específicamente, el plan es el siguiente:
\begin{itemize}
    \item
    Definimos una subcobertura $C_f\subseteq C$ en $A$ y
    construimos la proyección $\cal LA\to C_f\Idl$.
    Es decir, dado $S\in\cal L(A)$,
    construimos el $C_f$-ideal $FS$ generado por $S$.
    \item 
    Vemos que el conjunto $\cal DS$ de supremos dirigidos
    de $S$ está contenido en el $C$-ideal generado por $S$.
    \item
    Probamos que, si $S$ es un $C_f$-ideal, entonces
    $\cal DS$ también lo es.
    Por lo tanto, $\cal D(FS)$ es un $C_f$ ideal que
    contiene a $S$ y está contenido en el $C$-ideal $j(S)$
    generado por $S$.
    \[
        S\subseteq \cal D(FS) \subseteq j(S)
    .\]
    Sin embargo, la última contención puede ser propia.
    \item
    Para saltar de $\cal D(FS)$ a $j(S)$,
    iteramos $\cal D$ para construir una
    cadena de $C_f$-ideales que contienen a $D(FS)$
    \[
        FS\subseteq \cal D(FS) \subseteq \cal D^2(FS)
        \subseteq \dots
    \]
    y probamos que, eventualmente, se alcanza $j(S)$.
    \item
    Con nuestra construcción de $j(S)$ veremos
    que, dada una familia $P\subseteq C\Idl(A)$,
    la igualdad
    $j(\bigcup P)=A$ implica $F(\bigcup P)=A$.
    En otras palabras,
    $\Sup P = 1_{C\Idl(A)}$ implica $1\in F(\bigcup P)$.
    \item
    Con esta última herramienta, podremos demostrar
    la implicación $\impliedby$ del teorema.
    \item
    La implicación $\implies$ es directa.
\end{itemize}

\subsection{La cobertura de cubiertas finitas}

\begin{lemma}
La función $C_f:A\to\cal P(\cal P(A))$
definida para todo $a\in A$ como
\[C_f(a)=\{S\subseteq A\mid S\in C(a), S\textit{ finito}\}\]
es una cobertura sobre $A$.
\end{lemma}

Nuestra primera tarea será obtener una construcción para
el núcleo asociado al cociente $\cal LA\to C_f\Idl$.
Para esto, introduciremos algo de notación.

\begin{definition}
Sean $\lambda_1,\dots,\lambda_n\in \scr I$ índices distintos y
$x_1,\dots,x_n$ elementos tales que
$x_i\in A_{\lambda_i}$ para todo $i$.
En otras palabras, $x=(x_1,\dots,x_n)$ es un elemento del producto
$A_{\lambda_1}\times\dots\times A_{\lambda_n}$.

Para cualquier elemento $a\in A$ del coproducto de los $A_\lambda$
en $\Pos^{\inf}$, denotaremos como
$a(x)=a(x_1,\dots,x_n)$ al elemento de $A$ que
es igual a $a$ pero con las entradas $\lambda_1,\dots,\lambda_n$
reemplazadas por $x_1,\dots,x_n$.
Es decir,
\[
    p_\lambda(a(x_1,\dots,x_n))
    =
    a(x_1,\dots,x_n)_\lambda
    =
    \begin{cases}
        x_i, & \lambda = \lambda_i \\
        a_\lambda & \lambda\not\in\{\lambda_1,\dots\lambda_n\}
    \end{cases}
.\]
En particular, si $x_i\in A_{\lambda_i}$, se cumple que
\begin{align*}
  1(x_i)
    &= q_{\lambda_i}(x_i) \\
  1(x_1,\dots,x_n)
    &= \Inf\{q_{\lambda_i}(x_i) \mid i=1,\dots,n\}.
\end{align*}
%
%Estrictamente hablando,
%habría que denotar no solo los elementos $x_1,\dots,x_n$
%que van a reemplazar las
%entradas de $a$, sino también exactamente en qué entradas se
%está haciendo la sustitución
%(ya que podría darse el caso que $A_\lambda=A_\mu$ aún cuando
%$\lambda\neq\mu$).
%Sin embargo, no lo haremos así, pues todas las
%sustituciones se harán en un contexto que permita saber
%el conjunto exacto de índices.
\end{definition}

\begin{definition}
  Dado un conjunto $S\subseteq A$, defino el conjunto $FS$
  especifivando que $a\in FS$ si, y solo si,
  existe un conjunto finito no vacío de índices
  $\Gamma=\{\lambda_1,\dots,\lambda_n\}$ que contiene al
  soporte de $a$ y conjuntos $S_1,\dots,S_n$ con
  $S_i\subseteq A_{\lambda_i}$ tales que $\Sup S_i =
  a_{\lambda_i}$ y, para toda tupla
  $(s_1,\dots,s_n)\in S_1\times\cdots\times S_n$, el elemento
  \[
    a(s_1,\dots,s_n)
  \]
  está en $S$.
  En tal caso, decimos que $a$ es $\Gamma$-generado por $S$ y
  que $(\Gamma,S_1,\dots,S_n)$ es el testigo de $a\in FS$,
  ya que $(\Gamma,S_1,\dots,S_n)$ atestigua (o prueba) que
  $a$ pertenece a $FS$.
\end{definition}

\subsection{Si \tps{$S$}{S} es sección inferior, \tps{$FS$}{FS} también.}

Sea $S$ una sección inferior, $a\in FS$ y $b\leq a$.

Sea $a\in FS$ y sea $(\Gamma,S_1,\dots,S_n)$ su testigo.
Sin perder generalidad, podemos suponer que $\Gamma$ también
contiene al soporte de $b$.
En efecto, si $b_\gamma\neq 1_\gamma$ y $\gamma\not\in\Gamma$,
podemos agregar $\gamma$ a $\Gamma$ y poner
$S_\gamma = \{1_\gamma\}$,
de modo que $a_\gamma = 1_\gamma = \Sup S_\gamma$
y, para todo
$(s_1,\dots,s_n,1_\gamma)\in\prod_{i=1}^nS_i\times S_\gamma$
se tiene
\[
  1(s_1,\dots,s_n,1_\gamma) = 1(s_1,\dots,s_n) \in S
.\]

Supongamos, pues, que $b$ también tiene soporte contenido en
$\Gamma$.
Entonces haciendo $T_i=\{b_{\lambda_i}\inf s \mid s\in S_i\}$
para todo $i=1,\dots,n$, tenemos que
\begin{align*}
  \Sup T_i
  &= b_{\lambda_i}\inf\Sup S_i \\
  &= b_{\lambda_i}\inf a_{\lambda_i} \\
  &= b_{\lambda_i}.
\end{align*}

Más aún, dado $(b_{\lambda_1}\inf s_1,\dots,b_{\lambda_n}\inf
s_n)\in \prod_{i=1}^n T_\lambda$, tenemos
\begin{align*}
  1(b_{\lambda_1}\inf s_1,\dots,b_{\lambda_n}\inf s_n)
  \leq 1(s_1,\dots,s_n) \in S.
\end{align*}
y, como $S$ es sección inferior, tenemos
\[
  1(b_{\lambda_1}\inf s_1,\dots,b_{\lambda_n}\inf s_n) \in S
.\]
Se sigue que $b$ es finitamente generado por $S$ con testigo
$(\Gamma,T_1,\dots,T_n)$, es decir: $b\in FS$,
así que $FS$ es sección inferior.

\subsection{Si \tps{$S$}{S} es sección inferior, \tps{$FS$}{FS} está contenido en todos los \tps{$C_f$}{Cf}-ideales que contienen a \tps{$S$}{S}.}

Sea $J$ un $C_f$-ideal que contiene a $S$.
Tomemos un elemento $a\in FS$ con testigo $(\Gamma,S)$, donde
$\Gamma=\{\lambda_1,\dots,\lambda_n\}$.

Probaremos, por inducción sobre $k\leq n$, que, para cada tupla
$(s_k,\dots,s_n)\in S_k\times\cdots\times S_n$, el elemento
$a(s_k,\dots,s_n)$ está en $J$.

Para toda tupla
$(s_1,\dots,s_n)\in S_1\times\cdots\times S_n$, tenemos que
\[
  a(s_1,\dots,s_n)\in S\subseteq J
\]
ya que $a\in FS$.
Esto prueba el caso base ($k=1$).

La hipótesis de inducción ($k$) nos dice que, para cada tupla
$(s_k,\dots,s_n)\in S_k\times\cdots\times S_n$,
el elemento $a(s_k,\dots,s_n)$ está en $J$.

Ahora (el paso de inducción)
sea $(s_{k+1},\dots,s_n)\in S_{k+1}\times\cdots\times S_n$
una tupla arbitraria y consideremos el conjunto
\[
  \{a(s,s_{k+1},\dots,s_n) \mid s\in S_k\}
  =
  S_k(a(s_{k+1},\dots,s_n),\lambda_k)
  \in
  C_f(a(s_{k+1},\dots,s_n))
.\]
Esta es, en efecto, una cubierta, ya que la $\lambda_k$-ésima
coordenada de $a(s_{k+1},\dots,s_n)$ es $a_{\lambda_k}=\Sup S_k$.
Más aún: por hipótesis de inducción, cada elemento
de la cubierta está en $J$.
Como $J$ es un $C_f$-ideal, se sigue que
\[
  a(s_{k+1},\dots,a_n) \in J
.\]
Esto concluye la inducción para $k\leq n$.
En particular, tenemos $a(s_n)\in J$ para cada $s_n\in S_n$,
pero esta es una $C_f$-cubierta de $a$ contenida en $J$,
así que $a\in J$.
Por lo tanto, $FS\subseteq J$.

\subsection{Si \tps{$S$}{S} es sección inferior, \tps{$FS$}{FS} es cerrado bajo
cubiertas y, por lo tanto, es un \tps{$C_f$}{Cf}-ideal.}

Si $a\in A$ tiene una cubierta vacía, entonces tiene una entrada
cero, digamos $a_{\lambda_1}=0$.
Sea $\lambda_2,\dots,\lambda_n$ el resto de su soporte.
Poniendo $S_1=\{\}$ y $S_i=\{a_{\lambda_i}\}$ para $i=2,\dots,n$,
obtenemos $\Sup S_i = a_{\lambda_i}$.
Como $S_1\times S_2\times\cdots\times S_n=\emptyset$,
se cumple que $a\in FS$ por vacuidad.

Luego, basta considerar cubiertas de dos elementos.

Sean $a,b\in FS$ con $a_\gamma=b_\gamma$ para todo
$\gamma\neq\gamma_0$.
Debemos mostrar que $a\sup b\in FS$.

Como antes, podemos suponer que tanto $a$ como $b$ tienen soporte
en el mismo $\Gamma=\{\gamma_0,\gamma_1,\dots,\gamma_n\}$.

Por definición, tenemos $F_\gamma,G_\gamma$ tales que $\Sup
F_\gamma = a_\gamma$, $\Sup G_\gamma = b_\gamma$ para todo
$\gamma\in\Gamma$ y
\begin{align*}
  1(f_0,\dots,f_n)\in S && 1(g_0,\dots,g_n)\in S.
\end{align*}
siempre que $f_i\in F_{\gamma_i}$ y $g_i\in G_{\gamma_i}$.

Definamos $H_{\gamma_0}=F_{\gamma_0}\cup G_{\gamma_0}$ y
$H_\gamma = \{f\inf g\mid f\in F_\gamma,g\in G_\gamma\}$ para
todo $\gamma\in\Gamma,\gamma\neq\gamma_0$.
Entonces
\begin{align*}
  \Sup H_{\gamma_0}
  &= \Sup(F_{\gamma_0}\cup G_{\gamma_0}) \\
  &= \Sup F_{\gamma_0} \sup \Sup G_{\gamma_0} \\
  &= a_{\gamma_0} \sup b_{\gamma_0}
  \\
  \Sup H_\gamma
  &= \Sup\{f\inf g \mid f\in F_\gamma, g\in G_\gamma\} \\
  &= \Sup F_\gamma \inf \Sup G_\gamma \\
  &= a_\gamma \inf b_\gamma \\
  &= a_\gamma\sup b_\gamma
  \text{ para } \gamma\neq\gamma_0,
\end{align*}
lo cual son las entradas de $a\sup b$.

Ahora, las tuplas $x=\prod_{i=0}^n H_{\gamma_i}$ son de
dos formas posibles.
\begin{itemize}
  \item Si $x=(f_0,f_1\inf g_1,\dots,f_n\inf g_n)$,
  entonces
  \begin{align*}
    1(f_0,f_1\inf g_1,\dots,f_n\inf g_n)
    &\leq 1(f_0,f_1,\dots,f_n) \in S.
  \end{align*}
  \item de otro modo, $x=(g_0,f_1\inf g_1,\dots,f_n\inf g_n)$,
  así que
  \begin{align*}
    1(g_0,f_1\inf g_1,\dots,f_n\inf g_n)
    &\leq 1(g_0,g_1,\dots,g_n)\in S.
  \end{align*}
\end{itemize}
Como $S$ es sección inferior, se sigue que $1(x)$ siempre está en
$S$, así que $a\sup b$ está en $FS$.

\subsection{El conjunto de supremos dirigidos \tps{$\D S$}{DS} está contenido en todos los \tps{$C$}{C}-ideales que contienen a \tps{$S$}{S}.}

Sea $S$ una sección inferior y $\D S$ su conjunto de supremos
dirigidos.

Sea $J$ un $C$-ideal que contiene a $S$.
Como todo $a\in\D S$ es el supremo de un $D\subseteq S$ dirigido
y $D\subseteq S\subseteq J$, basta ver que $J$ es cerrado bajo
supremos dirigidos.

Tomemos un subconjunto dirigido $D\subseteq J$.
Como $D$ es dirigido, para cualquier $d\in D$ tenemos
\[
  \Sup D = \Sup(D\cap{\uparrow}d)
.\]
($\geq$ es obvia, para $\leq$ basta observar que todo $a\in D$
está por debajo de un $c\in D$ con $\geq a\sup d$).
Luego, podemos suponer que $D$ está contenido en una sección
superior principal.
En particular podemos suponer que todos los elementos de $D$
tienen soporte en $\Gamma=\{\lambda_1,\dots,\lambda_n\}$.

Sea $a=\Sup D$.
Como $D$ es dirigido, entonces sus proyecciones $D_1,\dots,D_n$ a
los marcos $A_{\lambda_1},\dots,A_{\lambda_2}$ son dirigidas
(todos los $x_1,y_1\in D_1$ vienen de elementos $x,y\in D$;
luego, existe $z\in D$ con $x,y\leq z$, por lo cual $x_1,y_1\leq
z_1$).

Por definición del supremo, tenemos
\[
  a = 1(\Sup D_1,\dots,\Sup D_n)
.\]

Por inducción sobre $k\leq n$, probaremos que, para toda tupla
$(d_k,\dots,d_n)\in D_k\times\cdots\times D_n$, el elemento
\[
  a(d_k,\dots,d_n)
  =
  1(\Sup D_1,\dots,\Sup D_{k+1},d_k,\dots,d_n)
\]
está en $J$.

Para cualquier tupla $(d_1,\dots,d_n)\in D_1\times\cdots\times
D_n$, cada entrada $d_i$ es la $\lambda_i$-ésima
proyección de algún $x_i\in D$.
Como $D$ es dirigido, existe $z\in D$ con $x_1,\dots,x_n\leq z$.
Luego,
\[
  1(d_1,\dots,d_n)\leq z\in D \subseteq J
.\]
Como $J$ es sección inferior, tenemos $1(d_1,\dots,d_n)\in J$.
Esto prueba el caso base ($k=1$).

La hipótesis de inducción ($k$) dice que, para cada tupla
$(d_k,\dots,d_n)\in D_k\times\cdots\times D_n$, el elemento
$a(d_k,\dots,d_n)$ está en $J$.

Ahora (paso de inducción) sea
$(d_{k+1},\dots,d_n)\in D_{k+1}\times\cdots\times D_n$ una tupla
arbitraria y consideremos el conjunto
\
\[
  \{a(d,d_{k+1},\dots,d_n) \mid d\in D_k\}
  =
  D_k(a(d_{k+1},\dots,d_n),\lambda_k)
  \in
  C(a(d_{k+1},\dots,d_n))
.\]
Esta es, en efecto, una cubierta, ya que la $\lambda_k$-ésima
coordenada de $a(d_{k+1},\dots,d_n)$ es $\Sup D_k$.
Más aún: por hipótesis de inducción, cada elemento de la cubierta
está en $J$.
Como $J$ es un $C$-ideal, se sigue que
$a(d_{k+1},\dots,d_n)\in J$.
Esto concluye la inducción.
En particular, para $k=n$, mostramos que $a(d_n)\in J$ para todo
$d_n\in D_n$, pero este conjunto es una $C$-cubierta de $a$.
Se sigue que $a\in J$, así que $J$ es cerrado bajo supremos
dirigidos.
Por lo tanto, $\D S\subseteq J$.

\subsection{Si \tps{$S$}{S} es un \tps{$C_f$}{Cf}-ideal,
entonces \tps{$\D S$}{DS} también.}

Dado $a\in S$, $\{a\}\subseteq S$ es dirigido, así que
$a=\Sup\{a\}\in \D S$, por lo que $S\subseteq\D S$.

Primero veamos que $\D S$ es sección inferior.
Sean $a\in \D S$ y $b\leq a$.
Podemos suponer que $a$ y $b$ tienen soporte en $\Gamma$.
Tenemos $a=\Sup D$ con $D\subseteq S$ dirigido.
Es claro que
\[
  b = \Sup\{b\inf d\mid d\in D\}
.\]
Como $S$ es sección inferior, este último conjunto esta en $S$,
así que basta probar que es dirigido.
Sean $b\in d,b\inf d'$ con $d,d'\in D$.
Entonces existe $d''\in D$ con $d\sup d'\leq d''$, por lo
cual $b\inf d''$ cumple
$(b\inf d)\sup(b\inf d')=b\inf(d\sup d')\leq b\inf d''$ y, así,
el conjunto es dirigido.
Luego $b\in\D S$ y $\D S$ es sección inferior.

Ahora veamos que $\D S$ es cerrado bajo $C_f$-cubiertas.
Si $a\in A$ tiene una cubierta vacía, tenemos $a\in S$, así que
$a\in\D S$ (pues $S$ es $C_f$-ideal),
por lo cual basta ver que $\D S$ es cerrado
bajo $C_f$-cubiertas de dos elementos.
Sean $a,b\in\D S$ con todas sus entradas iguales excepto
$\gamma_0$.
Por definición, existen $D,E\subseteq S$ dirigidos con
$a=\Sup D$ y $b=\Sup E$.

Definimos $*:D\times E\to A$ como
\begin{align*}
  (d*e)_{\gamma_0} &= d_{\gamma_0} \sup e_{\gamma_0} \\
  (d*e)_{\gamma} &= d_{\gamma} \inf e_{\gamma} \text{ para }
  \gamma\neq\gamma_0
\end{align*}
y sea $F=\{d*e \mid d\in D, e\in E\}$.

Ahora veamos que $F$ es dirigido.
Si $d*e,d'*e'\in F$, entonces existen $d''\in D$ y $e''\in E$ con
$d\sup d'\leq d''$ y $e\sup e'\leq e''$.
Luego, $(d*e)\sup(d'*e')\leq d''*e''$, pues la comparación es
puntual y $*$ es monótono en cada coordenada; en efecto:
\begin{align*}
  ((d*e)\sup(d'*e'))_{\gamma_0}
  &= (d*e)_{\gamma_0} \sup (d'*e')_{\gamma_0} \\
  &= d_{\gamma_0}\sup e_{\gamma_0}\sup d'_{\gamma_0}\sup
    e'_{\gamma_0} \\
  &= (d_{\gamma_0}\sup d'_{\gamma_0})
    \sup(e_{\gamma_0}\sup e'_{\gamma_0}) \\
  &\leq d''_{\gamma_0} \sup e''_{\gamma_0}
  \\
  ((d*e)\sup(d'*e'))_{\gamma}
  &= (d*e)_{\gamma} \sup (d'*e')_{\gamma} \\
  &= (d_\gamma\inf e_\gamma)\sup(d'_\gamma\inf e'_\gamma) \\
  &\leq d_\gamma \sup e'_\gamma \\
  &\leq d''_\gamma \sup e''_\gamma.
\end{align*}
Se sigue que $F$ es dirigido.
Mas aún, $\Sup F = a\sup b$, pues
\begin{align*}
  (\Sup F)_{\gamma_0}
  &= \Sup\{(d*e)_{\gamma_0} \mid d\in D,e\in E\} \\
  &= \Sup\{d_{\gamma_0}\sup e_{\gamma_0} \mid d\in D,e\in E\} \\
  &= \Sup\{d_{\gamma_0}\mid d\in D\}
    \sup \Sup\{e_{\gamma_0}\mid e\in E\} \\
  &= (\Sup D)_{\gamma_0} \sup (\Sup E)_{\gamma_0} \\
  &= a_{\gamma_0} \sup b_{\gamma_0}
  \\
  (\Sup F)_\gamma
  &= \Sup\{(d*e)_\gamma\mid d\in D,e\in E\} \\
  &= \Sup\{d_\gamma\inf e_\gamma\mid d\in D,e\in E\} \\
  &= \Sup\{d_\gamma\mid d\in D\}
    \inf \Sup\{e_\gamma\mid e\in E\} \\
  &= (\Sup D)_\gamma \inf (\Sup E)_\gamma \\
  &= a_\gamma \inf b_\gamma \\
  &= a_\gamma \sup b_\gamma \text{ para }\gamma\neq\gamma_0,
\end{align*}
lo cual son las entradas de $a\sup b$.

Para mostrar que $a\sup b\in\D S$,
solo queda ver que $F\subseteq S$.
Para esto, usaremos que $S$ es $C_f$-ideal (nótese que no lo
habíamos usado excepto en un caso muy simple).

Sea $d*e\in F$; es decir: $d\in D\subseteq S$ y $e\in E\subseteq S$.
Queremos construir una $C_f$-cubierta de $d*e$ que esté contenida
en $S$.
Consideremos  $r,s\in A$ dados por
\begin{align*}
  r_\gamma = s_\gamma
    &= d_\gamma \inf e_\gamma \text{ para } \gamma\neq\gamma_0 \\
  r_{\gamma_0} &= d_\gamma \\
  s_{\gamma_0} &= e_\gamma.
\end{align*}
Observemos que $r\leq d\in D\subseteq S$ y que $s\leq e\in
E\subseteq S$.
Como $S$ es sección inferior, tenemos $r,s\in S$.
Finalmente, es claro que $r\sup s = d*e$ donde el supremo se
concentra en una entrada.
Es decir, $\{r,s\}\in C_f(d*e)$ y $\{r,s\}\subseteq S$.
Como $S$ es $C_f$-ideal, se sigue que $d*e\in S$.
Así, $F\subseteq S$, por lo cual $a\sup b\in\D S$.
Es decir, $\D S$ es cerrado bajo $C_f$-cubiertas, por lo cual
también es un $C_f$-ideal.

\subsection{Lema: saltar de \tps{$\D(FS)$}{D(FS)} a \tps{$j(S)$}{j(S)}.}
    Sea $I$ un $C_f$-ideal de $A$.
    Como ya vimos, $\D I$ es un $C_f$-ideal contenido
    entre $I$ y el $C$-ideal $j(I)$ generado por $I$.
    Aplicando esto a $\D I$, obtenemos
    \[
        I\subseteq \D I
        \subseteq \D(\D I)
        \subseteq j(\D I)=j(I)
    .\]
    Poniendo $I=FS$ (el $C_f$-ideal generado por una sección
    inferior $S\subseteq A$) e iterando $\D$, obtenemos una
    cadena de $C_f$-ideales
    \[
        FS \subseteq \cal D(FS)
        \subseteq D^2(FS)
        \subseteq \dots
    \]
    contenidos en $j(FS)=j(S)$.


    La extensión de esta cadena a todos los ordinales
    \begin{align*}
        I_0 &= FS \\
        I_{\alpha+1} &= \D(I_\alpha) \\
        I_\lambda
        &= \bigcup\{I_\alpha \mid \alpha<\lambda\}
            && \text{ si $\lambda$ es límite }
    \end{align*}
    se estaciona en el $C$-ideal $j(S)$ generado por $S$. \\
    En efecto, sea $\gamma$ el primer ordinal donde la cadena
    se detiene.
    Entonces $I_\gamma=I_{\gamma+1}=\D(I_\alpha)$.
    Es decir, $I_\gamma$ es cerrado bajo supremos dirigidos.
    Como también es cerrado bajo supremos finitos
    (entrada por entrada), se sigue
    que es cerrado bajo supremos arbitrarios
    (entrada por entrada), así que es
    un $C$-ideal.
    Se sigue que $I_\gamma=j(S)$, como se afirmó.


\subsection{Lema: generar como \tps{$C$}{C}-ideal implica generarlo como \tps{$C_f$}{Cf}-ideal}
    Tomemos una familia $A_\lambda$ de marcos compactos
    y sea $A$ su coproducto como $\inf$-semiretículas.
    Equipémoslo con los cubrientes $C$ y $C_f$.
    
    Si $P\subseteq C\Idl(A)$ es tal que $\Sup P=1_{C\Idl(A)}$,
    entonces $1\in F(\bigcup P)$.
    En otras palabras, si
    $j(\bigcup P)=A$, entonces $F(\bigcup P)=A$.
    \\
    %\pause
    \textbf{\emph{Demostración:}} \\
    Supongamos que $A=j(\bigcup P)$.
    De la construcción anterior, tenemos
    $I_\gamma=j(\bigcup P)$ (donde $\gamma$ es el primer
    ordinal donde se detiene la cadena de iteraciones
    $\D^\alpha(\bigcup P)$),
    así que $1\in I_\gamma$. \\
    Afirmamos que $\gamma$ no puede ser ordinal límite.
    En efecto, si lo fuera, tendríamos
    \[
        1 \in I_\gamma = \bigcup\{I_\alpha \mid \alpha<\gamma\}
    ,\]
    por lo cual $1\in I_\alpha$ para algún $\alpha<\gamma$, lo
    cual no puede suceder por la minimalidad de $\gamma$.

    Más aún, $\gamma$ no puede ser sucesor.
    Si fuera el caso que $\gamma=\beta+1$, tendríamos
    $1\in I_\gamma = \D(I_\beta)$;
    es decir: $1=\Sup D$ para algún conjunto dirigido
    $D\subseteq I_\beta$.
    Dado que $D$ es dirigido, podemos tomar cualquier
    $d\in D$ y obtener
    \[
        1
        = \Sup D
        = \Sup\{a\in D\mid d\leq a\}
        = \Sup(D\cap{\uparrow}d)
    ,\]
    lo cual nos dice que,
    reemplazando a $D$ por $D\cap{\uparrow}d$ en
    caso de ser necesario,
    podemos suponer que $D$ está contenido en una
    sección superior principal.
    En particular, podemos suponer que todos los elementos
    de $D$ tienen soporte contenido en
    un conjunto finito de índices $\Gamma$.

    Para cada $\lambda\in\Gamma$, sea $D_\lambda=p_\lambda(D)$.
    Como $1=\Sup D$, tenemos $1_\lambda=\Sup D_\lambda$ y,
    por compacidad de los $A_\lambda$, esto nos da familias finitas
    $E_\lambda\subseteq D_\lambda$ tales que
    $1_\lambda=\Sup E_\lambda$.
    
    Luego, todos los elementos de todos los $E_\lambda$ aparecen
    como entradas de los elementos de un conjunto finito
    $E\subseteq D$.
    Como $D$ es dirigido y $E$ es finito,
    existe $a\in D$ tal que $\Sup E\leq a$,
    pero $\Sup E=1$, así que
    \[
        1 = a \in D\subseteq I_\beta
    ,\]
    lo cual contradice la minimalidad de $\gamma$.
    
    Se sigue que $\gamma=0$.
    Esto es, $A=j(\Sup P) = I_0 = F(\bigcup P)$, como se deseaba.

\subsection{Una implicación (Tychonoff)}
    Con toda esta herramienta, podemos demostrar el teorema
    de Tychonoff.
    Si $A_\lambda$ es una familia de marcos compactos y $A$
    es su coproducto como $\inf$-semiretículas,
    debemos mostrar que $C\Idl(A)$ es compacto.
    Sea $P\subseteq C\Idl(A)$ una familia de $C$-ideales tal que
    $\Sup P = 1_{C\Idl(A)}$.
    En otras palabras: $\bigcup P$ genera a $A$ como $C$-ideal.
    \[
        j(\bigcup P) = A
    .\]
    
    Como acabamos de probar, esto implca que $\bigcup P$ también
    genera a $A$ como $C_f$-ideal:
    \[
        F(\bigcup P)=A
    .\]
    En particular, $1\in F(\bigcup P)$; esto es:
    existe algún conjunto finito no vacío de índices $\Gamma$
    tal que $1\in A$ es $\Gamma$-finitamente generado por
    $\bigcup P$.

    Es decir: para cada $\lambda\in\Gamma$ existe un conjunto
    finito $S_\lambda\subseteq A_\lambda$
    con $\Sup S_\lambda = 1_\lambda$
    y, siempre que se tenga una tupla
    $x=(x_\lambda)_{\lambda\in\Gamma}$
    con $x_\lambda\in A_\lambda$,
    se cumple
    \[
        1(x)
        =\Inf\{q_\lambda(x_\lambda)\mid\lambda\in\Gamma\}
        \in\bigcup P
    .\]
    Ahora, como $\Gamma$ es finito y cada $S_\lambda$
    también, solo se puede formar
    una cantidad finta de tuplas
    $x=(x_\lambda)_{\lambda\in\Gamma}$.
    Así, el conjunto de los $1(x)$ es finito y, por lo tanto,
    está contenido en una cantidad finita de factores
    $P_1,\dots,P_n\in P$ de $\bigcup P$.
    
    Luego, $1\in A$ es $\Gamma$-finitamente generado
    por $\bigcup_{i=1}^nP_i$.
    Esto es $F(\bigcup_{i=1}^n P_i)=A$ y, así,
    \[
        \Sup\{P_1,\dots,P_n\}
        = j(\bigcup_{i=1}^nP_i)
        = A
        = 1_{C\Idl(A)}
    .\]
    Luego, $C\Idl(A)$ es compacto.

%\section*{SESIÓN 22: 30 NOV (Expo Alfredo, dudas tarea)}

\subsection{La otra implicación}

Necesitamos este pequeño resultado.
\begin{lemma}
Dada un elemento $x\in A_\lambda$, entonces
\[
    j(\down q_\lambda(x)) = \down q_\lambda(x)
.\]
\end{lemma}
\begin{proof}
    En efecto, sea $a\in A$ y $S(a,\mu)\subseteq\down q_\lambda(x)$
    una $C$-cubierta de $a$.
    Queremos mostrar que $a\leq q_\lambda(x)$.
    Si $\mu\neq\lambda$, entonces $S(a,\mu)\subseteq \down q_\lambda(x)$
    nos dice que $a_\lambda\leq x$, así que $a\leq q_\lambda(x)$.
    Por otro lado, si $\mu=\lambda$, entonces
    $S(a,\lambda) \subseteq \down q_\lambda(x)$ nos dice que, para todo
    $s\in S$ tenemos $s\leq x$.
    Por lo tanto, $a_\lambda=\Sup S\leq x$, así que $a\leq q_\lambda(x)$.
    Se sigue que $\down q_\lambda(x)$ es un $C$-ideal.
\end{proof}
Ahora sí.

\begin{lemma}
    Si el coproducto $C\Idl(A)$ de una familia de marcos
    $A_\lambda$ es compacto, entonces cada $A_\lambda$
    también es compacto.
\end{lemma}
\begin{proof}
    Supongamos que $C\Idl$ es compacto.
    Si $S\subseteq A_\lambda$ es tal que $\Sup S=1_\lambda$,
    entonces $S(\lambda,1)\in C(1)$.
    
    Para cada $s\in S$, sea $Q_\lambda(s)
    = j(\down q_\lambda(s))$
    el $C$-ideal generado por $s$ bajo $q_\lambda$.
    Luego,
    \[
        \Sup\{Q_\lambda(s) \mid s\in S\} = 1_{C\Idl(A)} = A
    .\]
    Por la compacidad de $C\Idl(A)$,
    existen $s_1,\dots,s_n\in S$ tales que
    \[
        \Sup\{Q_\lambda(s_i) \mid i=1,\dots,n\} = A
    .\]
    Como $Q_\lambda$ es morfismo de marcos, esto es
    \[
        Q_\lambda(\Sup\{s_i \mid i=1,\dots,n\}) = A
    .\]
    
    Ahora, dado que $Q_\lambda(x) = \down q_\lambda(x)$, tenemos
    \[
        1 \in \down q_\lambda(\Sup\{s_i \mid i=1,\dots,n\})
    ,\]
    o bien
    \[
        1 = q_\lambda(\Sup\{s_i \mid i=1,\dots,n\})
    .\]
    Proyectando a la $\lambda$-ésima coordenada, obtenemos
    \[
        1_\lambda = \Sup\{s_i \mid i=1,\dots,n\}
    .\]
    Se sigue que $A_\lambda$ es compacto.
\end{proof}

