\chapter{Cocientes}
\label{ch:cocientes}

%\section*{(SESIÓN 6: 28 SEP)}
En este capítulo estudiaremos los cocientes de un marco.

Si $A$ es un conjunto, grupo, anillo, módulo (o alguna estructura
algebraica similar) un cociente de $A$ está dado por una relación de
equivalencia $(\simr)\subseteq A\times A$ que respeta las operaciones.
Esto resulta ser equivalente a que $A/\simr$ tiene una estructura del
mismo tipo que $A$ (grupo, anillo, etc) tal que la proyección
$\pi:A\to A/\simr$ es un homomorfismo.
Estas relaciones de
equivalencia se llaman congruencias. En el caso de conjuntos, donde no
hay estructura adicional, cualquier relación de equivalencia es una
congruencia.

Las congruencias de $A$, siendo subconjuntos de $A\times A$, suelen ser
difíciles de manejar directamente. En los casos de grupos, anillos y
módulos, la situación se puede mejorar, pues las congruencias quedan
codificadas en algunos subconjuntos distinguidos de $A$:
grupos normales, ideales bilaterales y submódulos, respectivamente.
En cada caso, el subconjunto distinguido es el ``núcleo'' (en el
sentido adecuado) del morfismo proyección $\pi:A\to A/\simr$.

La situación para marcos es aún mejor. Las congruencias de un marco
$A$ corresponden a algunos subconjuntos de $A$ (llamados subconjuntos
implicativos), pero hay otra correspondencia más conveniente:
tanto las congruencias como los subconjuntos implicativos están
codificados en ciertos operadores $A\to A$ que llamamos núcleos.

Además, como en otras categorías, los cocientes de marcos satisfacen
el llamado ``teorema del cociente'': si un morfismo $f$ respeta
una congruencia $\simr$:
\begin{equation}
  a\sim b \implies fa = fb,
\end{equation}
entonces $f$ se factoriza a través del cociente $\pi:A\to A/\simr$.
Más aún, cualquier morfismo $f:A\to B$ induce una congruencia
$(\simr)$ en $A$ con la fórmula
\begin{equation}
  a\sim_f b \iff fa = fb.
\end{equation}
En particular, $f$ respeta la congruencia $(\simr_f)$, así que se
factoriza a través de $\pi:A\to A/\simr_f$, donde el marco cociente
$A/\simr_f$ eses isomorfo a la imagen de $f$.

\section{Cocientes de conjuntos}
Cualquier función de conjuntos $\phi:A\to B$ induce una relación de
equivalencia en $A$ como sigue:
\[
    a\sim b \ssi \phi(a) = \phi(b)
.\]
Como es usual, denotamos al bloque de un elemento $x\in A$ como
\begin{equation}
  [a]=\{x\in A\mid x\sim a\}
\end{equation}
y al conjunto de bloques como
\begin{equation}
  A/\simr
  =
  \{[x] \mid x\in A\}
.\end{equation}
Entonces tenemos la función $\eta: A\to A/\simr$ que manda un
elemento $a\in A$ a su bloque correspondiente:
\begin{align*}
    \eta:A&\to A/\simr \\
    a&\mapsto [a]
\end{align*}
En particular, se tiene que
\[
    \eta(x) = \eta(y) \ssi x\sim y
.\]
Una consecuencia es que tenemos una función
$\phi^\sharp:A/\simr \to B$ que cierra el triángulo:
\[
\begin{tikzcd}
    A\ar[rr,"\phi"] \ar[dr,"\eta"'] && B \\
    & A/\simr \ar[ur,"\phi^\sharp"']
\end{tikzcd}
\]
dada como $\phi^\sharp([a]) = \phi(a)$.
El hecho de que $\phi^\sharp$ esté bien definida como función se debe,
precisamente, a la definición de la relación de equivalencia:
si $[a]=[b]$, entonces $a\sim b$, por lo cual $\phi(a)=\phi(b)$, i.e.,
$\phi^\sharp([a])=\phi^\sharp([b])$.
Más aún, tenemos $A/\simr\simeq \img \phi$.
En particular, si $\phi$ es suprayectiva, entonces $A/\simr \simeq B$.

También aquí sucede que $\eta$ es universal con respecto a esta
propiedad: si $\psi:A\to B'$ es una función tal que
$\phi(a)=\phi(b)\implies \psi(a)=\psi(b)$,
entonces existe una única $\psi^\sharp:A/\simr\to B'$
que factoriza a $\psi$ a través de $\eta$.
Nos gustaría decir que el ``núcleo'' de $\phi$ es la relación de
equivalencia $\sim$ inducida por $\phi$.

\section{Cocientes en \tps{$\SupLat$}{SupLat}}
Antes de estudiar los cocientes de marcos, veremos que es más fácil
comenzar estudiando los cocientes de $\Sup$-semiretículas.
(Recordemos que tenemos un funtor fiel $\Frm\to\SupLat$).

Primero veremos que todo morfismo induce una congruencia.
Luego, veremos que toda congruencia produce un morfismo cociente y
que este morfismo induce la congruencia original.

En la categoría de conjuntos, todo morfismo $\phi:A\to B$
induce una relación de equivalencia $\simr$ en $A$  y,
recíprocamente, toda relación de equivalencia $\simr$ en $A$ está
inducida por un morfismo (porque podemos formar el cociente
$A/\simr$).
Queremos obtener un resultado análogo en $\SupLat$.

Un $\Sup$-morfismo $f:A\to B$ es, en particular,
una función, así que induce una relación de equivalencia en $A$.
Sin embargo, no todas las relaciones de equivalencia son
interesantes. Nos enfocaremos en las relaciones de equivalencia
que respeten la estructura de $\Sup$-semiretícula, las cuales
llamaremos $\Sup$-congruencias.

Supongamos que tenemos una relación de equivalencia
$\simr$ en una $\Sup$-semirre\-tícula $A$.
Si $X,Y\subseteq A$ son dos subconjuntos 
indicados por un conjunto $I$
\begin{align*}
    X &= \{x_i \mid i\in I\} &
    Y &= \{y_i \mid i\in I\}
\end{align*}
tales que para cada índice $i\in I$ se tiene $x_i\sim y_i$,
entonces usaremos la notación $X\sim Y$.

Decimos que una relación de equivalencia $\simr\subseteq A\times
A$ es una $\Sup$-congruencia si, para todo par de subconjuntos
$X,Y\subseteq A$ con $X\sim Y$, se tiene $\Sup X\sim \Sup Y$.

El primer resultado es
\begin{lemma}[Todo morfismo induce una congruencia]
  Sea $\phi:A\to B$ un $\Sup$-morfismo.
  Entonces la relación de equivalencia en $A$ inducida por
  $\phi$:
  \[
      a\sim b \ssi \phi(a) = \phi(b)
  \]
  es una $\Sup$-congruencia.
\end{lemma}
\begin{proof}
  Tomemos subconjuntos $X\sim Y\subseteq A$ indicados por $I$.
  Esto es, para todo $i\in I$ tenemos $x_i\sim y_i$,
  por lo cual $\phi(x_i)=\phi(y_i)$.
  Luego,
  \begin{align*}
      \phi(\Sup X)
      &= \Sup\{\phi(x_i)\mid i\in I\} \\
      &= \Sup\{\phi(y_i)\mid i\in I\} \\
      &= \phi(\Sup Y).
  \end{align*}
  Luego, $\Sup X \sim \Sup Y$, por lo cual $\sim$ es una
  $\Sup$-congruencia.
\end{proof}

%  Si $X,Y\subseteq A$ son dos subconjuntos con la misma
%  imagen $f(X)=f(Y)$, entonces $X\sim Y$.
%  La condición $f(X)=f(Y)$ implica que,
%  para cada $x\in X$, existe $y(x)\in Y$
%  tal que $x\sim y(x)$ y, recíprocamente,
%  para cada $y\in Y$ existe $x(y)\in X$ tal que $y\sim x(y)$.
%  Ahora tomemos $I=X\cup Y$ y definamos
%  \begin{align*}
%      x_i &=
%          \begin{cases}
%              i & i\in X \\
%              x(i) & i\nin X
%          \end{cases}
%      &
%      y_i &=
%          \begin{cases}
%              y(i) & i\in X \\
%              i & x\nin X
%          \end{cases}
%  \end{align*}
%  para todo $i\in I$.
%  Notemos que $X=\{x_i\mid i\in I\}$, $Y=\{y_i\mid i\in I\}$ y
%  $X\sim Y$.

\begin{lemma}[Toda congruencia produce un cociente]
  \label{lemma:congruencia-produce-cociente}
  %[Alfredo $\checkmark$]
  Dada una $\Sup$-semiretícula $A$ y una $\Sup$-congruencia
  $\simr$ sobre $A$, la estructura de semirretícula superior en
  $A/\simr$ inducida por $\eta:A\to A/\simr$ es el único orden
  que convierte a $A/\simr$ en una $\Sup$-semirretícula y
  a $\eta:A\to A/\simr$ en un $\Sup$-morfismo.
\end{lemma}
\begin{proof}
  Sean $[a],[b]\in A/\simr$.
  Como $\simr$ es una $\Sup$-congruencia, el bloque de
  $[a\sup b]$ no depende de los representantes,
  ya que $a\sim a'$ y $b\sim b'$ implica
  que $a\sup b\sim a'\sim b'$.
  Luego, podemos definir una operación en $A/\simr$ como
  \begin{align*}
      [a]\sup[b]
      &= \Sup\{[a],[b]\} \\
      &= \Sup\eta(\{a,b\}) \\
      &= \eta(a\sup b) \\
      &= [a\sup b].
  \end{align*}
  para cualesquiera $[a],[b]\in A/\simr$.
  Como $A$ es un monoide conmutativo idempotente, entonces esta
  operación en $A/\simr$ también lo convierte en uno,
  por lo cual la relación
  \[
      [a]\leq[b] \ssi [a\sup b]=[b]
  \]
  es un orden parcial en $A/\simr$ cuyo supremo (binario) es
  $\sup$. (Ver \ref{ss:semireticulas}).
  Más aún, la igualdad $[a]\sup[b]=[a\sup b]$ nos dice que $\eta$
  es un morfismo de semirretículas superiores y, de hecho, esta
  es la única estructura de semirretícula superior en $A/\simr$
  que lo logra, ya que si $\sup'$ es otra estructura en $A/\simr$
  compatible con $\eta$, entonces tenemos
  \begin{align*}
    [a]\sup'[b]
    &= \eta(a)\sup'\eta(b) \\
    &= \eta(a\sup b) \\
    &= [a\sup b] \\
    &= [a]\sup[b].
  \end{align*}
  
  Ahora nos falta ver que $A/\simr$ es superiormente completo
  y que $\eta$ preserva los supremos arbitrarios.
  Notemos que cualquier subconjunto $Y\subseteq A/\simr$ es de la
  forma $Y=\eta(X)$ para algún $X\subseteq A$: por ejemplo,
  $Y=\eta(\eta^{-1}(Y))$.

  Si probamos que el supremo de $\eta(X)$ es $\eta(\Sup X)$,
  esto mostrará, al mismo tiempo, que $A/\simr$ tiene supremos
  arbitrarios y que la función $\eta:A\to A/\simr$ es un
  $\Sup$-morfismo.
  
  Tomemos $\eta(X)\subseteq A/\simr$.
  Como $\eta$ preserva supremos binarios,
  se sigue que $\eta$ es monótona.
  Luego, $\eta(\Sup X)$ es cota superior de $\eta(X)$.
  Ahora sea $[a]\in A/\simr$ una cota superior de $\eta(X)$;
  esto es: $[x]\leq[a]$
  (o bien, $[x\sup a]=[a]$, o bien, $x\sup a\sim a$)
  para todo $x\in X$.
  Luego, poniendo
  \begin{align*}
      Y &= \{x\sup a\mid x\in X\}
      &
      Z &=\{a\mid x\in X\},
  \end{align*}
  (los cuales son conjuntos indicados por $X$)
  tenemos $Y\sim Z$.
  Como $\simr$ es una $\Sup$-congruencia, tenemos
  $\Sup Y\sim\Sup Z$, pero esto es
  \[
      \Sup X\sup a\sim a
  .\]
  Se sigue que $[\Sup X\sup a]=[a]$. Es decir, $[\Sup
  X]\sup[a]=[a]$ o, equivalentemente $[\Sup X]\leq[a]$.
  Así,
  \[
      \eta(\Sup X)\leq [a]
  .\]
  Esto muestra que $\eta(\Sup X)$ es el supremo de $\eta(X)$
  en $A/\simr$:
  \[
    \eta(\Sup X) = \Sup \eta(X)
  .\]
  Así, $A/\simr$ es superiormente completa y
  $\eta$ es un $\Sup$-morfismo.
\end{proof}

El resultado anterior nos dice que en la
categoría de $\Sup$-semirretículas también existe la noción de
cociente, al igual que en las categorías más familiares de
anillos, grupos, etc. Una $\Sup$-semirretícula $B$ es cociente de
$A$ si es de la forma $A/\simr$ para alguna $\Sup$-congruencia
$\simr$ en $A$. Más formalmente, hacemos la siguiente definición:
\begin{defn}
  Decimos que un morfismo $f:A\to B$ es un morfismo cociente de
  $A$, o que exhibe a $B$ como cociente de $A$, si hay una
  congruencia $\simr$ en $A$ y un isomorfismo $A/\simr\simeq B$
  de forma que el triángulo
  \[
    \begin{tikzcd}
      A \ar[d,"\eta"'] \ar[r,"f"] & B \\
      A/\simr \ar[ur,"\simeq"']
    \end{tikzcd}
  \]
  es conmutativo.
  En este caso, decimos que $B$ es el cociente de $A$ por
  $\simr$.
\end{defn}
Con esta definición se entiende el título del lema anterior:
``toda congruencia produce un cociente''.

\begin{obs}[Correspondencia entre congruencias y cocientes]
\leavevmode
  \begin{itemize}
    \item
    Toda $\Sup$-congruencia $\simr$ es la $\Sup$-congruencia
    inducida por su morfismo cociente $\eta:A\to A/\simr$.
    \item
    Todo morfismo cociente $f:A\to B$ es el cociente de $A$ por la
    congruencia $\simr$ inducida por $f$.
  \end{itemize}
\end{obs}
Así, los morfismos cociente son exactamente los
morfismos suprayectivos.

\begin{prop}[Teorema del factor, versión 1]
  Sea $A$ una $\Sup$-semirretícula y $\simr$ una
  $\Sup$-congruencia en $A$.
  Si $f:A\to B$ un $\Sup$-morfismo que respeta a $\simr$,
  entonces $f$ se factoriza de manera única
  a través de $\eta:A\to A/\simr$.
\end{prop}
\begin{proof}
  Como $f$ respeta a $\simr$ como relación de equivalencia,
  entonces se factoriza de manera única a través de $A/\simr$ en
  la categoría de conjuntos:
  \[
    \begin{tikzcd}
      A \ar[rr,"f"] \ar[dr,"\eta"'] & & B \\
      & A/\simr \ar[ur,"f^\sharp"']
    \end{tikzcd}
  \]
  Notemos que tanto $f$ como $\eta$ son $\Sup$-morfismos, así
  que basta ver que $f^\sharp$ también es un $\Sup$-morfismo,
  pero esto es fácil: cualquier subconjunto de $A/\simr$ tiene la
  forma $\eta(X)$ para algún $X\subseteq A$, así que
  \begin{align*}
    f^\sharp(\Sup \eta(X))
    &= f^\sharp(\eta(\Sup X)) \\
    &= f(\Sup X) \\
    &= \Sup f(X) \\
    &= \Sup f^\sharp(\eta(X)).
  \end{align*}
\end{proof}
Además, si $\simr$ es la congruencia inducida por $f$,
entonces $f$ se factoriza como $f=f^\sharp\eta$, con $f^\sharp$
inyectivo y $\eta$ suprayectivo.

\subsection{Operadores cerradura}

El apartado anterior vimos que las $\Sup$-semirretículas tienen
muchos paralelismos con los grupos, anillos, módulos, etc.
Sin embargo, en éstas últimos casos, las congruencias tienen
representaciones alternativas: las congruencias de un grupo están
en correspondencia con sus subgrupos normales,
las congruencias de un módulo están en correspondencia con sus
submódulos, las congruencias de un anillo están en
correspondencia con sus ideales bilaterales, etc.

En vista de esto, queremos encontrar una representación
alternativa de las congruencias de una $\Sup$-semirretícula.
De hecho, encontraremos no una, sino dos de estas
representaciones: los operadores cerradura y los
subconjuntos $\Inf$-cerrados.

Dada una $\Sup$-congruencia $\simr$ en $A$, sabemos construir la
$\Sup$-semirretícula cociente $A/\simr$.
Sin embargo, los elementos de $A/\simr$ son los bloques $[x]$ de
$A$ con respecto a $\simr$ y a veces es molesto trabajar con
clases de equivalencia. En otro tipo de estructuras como espacios
vectoriales y anillos, podríamos pensar en tomar un representante
de cada bloque y trabajar con ellos en vez de trabajar con los
bloques, pero no hay una eleccción canónica de representantes.
Sin embargo, en el caso de las $\Sup$-semirretículas cada bloque
tiene un representante natural: su supremo.
Para que esta idea funcione bien, lo primero que hay que
verificar es que el supremo de cada bloque sigue estando en el
bloque:
\begin{lemma}
  Si $\sim$ es una $\Sup$-congruencia en $A$, entonces cada
  bloque de $\simr$ tiene un mayor elemento.
  Es decir, para cada $[a]\in A/\simr$, tenemos $\Sup[a]\in [a]$.
\end{lemma}
\begin{proof}
  Tomemos $[a]\in A/\simr$ y definamos nuestro conjunto de
  índices como $I=[a]$.
  Sean
  \begin{align*}
      X &= \{x_i \mid i\in I\} \\
      Y &= \{y_i \mid i\in I\}
  \end{align*}
  donde $x_i=i$ y $y_i=a$ para cada $i\in I=[a]$.
  Por construcción, tenemos $x_i\sim y_i$ para cada $i\in I$:
  esto es, $X\sim Y$.
  Luego, como $\sim$ es una $\Sup$-congruencia, se sigue que
  $\Sup X\sim\Sup Y$.
  Esto es,
  \[
      \Sup[a] = \Sup X \sim \Sup Y = a
  .\]
  Por lo tanto, $\Sup [a]\in [a]$.
\end{proof}
Este resultado dice que, si tomamos un bloque $[a]\in A/\simr$,
nos fijamos en su supremo y luego volvemos a bajar con $\eta:A\to
A/\simr$, caemos en el bloque en el que empezamos.
Es decir:
\[
    \eta(\Sup [a]) = [a]
.\]
Ahora consideramos la otra composición:
\begin{defn}
    Sea $\simr$ una $\Sup$-congruencia en $A$.
    El \emph{selector} de $\simr$ es la función $j:A\to A$ dada por
    \[
        j(a) = \Sup\eta(a) = \Sup[a] = \Sup\{x\in A \mid x\sim a\}
    .\]
\end{defn}
El selector tiene algunas propiedades interesantes.
\begin{itemize}
    \item 
    Si $x\sim a$, entonces $x\leq j(a)$.
    En particular, $a\leq j(a)$, ya que $a\sim a$.
    Decimos que $j$ infla.
    \item
    Supongamos que $a\leq b$.
    Dado que $a\sim j(a)$ y $b\sim j(b)$, entonces
    \[
        b = a\sup b \sim j(a) \sup j(b)
    \]
    porque $\simr$ es $\Sup$-congruencia.
    Luego, $j(a) \sup j(b) \leq j(b)$.
    Entonces
    \[
        j(a)\leq j(a)\sup j(b) \leq j(b)
    .\]
    Es decir, $j$ es monótona.
    \item
    Ahora tomemos $a\in A$.
    Sabemos que $a\sim j(a)$ y que $j(a)\sim j(j(a))$,
    por lo cual $a\sim j(j(a))$.
    Luego, $j(j(a))\leq j(a)$.
    Por otro lado, también tenemos $j(a)\leq j(j(a))$ (porque $j$ infla) así que
    $j(a)=j(j(a))$.
    En otras palabras, $j$ es idempotente.
\end{itemize}
\begin{defn}[Operador cerradura]
  \label{def:operador-cerradura}
  Si $A$ es un copo, un operador cerradura en $A$ es una función
  $j:A\to A$ tal que
  \begin{itemize}
    \item Infla: $a\leq j(a)$ para todo $a\in A$
    \item Es monótona: $a\leq b$ implica que $j(a)\leq j(b)$.
    \item Es idempotente: $j(j(a))=a$.
  \end{itemize}
\end{defn}
Así, lo que acabamos de mostrar es que, si $j$ es el selector de
una $\Sup$-congruencia $\simr$, entonces $j$ es un operador
cerradura.
Ahora queremos ver el camino de regreso: dado un operador
cerradura $j$ en $A$, ¿podemos obtener una $\Sup$-congruencia de
manera natural? La respuesta es sí y, de hecho, estas
construcciones son inversas una de la otra.
\begin{lemma}%[Juan $\checkmark$ ]
  Sea $A$ una $\Sup$-retícula. Los selectores de $A$ son
  precisamente los operadores cerradura. Además cada operador
  cerradura es selector para una única $\Sup$-congruencia. Es
  decir, hay una relación biyectiva entre $\Sup$-congruencias y
  operadores cerradura en $A$.
\end{lemma}
\begin{proof}
  Ya sabemos que el selector de una $\Sup$-congruencia es un
  operador cerradura.

  Ahora tomemos un operador cerradura $j:A\to A$.
  Como $j$ es una función, sabemos que la relación
  \[
    x\sim y\ssi j(x)=j(y)
  \]
  es de equivalencia.
  Veamos que es una $\Sup$-congruencia.
  Supongamos que $X,Y\subseteq A$ son subconjuntos indicados por
  un conjunto $I$ y que $X\simr Y$.
  Entonces para cualquier $y_i\in Y$ tenemos $y_i\sim x_i$ con
  $x_i\in X$. Como $j$ infla y es monótono, tenemos
  \[
    y_i\leq j(y_i)=j(x_i)\leq j(\Sup X)
  \]
  Así, $\Sup Y\leq j(\Sup X)$.
  Aplicando $j$ obtenemos $j(\Sup Y)\leq j(\Sup X)$, pues $j$ es
  idempotente.
  Análogamente, tenemos $j(\Sup X)\leq j(\Sup Y)$, por lo cual
  $j(\Sup X)=j(\Sup Y)$ y así $\Sup X\sim\Sup Y$, así que
  $\simr$ es una $\Sup$-congruencia.

  Ahora veamos que estas correspondencias son inversas una de la
  otra.
  
  Sea $j$ un operador cerradura y $\simr$ su $\Sup$-congruencia
  asociada.
  Para ver que $j$ es el selector de $\simr$, basta ver que
  $j(y)$ es el supremo del bloque de $y$ con respecto a $\simr$.
  Notemos que, si $x\sim y$, entonces $x\leq j(x)=j(y)$ porque
  $j$ infla, así que $j(y)$ es cota superior del bloque de $y$.
  Además, tenemos $y\sim j(y)$, ya que $j(y)=j(j(y))$. Por lo
  tanto, si $z$ es una cota superior del bloque de $y$, entonces
  en particular $j(y)\leq z$.

  Ahora sea $\simr$ una $\Sup$-congruencia y $j$ su selector.
  Debemos ver que $\simr$ es la congruencia de $j$. Es decir,
  mostraremos que
  \[
    x\sim y \ssi j(x)=j(y)
  .\]
  Por un lado $x\sim y$ significa que $[x]=[y]$, por lo cual sus
  supremos $j(x)$ y $j(y)$ son iguales.
  Por otro lado, dado que $x\simr j(x)$ para todo $x\in A$, si
  $j(x)=j(y)$ entonces tenemos $x\sim j(x)=j(y)\sim y$, con lo
  cual terminamos.
\end{proof}

\subsubsection{Conjuntos \tps{$\Inf$}{Inf}-cerrados}

Vimos que, en una $\Sup$-semirretícula, hay una correspondencia
entre operadores cerradura y $\Sup$-congruencias.
De hecho, ahora veremos que la correspondencia es triple:
los operadores cerradura también están en correspondencia con los
subconjuntos de $A$ que son cerrados bajo ínfimos arbitrarios.

\begin{defn}
    Si $A$ es una $\Sup$-semirretícula,
    un subconjunto $F\subseteq A$ se dice que es $\Inf$-cerrado si,
    para todo subconjunto $X\subseteq F$, se tiene $\Sup X\in F$.
\end{defn}

Ahora mostraremos que hay una correspondencia entre operadores
cerradura y conjuntos $\Inf$-cerrados.

\begin{lemma}\leavevmode
  \begin{enumerate}
    \item
    Dado un operador cerradura $j$ en $A$, el conjunto de sus
    puntos fijos
    \[
        A_j = \{x\in A \mid j(x) = x\}
    \]
    es $\Inf$-cerrado.
    \item
    Dado un conjunto $\Inf$-cerrado $F\subseteq A$, la función
    $j_F:A\to A$ dada por
    \[
        j_F(a) = \Inf\{x\in F \mid a\leq x\}
    \]
    es un operador cerradura.
    \item
    Las construcciones $F\mapsto j_F$
    y $j\mapsto A_j$ son inversas una de la otra y, por lo tanto,
    establecen una biyección entre operadores
    cerradura en $A$ y los conjuntos $\Inf$-cerrados de $A$, que,
    además, están en biyección con las $\Sup$-congruencias de
    $A$.
  \end{enumerate}
\end{lemma}
\begin{proof}
  \begin{enumerate}
    \item
    Tomemos un subconjunto $X\subseteq A_j$.
    Para cualquier $x\in X$ tenemos $\Inf X\leq x$, por lo cual
    \[
        j(\Inf X) \leq j(x) = x
    \]
    ya que $j$ es monótona y $x\in A_j$.
    Esto significa que $j(\Inf X)$ es cota inferior de $X$, por
    lo cual
    \[
        j(\Inf X)\leq \Inf X
    \]
    y, además, la otra desigualdad $\Inf X\leq j(\Inf X)$ se da
    porque $j$ infla.
    Luego, $j(\Inf X) = \Inf X$, pero esto es $\Inf X \in A_j$.
    \item
    En efecto, dado que cualquier $a\in A$ es cota inferior de
    $\{x\in F \mid a\leq x\}$, se sigue que $a\leq j_F(a)$.

    Ahora, dados $a\leq b\in A$, tenemos
    \[
        \{x\in F \mid a\leq x\} \supseteq \{x\in F \mid b\leq x\}
    \]
    lo cual, tomando ínfimos, nos da $j_F(a) \leq j_F(b)$.

    Finalmente, como $F$ es $\Inf$-cerrado, para cualquier $a\in A$
    tenemos $j_F(a)\in F$ (por definición de $j_F$),
    de modo que $j_F(a) \in\{x\in F \mid j_F(a) \leq x\}$.
    Una vez más, por la definición de $j_F$ se sigue que
    $j_F(j_F(a)) \leq j_F(a)$
    y la otra desigualdad $j_f(a)\leq j_F(j_F(a))$ es porque
    $j_F$ infla.
    \item
    Sea $F\subseteq A$ un conjunto $\Inf$-cerrado.
    Entonces
    \begin{align*}
        j_F(a) &= \Inf\{x\in F \mid a\leq x\} \\
        A_{j_F} &= \{a\in A \mid j_F(a)=a\}.
    \end{align*}
    Observemos que $F\subseteq A_{j_F}$, ya que
    $a=\Inf\{x\in F\mid a\leq x\}$ para todo $a\in F$.
    Por otro lado,
    como $F$ es $\Inf$-cerrado, tenemos $j_F[A]\subseteq F$.
    Luego, $A_{j_F}=j_F[A_{j_F}] \subseteq F$ y, así, $F=A_{j_F}$.
    
    Recíprocamente, sea $j:A\to A$ un operador cerradura.
    Entonces
    \begin{align*}
        A_j &= \{a\in A \mid j(a)=a \} \\
        j_{A_j}(a) &= \Inf\{x\in A_j \mid a\leq x\}.
    \end{align*}
    Como $j$ infla y es idempotente, todo $a\in A$ satisface
    \[
        j(a) \in \{x\in A_j \mid a\leq x\}
    .\]
    Además, $j(a)$ es una cota inferior del conjunto,
    pues para todo
    $x\in A_j$ con $a\leq x$ tenemos $j(a)\leq j(x)=x$.
    Por lo tanto,
    \[
        j(a) = \Inf\{x\in A_j \mid a\leq x\} = j_{A_j}(a)
    .\]
    Esto muestra que $j=j_{A_j}$.
    Luego, las construcciones $F\mapsto j_F$ y $j\mapsto A_j$
    son inversas, lo cual establece la biyección deseada.
  \end{enumerate}
\end{proof}

\subsubsection{La correspondencia entre operadores cerradura y cocientes}

Al principio de esta sección,
vimos que todo $\Sup$-morfismo $f:A\to B$ induce una
$\Sup$-congruencia en $A$. Por lo tanto, también induce un
operador cerradura $k:A\to A$

\begin{defn}
    Para cada $\Sup$-morfismo $f:A\to B$, el núcleo de $\phi$
    es el operador cerradura $k:A\to A$ que le corresponde a
    la $\Sup$-congruencia inducida por $f$.
    Es decir $k$ está determinado por la condición
    \[
        k(x) = k(y) \ssi f(x) = f(y)
    .\]
\end{defn}

También vimos que toda $\Sup$-congruencia en $A$ produce un
cociente de $A$. Por lo tanto, todo operador cerradura $k:A\to A$
produce un cociente de $A$.

Como estas correspondencias son biyecciones,
cualquier cociente $f:A\to B$ es el cociente producido por
su núcleo, mientras que todo operador cerradura es el
núcleo del cociente que produce.

Ahora veremos que estas correspondencias tienen descripciones más
directas.

Si $f:A\to B$ es un $\Sup$-morfismo, entonces preserva supremos,
por lo cual tiene un adjunto derecho $f_*:B\to A$.
(Ver \ref{ss:adj-copos}).
Es decir, $f_*$ es un morfismo de copos determinado por la
condición
\[
    a\leq f_*(b) \ssi f(a) \leq b
\]
y está dado por la fórmula
\[
    f_*(b) = \Sup\{x\in A \mid f(x) \leq b\}
.\]
\begin{prop}[El operador cerradura de un morfismo]
  Si $f:A\to B$ es un $\Sup$-morfismo y $f_*:B\to A$ es su adjunto
  derecho, entonces la función $k=f_*f:A\to A$ es un operador
  cerradura y, de hecho es el núcleo de $f$.
\end{prop}
\begin{proof}
  Primero veremos que $k=f_*f$ es un operador cerradura.
  \begin{itemize}
    \item
    $k$ es monótona porque $f$ y $f_*$ lo son.
    \item 
    Usando la fórmula para $f_*$, tenemos
    \[
        k(a) = f_*f(a) = \Sup\{x\in A \mid f(x) \leq f(a)\} \geq
        a
    ,\]
    pues $a\in\{x\in A\mid f(x)\leq f(a)\}$.
    Así, $k$ infla.
    \item
    Para ver la idempotencia de $k$, calculemos $ff_*f:A\to B$.
    \begin{align*}
        ff_*f(a)
        &= f(k(a)) \\
        &= f(\Sup\{x\in A \mid f(x) \leq f(a) \}) \\
        &= \Sup\{f(x) \mid f(x) \leq f(b) \} \\
        &= f(a)
    \end{align*}
    Es decir, $ff_*f=f$. Por lo tanto, $k^2 = f_*ff_*f=f_*f=k$,
    como se quería
\end{itemize}

Ahora veamos que $k=f_*f$ cumple la propiedad de ser el núcleo de
$f$; es decir:
\[
  k(x)=k(y) \ssi f(x)=f(y)
.\]
Tomemos $x,y\in A$ tales que $k(x)=k(y)$,
esto es, $f_*f(x) = f_*f(y)$, o bien
\begin{align*}
  f_*f(x) &\leq f_*f(y)
  &
  f_*f(y) &\leq f_*f(x).
\end{align*}
Como $f_*$ es adjunto derecho de $f$, esto sucede si, y solo si,
\begin{align*}
  ff_*f(x) &\leq f(y)
  &
  ff_*f(y) &\leq f(x).
\end{align*}
Recordando que $ff_*f=f$ (de la demostración de idempotencia),
esto es equivalente a
\begin{align*}
  f(x) &\leq f(y)
  &
  f(y) &\leq f(x),
\end{align*}
es decir, $f(x)=f(y)$. Esto prueba que $k=f_*f$ es el núcleo de
$f$, como se quería.
\end{proof}

Esto nos permite probar otra caracterización del nucleo de un
$\Sup$-morfismo:

\begin{cor}
  Dado un $\Sup$-morfismo $f:A\to B$, el núcleo de $f$
  es el operador cerradura $k:A\to A$ determinado por la condición
  \[
      x\leq k(a) \ssi f(x) \leq f(a)
  .\]
\end{cor}
\begin{proof}
  Recordemos que la adjunción nos da la equivalencia
  \[
      x\leq f_*(y) \ssi f(x) \leq y
  .\]
  Tomando $y=f(a)$, tenemos la equivalencia deseada.
\end{proof}

Ahora veamos el regreso. Dado un operador cerradura $j$, podemos
obtener una descripción sencilla del morfismo cociente asociado.
Sabemos que el conjunto de puntos fijos $A_j$
de $j$ es $\Inf$-cerrado y, por lo tanto, es una retícula
completa con el orden inducido de $A$, aunque los supremos,
en general, no coinciden con lo supremos en $A$.

Por otro lado, notemos que la función $j^*:A\to A_j$ definida
como $j^*(a)=j(a)$ es suprayectiva e induce en $A$ la relación de
equivalencia
\[
  a\sim b \ssi j(a)=j(b)
,\]
que es, precisamente, la congruencia en $A$ inducida por $j$.
Por lo tanto, se sigue que el orden en $A_j$ inducido por $j^*$
es el únco orden que hace que $j^*$ sea un $\Sup$-morfismo.

\begin{lemma}
  Sean $j:A\to A$ un operador cerradura y $j^*:A\to A_j$ la
  restricción de $j$ a su imagen.
  Entonces el orden en $A_j$ heredado de $A$, que convierte a
  $A_j$ en una $\Sup$-semirretícula, concide con el
  orden inducido por $j^*$ que convierte a $j^*$ en un
  $\Sup$-morfismo.
\end{lemma}
\begin{proof}
  Gracias al lema \ref{lemma:congruencia-produce-cociente},
  basta ver que el orden en $A_j$ heredado de $A$ convierte
  a $j^*$ en un morfismo de semirretículas superiores.
  Es decir, denotando como $\sup_j$ al supremo calculado en $A_j$
  con el orden inducido de $A$, basta ver que, para cualesquiera
  $a,b\in A_j$, tenemos $a\sup_j b=j(a\sup b)$.

  Por un lado, como $j$ infla, tenemos
  $a,b\leq a\sup b\leq j(a\sup b)$.
  Así, $j(a\sup b)\in A_j$ es cota superior de $a$ y $b$.
  Además, si $z\in A_j$ está por encima de $a$ y $b$, tenemos
  $a\sup b\leq z$, por lo cual $j(a\sup b)\leq j(z)=z$.
  Se sigue que $j(a\sup b)$ es el supremo de $a$ y $b$ en $A_j$.
\end{proof}
\begin{cor}
  Para cualquier subconjunto $Y\subseteq A_j$, se tiene
  \[
      \Sup_j Y = j(\Sup Y)
  ,\]
  donde $\Sup_j$ denota como al supremo calculado en $A_j$.
\end{cor}
\begin{proof}
  Considerando a $Y$ como subconjunto de $A$, tenemos
  \[
    j(\Sup Y)=\Sup_j j(Y) =\Sup_j Y
  \]
  porque $j^*$ es $\Sup$-morfismo y $j(Y)=Y$.
\end{proof}

Con toda la información que tenemos,
podemos probar el teorema del factor
en la categoría de $\Sup$-semiretículas.
\begin{thm}[Teorema del factor en $\SupLat$]
    Sea $A$ una $\Sup$-semiretícula y $k$ un operador cerradura en $A$.
    Si $f:A\to B$ es un morfismo cuyo núcleo $k$ satisface $j\leq k$,
    entonces existe un único morfismo $f^\sharp:A_j\to B$ tal que
    el siguiente triángulo conmuta.
    \[
        \begin{tikzcd}
            A \ar[rr,"f"] \ar[dr,"j^*"'] &&  B \\
            &A_j \ar[ur,"f^\sharp"']
        \end{tikzcd}
    .\]
\end{thm}
    
%\section*{(SESIÓN 7: 30 SEP)}

\section{Cocientes en \tps{$\Frm$}{Frm}: núcleos}
En el caso de $\Sup$-semiretículas, teníamos la correspondencia
entre operadores cerradura, conjuntos $\Inf$-cerrados
y $\Sup$-congruencias.
Cuando trabajamos en un marco, esta correspondencia se puede refinar
más.

\begin{defn}[Núcleo en un marco]
\label{def:nucleo}
Un núcleo en un marco $A$ es un operador cerradura $j:A \to A$ que cumple con $j(a\wedge b)=j(a)\wedge j(b) \ \forall \ a,b \ \in \ A$.
Al conjunto de núcleos de $A$ lo denotamos como $NA$.
\end{defn}

Nótese que, para un morfismo de marcos $f: A\to B$, el operador
cerradura $k$ inducido por $f$ (el núcleo de $f$, visto como morfismo
de $\Sup$-retículas) también es un núcleo de marcos, ya que
\begin{align*}
x\leq k(a\wedge b)&\iff f(x)\leq f(a\wedge b)=f(a)\wedge f(b)\\
&\iff f(x)\leq f(a) , f(x)\leq f(b) \\
&\iff x\leq k(a), x\leq k(b)\\
& \iff x\leq k(a)\wedge k(b)
\end{align*}
Siendo $k$ un operador cerradura, el conjunto de puntos fijos $A_k$ es
un conjunto $\Inf$-cerrado, así que es una retícula completa. Sin
embargo, como $k$ es un núcleo (de marcos), entonces $A_k$ tiene
una propiedad extra: se comporta bien con respecto a la implicación.
Esto queda capturado en la siguiente definición y el lema que la
acompaña.

\begin{defn}[Conjunto implicativo]
Decimos que un subconjunto $F$ de un marco $A$ es un conjunto
implicativo si es $\Inf$-cerrado y
$a\in F \implies  \forall x \in A,\; (x\succ a)\in F$.
\end{defn}
\begin{lemma}
  Sea $A$ un marco. Entonces un operador cerradura $k:A\to A$
  preserva ínfimos (es un núcleo de marcos) si, y
  sólo si, su conjunto de puntos fijos $A_k$ es implicativo.
\end{lemma}
\begin{proof}
Sea $k:A\to A$ un operador cerradura.

($\Leftarrow$). Supongamos que $A_k$ es implicativo.
Sean $x,y \in A$.
Por un lado, $k(x\inf y)\leq k(x)\inf k(y)$ por la monotonía de $k$:
Para ver la otra comparación, sea $a=k(x\inf y)\in A_k$. Entonces
\begin{align*}
    x\inf y\leq a
    & \iff y\leq (x\succ a)\in A_k\\
    &\iff k(y)\leq (x\succ a)\\
    &\iff k(y)\inf x\leq a \\
    &\iff x\leq (k(y)\succ a) \\
    &\iff k(x)\leq (k(y)\succ a)\in A_k \\
    &\iff k(x)\inf k(y) \leq a.
\end{align*}
Esto es $k(x\inf y)=k(x)\inf k(y)$. Así, $k\in NA$.

($\Rightarrow$). Supongamos que $k$ es un núcleo. Tomemos $a\in A_k$ y
$x\in A$.
Por un lado, $(x\succ a)\leq k(x\succ a)$. Para ver la otra
comparación, sea $y=(x\succ a)$.
Así, $x\inf y\leq a$, y se cumple que 
$$x\inf k(y)\leq k(x)\inf k(y) =k(x\inf y)\leq k(a)=a$$
lo cual implica que $k(y)\leq (x\succ a)=y$. Por lo tanto, $k(y)=y$ y
$y\in A_k$.
\end{proof}

Recordemos que, si $j\in CA$, entonces $A_j$ es una $\Sup$-semiretícula
y $j^*:A\to A_j$ dada como $j^*(a)=j(a)$
es un morfismo de $\Sup$-semiretículas; esto es:
$$j(\Sup X)=\Sup_jX$$
para cualesquiera $j\in NA, X\subset A_j$, donde $\Sup_j$ denota
al supremo en $A_j$.
Nótese que, si $j\in NA$, entonces $A_j$ es una retícula completa
que, por el lema anterior, tiene implicación, así que $A_j$ es un
marco y $j$ es un morfismo de marcos, ya que $j$ preserva
ínfimos.
Poniendo juntas estas observaciones, obtenemos el siguiente
resultado.
\begin{lemma}
    Sea $j\in CA$.
    Entonces $j\in NA$ si, y solo si,
    $A_j$ es un marco (con el orden heredado de $A$)
    y $j^*:A\to A_j$ es un morfismo de marcos suprayectivo.
\end{lemma}
\begin{defn}
  Un cociente de un marco $A$ es un marco $B$ equipado con un morfismo
  suprayectivo $f:A\to B$.
\end{defn}

\begin{exe}%[Dante $\checkmark$ ]
Para cualquier marco $A$, existe una biyección entre el conjunto de cocientes de $A$ y $NA$.
\end{exe}
\begin{sol}
    Defínase $\mathcal{C}=\{\text{Cocientes de } \ A\}$. Sea $f:A\to B\in \mathcal{C}$. Así, $F=f_*\circ f\in NA$, y si $g:A\to B\in\mathcal{C}$ es tal que $G=g_*\circ g=f_*\circ f=F$, entonces $A_F=A_G$, y $F^*=G^*$, donde $F^*:A\to A_F$ se define como $F^*(a)=F(a)$. Sabemos entonces que existe un único morfismo $f^\sharp:A_F\to B$ tal que el diagrama
    \[
        \begin{tikzcd}
            A \ar[rr,"f"] \ar[dr,"F^*"'] &&  B \\
            &A_j \ar[ur,"f^\sharp"']
        \end{tikzcd}
    \]
    conmuta, así como $g^\sharp$ que cumple lo mismo para $g$ y $A_G$. Sin embargo, como $G=F$, $A_F=A_G$ y $F^*=G^*$, entonces $f=F^\sharp\circ F^*=G^\sharp\circ G^*=G$, y por lo tanto la función $\phi:\mathcal{C}\to NA$ definida como $\phi(f)=F=f_*\circ f$ es inyectiva. También, si $j\in NA$, claramente $j^*:A\to A_j$ es un cociente de $A$, con $j_*\circ j =j$. Por lo tanto, $\phi$ es una biyección entre $\mathcal{C}$ y $NA$.
\end{sol}

\begin{thm}[Teorema del factor]%[Dante $\checkmark$]
\label{thm:factor-frm}
Sean $A\in \Frm$ y $j\in NA$, y considérese $f:A\to B$ morfismo de marcos con núcleo $k\geq j$. Entonces existe un único morfismo $f^{\sharp}:A_j \to B$ tal que el diagrama
    \[
        \begin{tikzcd}
            A \ar[rr,"f"] \ar[dr,"j^*"'] &&  B \\
            &A_j \ar[ur,"f^\sharp"']
        \end{tikzcd}
    \]
conmuta.
\end{thm}
\begin{proof}
    Dado que $j$ es núcleo, $A_j$ es un marco, con $A_j=j(A)$, y $j^*$ es un morfismo de marcos suprayectivo. Tomando en cuenta lo anterior, defínase $f^\sharp:A_j\to B$ como
    $$f^\sharp(j(a))=f(a)$$
    Como $A_j$ es un marco, es cerrado bajo supremos e ínfimos arbitrarios, por lo que 
    \begin{itemize}
    \item Sean $a,b\in A$ tales que $j(a)=j(b)$. Nótese que como $j\in NA$, entonces $a\leq j(a)$ y $b\leq j(b)$, por lo que a su vez $f(a)\leq f(j(a))$ y $f(b)\leq f(j(b))$. También, dado que $k\geq j$, se tiene que
    \begin{align*}
        j(a)=j(b)\leq k(b) \ & \text{de donde, por la definición de} \ k\\
         j(a)\leq k(b) & \iff f(j(a))\leq f(b) \ \text{y}\\
        j(b)\leq k(a)&\iff f(j(b))\leq f(a)
    \end{align*}
    Por lo anterior, se cumplen las desigualdades 
    $f(a)\leq f(b)$ y $f(b)\leq f(a)$, por lo que $f(a)=f(b)$ y $f^\sharp$ está bien definida.
    \item Para $j(a), j(b) \in A_j$ cualesquiera, 
    \begin{align*}
    f^\sharp(j(a)\inf j(b))&=f^\sharp(j^*(a\inf b))\\
    &=f(a\inf b)\\
    &=f(a)\inf f(b)\\
    &=f^\sharp(j^*(a))\inf f^\sharp(j^*(b))
    \end{align*}
    \item Sea $X\subset A_j$. Así, 
    \begin{align*}
        f^\sharp(\Sup X)&=f^\sharp(\Sup j^*(X^\prime))\\
        &=f^\sharp(j^*(\Sup X^\prime))\\
        &=f(\Sup X^\prime)\\
        &=\Sup f(X^\prime)\\
        &=\Sup f^\sharp(X)
    \end{align*}
    Por lo anterior, $f^\sharp$ es morfismo de marcos. Claramente, $f^\sharp\circ j^*=f$, y si $g:A_j\to B$ es tal que $g\circ j^*=f$, entonces $g(j(a))=f(a)$ $\forall j(a)\in A_J$, por lo que $g=f^\sharp$. Así, $f^\sharp$ es único.
\end{itemize}
\end{proof}

Esto termina de establecer las correspondencias que queríamos.

\subsection{Algunos núcleos particulares}

En cualquier marco $A$, es fácil encontrar núcleos.
Por ejemplo, los lemas \ref{lemma:w-cerradura}
y \ref{lemma:modalidad-w} implican que, para cualquier elemento
$a\in A$ de un marco $A$, la función $((-\succ a)\succ a):A\to A$
es un núcleo.
Además hay, al menos, otras dos maneras de obtener núcleos
a partir de $a$.

\begin{defn}[Núcleos cerrado, abierto y regular]
Sea $a\in A$ fijo, y considérense los operadores siguientes.
  \begin{itemize}
    \item $\unuc a:A\to A$, definido como $\unuc a(x)= a\sup x$.
    \item $\vnuc a:A\to A$, definido como $\vnuc a(x)=(a\succ
    x)$.
    \item $\wnuc a:A\to A$, definido como
      $\wnuc a(x)=((x\succ a)\succ a)$.
  \end{itemize}
  Los operadores $\unuc a$, $\vnuc a$ y $\wnuc a$ son llamados
  los núcleos cerrado, abierto y regular de $a$, respectivamente.
\end{defn}
\begin{prop}
  Los operadores $\unuc a$, $\vnuc a$ y $\wnuc a$ son núcleos, para
  cualquier $a\in A$.
\end{prop}
\begin{proof}
  En efecto, $\unuc a$ infla, es idempotente
  y monótono. Además, la afirmación de que preserva ínfimos es
  precisamente la ley distributiva.

  Para $\vnuc a$,...
  \todo{termi\-nar}

  Finalmente, la demostración para $\wnuc a$ está
  dada por los lemas \ref{lemma:w-cerradura} y
  \ref{lemma:modalidad-w}.
\end{proof}

\begin{exa}%[Alfredo $\checkmark$ ]
  Encontremos todos los núcleos del marco
  \[
    A\hspace{10mm} = \hspace{10mm}
    \begin{tikzcd}
      & 1 \\
      a \ar[ur,no head] && b\ar[ul,no head] \\
      & 0 \ar[ur,no head] \ar[ul,no head]
    \end{tikzcd}
  .\]
  Dado que todos los núcleos mandan el $1$ al $1$ y
  también preservan ínfimos, cada núcleo está determinado por
  sus valores en $a$ y en $b$, pues $0=a\inf b$.
  Como los núcleos inflan, cada núcleo solo puede mandar $a$ a sí
  misma o al $1$, y $b$ a sí misma o al $1$, así que hay, a lo
  más, $4$ núcleos.
  Los núcleos cerrados, abiertos y regulares
  asociados a $a$ y a $b$ son
  \[ 
      \begin{array}{|c|}
          \hline
          \unuc a \\
          a\mapsto a\sup a = a \\
          b\mapsto a\sup b = 1 \\ \hline
          \unuc b \\
          a\mapsto b\sup a = 1 \\
          b\mapsto b\sup b = b \\
          \hline
      \end{array}
      \begin{array}{c|}
          \hline
          \vnuc a \\
          a\mapsto (a\succ a) = 1 \\
          b\mapsto (a\succ b) = b \\ \hline
          \vnuc b \\
          a\mapsto (b\succ a) = a \\
          b\mapsto (b\succ b) = 1 \\
          \hline
      \end{array}
      \begin{array}{c|}
          \hline
          \wnuc a \\
          a\mapsto ((a\succ a)\succ a) = (1\succ a) = a \\
          b\mapsto ((b\succ a)\succ a) = (a\succ a) = 1 \\
          \hline
          \wnuc b \\
          a\mapsto ((a\succ b)\succ b) = (b\succ b) = 1 \\
          b\mapsto ((b\succ b)\succ b) = (1\succ b) = b \\
          \hline
      \end{array}
  \]
  así,
  \begin{align*}
      \unuc a = \vnuc b = \wnuc a \\
      \unuc b = \vnuc a = \wnuc b
  \end{align*}
  son dos núcleos diferentes, mientras que
  los otros dos núcleos son la identidad $\id=\unuc 0=\wnuc
  0=\vnuc 1$ y la función constante $1$, denotada como $\tp$.
  Notemos que $\tp=\unuc 1=\wnuc 1=\vnuc 0$.
  Para ver la estructura ordenada de $NA$, notemos que
  $\unuc a$ y $\unuc b$ no son comparables, por lo cual
  \[
    NA\hspace{10mm} = \hspace{10mm}
    \begin{tikzcd}
      & \tp \\
      \unuc a \ar[ur,no head] && \unuc b \ar[ul,no head] \\
      & \id \ar[ur,no head] \ar[ul,no head]
    \end{tikzcd}
    \hspace{10mm} \simeq \hspace{10mm} A
  .\]
\end{exa}

\begin{exa}%[Juan $\checkmark$ ]
Sea $S$ un espacio topológico y considemos el marco $\cal O S$.
Para cualesquiera abiertos $A,X\in \cal OS$, tenemos
  \begin{itemize}
    \item $\unuc A(X)=A\cup X$
    \item $\vnuc A(X)=(A\succ X)=(A'\cup X)^\circ$
    \item
      $\wnuc A(X)
      =((X\succ A)\succ A)
      =(((X'\cup A)^\circ)'\cup A)^\circ
      =(\ol{X\cap A'}\cup A)^\circ$
  \end{itemize}
\end{exa}

La razón de la nomenclatura para los núcleos cerrado y abierto se
ilustra en el siguiente ejemplo.

\begin{exa}%[Alfredo $\checkmark$ ]
    \label{exe:cociente-abierto-cerrado}
    Sea $S$ un espacio topológico y $U\in\cal OS$ un abierto.
    Observemos que
    \[
        V \in(\cal OS_{\unuc U}) \iff U\cup V=V \iff U\subset V
    .\]
    Luego,
    \[
        (\cal OS)_{\unuc U} = \{V\in \cal OS \mid U\subset V\}
    .\]
    Es decir, los puntos fijos de $\unuc U$ son los abiertos
    que contienen a $U$.
    Ahora consideremos el cerrado $U^c$ de $S$ y su topología
    como subespacio:
    \[
      \cal OU^c = \{W\cap U^c \mid W\in\cal OS\}
    .\]
    Las funciones
    $\cal OU^c\rightleftarrows(\cal OS)_{\unuc U}$
    dadas como
    \begin{align*}
      V\cap U^c &\mapsto (V\cap U^c)\cup U = V\cup U \\
      W\cap U^c &\mapsfrom W
    \end{align*}
    constituyen un isomorfismo de marcos.
    En efecto, para cualesquiera $V\cap U^c\in\cal OU^c$ y
    $W\in(\cal OS)_{\unuc U}$, tenemos
    \begin{align*}
      ((V\cap U^c)\cup U)\cap U^c
      &= (V\cup U)\cap U^c \\
      &= V\cap U^c, \\
      (W\cap U^c)\cup U
      &= W\cup U \\
      &= W.
    \end{align*}
    Como estas funciones son monótonas, son un isomorfismo de
    copos y, por lo tanto, un isomorfismo de marcos.

    Similarmente, $\vnuc U$ se relaciona con la topología de
    subespacio de $U$:
    \[
      \cal OU
      =
      \{V\in \cal OS : V\subseteq U\}
    .\]
    Las funciones dadas por
    $\cal OU\rightleftarrows(\cal OS)_{\vnuc U}$
    \begin{align*}
         V &\mapsto (U\succ V) \\
         W\cap U &\mapsfrom W.
    \end{align*}
    son un isomorfismo de marcos.
    En efecto, para cualesquiera $V\in \cal OU$
    y $W\in (\cal OS)_{\vnuc U}$, tenemos
    \begin{align*}
        (U\succ V)\cap U
        &= U\cap V \\
        &= V \\
        (U\succ (W\cap U))
        &= (U\succ W)\cap(U\succ U) \\
        &= U\succ W \\
        &= W
    \end{align*}
    Como estas funciones son monótonas,
    son un isomorfismo de copos y, por lo tanto,
    un isomorfismo de marcos.
\end{exa}

\begin{exa}
En general, dado un marco $A$ y un elemento $a\in A$,
los mismos argumentos del ejemplo anterior demuestran que
\begin{align*}
    A_{\unuc a}
    &= \{x\in A\mid a\leq x\} = [a,1]\\
    A_{\vnuc a}
    &\simeq \{x\in A\mid x\leq a\} = [0,a],
\end{align*}
donde el último isomorfismo está dado como $y\mapsto y\inf a$
y $(a\succ x)\mapsfrom x$.
\end{exa}

Por otro lado, los $\wnuc a$ se llaman núcleos regulares por el
caso $a=0$ de un núcleo espacial:

\begin{exa}
  Considerando $\wnuc {\emptyset}:\cal O S\to \cal O S$, definido
  como
  \[
    \wnuc {\emptyset}(U)
    =((U\succ\emptyset)\succ\emptyset)
    =(\overline{U})^{\circ}
  \]
  se tiene que los puntos fijos son 
  $$(\cal OS)_{\wnuc {\emptyset}}
  =\{U\in \cal O S : (\overline{U})^{\circ}=U\}
  =\{\text{Abiertos regulares de }S\}.$$
\end{exa}

Ahora estudiaremos más sobre los núcleos regulares de un marco.
El ejemplo anterior nos mostró que, en un espacio topológico $S$,
los puntos fijos $(\cal OS)_{\wnuc \emptyset}$ son el marco de
abiertos regulares de $S$.
Del siguiente lema se sigue, en particular, que
$(\cal OS)_{\wnuc\emptyset}$ es un álgebra booleana completa.

\begin{lemma}
  \label{lemma:cocientes-w-bool}
  Si $a\in A \in \Frm$, entonces $A_{\wnuc a} \in \CBA$.
\end{lemma}
\begin{proof}
$$\wnuc a (0)=((0\succ a)\succ a)=(1\succ a)=a$$
por lo tanto, $a \in A_{\wnuc a}$ es el menor elemento de $A_{\wnuc a}$.
Mostraremos que todo $x\in A_{\wnuc a}$ es complementado.
Dado $x \in A_{\wnuc a}$, considérese $y=(x\succ a)$.
Por un lado, 
\begin{align*}
    \wnuc a(y) & =\wnuc a(x\succ a)\\
    &=(\wnuc a(x)\succ a)\\
    &=(x\succ a)\\
    &=y
\end{align*}
por lo que $y \in A_{\wnuc a}$. Ahora bien, 
\begin{align*}
    x\inf y&=x\inf(x\succ a)\\
    &=x\inf a\\
    &=a
\end{align*}
y también, 
\begin{align*}
    \wnuc a(x\sup y)&=((x\sup y)\succ a)\succ a)\\
    &=(((x\succ a)\inf(y\succ a))\succ a)\\
    &=((y\inf(y\succ a))\succ a)\\
    &=((y\inf a)\succ a)\\
    &=1.
\end{align*}
Luego, $y$ es el complemento de $x$ en $A_{\wnuc a}$.
Se sigue que $A_{\wnuc a} \in \CBA$.
\end{proof}
Un caso particular del resultado anterior es cuando $a=0$.
En este caso, el álgebra booleana completa $A_{\wnuc 0}$ se
denota como $A_{\neg \neg}$. Como se mencionó antes del lema,
en el caso espacial, esto nos dice que los abiertos regulares de
un espacio topológico forman un álgebra booleana completa.

%\section*{(SESIÓN 8: 5 OCT)}

\begin{lemma}%[Juan $\checkmark$ ]
  \label{lemma:nucleos-densos}
  Para cualquier núcleo $j\in NA$ se tiene la equivalencia
  \[
    j(0)=0 \ssi j\leq \wnuc 0
  .\]
  (C.f. con el lema \ref{lemma:equivalencias}).
  Si un núcleo $j\in NA$ cumple $j\leq\wnuc 0$,
  decimos que $j$ es un núcleo denso.
  Así, $\wnuc 0$ es el mayor núcleo denso en $A$.
\end{lemma}
\begin{proof}
    Supongamos que $j\leq \wnuc 0$.
    Entonces $0\leq j(0)\leq \wnuc 0(0)=((0\succ 0)\succ 0)=0$. Así, $j(0)=0$.

    Supongamos ahora que $j(0)=0$.
    Dado cualquier $x\in A$, queremos mostrar que $j(x)\leq\wnuc
    0 x=((x\succ 0)\succ 0)$ o, equivalentemente, que
    $j(x)\inf(x\succ 0)\leq 0$.
    Como $j$ infla, tenemos
    \begin{align*}
      j(x)\inf(x\succ 0)
      &\leq j(x)\inf j(x\succ 0) \\
      &= j(x\inf (x\succ 0)) \\
      &= j(x\inf 0) \\
      &= j(0) \\
      &= 0,
    \end{align*}
    que es lo que se quería.
\end{proof}

El lema \ref{lemma:cocientes-w-bool} nos dice que los núcleos
regulares producen cocientes booleanos.
Ahora tenemos el recíproco.
\begin{thm}
  Sean $A$ un marco y $j\in NA$.
  Si el cociente $A_j$ es booleano, entonces
  $j=\wnuc a$, donde $a=j(0)$.
  Así, los cocientes booleanos de $A$ están dados, precisamente,
  por los núcleos regulares $\wnuc a$.
\end{thm}
\begin{proof}
    Sea $x\in A$. Como $B$ es booleano, $f(x)$ es complementado.
    Basta mostrar que $f(x\succ a)$ es el complemento $\neg f(x)$
    de $f(x)$, ya que, en este caso, tenemos
    \begin{align*}
        y\leq j(x)
        &\iff f(y) \leq f(x) \\
        &\iff f(y) \inf f(x\succ a) \leq 0
            & \text{caballo de batalla} \\
        &\iff f(y\inf(x\succ a)) \leq 0 = f(0) \\
        &\iff y\inf(x\succ a) \leq j(0) = a \\
        &\iff y\leq ((x\succ a)\succ a) = \wnuc a(x)
    \end{align*}
    y, así, $j(x)=\wnuc a(x)$.
    
    Ahora bien, como $a=j(0)=f_*(f(0))$, se sigue que $f(a)=0$.
    Luego,
    \begin{align*}
        f(x)\inf f(x\succ a)
        &= f(x\inf(x\succ a)) \\
        &= f(x\inf a) \\
        &\leq f(a) \\
        &= 0.
    \end{align*}
    Además, como $f$ es suprayectiva, podemos tomar
    $\tilde x\in A$ con $f(\tilde x)=\neg f(x)$.
    Luego,
    \begin{align*}
        f(x\inf \tilde x)
        &= f(x)\inf\neg f(x) \\
        &= 0 \\
        &= f(0),
    \end{align*}
    lo cual sucede si, y solo si, $x\inf \tilde x\leq j(0)=a$.
    Esto es, $\tilde x\leq(x\succ a)$, así que
    $f(\tilde x)\leq f(x\succ a)$.
    Aplicando supremo con $f(x)$, tenemos
    \[
        1\leq f(x)\sup f(x\succ a)
    .\]
    Concluimos que $f(x\succ a)=\neg f(x)$, lo cual finaliza la prueba.
\end{proof}

\begin{thm}
  Sea $A$ un marco.
  Si $A$ es un álgebra booleana completa y $j\in NA$,
  entonces $j=\unuc a$, donde $a=j(0)$.
  
  Dado que $A$ es booleano, entonces $A_j$ también lo es,
  así que el resultado anterior también nos da $j=\wnuc a$.
\end{thm}
\begin{proof}
    Dado $x\in A$, debemos mostrar que $j(x)=\unuc a(x)=x\sup a$.
    Por un lado, tenemos $0\leq x$, así que $a=j(0)\leq j(x)$.
    Como $x\leq j(x)$, se sigue que $x\sup a\leq j(x)$.
    Por nuestro caballo de batalla, la otra desigualdad
    ($j(x)\leq x\sup a$) es equivalente a $j(x)\inf \neg x\leq a$
    (ya que $x$ es complementado), pero esto es sencillo:
    \begin{align*}
        j(x)\inf\neg x
        &\leq j(x)\inf j(\neg x) \\
        &= j(x\inf\neg x) \\
        &= j(0) \\
        &= a.
    \end{align*}
\end{proof}

\subsection{Núcleos espacialmente inducidos en una topología}

Sea $\phi:T\to S$ un morfismo de espacios topológicos.
El morfismo de marcos $\phi^*=\cal O\phi:\cal OS\to\cal OT$
tiene adjunto derecho dado por
\[
  \phi_*(W)
  = \bigcup\{U\in\cal OS \mid \phi^*(U) \leq W\}
.\]
En el ejemplo \ref{exa:adjunto-derecho-top}, vimos que
$\phi_*:\cal OT\to\cal OS$ se puede calcular como
\[
  \phi_*(W) = \ol{\phi[W']}' = (\phi[W']')^\circ
.\]
De todos modos, aquí repetimos una demostración alternativa.
\begin{proof}
  Recordemos que tenemos la adjunción $\phi[-]\dashv\phi^{-1}$
  entre la imagen directa $\phi[-]:\cal PS\to\cal PT$ y la imagen
  inversa $\phi^{-1}:\cal PT\to\cal PS$.
  Sea $U\in\cal OS$.
  Dado que $\phi^*:\cal OS\to\cal OT$ es la restricción de
  $\phi^{-1}:\cal PS\to\cal PT$, para todo $W\in\cal OT$, tenemos
  \begin{align*}
      \phi^*(U)\leq W
      &\iff \phi^{-1}(U)\leq W \\
      &\iff W'\leq \phi^{-1}(U)' \\
      &\iff W'\leq \phi^{-1}(U') \\
      &\iff \phi[W'] \leq U' \\
      &\iff U\leq \phi[W']' \\
      &\iff U\leq (\phi[W']')^\circ = \ol{\phi[W']}',
  \end{align*}
  como se quería.
\end{proof}

\begin{thm}
  Si $\phi:T\to S$ es un morfismo de espacios topológicos,
  el núcleo $k:\cal OS\to\cal OS$ del morfismo de marcos
  $\phi^*:\cal OS\to\cal OT$ tiene la descripción
  \[
    k(U) = (\phi[T]' \cup U)^\circ
  .\]
\end{thm}
\begin{proof}
    Sabemos que $k=\phi_*\phi^*$.
    Observando que $\phi[\phi^*(U)]=U\cap\phi[T]$, tenemos
    \begin{align*}
        U\leq k(V)
        &\iff U \leq \phi_*(\phi^*(V)) \\
        &\iff \phi^*(U)\leq\phi^*(V) \\
        &\iff \phi[\phi^*(U)] \leq V \\
        &\iff U\cap \phi[T] \leq V \\
        &\iff U\leq \phi[T]'\cup V
            &\text{caballo de batalla en $\cal PS$} \\
        &\iff U\leq {(\phi[T]'\cup V)}^\circ.
    \end{align*}
    Luego, $k(V)={(\phi[T]'\cup V)}^\circ$.
\end{proof}

Sea $S$ un espacio topológico. Para cualquier subconjunto
$E\subset S$, definimos el operador $[E]:\cal O S\to \cal O S$
como $[E](U)={(E\cup U)}^{\circ}$.
Nótese que, si $U \in \cal OS$, entonces $[U]=\unuc U$
y $[U']=\vnuc U$.
Además, dada una función continua $\phi:T\to S$,
el teorema anterior dice que el núcleo
$k:\cal OS\to\cal OS$ de $\phi^*:\cal OS\to\cal OT$ es
$k=[\phi[T]']$. En general, $[E]$ siempre es un núcleo.

\begin{prop}
  Si $S$ es un espacio topológico y $E\subseteq S$ un
  subconjunto, entonces $[E]:\cal OS\to\cal OS$ es un núcleo.
\end{prop}
\begin{proof}
  Sean $U,V\in\cal OS$.
  Primero, $[E]$ es inflacionario, pues
  \begin{align*}
    [E](U)
    &= (E\cup U)^\circ \\
    &\geq (U)^\circ \\
    &= U.
  \end{align*}
  Además, es idempotente: por un lado, por el punto anterior
  tenemos $[E](U)\leq[E]([E](U))$, mientras que
  \begin{align*}
    [E]([E](U))
    &= (E\cup (E\cup U)^\circ)\circ \\
    &\leq (E\cup E\cup U)^\circ \\
    &= (E\cup U)^\circ \\
    &= [E](U).
  \end{align*}
  Además, preserva ínfimos binarios, pues
  \begin{align*}
    [E](U\cap V)
    &= (E\cup(U\cap V))^\circ \\
    &= ((E\cup U)\cap(E\cap V))^\circ \\
    &= ((E\cup U)^\circ\cap(E\cap V))^\circ \\
    &= [E](U)\cap[E](V).
  \end{align*}
  Finalmente, $[E]$ es monótono, ya que, si $U\leq V$, entonces
  \begin{align*}
    [E](U)
    &= (E\cup U)^\circ \\
    &\leq (E\cup V)^\circ \\
    &= [E](V).
  \end{align*}
\end{proof}

\begin{defn}
  Sea $S$ un espacio topológico.
  Un núcleo $j\in N\cal O S$ es \emph{espacialmente
  inducido} si es de la forma $j=[E]$ para algún
  $E\subseteq S$.
\end{defn}

