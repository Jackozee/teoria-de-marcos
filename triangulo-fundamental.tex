\chapter{El triangulo fundamental de un espacio}
\section{La topología de Skula o topología frontera}

Recordemos que, dado un marco $A$, el ensamble $NA$ viene equipado con
un morfismo $\eta_A:A\to NA$ que resuelve el problema de complementación
(de hecho, $\eta_A$ es la solución universal).

Ahora queremos explorar una situación parecida, para una topología
$\cal OS$, aunque solo consideraremos topologías en el conjunto $S$.

\begin{definition}
    Sea $S$ un espacio topológico y $\cal OS$ su topología.
    Definimos $\cal O\prs f{}S$ como la topología más pequeña
    en $S$ tal que todo $u\in\cal OS$ es abierto y cerrado.
    Es decir, $\cal O\prs f{}S$ es la topología generada por la subbase
    \[
        \cal OS\cup\{u'\mid u\in\cal OS\}
    .\]
    En otras palabras, $\cal O\prs f{}S$
    es la topología más pequeña sobre $S$
    tal que la inclusión $\cal OS\to\cal O\prs f{}S$ resuelve
    el problema de complementación.
    $\cal O\prs f{}S$ se llama la \emph{topología de Skula}
    de $S$, y al conjunto $S$ equipado con $\cal O\prs f{}S$
    lo denotamos $\prs f{}S$.
\end{definition}

\begin{example}
    Sea $\R$ el conjunto de los números reales.
    Consideraremos los espacios topológicos
    Además, sean $\prs l{}\R$, $\prs{}m\R$
    y $\prs r{}\R$ los espacios topológicos dados por $\R$
    equipado con la topologías generadas por
    \begin{align*}
        \{(-\infty,a)\mid a\in\R\}, \\
        \{(a,b)\mid a,b\in\R\}, \\
        \{[a,b)\mid a,b\in\R\},
    \end{align*}
    respectivamente.
    Nótese que $\cal O\prs{}m\R$ es la topología estandar.
    Además, tenemos
    \[
        \cal O\prs l{}\R
        \subseteq
        \cal O\prs{}m\R
        \subseteq
        \cal O\prs r{}\R.
    \]
    No es difícil ver que $\prs l{}\R$ es $T_0$,
    aunque no es $T_0$.
    
    Entonces $\cal O\prs r{}\R$ es la topología de Skula
    en $\cal O\prs l{}\R$:
    \[
        \cal O\prs f{}(\prs l{}\R)
        = \cal O\prs r{}\R
    .\]
\end{example}

Dado que la inclusión $\iota:\cal OS\to\cal O\prs f{}S$
es un morfismo de marcos que resuelve el problema de complementación,
se factoriza de manera única a través de $\eta_{\cal OS}$:
\[
    \begin{tikzcd}
        \cal OS \ar[d,"\eta_{\cal OS}"'] \ar[r,"\iota"]
        & \cal O\prs f{}S \\
        N\cal OS \ar[ur,"\iota^!"']
    \end{tikzcd}
\]
Por la adjunción entre $\cal O$ y $\pt$,
el morfismo $\iota^!:N\cal OS\to\cal O\prs f{}S$ corresponde
a un único morfismo $\phi:\prs f{}S\to\pt N\cal OS$.
Reemplazando a $\cal OS$ por un marco arbitrario $A$
y a $S$ por $\pt A$,
se puede demostrar que $\phi$ es un isomorfismo.