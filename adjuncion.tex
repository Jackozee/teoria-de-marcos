\chapter{La adjunción entre \tps{$\Frm$}{Frm} y \tps{$\Top$}{Top}}

%\section*{VIDEO 5: El espacio de puntos, parte 1. (18 OCT)}
Uno de los primeros ejemplos de marcos, y de hecho, la
motivación para la definición, fue que los abiertos de un espacio
topológico $S$ forman un marco $\cal OS$.
En este capítulo desarrollaremos más a fondo la relación entre marcos
y espacios topológicos. Sabemos que la asignación $S\mapsto\cal OS$ es
un funtor contravariante $\cal O\colon\Top\to\Frm$.
Ahora construiremos un funtor de regreso $\pt:\Frm\to\Top$
y probaremos que ambos funtores forman una adjunción.
\[
  \begin{tikzcd}
    \Top \ar[d,shift right=2,"\cal O"'{name=L}]
    \\
    \Frm^\op \ar[u,shift right=2,"\pt"'{name=R}]
    \adj{L}{R}
  \end{tikzcd}
\]

Antes de continuar, vale la pena mencionar que todos nuestros espacios
(de carne y hueso) ser\'an al menos $T_{0}$. Recordemos que un espacio
$T_0$ es un espacio donde cada par de puntos se pueden separar por al
menos un abierto. Más aún:
\begin{theorem}\label{tcero}
    La categoría $\Top_0$ de espacios topológicos
    $T_0$  es \emph{reflexiva} en $\Top$.
    Es decir, el funtor de inclusión $\Top_0\to\Top$
    tiene un adjunto izquierdo.
\end{theorem}
\begin{proof}
Es un ejercicio sencillo.
\end{proof}
La razón de esta decisión es que, en un espacio que no es $T_0$, hay
pares de puntos que tienen exactamente las mismas vecindades de
abiertos, así que no tienen mucha esperanza de ser caracterizados por
sus marcos de abiertos.
Otro ejercicio sencillo que ayuda a familiarizarse con los espacios
$T_0$ es el siguiente. En cualquier espacio topológico, los puntos
tienen un preorden (es decir, una relación reflexiva y transitiva)
dado por
  \[q\sqsubseteq p\iff \overline{q}\subseteq \overline{p}\]
Este se llama preorden de especialización.
\begin{exercise}
  \label{exe:especializacion}
  Un espacio $S$ es $T_0$ si, y solo si, su preorden de
  especialización es antisimétrico y, por lo tanto, un orden parcial.
\end{exercise}

\section{El espacio de puntos}
Dado un espacio con un punto $\{*\}$, el conjunto de
puntos de $S$ está en biyección con las funciones continuas
$\{*\}\to S$:
\[
  S \simeq \Top(\{*\},S)
\]
donde cada punto $s\in S$ está asociado a la función $*\mapsto
s$.
Así, si $2$ es el marco de dos elementos $2=\{0<1\}$,
cada de estas funciones $s:\{*\}\to S$ induce un morfismo de
marcos $\chi_s:\cal OS\to\cal O\{*\}\simeq 2$ dado como
\[
  \chi_s(u) =
  \begin{cases}
    1 & s\in u \\
    0 & s\not\in u
  \end{cases}
.\]
Así, para cada marco $A$, tiene sentido definir los puntos de $A$
como morfismos $\Frm(A,2)$.
En efecto, más adelante consideraremos esta construcción.
Sin embargo, primero consideraremos otra construcción equivalente:
representaremos cada morfismo $\chi:A\to 2$ con un elemento de
$A$ de manera canónica: el elemento
\[
    p = \Sup\{x\in A\mid \chi(x)=0\}
\]
es el único elemento de $A$ que cumple
\[
    x\leq p \ssi \chi(x) = 0
.\]
En particular, dado que $\chi(1)=1$, tenemos $p\neq 1$.
Por otro lado, para cualesquiera $x,y\in A$ con
$x\inf y\leq p$, tenemos
\[
    \chi(x)\inf\chi(y)=\chi(x\inf y)=0
,\]
así que $\chi(x)=0$ o bien $\chi(x)=0$, pues $\chi$ toma valores
en el marco $2$. es decir: $x\leq p$ o bien $y\leq p$.

\begin{definition}
  Sea $A\in \Frm$. Un punto o elemento $\inf$-irreducible de A es un elemento $p\in A$ con $p\neq 1$ tal que si $x\inf y\leq p$, entonces $x\leq p$ ó $y\leq p$. Denotamos por $\pt A$ al conjunto de todos los puntos de $A$.
\end{definition}


\begin{lemma}
  Sea $A\in \Frm$.
  \begin{itemize}
      \item Cada máximo de $A$ es $\inf$-irreducible.
      \item Si A es booleano, entonces todo elemento $\inf-$irreducible de A es máximo.
      \item Si A es una cadena, entonces cada elemento propio de A es $\inf-$irreducible.
  \end{itemize}
\end{lemma}
\begin{proof}\quad
  \begin{itemize}
      \item Sea $p\in A$ máximo, entonces $p<1$. Si $x\inf y\leq p$ y suponiendo que $x\not\leq p$, entonces $p<x\sup p$ y, por la maximalidad de $p$, tenemos que $p\sup x=1$. Similarmente, $y\not\leq p$ implica $p\sup y=1$. Si $x\not\leq p$ y $y\not\leq p$, se tiene que 
      \[p=p\sup (x\inf y)=(p\sup x)\inf(p\sup y)=1.\]
      Esto es una contradicción ya que $p<1$.
      \item Supongamos que $A$ es booleano. Sean $p\in\pt A$ y $x,y\in A$ con $p<x$ y $y=\neg x$. Tenemos que $x\inf y=0\leq p$, entonces $x\leq p$ ó $y\leq p$ ya que p es $\inf-$irreducible. Además $y\leq p<x$ puesto que $p<x$. En consecuencia, $x\sup y=1=x$, así, $p$ es máximo.
      \item Supongamos que $A$ es una cadena. Para cualesquiera $x,y\in A$, tenemos que $x\leq y$ ó $y\leq x$, es decir, $x\inf y\leq x$ ó $x\inf y\leq y$. Sea $p\in A$ con $p<1$. Si $x\inf y\leq p$, entonces $x\leq p$ ó $y\leq p$.
  \end{itemize}
\end{proof}

Sean $A\in \Frm$ y $a\in A$. Decimos que un punto $p\in \pt A$ está en $U_A(a)\subseteq \pt A$ si, y sólo si $a\not\leq p$.
\begin{lemma}
  Sean $A\in \Frm$ y $a,b\in A$.
  \begin{itemize}
      \item $U_A(1)=\pt A$.
      \item $U_A(0)=\emptyset$.
      \item $U_A(a\inf b)=U_A(a)\cap U_A(b)$.
      \item $U_A(\Sup X)=\bigcup \{U_A(x)|x\in X\}$, $\forall X\subseteq A$.
  \end{itemize}
\end{lemma}
\begin{proof}%[Yareli $\checkmark$ ]
  Sean $A\in \Frm$ y $a,b\in A$.
\begin{itemize}
\item Por definición $U_A(1)\subseteq \pt A$. Sea $p\in \pt A$, entonces $p\neq 1$. Además $1\not\leq p$, por lo que $p\in U_A(1)$. Así, $U_A(1)=\pt A$.
\item Supongamos que $U_A(0)\neq \emptyset$. Sea $p\in U_A(0)$. Por definición, $0\not\leq p$ pero $0\leq a, \forall a\in A$. Por lo tanto, $U_A(0)=\emptyset$.
\item Sea $p\in \pt A$. Tenemos que
\begin{align*}
p\in U_A(a\wedge b)&\iff a\wedge b\not\leq p\\
&\iff a\not\leq p\quad y\quad b\not\leq p\\
&\iff p\in U_A(a)\quad y\quad p\in U_A(b)\\
&\iff p\in U_A(a)\cap U_A(b).
\end{align*}
Por lo que $U_A(a\wedge b)=U_A(a)\cap U_A(b)$.
\item Sea $X\subseteq A$ y notemos que si $X=\emptyset$, entonces ocurre el segundo punto. En caso contrario,
\begin{align*}
p\in U_A(\bigvee X)&\Rightarrow \bigvee X\not\leq p\\
&\Rightarrow \textit{existe }x\in X\textit{ tal que }x\not\leq p\\
&\Rightarrow p\in U_A(x)\\
&\Rightarrow p\in \bigcup \{U_A(x)\mid x\in X\}.
\end{align*}
Además,
\begin{align*}
p\in \bigcup\{U_A(x)\mid x\in X\}&\Rightarrow p\in U_A(x)\textit{ para algún }x\in X\\
&\Rightarrow x\not \leq p\\
&\Rightarrow \bigvee X\not\leq p\\
&\Rightarrow p\in U_A(\bigvee X).
\end{align*}
Por lo tanto, $U_A(\Sup x)=\bigcup \{U_A(x)|x\in X\}$, $\forall X\subseteq A$.
\end{itemize}
\end{proof}
Se sigue que $U_A(A)=\{U_A(a)\mid a\in A\}$ es una topología en $\pt
A$. Al espacio topológico $(\pt A,U_A(A))$ lo llamamos
el \textit{espacio de puntos} de $A$.
Dado que la topología de $\pt A$ es $\cal O\pt A=U_A(A)$,
se sigue que $U_A:A\to\cal O\pt A$ es un
morfismo suprayectivo de marcos, al cual llamamos la \textit{reflexión
espacial} de $A$. Si $U_A$ es inyectivo (y, por lo tanto, un
isomorfismo) decimos que el marco $A$ es \textit{espacial}.

\begin{remark}
  \leavevmode
  \begin{enumerate}
    \item (La reflexión espacial como un cociente)
      Como $U_A:A\to\cal O\pt$ es suprayectivo, el marco $\cal O\pt A$ 
      es el cociente de $A$ bajo el núcleo de $U_A$: el núcleo
      $S\in NA$ dado como
      \[
        x\leq S(a) \iff U(x)\subseteq U(a)
      .\]
    \item
      Además, por el lema \ref{lemma:cerraduras-y-conjuntosfijos}, el
      núcleo $S$ de $U_A$ admite la descripción
      \[
        S(a)=\Inf \{p\in \pt A|a\leq p\}
      .\]
    \item
      Dados dos puntos $p,q\in \pt A$, se cumple
      \begin{align*}
        q\sqsubseteq p&\iff \overline{q}\subseteq \overline{p}\\
        &\iff (\forall x\in A)[q\in U(x)\Rightarrow p\in U(x)]\\
        &\iff (\forall x\in A)[x\leq p\Rightarrow x\leq q]\\
        &\iff p\leq q.
      \end{align*}
      Es decir, el preorden de especialización del espacio de puntos es
      el orden opuesto al orden heredado del marco:
      \[
        (\pt A,\sqsubseteq) = (\pt A,\leq)^\op.
      \]
      En particular, ya que su preorden de especialización es un orden
      parcial, esto prueba que el espacio de puntos es $T_0$.
  \end{enumerate}
\end{remark}

\subsection{Funtorialidad y naturalidad}
Queremos ver que la asignación $A\mapsto \pt A$ es un funtor y que
la reflexión espacial $U_A:A\to\cal O\pt A$
es una transformación natural
\[U_\bullet:\id_{\Frm}\to\cal O\pt(\_).\]

Lo primero es verificar que, dado un morfismo de marcos $f:A\to B$,
obtenemos una función continua $\pt f:\pt B\to\pt A$ entre los
espacios de puntos. De hecho, $\pt f$ será la restricción del adjunto
derecho $f_*:B\to A$ a los puntos de $B$, pero hay que verificar que
$f_*$ manda puntos a puntos.

Si $p$ es un punto de $\pt B$, veamos que $f_*p$ es un punto de $A$.
Primero, $f_*p$ no puede ser $1$, ya que en ese caso $1\leq f_\ast(p)$
implicaría $f(1)\leq p$ por la adjunción, pero esto es imposible ya
que $p\neq 1$.
Ahora veamos que $f_*p$ es $\inf$-irreducible. Si $x,y\in A$ son tales
que $x\inf y\leq f_\ast(p)$, por adjunción tenemos $f(x)\inf f(y)\leq p$.
En consecuencia, $f(x)\leq p$ ó $f(y)\leq p$, i.e., $x\leq f_\ast (p)$
ó $y\leq f_\ast (p)$. Por lo tanto, $f_\ast (p)\in \pt A$.

En resumen, dado un morfismo de marcos $f\colon A\to B$, obtenemos una
función $\pt f \colon \pt B\to \pt A$ dada por la restricción de
$f_*:B\to A$.

Observemos que, para todo $p\in \cal \pt B$, tenemos
\begin{align*}
    p\in (\pt f)^{-1} \left(U_A(a)\right)&\iff f_\ast (p)\in U_A(a)\\
    &\iff a\not\leq f_\ast (p)\\
    &\iff f(a)\not\leq p\\
    &\iff p\in U_B\left(f(a)\right).
\end{align*}
Por lo tanto $\pt f\colon \pt B\to \pt A$ es continua.
Es fácil ver que, dados morfismos $k:C\to B$ y $h:B\to A$,
se satisface $(hk)_*=k_*h_*$.
Además, el adjunto derecho de $\id:A\to A$ también es la identidad
de $A$.
De estas observaciones se sigue que la asignación $\pt$
es un funtor (contravariante) $\pt:\Frm\to\Top$.

Además, en el párrafo anterior probamos que
\[
    \cal O(\pt f)(U_A(a)) = U_B(f(a))
\]
para todo $a\in A$.
Es decir: el diagrama
\[
    \begin{tikzcd}
        A \ar[r,"f"] \ar[d,"U_A"'] & B \ar[d,"U_B"] \\
        \cal O\pt A \ar[r,"\cal O\pt f"'] & \cal O\pt B
    \end{tikzcd}
\]
es conmutativo, así que $U_\bullet=(U_A\mid A\in\Frm)$
es una transformación
natural $U_\bullet:\id_\Frm\to\cal O\pt$.

\subsection{El espacio de puntos del marco de abiertos}

¿Qué tanta información acerca de un espacio topológico se
puede recuperar a través de su marco de abiertos?
Comenzaremos preguntándonos
¿cómo se relacionan los puntos de un espacio $S$ con los puntos de
$\pt\cal OS$?
Como dijimos al principio, un punto $s\in S$ se puede ver como una
función continua $\{s\}\to S$, la cual induce un morfismo
$\chi_s:\cal O S \to 2$ como
\begin{equation}
  \chi_s(u) =
  \begin{cases}
    1 & s\in u  \\
    0 & s\not\in u.
  \end{cases}
\end{equation}
Por lo tanto, el $\inf$-irreducible que le corresponde a $s$ es
\begin{equation}
  \Phi_S(s) = \Sup\{u\in\cal OS\mid \chi_s(u)=0\} \in \pt\cal O S
.\end{equation}
Esto nos da una función $\Phi_S:S\to\pt\cal OS$. Esta descripción
se puede simplificar. Notemos que, dados $s\in S$ y $u\in\cal OS$, tenemos
\[
  \chi_s(u) = 0
  \iff
  s\not\in u
  \iff
  s\in u'
  \iff
  \ol s \subseteq u'
  \iff
  u \subseteq {\ol s}'.
\]
Luego,
\begin{equation}
  \Phi_S(s)
  =
  \Sup\{u\in\cal OS\mid u\subseteq {\ol s}'\}
  =
  {\ol s}'
  \in
  \pt\cal OS
.\end{equation}
Ahora, recordemos que un abierto de $\pt\cal OS$ es de la forma
\begin{align*}
  U_{\cal OS}(u)
  &= \{p\in\pt\cal OS \mid u\not\subseteq p\}
.\end{align*}
Tenemos
\begin{align*}
    s\in (\Phi_S)^{-1}(U_{\cal OS}(u))
    &\iff \Phi_S(s) \in U_{\cal OS}(u) \\
    &\iff {\ol s}' \in U_{\cal OS}(u) \\
    &\iff u \nsubseteq {\ol s}' \\
    &\iff s\in u.
\end{align*}
Es decir,
\[
  (\Phi_S)^{-1}(U_{\cal OS}(u)) = u
.\]
En particular, $(\Phi_S)^{-1}$ manda abiertos de $\pt\cal OS$ en
abiertos de $S$, así que la función $\Phi_S:S\to\pt\cal OS$ es continua.
Por último, observemos que, dada una función continua $\psi:S\to T$,
las funciones $\Phi_S:S\to\pt\cal OS$ hacen conmutar el diagrama
\[
    \begin{tikzcd}
        S \ar[r,"\psi"] \ar[d,"\Phi_S"'] & T \ar[d,"\Phi_T"] \\
        \pt\cal OS \ar[r,"\pt\cal O\psi"'] & \pt\cal OT
    \end{tikzcd}
\]
En efecto, para todo $v\in\cal OT$, tenemos
\begin{align*}
    v\subseteq (\pt\cal O\psi)(\Phi_S(s))
    &\iff v\subseteq (\cal O\psi)_*(\Phi_S(s)) \\
    &\iff (\cal O\psi)(v) \subseteq \Phi_S(s) \\
    &\iff \psi^{-1}(v) \subseteq \ol{s}' \\
    &\iff s\nin \psi^{-1}(v) \\
    &\iff \psi(s)\nin v \\
    &\iff v\subseteq \ol{\psi(s)}' \\
    &\iff v\subseteq \Phi_T(\psi(s)),
\end{align*}
por lo cual $(\pt\cal O\psi)(\Phi_S(s))=\Phi_T(\psi(s))$.
Luego, la familia de funciones continuas
\[
    \Phi_\bullet=(\Phi_S:S\to\pt\cal OS\mid S\in \Top)
\]
es una transformación natural
\[
    \Phi_\bullet : \id_\Top\to\pt\cal O
.\]

%\section*{VIDEO 6: El espacio de puntos, 2da parte (25 OCT)}
\section{La adjunción}
    \label{ssec:adjuncion}

En la primera parte, vimos que todo morfismo de marcos
$f:A\to B$ induce una función continua $\pt f:\pt B\to\pt A$ dada
como la restricción del adjunto derecho $f_*:B\to A$ de $f$ y
probamos que esta asignación es un funtor $\pt:\Frm\to\Top$.
Ahora veremos que $\pt$ y el funtor de abiertos
$\cal O:\Top\to\Frm$ son las mitades de una adjunción contravariante
entre $\Top$ y $\Frm$. En particular, construiremos un isomorfismo
\begin{equation}
    \label{eqn:adj_frm_top}
    \Frm(A,\cal OS) \simeq \Top(S,\pt A)
\end{equation}
natural en $A$ y en $S$.

\iffalse
\subsubsection{Usando morfismos iniciales}

Veremos que, para cada espacio topológico $S$, la función continua
$\Phi_S:S\to\pt\cal OS$ es el morfismo inicial de $S$ hacia marcos $A$
a través de $\pt$. En efecto, tomemos cualquier función continua
$\phi:S\to \pt A$. Tomando $f$ como la composición
\[
  A\xto{U_A}\cal O\pt A\xto{\cal O\phi}\pt S,
\]
tenemos la siguiente situación:
\[
  \begin{tikzcd}
    S \ar[r,"\Phi_S"] \ar[dr,"\phi"'] & \pt\cal OS \ar[d,"\pt f"] & \cal OS \\
                      & \pt A & A \ar[u,"f"'].
  \end{tikzcd}
\]
Veremos que el triángulo de la izquierda conmuta y que
$f:A\to\cal OS$ es el único morfismo de marcos que, bajo
$\pt$, lo hace conmutar. Primero notemos que
\[
  (\pt f)(\Phi_S(s))
  = \pt f({\ol s}')
  = f_*({\ol s}')
,\]
así que, para todo $u\in A$,
\begin{align*}
  u\leq (\pt f)(\Phi_S(s))
  &\iff u\leq f_*({\ol s}') \\
  &\iff fu \subseteq {\ol s}' \\
  &\iff \cal O\phi(U_A(u)) \subseteq  {\ol s}' \\
  &\iff \phi^{-1}(U_A(u)) \subseteq  {\ol s}' \\
  &\iff \ol s \subseteq \phi^{-1}(U_A(u)') \\
  &\iff s\in \phi^{-1}(U_A(u)') \\
  &\iff \phi(s)\in U_A(u)' \\
  &\iff \phi(s)\in \{p\in\pt A\mid u\leq p\} \\
  &\iff u\leq \phi(s).
\end{align*}
Es decir, $(\pt f^{\sharp})\circ\Phi_S=\phi$.
Ahora, si $f':A\to\cal OS$ es cualquier morfismo que, al aplicarle
$\pt$, hace conmutar el triángulo (es decir, $(\pt
f')\circ\Phi_S=\phi$), tenemos $(\pt f')\circ\Phi_S = (\pt
f')\circ\Phi_S$.

\subsubsection{Otra manera}
\fi

Cuando aprendimos sobre adjunciones,
vimos el caso covariante, en el cual el isomorfismo de
adjunción es equivalente a la existencia de dos transformaciones
naturales que satisfacen las identidades triangulares.

Ahora veremos que, en el caso contravariante,
tenemos el resultado análogo:
las identidades triangulares adecuadas
implican el isomorfismo natural (\ref{eqn:adj_frm_top}).

Recordemos que las transformaciones naturales
$U_\bullet:\id_\Frm\to\cal O\pt$ y
$\Phi_\bullet:\id_\Top\to\pt\cal O$
tienen componentes dadas como
\begin{align*}
    U_A:A&\to \cal O\pt A \\
    a &\mapsto U_A(a) = \{p\in \pt A \mid a\nleq p\}, \\
    \Phi_S:S&\to \pt\cal O S \\
    s &\mapsto \Phi_S(s)=\ol{s}'.
\end{align*}
Primero veremos que se cumplen las identidades triangulares
\[
    \begin{tikzcd}[row sep=15mm]
        & \cal OS \ar[d,"U_{\cal OS}"] \ar[dl,"\id_{\cal OS}"']
        \\
        \cal OS
        & \cal O\pt\cal OS \ar[l,"\cal O\Phi_S"]
    \end{tikzcd}
    \hspace{10mm}
    \begin{tikzcd}[row sep=15mm]
        & \pt A \ar[d,"\Phi_{\pt A}"] \ar[dl,"\id_{\pt A}"']
        \\
        \pt A
        & \pt \cal O\pt A \ar[l,"\pt U_A"]
    \end{tikzcd}
\]
En efecto, usando las equivalencias
\begin{align*}
    u\subseteq \Phi_S(s) &\ssi s\nin u, \\
    x\in U_A(a) &\ssi a\nleq x,
\end{align*}
tenemos
\begin{align*}
    x\in (\cal O\Phi_S)(U_{\cal OS}(u))
    &\iff \Phi_S(x) \in U_{\cal OS}(u) \\
    &\iff u\nleq \Phi_S(x) \\
    &\iff x\in u,
    \\
    a\leq (\pt U_A)(\Phi_{\pt A}(x))
    &\iff U_A(a) \leq \Phi_{\pt A}(x) \\
    &\iff x\nin U_A(a) \\
    &\iff a\leq x.
\end{align*}
Es decir, $(\cal O\Phi_S)(U_{\cal OS}(u))=u$
y $(\pt U_A)(\Phi_{\pt A}(x))=x$, como se quería.

Ahora, afirmamos que las funciones
\begin{align*}
    \Frm(A,\cal OS) &\to \Top(S,\pt A) \\
    f &\mapsto \bar f = (\pt f)\Phi_S,
    \\
    \Frm(A,\cal OS) &\leftarrow \Top(S,\pt A) \\
    (\cal O\phi)U_A = \bar\phi &\mapsfrom \phi.
\end{align*}
conforman una biyección.
En efecto,
la naturalidad de $\Phi_\bullet$, $U_\bullet$ y las identidades
triangulares implican la conmutatividad de los diagramas
\[
    \begin{tikzcd}[row sep=15mm]
        & \cal OS \ar[d,"U_{\cal OS}"] \ar[dl,"\id_{\cal OS}"']
        & A \ar[l,"f"'] \ar[d,"U_A"]
        \\
        \cal OS
        & \cal O\pt\cal OS \ar[l,"\cal O\Phi_S"]
        & \cal O\pt A \ar[l,"\cal O\pt f"]
    \end{tikzcd}
    \hspace{10mm}
    \begin{tikzcd}[row sep=15mm]
        & \pt A \ar[d,"\Phi_{\pt A}"] \ar[dl,"\id_{\pt A}"']
        & S \ar[l,"\phi"'] \ar[d,"\Phi_S"]
        \\
        \pt A
        & \pt \cal O\pt A \ar[l,"\pt U_A"]
        & \pt \cal OS \ar[l,"\pt\cal O\phi"]
    \end{tikzcd}
\]
por lo cual tenemos
\[
    \begin{aligned}
        \bar{\bar f}
        &= \ol{(\pt f)\Phi_S} \\
        &= \cal O((\pt f)\Phi_S)U_A \\
        &= (\cal O\Phi_S)(\cal O\pt f)U_A \\
        &= f,
    \end{aligned}
    \hspace{20mm}
    \begin{aligned}
        \bar{\bar\phi}
        &= \ol{(\cal O\phi)U_A} \\
        &= \pt((\cal O\phi)U_A)\Phi_S \\
        &= (\pt U_A)(\pt\cal O\phi)\Phi_S \\
        &= \phi.
    \end{aligned}
\]
Esto nos da la biyección (\ref{eqn:adj_frm_top}).
De manera explícita, la biyección está dada como
$\Frm(A,\cal OS)\ni f\leftrightarrow \phi\in \Top(S,\pt A)$, donde
\[
    s\in f(a) \ssi a\nleq \phi(s)
\]
para cualesquiera $s\in S$, $a\in A$, puesto que
\begin{align*}
    s\in f(a)
    &\iff f(a)\nleq \Phi_S(s) \\
    &\iff a \nleq (\pt f)(\Phi_S(s))=\bar f(s),
    \\
    a\nleq \phi(s)
    &\iff \phi(s) \in U_A(a) \\
    &\iff s \in (\cal O\phi)(U_A(a)) = \bar\phi(a).
\end{align*}
Finalmente, veamos que la biyección (\ref{eqn:adj_frm_top})
es natural en $A$ y en $S$.
Dado un morfismo de marcos $g:A\to B$, el diagrama
\[
    \begin{tikzcd}
        \Frm(B,\cal OS) \ar[d,"{-}\circ g"'] \ar[r,"f\mapsto\bar f"']
        & \Top(S,\pt B)
        \ar[d,"\pt g\circ{-}"]
        \\
        \Frm(A,\cal OS) \ar[r,"h\mapsto\bar h"']
        & \Top(S,\pt A)
    \end{tikzcd}
\]
es conmutativo:
\begin{align*}
    \ol{fg}
    &= \pt(fg)\Phi_S \\
    &= (\pt g)(\pt f)\Phi_S \\
    &= (\pt g)\bar f.
\end{align*}
Similarmente, dada una función continua $\psi:S\to T$,
el diagrama
\[
    \begin{tikzcd}
        \Frm(A,\cal OT)
        \ar[d,"\cal O\psi\circ{-}"']
        & \Top(T,\pt A) \ar[l,"\bar \phi\mapsfrom \phi"']
        \ar[d,"{-}\circ \psi"]
        \\
        \Frm(A,\cal OS)
        & \Top(S,\pt A) \ar[l,"\bar \xi\mapsfrom \xi"']
    \end{tikzcd}
\]
es conmutativo:
\begin{align*}
    \ol{\phi\psi}
    &= \cal O(\phi\psi)U_A \\
    &= (\cal O\psi)(\cal O\phi)U_A \\
    &= (\cal O\psi)\bar\phi.
\end{align*}

\section{La propiedad universal de las reflexiones}
La biyección
\begin{align*}
    \Frm(A,\cal OS) &\simeq \Top(S,\pt A) \\
    f &\mapsto \bar f = (\pt f)\Phi_S \\
    (\cal O\phi)U_A = \bar\phi &\mapsfrom \phi.
\end{align*}
se puede leer como sigue:
dado un morfismo $f:A\to\cal OS$, existe una única función continua
$\phi:S\to\pt A$ tal que el diagrama
\[
    \begin{tikzcd}
        A \ar[r,"f"] \ar[d,"U_A"'] & \cal OS \\
        \cal O\pt A \ar[ur,"\cal O\phi"']
    \end{tikzcd}
\]
conmuta.
Similarmente, dada una función continua $\phi:S\to\pt A$, existe
un único morfismo $f:A\to\cal OS$ tal que el diagrama
\[
    \begin{tikzcd}
        S \ar[r,"\phi"] \ar[d,"\Phi_S"'] & \pt A \\
        \pt\cal OS \ar[ur,"\pt f"']
    \end{tikzcd}
\]
conmuta.

