\chapter{La adjunción entre \tps{$\Frm$}{Frm} y \tps{$\Top$}{Top}}

test de pull request
%\section*{VIDEO 5: El espacio de puntos, parte 1. (18 OCT)}
La asignación que manda cada espacio topológico $S$ a su marco de
abiertos $\cal OS$ es un funtor contravariante
$\cal O:\Top\to\Frm$.

Queremos ver que hay un funtor de regreso $\pt:\Frm\to\Top$, que
a cada marco $A$ le asigna un espacio topológico cuyos elementos
se verán como los puntos del marco.
Veremos que $\cal O$ y $\pt$ forman una adjunción
\[
\begin{tikzcd}
\Top \ar[d,shift right=2,"\cal O"'{name=L}]
\\
\Frm^\op \ar[u,shift right=2,"\pt"'{name=R}]
\adj{L}{R}
\end{tikzcd}
\]

Dado un espacio con un punto $\{*\}$, el conjunto de
puntos de $S$ está en biyección con las funciones continuas
$\{*\}\to S$:
\[
  S \simeq \Top(\{*\},S)
\]
donde cada punto $s\in S$ está asociado a la función $*\mapsto
s$.
Así, si $2$ es el marco de dos elementos $2=\{0<1\}$,
cada de estas funciones $s:\{*\}\to S$ induce un morfismo de
marcos $\chi_s:\cal OS\to\cal O\{*\}\simeq 2$ dado como
\[
  \chi_s(u) =
  \begin{cases}
    1 & s\in u \\
    0 & s\not\in u
  \end{cases}
.\]
Así, para cada marco $A$, tiene sentido definir los puntos de $A$
como morfismos $\Frm(A,2)$.
En efecto, más adelante consideraremos esta construcción.
Sin embargo, primero consideraremos otra construcción equivalente:
representaremos cada morfismo $\chi:A\to 2$ con un elemento de
$A$ de manera canónica: el elemento
\[
    p = \Sup\{x\in A\mid \chi(x)=0\}
\]
es el único elemento de $A$ que cumple
\[
    x\leq p \ssi \chi(x) = 0
.\]
En particular, dado que $\chi(1)=1$, tenemos $p\neq 1$.
Por otro lado, para cualesquiera $x,y\in A$ con
$x\inf y\leq p$, tenemos
\[
    \chi(x)\inf\chi(y)=\chi(x\inf y)=0
,\]
así que $\chi(x)=0$ o bien $\chi(x)=0$, pues $\chi$ toma valores
en el marco $2$. es decir: $x\leq p$ o bien $y\leq p$.

\begin{defn}
  Sea $A\in \Frm$. Un punto o elemento $\inf$-irreducible de A es un elemento $p\in A$ con $p\neq 1$ tal que si $x\inf y\leq p$, entonces $x\leq p$ ó $y\leq p$. Denotamos por $\pt A$ al conjunto de todos los puntos de $A$.
\end{defn}


\begin{lemma}
  Sea $A\in \Frm$.
  \begin{itemize}
      \item Cada máximo de $A$ es $\inf$-irreducible.
      \item Si A es booleano, entonces todo elemento $\inf-$irreducible de A es máximo.
      \item Si A es una cadena, entonces cada elemento propio de A es $\inf-$irreducible.
  \end{itemize}
\end{lemma}
\begin{proof}\quad
  \begin{itemize}
      \item Sea $p\in A$ máximo, entonces $p<1$. Si $x\inf y\leq p$ y suponiendo que $x\not\leq p$, entonces $p<x\sup p$ y, por la maximalidad de $p$, tenemos que $p\sup x=1$. Similarmente, $y\not\leq p$ implica $p\sup y=1$. Si $x\not\leq p$ y $y\not\leq p$, se tiene que 
      \[p=p\sup (x\inf y)=(p\sup x)\inf(p\sup y)=1.\]
      Esto es una contradicción ya que $p<1$.
      \item Supongamos que $A$ es booleano. Sean $p\in\pt A$ y $x,y\in A$ con $p<x$ y $y=\neg x$. Tenemos que $x\inf y=0\leq p$, entonces $x\leq p$ ó $y\leq p$ ya que p es $\inf-$irreducible. Además $y\leq p<x$ puesto que $p<x$. En consecuencia, $x\sup y=1=x$, así, $p$ es máximo.
      \item Supongamos que $A$ es una cadena. Para cualesquiera $x,y\in A$, tenemos que $x\leq y$ ó $y\leq x$, es decir, $x\inf y\leq x$ ó $x\inf y\leq y$. Sea $p\in A$ con $p<1$. Si $x\inf y\leq p$, entonces $x\leq p$ ó $y\leq p$.
  \end{itemize}
\end{proof}

\section{La reflexión espacial de un marco}

Sean $A\in \Frm$ y $a\in A$. Decimos que un punto $p\in \pt A$ está en $U_A(a)\subseteq \pt A$ si, y sólo si $a\not\leq p$.
\begin{exe}%[Yareli $\checkmark$ ]
Demostrar el siguiente lema:
  \begin{lemma}
    Sean $A\in \Frm$ y $a,b\in A$.
    \begin{itemize}
        \item $U_A(1)=\pt A$.
        \item $U_A(0)=\emptyset$.
        \item $U_A(a\inf b)=U_A(a)\cap U_A(b)$.
        \item $U_A(\Sup X)=\bigcup \{U_A(x)|x\in X\}$, $\forall X\subseteq A$.
    \end{itemize}
  \end{lemma}
\end{exe}
\begin{proof}
  Sean $A\in \Frm$ y $a,b\in A$.
\begin{itemize}
\item Por definición $U_A(1)\subseteq \pt A$. Sea $p\in \pt A$, entonces $p\neq 1$. Además $1\not\leq p$, por lo que $p\in U_A(1)$. Así, $U_A(1)=\pt A$.
\item Supongamos que $U_A(0)\neq \emptyset$. Sea $p\in U_A(0)$. Por definición, $0\not\leq p$ pero $0\leq a, \forall a\in A$. Por lo tanto, $U_A(0)=\emptyset$.
\item Sea $p\in \pt A$. Tenemos que
\begin{align*}
p\in U_A(a\wedge b)&\iff a\wedge b\not\leq p\\
&\iff a\not\leq p\quad y\quad b\not\leq p\\
&\iff p\in U_A(a)\quad y\quad p\in U_A(b)\\
&\iff p\in U_A(a)\cap U_A(b).
\end{align*}
Por lo que $U_A(a\wedge b)=U_A(a)\cap U_A(b)$.
\item Sea $X\subseteq A$ y notemos que si $X=\emptyset$, entonces ocurre el segundo punto. En caso contrario,
\begin{align*}
p\in U_A(\bigvee X)&\Rightarrow \bigvee X\not\leq p\\
&\Rightarrow \textit{existe }x\in X\textit{ tal que }x\not\leq p\\
&\Rightarrow p\in U_A(x)\\
&\Rightarrow p\in \bigcup \{U_A(x)\mid x\in X\}.
\end{align*}
Además,
\begin{align*}
p\in \bigcup\{U_A(x)\mid x\in X\}&\Rightarrow p\in U_A(x)\textit{ para algún }x\in X\\
&\Rightarrow x\not \leq p\\
&\Rightarrow \bigvee X\not\leq p\\
&\Rightarrow p\in U_A(\bigvee X).
\end{align*}
Por lo tanto, $U_A(\Sup x)=\bigcup \{U_A(x)|x\in X\}$, $\forall X\subseteq A$.
\end{itemize}
\end{proof}
Por el lema anterior, $U_A(A)=\{U_A(a)\mid a\in A\}$
es una topología en $\pt A$ y $U_A:A\to\cal O\pt A$ es un morfismo
suprayectivo de marcos.\\
Al espacio topológico $(\pt A,U_A(A))$ lo llamamos
el \textit{espacio de puntos} de $A$,
mientras que al morfismo $U_A\colon A\to \cal O\pt A$
lo llamamos la \textit{reflexión espacial} de $A$.
Además, si $U_A$ es un isomorfismo (basta con que sea inyectivo),
decimos que el marco $A$ es \textit{espacial}.

Observemos que, como $U_A$ es suprayectivo,
existe un núcleo $S\in NA$ tal que $A_S\cong \cal O\pt A$.\par
Sabemos que $S$ está caracterizado como $x\leq S(a) \iff U(x)\subseteq U(a)$. Probaremos que $S(a)=\Inf \{p\in \pt A|a\leq p\}$.
\begin{align*}
    x\leq S(a) &\iff U(x)\subseteq U(a)\\
    &\iff (\forall p\in \pt A)[x\not\leq p\Rightarrow a\not\leq p]\\
    &\iff (\forall p\in \pt A)[a\leq p \Rightarrow x\leq p]\\
    &\iff x\leq \Inf \{p\in \pt A|a\leq p\}
\end{align*}

\section{El orden de especialización}
\begin{exe}[Para el lector]%[Juan]
    \label{exe:especializacion}
  Sean $S\in Top$ y $p,q\in S$. Probar que la relación
  \[q\sqsubseteq p\iff \overline{q}\subseteq \overline{p}\]
  es un preorden. Si el espacio es $T_0$, entonces es un orden parcial.
\end{exe}
A la relación del ejercicio \ref{exe:especializacion}
le llamamos el \textit{orden de especialización}.\par 
Sean $p,q\in \pt A$. Notemos que
\begin{align*}
    q\sqsubseteq p&\iff \overline{q}\subseteq \overline{p}\\
    &\iff (\forall x\in A)[q\in U(x)\Rightarrow p\in U(x)]\\
    &\iff (\forall x\in A)[x\leq p\Rightarrow x\leq q]\\
    &\iff p\leq q.
\end{align*}
Es decir, el orden de especialización del espacio de puntos es el orden opuesto del marco original.
\[
    (\pt A,\sqsubseteq) = (\pt A,\leq)^\op.
\]
En particular, esto prueba que el espacio de puntos, $\pt A$, es $T_0$.

\section{El espacio de puntos del marco de abiertos}
Tomemos un espacio topológico $S\in\Top$.
¿Cómo se relacionan los puntos $s\in S$ con los puntos $p\in\cal OS$?
Dado $s\in S$ y $u\in\cal OS$ con $u\subseteq \overline{s}'$. Tenemos que
\[u\subseteq \overline{s}' \iff \overline{s}\subseteq u' \iff s\in u' \iff s\notin u.\]
Primero, observemos que $\overline{s}'\neq S$ ya que, de otro modo,
tendríamos $s\in\ol{s}=\emptyset$.\par 
Además, si $u,v\in \cal OS$ son tales que
$u\cap v\subseteq \ol{s}'$, entonces $s\notin u\cap v$.
Esto implica que $s\notin u$ ó $s\notin v$, es decir, $u\subseteq \ol{s}'$ ó $v\subseteq \ol{s}'$.
Esto nos dice que $\ol{s}'$ es un punto de $\cal OS$.
Por lo tanto, tenemos una función $\Phi_S:S\to\pt\cal OS$ dada
como $\Phi_S(s) = \ol{s}'$.
Además, si consideramos a $\pt\cal OS$ con la topología
$U_{\cal OS}(\cal OS)$, tenemos que
\begin{align*}
    s\in (\Phi_S)^{-1}(U_{\cal OS}(u))
    &\iff \Phi_S(s) \in U_{\cal OS}(u) \\
    &\iff u \nsubseteq \Phi_{\cal OS}(s) \\
    &\iff u \nsubseteq \ol{s}' \\
    &\iff s\in u.
\end{align*}
Es decir, $(\Phi_S)^{-1}$ manda abiertos de $\pt\cal OS$ en abiertos
de $S$, así que $\Phi_S:S\to\pt\cal OS$ es continua.

\section{La funtorialidad del espacio de puntos}
Queremos ver que la asignación $A\mapsto \pt A$ es un funtor.
Además comprobaremos que, si varía $A$, la reflexión espacial
es una transformación natural
\[U_\bullet:\id_{\Frm}\to\cal O\pt(\_).\]

Notemos que, para un morfismo de marcos $f\colon A\to B$ y un punto $p\in \pt B$, $z\leq f_\ast (p)\iff f(z)\leq p$, donde $f_\ast$ es adjunto derecho de $f$. En particular, si $1\leq f_\ast(p)\iff f(1)=1\leq p$. Esto es imposible ya que $p\in \pt B$. Por lo que $f_\ast(p)\neq 1$.\\
Sean $x,y\in A$ tales que $x\inf y\leq f_\ast(b)$. Esto pasa si, y sólo si $f(x)\inf f(y)\leq p$, en consecuencia, $f(x)\leq p$ ó $f(y)\leq p$, i.e., $x\leq f_\ast (p)$ ó $y\leq f_\ast (p)$. Por lo que $f_\ast (p)\in \pt A$.\par 
En resumen, dado un morfismo de marcos $f\colon A\to B$, obtenemos una función $\pt f \colon \pt B\to \pt A$ dada por la restricción
de $f_*:B\to A$.\par
Observemos que, para todo $p\in \cal \pt B$, tenemos
\begin{align*}
    p\in (\pt f)^{-1} \left(U_A(a)\right)&\iff f_\ast (p)\in U_A(a)\\
    &\iff a\not\leq f_\ast (p)\\
    &\iff f(a)\not\leq p\\
    &\iff p\in U_B\left(f(a)\right).
\end{align*}
Por lo tanto $\pt f\colon \pt B\to \pt A$ es continua.
Es fácil ver que, dados morfismos $k:C\to B$ y $h:B\to A$,
se satisface $(hk)_*=k_*h_*$.
Además, el adjunto derecho de $\id:A\to A$ también es la identidad
de $A$.
De estas observaciones se sigue que la asignación $\pt$
es un funtor (contravariante) $\pt:\Frm\to\Top$.

Además, en el párrafo anterior probamos que
\[
    \cal O(\pt f)(U_A(a)) = U_B(f(a))
\]
para todo $a\in A$.
Es decir: el diagrama
\[
    \begin{tikzcd}
        A \ar[r,"f"] \ar[d,"U_A"'] & B \ar[d,"U_B"] \\
        \cal O\pt A \ar[r,"\cal O\pt f"'] & \cal O\pt B
    \end{tikzcd}
\]
es conmutativo, así que $U_\bullet=(U_A\mid A\in\Frm)$
es una transformación
natural $U_\bullet:\id_\Frm\to\cal O\pt$.

Por último, hagamos la siguiente observación.
Dada una función continua $\psi:S\to T$,
las funciones $\Phi_S:S\to\pt\cal OS$ hacen conmutar el diagrama
\[
    \begin{tikzcd}
        S \ar[r,"\psi"] \ar[d,"\Phi_S"'] & T \ar[d,"\Phi_T"] \\
        \pt\cal OS \ar[r,"\pt\cal O\psi"'] & \pt\cal OT
    \end{tikzcd}
\]
En efecto, para todo $v\in\cal OT$, tenemos
\begin{align*}
    v\subseteq (\pt\cal O\psi)(\Phi_S(s))
    &\iff v\subseteq (\cal O\psi)_*(\Phi_S(s)) \\
    &\iff (\cal O\psi)(v) \subseteq \Phi_S(s) \\
    &\iff \psi^{-1}(v) \subseteq \ol{s}' \\
    &\iff s\nin \psi^{-1}(v) \\
    &\iff \psi(s)\nin v \\
    &\iff v\subseteq \ol{\psi(s)}' \\
    &\iff v\subseteq \Phi_T(\psi(s)),
\end{align*}
por lo cual $(\pt\cal O\psi)(\Phi_S(s))=\Phi_T(\psi(s))$.
Luego, la familia de funciones
\[
    \Phi_\bullet=(\Phi_S:S\to\pt\cal OS\mid S\in \Top)
\]
es una transformación natural
\[
    \Phi_\bullet : \id_\Top\to\pt\cal O
.\]

%\section*{VIDEO 6: El espacio de puntos, 2da parte (25 OCT)}
\section{La adjunción}
    \label{ssec:adjuncion}
En la primera parte, vimos que todo morfismo de marcos
$f:A\to B$ induce una función continua $\pt f:\pt B\to\pt A$ dada
como la restricción del adjunto derecho $f_*:B\to A$ de $f$ y
probamos que esta asignación es un funtor $\pt:\Frm\to\Top$.
Ahora veremos que $\pt$ y el funtor de abiertos
$\cal O:\Top\to\Frm$ son las mitades de una adjunción contravariante
entre $\Top$ y $\Frm$; es decir: que existe un isomorfismo
\begin{equation}
    \label{eqn:adj_frm_top}
    \Frm(A,\cal OS) \simeq \Top(S,\pt A)
\end{equation}
natural en $A$ y en $S$.

Cuando aprendimos sobre adjunciones,
vimos el caso covariante, en el cual el isomorfismo de
adjunción es equivalente a la existencia de dos transformaciones
naturales que satisfacen las identidades triangulares.

Ahora veremos que, en el caso contravariante,
tenemos el resultado análogo:
las identidades triangulares adecuadas
implican el isomorfismo natural (\ref{eqn:adj_frm_top}).

Recordemos que las transformaciones naturales
$U_\bullet:\id_\Frm\to\cal O\pt$ y
$\Phi_\bullet:\id_\Top\to\pt\cal O$
tienen componentes dadas como
\begin{align*}
    U_A:A&\to \cal O\pt A \\
    a &\mapsto U_A(a) = \{p\in \pt A \mid a\nleq p\}, \\
    \Phi_S:S&\to \pt\cal O S \\
    s &\mapsto \Phi_S(s)=\ol{s}'.
\end{align*}
Primero veremos que se cumplen las identidades triangulares
\[
    \begin{tikzcd}[row sep=15mm]
        & \cal OS \ar[d,"U_{\cal OS}"] \ar[dl,"\id_{\cal OS}"']
        \\
        \cal OS
        & \cal O\pt\cal OS \ar[l,"\cal O\Phi_S"]
    \end{tikzcd}
    \hspace{10mm}
    \begin{tikzcd}[row sep=15mm]
        & \pt A \ar[d,"\Phi_{\pt A}"] \ar[dl,"\id_{\pt A}"']
        \\
        \pt A
        & \pt \cal O\pt A \ar[l,"\pt U_A"]
    \end{tikzcd}
\]
En efecto, usando las equivalencias
\begin{align*}
    u\subseteq \Phi_S(s) &\ssi s\nin u, \\
    x\in U_A(a) &\ssi a\nleq x,
\end{align*}
tenemos
\begin{align*}
    x\in (\cal O\Phi_S)(U_{\cal OS}(u))
    &\iff \Phi_S(x) \in U_{\cal OS}(u) \\
    &\iff u\nleq \Phi_S(x) \\
    &\iff x\in u,
    \\
    a\leq (\pt U_A)(\Phi_{\pt A}(x))
    &\iff U_A(a) \leq \Phi_{\pt A}(x) \\
    &\iff x\nin U_A(a) \\
    &\iff a\leq x.
\end{align*}
Es decir, $(\cal O\Phi_S)(U_{\cal OS}(u))=u$
y $(\pt U_A)(\Phi_{\pt A}(x))=x$, como se quería.

Ahora, afirmamos que las funciones
\begin{align*}
    \Frm(A,\cal OS) &\to \Top(S,\pt A) \\
    f &\mapsto \bar f = (\pt f)\Phi_S,
    \\
    \Frm(A,\cal OS) &\leftarrow \Top(S,\pt A) \\
    (\cal O\phi)U_A = \bar\phi &\mapsfrom \phi.
\end{align*}
conforman una biyección.
En efecto,
la naturalidad de $\Phi_\bullet$, $U_\bullet$ y las identidades
triangulares implican la conmutatividad de los diagramas
\[
    \begin{tikzcd}[row sep=15mm]
        & \cal OS \ar[d,"U_{\cal OS}"] \ar[dl,"\id_{\cal OS}"']
        & A \ar[l,"f"'] \ar[d,"U_A"]
        \\
        \cal OS
        & \cal O\pt\cal OS \ar[l,"\cal O\Phi_S"]
        & \cal O\pt A \ar[l,"\cal O\pt f"]
    \end{tikzcd}
    \hspace{10mm}
    \begin{tikzcd}[row sep=15mm]
        & \pt A \ar[d,"\Phi_{\pt A}"] \ar[dl,"\id_{\pt A}"']
        & S \ar[l,"\phi"'] \ar[d,"\Phi_S"]
        \\
        \pt A
        & \pt \cal O\pt A \ar[l,"\pt U_A"]
        & \pt \cal OS \ar[l,"\pt\cal O\phi"]
    \end{tikzcd}
\]
por lo cual tenemos
\[
    \begin{aligned}
        \bar{\bar f}
        &= \ol{(\pt f)\Phi_S} \\
        &= \cal O((\pt f)\Phi_S)U_A \\
        &= (\cal O\Phi_S)(\cal O\pt f)U_A \\
        &= f,
    \end{aligned}
    \hspace{20mm}
    \begin{aligned}
        \bar{\bar\phi}
        &= \ol{(\cal O\phi)U_A} \\
        &= \pt((\cal O\phi)U_A)\Phi_S \\
        &= (\pt U_A)(\pt\cal O\phi)\Phi_S \\
        &= \phi.
    \end{aligned}
\]
Esto nos da la biyección (\ref{eqn:adj_frm_top}).
De manera explícita, la biyección está dada como
$\Frm(A,\cal OS)\ni f\leftrightarrow \phi\in \Top(S,\pt A)$, donde
\[
    s\in f(a) \ssi a\nleq \phi(s)
\]
para cualesquiera $s\in S$, $a\in A$, puesto que
\begin{align*}
    s\in f(a)
    &\iff f(a)\nleq \Phi_S(s) \\
    &\iff a \nleq (\pt f)(\Phi_S(s))=\bar f(s),
    \\
    a\nleq \phi(s)
    &\iff \phi(s) \in U_A(a) \\
    &\iff s \in (\cal O\phi)(U_A(a)) = \bar\phi(a).
\end{align*}

La naturalidad de la biyección se deja como ejercicio:
\begin{exe}%[Alfredo $\checkmark$]
    Verifica que la biyección (\ref{eqn:adj_frm_top})
    es natural en $A$ y en $S$.
\end{exe}
\begin{sol}
    Dado un morfismo de marcos $g:A\to B$, el diagrama
    \[
        \begin{tikzcd}
            \Frm(B,\cal OS) \ar[d,"{-}\circ g"'] \ar[r,"f\mapsto\bar f"']
            & \Top(S,\pt B)
            \ar[d,"\pt g\circ{-}"]
            \\
            \Frm(A,\cal OS) \ar[r,"h\mapsto\bar h"']
            & \Top(S,\pt A)
        \end{tikzcd}
    \]
    es conmutativo:
    \begin{align*}
        \ol{fg}
        &= \pt(fg)\Phi_S \\
        &= (\pt g)(\pt f)\Phi_S \\
        &= (\pt g)\bar f.
    \end{align*}
    Similarmente, dada una función continua $\psi:S\to T$,
    el diagrama
    \[
        \begin{tikzcd}
            \Frm(A,\cal OT)
            \ar[d,"\cal O\psi\circ{-}"']
            & \Top(T,\pt A) \ar[l,"\bar \phi\mapsfrom \phi"']
            \ar[d,"{-}\circ \psi"]
            \\
            \Frm(A,\cal OS)
            & \Top(S,\pt A) \ar[l,"\bar \xi\mapsfrom \xi"']
        \end{tikzcd}
    \]
    es conmutativo:
    \begin{align*}
        \ol{\phi\psi}
        &= \cal O(\phi\psi)U_A \\
        &= (\cal O\psi)(\cal O\phi)U_A \\
        &= (\cal O\psi)\bar\phi.
    \end{align*}
\end{sol}
El ejercicio anterior concluye la demostración.

\section{La propiedad universal de las reflexiones}
La biyección
\begin{align*}
    \Frm(A,\cal OS) &\simeq \Top(S,\pt A) \\
    f &\mapsto \bar f = (\pt f)\Phi_S \\
    (\cal O\phi)U_A = \bar\phi &\mapsfrom \phi.
\end{align*}
se puede leer como sigue:
dado un morfismo $f:A\to\cal OS$, existe una única función continua
$\phi:S\to\pt A$ tal que el diagrama
\[
    \begin{tikzcd}
        A \ar[r,"f"] \ar[d,"U_A"'] & \cal OS \\
        \cal O\pt A \ar[ur,"\cal O\phi"']
    \end{tikzcd}
\]
conmuta.
Similarmente, dada una función continua $\phi:S\to\pt A$, existe
un único morfismo $f:A\to\cal OS$ tal que el diagrama
\[
    \begin{tikzcd}
        S \ar[r,"\phi"] \ar[d,"\Phi_S"'] & \pt A \\
        \pt\cal OS \ar[ur,"\pt f"']
    \end{tikzcd}
\]
conmuta.

