%\section*{SESIÓN 14: 28 OCT}
\chapter{Espacios sobrios y marcos espaciales}

Recordemos que, por definición, un marco $A$ es espacial si
su reflexión espacial $U_A:A\to\cal O\pt A$ es un isomorfismo
(o, equivalentemente, si $U_A$ es inyectiva).
Los marcos espaciales, junto con los morfismos de marcos,
forman la subcategoría $\Sp$ de $\Top$.

Ahora nuestro objetivo es ver que $\Top$ tiene una subcategoría
dual a $\Sp$, de tal modo que la dualidad
está dada por la restricción de la adjunción.
La categoría que queremos es la de los espacios sobrios,
a la cual denotaremos como $\Sob$.
También veremos que, análogamente a lo que pasa
con los marcos espaciales, un espacio topológico $S$ es sobrio
si, y solo si, $\Phi_S:S\to\pt\cal OS$ es un isomorfimso.

\section{Espacios sobrios}
Comenzaremos por definir los espacios topológicos sobrios.

Recordemos que, dado un espacio topológico $S$,
$\pt\cal OS$ es el conjunto de elementos
$\inf$-irreducibles del marco $\cal OS$.
Además, $\pt\cal OS$ tiene una topología canónica,
donde los abiertos son los conjuntos de la forma
\[
    U_{\cal OS}(u) = \{p\in\pt\cal OS \mid u\nleq p\}
.\]
También recordemos que, dotando a $\pt A$ con esta topología,
la función $\Phi_S:S\to\pt\cal OS$ dada como
\[
    \Phi_S(x) = \ol{x}' = S-\ol{\{x\}}.
\]
es continua.
A $\Phi_S(x):S\to\pt\cal OS$ le llamamos
la reflexión sobria de $S$, debido a la siguiente definición.

\begin{definition}
    Decimos que un espacio topológico $S$ es sobrio si
    cada abierto $\inf$-irreducibles es de la forma
    $\Phi_S(x)$ para un único punto $x\in S$.
    En otras palabras, un espacio $S$ es sobrio
    si su reflexión sobria $\Phi_S:S\to\cal OS$ es biyectiva.
    
    Puede parecer extraño definir la sobriedad en términos de
    la biyectividad de $\Phi_S$, en lugar de pedir que $\Phi_S$
    sea un homeomorfismo.
    Más adelante, veremos que estas condiciones son equivalentes.
    
    Denotamos como $\Sob$ a la categoría formada por
    los espacios topológicos sobrios y sus funciones continuas.
\end{definition}

\begin{lemma}
    Sea $S$ un espacio topológico.
    Si $S$ es de Hausdorff, entonces $S$ es sobrio.
    Si $S$ es sobrio, entonces es de Kolmogorov.
    Esto es
    \[
        T_2 \implies \text{sobrio}\implies T_0.
    \]
\end{lemma}
\begin{proof}
    Supongamos que $S$ es Hausdorff y sea $u\in\pt\cal OS$.
    Es decir, $u$ es un elemento $\inf$-irreducible de $\cal OS$.
    Queremos ver que $u=\ol{x}'$ (i.e.: $u'=\ol x$) para un único
    punto $x\in S$.
    Como $S$ es Hausdorff, esto sucede si, y solo si, $u'=\{x\}$
    para algún $x\in S$.
    Por definición, $u\neq S$, así que existe,
    al menos, un punto en $u'$.
    Supongamos que hay dos puntos $x,y\in u'$ distintos.
    Como $S$ es Hausdorff, existen abiertos $v_1,v_2\in\cal OS$
    con $x\in v_1$, $y\in v_2$ y $v_1\inf v_2=\emptyset$.
    Luego,
    \begin{align*}
        u
        &= u\sup(v_1\inf v_2) \\
        &= (u\sup v_1)\inf (u\sup v_2).
    \end{align*}
    Como $u$ es $\inf$-irreducible, esto implica que $v_1\leq u$
    o que $v_2\leq v$, pero esto implicaría que $x\in u$ o que $y\in u$,
    lo cual es una contradicción.
    Se sigue que $u'=\{x\}$, que es lo que se quería demostrar.
    
    Ahora supongamos que $S$ es sobrio.
    Si $x,y\in S$ son puntos distintos, entonces
    Dados $x,y\in S$, tenemos
    \begin{align*}
        \Phi_S(x) = \Phi_S(y)
        &\iff \ol{x}' = \ol{y}' \\
        &\iff \ol x = \ol y \\
        &\iff \cal U(x) = \cal U(y),
    \end{align*}
    donde $\cal U(x)$ y $\cal U(y)$ son los filtros
    de vecindades abiertas de $x$ y $y$, respectivamente.
    Como $\Phi_S$ es inyectiva, dados $x,y\in S$ distintos, tenemos
    \[
        \cal U(x)\neq\cal U(y),
    \]
    así que existe una vecindad abierta de $x$ que no contiene a $y$
    o una vecindad abierta de $y$ que no contiene a $x$.
    Esto es, $S$ es $T_0$.
\end{proof}

Reformulando un poco la definición de sobriedad,
obtenemos el siguiente resultado.
\begin{lemma}
    Un espacio $S$ es sobrio si, y solo si,
    todo cerrado irreducible de $S$
    (esto es, todo elemento $\sup$-irreducible de $\cal CS$)
    tiene un único punto genérico.
\end{lemma}
\begin{proof}
    Los cerrados $\sup$-irreducibles son exactamente
    los de la forma $c=u'$ para $u$ un abierto $\inf$-irreducible.
    Tomando complementos, vemos que las siguientes afirmaciones
    son equivalentes:
    \begin{itemize}
         \item Todo abierto $\inf$-irreducible es de
         la forma $u=\ol{x}'$ para un único $x\in S$.
         \item Todo cerrado $\sup$-irreducible es de
         la forma $c=\ol x$ para un único $x\in S$.
    \end{itemize}
    La primera afirmación es la sobriedad de $S$,
    mientras que la segunda dice que todo cerrado
    irreducible de $S$ tiene un único punto genérico.
\end{proof}

\begin{example}
    Dado un anillo conmutativo $R$,
    el espectro primo $\mathrm{Spec}R$ es un espacio sobrio.
    
    Por ejemplo, los cerrados irreducibles de $\mathrm{Spec}\Z$ son
    de la forma $c=\{p\}$ (para un primo $p\in\Z$) o $c=\mathrm{Spec}\Z$.
    En el primer caso, el punto genérico es $p$.
    De otro modo, tenemos $\ol{\{0\}}=\mathrm{Spec}\Z$.
\end{example}

%\section*{SESIÓN 15: 4 NOV}



\section{Las distintas encarnaciones del espacio de puntos}

Recordemos que definimos el espacio de puntos de un marco
como el conjunto de sus elementos $\inf$-irreducibles.
Ahora veremos otras dos construcciones equivalentes.

\begin{definition}
    Sea $A$ es una retícula.
    Un subconjunto $F\subseteq A$ es un filtro si
    \begin{itemize}
        \item es sección superior (absorbe hacia arriba),
        \item es no vacío y
        \item es cerrado bajo ínfimos de dos elementos.
    \end{itemize}
    Además, un filtro $F$ es propio si $0\nin F$.
    Nótese que
    \[
        0\in F \ssi F=A
    .\]
    
    Decimos que un filtro $F\subseteq A$ es primo si,
    \begin{itemize}
        \item es propio y
        \item siempre que $a\sup b\in F$, entonces $a\in F$ o $b\in F$.
    \end{itemize}
    
    Si $A$ es una retícula completa (por ejemplo, un marco),
    entonces decimos que un filtro $F\subseteq A$ es completamente
    primo si,
    \begin{itemize}
        \item es propio y
        \item siempre que $\Sup X\in F$, entonces existe $x\in X\cap F$.
    \end{itemize}
\end{definition}

\begin{example}
    Sean $S$ un espacio topológico y $x\in S$.
    Entonces el conjunto de vecindades abiertas de $x$
    \[
        \cal U(x) = \{u\in \cal OS\mid x\in u\}
    \]
    es un filtro completamente primo.
    
    En efecto, $\cal U(x)$ es un filtro:
    \begin{itemize}
        \item si $x\in u\leq v$, entonces $x\in v$;
        \item $x\in S$;
        \item si $x\in u$ y $x\in v$, entonces $x\in u\inf v$.
    \end{itemize}
    Además,  $\cal U(x)$ es propio, ya que $\emptyset\nin\cal U(x)$.
    Finalmente, si $X\subseteq \cal OS$ es tal que $x\in \Sup X$,
    entonces existe $u\in X$ tal que $x\in u$.
\end{example}

%\section*{SESIÓN 16: 9 NOV}

\begin{definition}
    Un caracter en un marco $A$ es un morfismo
    \[
        \chi:A\to 2
    .\]
\end{definition}

\begin{theorem}
    Las funciones
    \[
        \begin{aligned}
            (\pt A,\sqsubseteq)
            =(\pt A,\leq)^\op &\rightleftarrows \Cp(A) \\
            p &\mapsto F_p=\{a\in A\mid a\nleq p\} \\
            \Sup\{a\in A\mid a\nin F\} = p_F &\mapsfrom F
        \end{aligned}
        \qquad
        \begin{aligned}
            \Cp(A) &\rightleftarrows \Frm(A,2) \\
            F &\mapsto \chi_F \\
            \chi^{-1}(1) = F_\chi &\mapsfrom \chi
        \end{aligned}
    \]
    son isomorfismos de copos.
\end{theorem}
\begin{proof}
    \textbf{Primera parte.}
    Veamos que las funciones
    \begin{align*}
        \pt A &\rightleftarrows \Cp(A) \\
        p &\mapsto F_p = \{a\in A\mid a\nleq p\} \\
        \Sup\{a\in A\mid a\nin F\} = p_F &\mapsfrom F
    \end{align*}
    están bien definidas.
    \begin{itemize}
        \item
        Dado un $p\in\pt A$, sea $F_p=\{a\in A\mid a\nleq p\}$.
        Como $p\neq 1$, tenemos $1\nleq p$, por lo cual
        $1\in F_p$.
        Además, $F_p$ es cerrado hacia arriba:
        Si $a\geq b$, entonces $a\leq p\implies b\leq p$,
        o bien $b\nleq p\implies a\nleq p$;
        es decir: $b\in F_p$ implica que $a\in F_p$.
        Si $a,b\in A$, la irreducibilidad de $p$ implica que
        $a\inf b\leq p \implies a\leq p \text{ o } b\leq p$.
        Tomando contrapuesta, tenemos
        $a,b\in F_p \implies a\inf b\in F_p$.
        Luego, $F_p$ es un ideal.
        
        Notemos que $F_p$ es propio, pues $0\leq p$
        implica $0\nin F_p$.
        Finalmente, si $X\subseteq A$ tenemos
        \[
            (\forall x\in X,\; x\leq p)\implies\Sup X\leq p
        \]
        o, tomando contrapuesta:
        \[
            \Sup X\nleq p \implies \exists x\in X,\; x\nleq p
        .\]
        Luego, $F_p$ es completamente primo.
        \item
        Dado un filtro completamente primo $F$,
        consideremos $p_F=\Sup\{x\in A\mid x\nin F\}$.
        Como $F$ es completamente primo, tenemos $p_F\nin F$.
        Además, todo $x\leq p$ cumple $x\nin F$
        (de otro modo, tendríamos $p\in F$,
        pues $F$ es sección superior).
        Es decir:
        \[
            x\nin F\iff x\leq p_F
        .\]
        Notemos que $p_F\neq 1$, pues $1\in F$.
        Como $F$ es filtro, tenemos que
        $a,b\in F\implies a\inf b\in F$.
        Tomando contrapuesta, obtenemos
        $a\inf b\leq p\implies a\leq p_F$ o $b\leq p_F$.
        Ahora sean $a,b\in A$ tales que $a\nleq p_F$, $b\nleq p_F$.
        Como $F$ es sección superior, se sigue que $a\in F$,
        por lo cual $a\nleq p_F$.
        Es decir, $p_F$ es $\inf$-irreducible,
        o bien $p_F\in\pt A$.
    \end{itemize}
    Ahora veamos que estas funciones son inversas mutuas.
    Dado $p\in\pt A$, tenemos $F_p=\{x\in A\mid x\nleq p\}$.
    Entonces para todo $y\in A$ tenemos
    \begin{align*}
        p_{F_p} \leq y
        &\iff \Sup\{x\in A\mid x\nin F_p\}\leq y \\
        &\iff (\forall x\in A,\; x\nin F_p\implies x\leq y) \\
        &\iff (\forall x\in A,\; x\leq p\implies x\leq y) \\
        &\iff p\leq y.
    \end{align*}
    Se sigue que $p_{F_p}=p$.
    Ahora sea $F\in\Cp(A)$ y consideremos
    $p_F=\Sup\{x\in A\mid x\nin F\}$.
    Arriba ya mostramos que, para todo $x\in A$, tenemos
    \[
        x\leq p \iff x\nin F_p
    ,\]
    de modo que $F_{p_F}=\{x\in A\mid x\nleq p\}=F$.
    Luego, las funciones forman una biyección.
    
    Finalmente, notemos que, dados $p,q\in\pt A$, tenemos
    \[
        p\leq q
        \implies
        F_q=\{x\in A\mid x\nleq q\}
        \subseteq
        \{x\in A\mid x\nleq p\}=F_p,
    \]
    por lo cual obtenemos un isomorfismo de copos
    \[
        (\pt A,\sqsubseteq)=(\pt A,\leq)^\op \simeq \Cp(A)
    .\]
    \textbf{Segunda parte.}
    Veamos que las otras dos funciones
    \begin{align*}
        \Cp(A) &\rightleftarrows \Frm(A,2) \\
        F &\mapsto \chi_F \\
        \chi^{-1}(1) = F_\chi &\mapsfrom \chi
    \end{align*}
    están bien definidas.
    \begin{itemize}
        \item
        Si $F$ es un filtro completamente primo de un marco $A$, entonces
        la función característica $\chi_F:A\to 2$ de $F$, dada como
        \[
            \chi_F(a)=
            \begin{cases}
                1 & a\in F \\
                0 & a\nin F,
            \end{cases}
        \]
        es un morfismo de marcos y, por lo tanto, es un caracter de $A$.
        
        En efecto, dados $x,y\in A$, tenemos
        \begin{align*}
            \chi_F(x\inf y) = 1
            &\iff x\inf y\in F \\
            &\iff x\in F, y\in F & \text{ya que $F$ es filtro} \\
            &\iff \chi_F(x)=1, \chi_F(y)=1 \\
            &\iff \chi_F(x)\inf\chi_F(y)=1.
        \end{align*}
        Como $\chi_F$ toma valores en $2$, esto muestra que
        $\chi_F$ preserva ínfimos de dos elementos.
        Además, $1\in F$, por lo cual $\chi_F(1)=1$.
        
        Más aún, dado $X\subseteq A$, tenemos
        \begin{align*}
            \chi_F(\Sup X) = 1
            &\iff \Sup X\in F \\
            &\iff \exists x\in X, x\in F
                & \text{ya que $F$ es comp. primo} \\
            &\iff \exists x\in X, \chi_F(x) = 1 \\
            &\iff \Sup\{\chi_F(x)\mid x\in X\} = 1
                & \text{ya que $\chi_F$ es $2$-valuado.}
        \end{align*}
        Luego, $\chi_F$ preserva supremos.
        Esto prueba que $\chi_F\in\Frm(A,2)$.
        \item
        Tomemos $\chi\in\Frm(A,2)$ y sea $F_\chi=\chi^{-1}(1)$.
        $F_\chi$ es no vacío, pues $1\in F_\chi$.
        $F_\chi$ es cerrado hacia arriba: si $a\geq b\in F_\chi$,
        entonces $\chi(a)\geq\chi(b)=1$, por lo cual $a\in F_\chi$.
        Ahora, dados $a,b\in F_\chi$ tenemos
        $\chi(a\inf b)=\chi(a)\inf\chi(b)=1\inf 1=1$, así que
        $a\inf b\in F_\chi$.
        Se sigue que $F_\chi$ es un ideal.
        
        Además $F_\chi$ es propio, pues $0\nin F_\chi$.
        Finalmente, si $X\subseteq A$ es tal que
        $\Sup X\in F_\chi$,
        tenemos $1=\chi(\Sup X)=\Sup\{\chi(x)\mid x\in X\}$,
        lo cual solo puede suceder si hay $x\in X$ con $\chi(x)=1$,
        así que $x\in X\cap F_\chi$.
        Por lo tanto, $F_\chi$ es completamente primo.
    \end{itemize}
    Ahora veamos que estas funciones son inversas mutuas.
    Si $\chi:A\to 2$, consideremos $F_\chi=\chi^{-1}(1)$.
    Tenemos
    \begin{align*}
        \chi_{F_\chi}(x) = 1
        &\iff x\in F_\chi \\
        &\iff \chi(x),
    \end{align*}
    por lo cual $\chi_{F_\chi}=\chi$.
    Dado un filtro completamente primo $F$, consideremos
    $\chi_F$.
    Entonces
    \begin{align*}
        F_{\chi_F}
        &= \chi_F^{-1}(1) \\
        &= \{x\in A\mid \chi_F(x)=1 \} \\
        &= \{x\in A\mid x\in F\} \\
        &= F.
    \end{align*}
    Luego, las funciones son inversas mutuas.
    Además, si $\chi\leq\xi$ son caracteres de $A$,
    tenemos que
    \begin{align*}
        F_\chi
        &= \{x\in A\mid \chi(x)=1\} \\
        &\subseteq \{x\in A\mid \xi(x)=1\} \\
        &= F_\xi,
    \end{align*}
    de modo que tenemos un isomorfismo de copos
    \[
        \Cp(A) \simeq \Frm(A,2).
    \]
    Esto concluye la demostración.
\end{proof}

Recordemos que los elementos de la topología en $\pt A$
se definen como los de la forma
\[
    U_A(a) = \{p\in\pt A\mid a\nleq p\}.
\]
El teorema anterior nos dice que, en vez de tomar
al espacio de puntos $\pt A$ formado por los elementos
$\inf$-irreducibles de $A$, a un marco le podemos asignar
su espacio de puntos formado por sus filtros completamente primos
\[
    \pt^{\cal F} A = \Cp A
\]
o su espacio de puntos formado por sus caracteres
\[
    \pt^\chi A = \Frm(A,2).
\]
cuyas topologías están dadas por la imagen de $\cal O\pt A$
bajo las biyecciones.\\
Explícitamente, las topologías en $\pt^{\cal F}A$
y en $\pt^\chi A$ están dadas por los conjuntos de la forma
\[
    U^{\cal F}_A(a) = \{F\in\Cp(A) \mid a\in F\}
\]
y
\[
    U^\chi_A(a) = \{\chi\in\Frm(A,2)\mid\chi(a)=1\},
\]
respectivamente.

En particular, dado un espacio topológico $S$, podemos formar
los tres espacios homeomorfos
$\pt\cal OS$, $\pt^{\cal F}\cal OS$ y $\pt^\chi \cal OS$,
y cada uno viene equipado con las reflexiones sobrias
que se obtienen al componer con los homeomorfismos.
Explícitamente, tenemos
\begin{align*}
    \Phi_S:S&\to \pt\cal OS \\
    x&\mapsto \ol{x}' \\
    \Phi^{\cal F}_S:S&\to\pt^{\cal F}\cal OS \\
    x&\mapsto \{u\in\cal OS \mid x\in u\} = \cal U(x) \\
    \Phi^\chi_S:S&\to\pt^\chi\cal OS \\
    x&\mapsto \chi_{\cal U(x)}.
\end{align*}
Dado que estas tres funciones tienen las mismas
propiedades, podemos ver que un espacio $S$ es sobrio si,
y solo si, todo filtro completamente primo de $\cal OS$
es de la forma $\cal U(x)$ para un único $x\in X$,
lo cual sucede si, y solo si, todo morfismo de marcos
$\cal OS\to 2$ es de la forma $\chi_{\cal U(x)}$ para un único
$x\in S$.

%\section*{SESIÓN 17: 11 NOV}

\section{La dualidad entre \tps{$\Sob$}{Sob} y \tps{$\Sp$}{Sp}}

Recordemos que un marco $A$ es espacial si su reflexión espacial
\[
    U_A:A\to\cal O\pt A
\]
es un isomorfismo, y que un espacio topológico $S$ es sobrio
si su reflexión sobria
\[
    \Phi_S:S\to\pt\cal OS
\]
es biyectiva (después veremos que este es equivalente a que
$\Phi_S$ sea un homeomorfismo).

\begin{lemma}
    \leavevmode
    \begin{itemize}
        \item Para todo espacio $S$,
            el marco $\cal OS$ es espacial.
        \item Para todo marco $A$,
            el espacio $\pt A$ es sobrio.
    \end{itemize}
    En otras palabras, los funtores
    $\cal O:\Top\to\Frm$ y $\pt:\Frm\to\Top$ se factorizan
    a través de $\Sp$ y $\Sob$, respectivamente:
    \[
        \begin{tikzcd}
            \Top \ar[rr,"\cal O"] \ar[dr] && \Frm \\
            & \Sp \ar[ur]
        \end{tikzcd}
        \hspace{10mm}
        \begin{tikzcd}
            \Frm \ar[rr,"\pt"] \ar[dr] && \Top \\
            & \Sob \ar[ur]
        \end{tikzcd}
    \]
\end{lemma}
\begin{proof}
    En \ref{ssec:adjuncion} probamos que
    las identidades triangulares
    \[
        \begin{tikzcd}[row sep=15mm]
            & \cal OS \ar[d,"U_{\cal OS}"] \ar[dl,"\id_{\cal OS}"']
            \\
            \cal OS
            & \cal O\pt\cal OS \ar[l,"\cal O\Phi_S"]
        \end{tikzcd}
        \hspace{10mm}
        \begin{tikzcd}[row sep=15mm]
            & \pt A \ar[d,"\Phi_{\pt A}"] \ar[dl,"\id_{\pt A}"']
            \\
            \pt A
            & \pt \cal O\pt A \ar[l,"\pt U_A"]
        \end{tikzcd}
    \]
    siempre se satisfacen.
    Como queremos probar que $U_{\cal OS}$ es un isomorfismo
    y que $\Phi_{\pt A}$ es biyectiva,
    basta probar que las otras composiciones también son
    las identidades en $\cal OS$ y en $\pt A$, respectivamente.
    Esto es:
    \begin{align*}
        (U_{\cal OS})(\cal O\Phi_S) &= \id_{\cal OS} \\
        (\Phi_{\pt A})(\pt U_A) &= \id_{\pt A}.
    \end{align*}
    Lo primero es fácil:
    todo abierto $U\in\cal O\pt\cal OS$ es de la forma
    $U=U_{\cal OS}(u)$ para algún $u\in\cal OS$.
    Luego,
    \begin{align*}
        U_{\cal OS}(\cal O\Phi_S(U))
        &= U_{\cal OS}(\cal O\Phi_S(U_{\cal OS}(u))) \\
        &= U_{\cal OS}(u) \\
        &= U.
    \end{align*}
    
    Por otro lado, todo punto $z\in\pt\cal O\pt A$,
    es elemento de $\cal O\pt A$ y, por lo tanto,
    es de la forma $z=U_A(x)$ para algún $x\in A$.
    Poniendo $y=(\pt U_A)(z)$, tenemos
    \begin{align*}
        y
        &= (\pt U_A)(z) \\
        &= (U_A)_*(z) \\
        &= \Sup\{a\in A\mid U_A(a)\leq z\} \\
        &= \Sup\{a\in A\mid U_A(a)\leq U_A(x)\},
    \end{align*}
    de modo que
    \begin{align*}
        U_A(y)
        &= \Sup\{U_A(a)\in A\mid U_A(a)\leq U_A(x)\} \\
        &= U_A(x) \\
        &= z.
    \end{align*}
    Luego, para todo $p\in\pt A$ tenemos
    \begin{align*}
        p\in\Phi_{\pt A}((\pt U_A)(z))
        &\iff p\in \Phi_{\pt A}(y) \\
        &\iff p\in \ol{y}' \\
        &\iff y\nleq p \\
        &\iff p\in U_A(y) = z.
    \end{align*}
\end{proof}

Esta prueba también muestra que, si $\Phi_S:S\to\pt\cal OS$
es biyectiva, entonces es un homeomorfismo,
pues $\cal O\Phi_S:\cal O\pt\cal OS\to\cal OS$
siempre es un isomorfismo de marcos.
Luego, un espacio $S$ es sobrio si, y solo si,
$\Phi_S:S\to\pt\cal OS$ es un isomorfismo.

Una consecuencia inmediata es que las composiciones
\begin{align*}
    \Sob \to \Frm \to \Sp \\
    \Sp \to \Top \to \Sob
\end{align*}
son esencialmente suprayectivas.
En efecto: para cualquier marco espacial $A$ tenemos
$A\simeq\cal O\pt A$ con $\pt A$ sobrio, mientras que,
para cualquier espacio sobrio $S$, tenemos $S\simeq\pt\cal OS$
con $\cal OS$ espacial.

Además, estos funtores son fielmente plenos, pues dados marcos
espaciales $A,B$ y espacios $S,T$ sobrios, tenemos
\begin{align*}
    \Sp(A,B)
    &= \Frm(A,B) \\
    &\simeq \Frm(A,\cal O\pt B) \\
    &\simeq \Frm(\pt B,\pt A) \\
    &\simeq \Sob(\pt B,\pt A),
    \\
    \Sob(S,T)
    &= \Top(S,T) \\
    &\simeq \Top(S,\pt\cal OT) \\
    &\simeq \Frm(\cal OT,\cal OS) \\
    &\simeq \Sp(\cal OT,\cal OS).
\end{align*}
Por lo tanto, tenemos el resultado
\begin{theorem}
    La adjunción contravariante
    \[
        \begin{tikzcd}
            \Frm \ar[r,"\cal O",shift left]
            & \Top \ar[l,"\pt",shift left]
        \end{tikzcd}
    \]
    se restringe a una equivalencia dual entre $\Sp$ y $\Sob$.
    Esto es:
    \[
        \Sp^\op \simeq \Sob.
    \]
\end{theorem}

\section{\tps{$\Sp$}{Sp} y \tps{$\Sob$}{Sob} son reflexivas.}
Ahora veremos que las subcategorías $\Sp$ de $\Frm$ y
$\Sob$ de $\Top$ son reflexivas.
Esto es: los funtores de inclusión
\begin{align*}
    \Sp&\to\Frm \\
    \Sob&\to\Top
\end{align*}
tienen adjuntos izquierdos.
De hecho, veremos que los adjuntos son (las restricciones de)
$\cal O\pt$ y $\pt\cal O$, respectivamente.
En realidad la demostración es bastante fácil.
Usaremos la adjunción
\[
    \Frm(A,\cal OS) \simeq \Top(S,\pt A).
\]
Denotemos, por un momento, como $i:\Sp\to\Frm$ al funtor
de inclusión.
Entonces, para todo marco $A$ y todo marco espacial $B$, tenemos
$\pt\cal O\pt A\simeq\pt A$ y $\cal O\pt B\simeq B$, por lo cual
tenemos isomorfismos
\begin{align*}
    \Frm(A,iB)
    &=\Frm(A,B) \\
    &\simeq \Frm(A,\cal O\pt B) \\
    &\simeq \Top(\pt B,\pt A) \\
    &\simeq \Top(\pt B,\pt\cal O\pt A) \\
    &\simeq \Frm(\cal O\pt A,\cal O\pt B) \\
    &\simeq \Frm(\cal O\pt A,B) \\
    &\simeq \Sp(\cal O\pt A,B)
\end{align*}
naturales en $A$ y en $B$.
Se sigue que $\cal O\pt\dashv i$.

Ahora denotemos como $i:\Sob\to\Top$ al funtor de inclusión.
Para todo espacio $S$ y todo espacio sobrio $T$,
tenemos $\cal O\pt\cal OS\simeq\cal OS$ y $\pt\cal O T\simeq T$,
por lo cual tenemos isomorfismos
\begin{align*}
    \Top(S,iT)
    &= \Top(S,T) \\
    &\simeq \Top(S,\pt\cal OT) \\
    &\simeq \Frm(\cal OT,\cal OS) \\
    &\simeq \Frm(\cal OT,\cal O\pt\cal OS) \\
    &\simeq \Top(\pt\cal OS,\pt\cal OT) \\
    &\simeq \Top(\pt\cal OS,T) \\
    &\simeq \Sob(\pt\cal OS,T)
\end{align*}
naturales en $S$ y en $T$.
Se sigue que $\pt\cal O\dashv i$.

\section{Dualidad de Stone}

\begin{definition}
    Sea $A$ una retícula completa.
    Decimos que un elemento $a\in A$ es compacto si,
    para todo $S\subseteq A$ tal que $a\leq\Sup S$,
    existe un subconjunto finito $T\subseteq S$ tal que
    $a\leq\Sup T$.
    
    Al conjunto de elementos compactos
    de $A$ lo denotamos como $\cal KA$.
\end{definition}

Nótese que $\cal KA$ es cerrado bajo supremos finitos.
En efecto, sean $a,b\in\cal KA$.
Si $X\subseteq A$ es tal que $a\sup b\leq\Sup X$,
entonces $a\leq\Sup X$ y $b\leq\Sup X$, por lo cual
existen subconjuntos finitos $S,T\subseteq X$
tales que $a\leq\Sup S$ y $b\leq\Sup T$.
Luego, $a\sup b\leq\Sup\{S\cup T\}$ con $S\cup T\subseteq X$ finito.
Claramente, $0\in\cal KA$.

\begin{definition}
    Decimos que un marco $A$ es \emph{coherente} si
    \begin{itemize}
        \item Todo elemento de $A$ es supremo de elementos
        en $\cal KA$.
        \item $\cal KA$ es cerrado bajo ínfimos finitos
        y $1\in\cal KA$.
        Es decir, $\cal KA$ es una subretícula de $A$.
    \end{itemize}
    En particular, dado un marco coherente $A$, la retícula
    $\cal KA$ es distributiva.
\end{definition}

\begin{definition}
    Sea $D$ una retícula distributiva.
    Un \emph{ideal} de $D$ es un subconjunto $I\subseteq D$ tal que
    \begin{itemize}
        \item Absorbe hacia abajo (esto es, $I\in\cal LA$).
        \item Es no vacío.
        \item Es cerrado bajo supremos finitos.
    \end{itemize}
    Al conjunto de ideales de $D$ lo denotamos como $\cal ID$.
\end{definition}

\begin{lemma}
    Sea $D$ una retícula distributiva.
    El conjunto $\cal ID$ de ideales de $D$ es un marco,
    con el orden de contención (el heredado del conjunto potencia
    $\cal PD$).
    
    Nótese que $\cal ID$ no es un submarco de $\cal PD$
    ni de $\cal LD$, pues $\emptyset\nin\cal ID$.
\end{lemma}
\begin{proof}
    Nótese que $D$ y $\{0\}$ son el mayor y menor elemento
    de $\cal ID$, respectivamente.
    
    Sea $X\subseteq\cal ID$ y notemos que $\bigcap X$
    es una sección inferior no vacía.
    Además, dados $a,b\in \bigcap X$, tenemos
    $a,b\in I$ para todo $I\in X$.
    Como cada $I$ es ideal, se sigue que
    $a\inf b\in I$ para todo $I\in X$; esto es:
    $a\inf b\in\bigcap X$.
    Luego, $\bigcap X$ es ideal de $D$ y, por lo tanto,
    es el ínfimo de $X$ en $\cal ID$
    \[
        \Inf X = \bigcap X,
    \]
    Es decir, $\cal ID$ es una sub-$\Inf$-semirretícula
    de $\cal PD$ (y de $\cal LD$).
    En particular, $\cal ID$ es una retícula completa.
    Sin embargo, los supremos en $\cal ID$ no se calculan
    como en $\cal PD$.
    
    Como $\cal ID$ es un conjunto $\Inf$-cerrado de $\cal PD$,
    le corresponde un operador cerradura $j:\cal PD\to\cal PD$.
    Afirmamos que $j$ está dado como
    \[
        j(S) = \{\Sup T\mid T\subseteq\down S\text{ finito}\},
    \]
    donde $\down:\cal PD\to\cal PD$ es la sección inferior
    generada por $S$ (el operador
    cerradura que le corresponde a $\cal LD$).
    En efecto, si $I$ es un ideal que contiene a $S$,
    entonces $\down S\subseteq I$ (porque $I$ es sección inferior)
    así que, para todo $T\subseteq\down S$ finito tenemos
    $\Sup T\in I$.
    Además, $\{\Sup T\mid T\subseteq\down S\text{ finito}\}$
    es ideal, ya que $D$ es distributiva.
    En particular, para un ideal $I$ y una sección inferior $L$,
    tenemos
    \begin{align*}
        I\cap j(L)
        &= I\cap\{\Sup T\mid T\subseteq\down L\text{ finito}\} \\
        &= I\cap\{\Sup T\mid T\subseteq L\text{ finito}\} \\
        &= \{\Sup T\mid\Sup T\in I,T\subseteq L\text{ finito}\}
        \\
        &= \{\Sup T\mid T\subseteq I,T\subseteq L\text{ finito}\}
            & \text{(}\Sup T\in I \iff T\subseteq I\text{)} \\
        &= \{\Sup T\mid T\subseteq(I\cap L)\text{ finito}\} \\
        &= j(I\cap L).
    \end{align*}
    Ahora recordemos que los supremos en $\cal ID$ se calculan como
    \[
        \Sup X = j(\bigcup X)
    \]
    para $X\subseteq\cal ID$.
    Luego, dado $I\in\cal ID$, tenemos
    \begin{align*}
        I\inf\Sup X
        &= I\cap j(\bigcup X) \\
        &= j(I\cap\bigcup X)
            &\text{pues $\bigcup X$ es sección inferior} \\
        &= j(\bigcup\{I\cap J\mid J\in X\}) \\
        &= \Sup\{I\inf J\mid J\in X\}.
    \end{align*}
    Se sigue que $\cal ID$ es marco.
\end{proof}

\begin{theorem}
    \begin{itemize}
        \item Si $A$ es un marco coherente, entonces
        \[
            A\simeq \cal I\cal KA
        .\]
        (Recordemos que $\cal KA$ es una retícula distributiva).
        \item Si $D$ es una retícula distributiva, entonces
        \[
            D \simeq \cal K\cal ID
        ,\]
        donde $\cal ID$ es un marco coherente.
    \end{itemize}
    Nótese: este teorema implica que los marcos coherentes
    son exactamente los marcos de ideales de las retículas
    distributivas.
\end{theorem}
\begin{proof}
    \begin{itemize}
        \item
        Sea $A$ un marco coherente.
        Entonces la función $\iota_A:A\to\cal I\cal KA$ dada como
        \[
            \iota_A(a) = \{c\in\cal KA\mid c\leq a\}
            =\down a\cap\cal KA
        \]
        es monótona y, para cualesquiera $a\in A, I\in\cal I\cal KA$,
        tenemos
        \begin{align*}
            \Sup\iota_A(a)
            &= \Sup\{c\in\cal KA\mid c\leq a\} \\
            &= a & \text{porque $A$ es coherente} \\
            \iota_A(\Sup I)
            &= \{c\in\cal KA\mid c\leq \Sup I\} \\
            &= I.
        \end{align*}
        Se sigue que $\iota_A:A\to\cal I\cal KA$ tiene inversa
        dada por
        \[
            \iota_A^{-1}(I) = \Sup I
        .\]
        
        \item
        Ahora sea $D$ una retícula distributiva.
        Por un lado, notemos que toda sección inferior
        principal $\down d$ es un ideal.
        Afirmamos que $\down d$ es compacto:
        \begin{quote}
            Supongamos que $X$ es una familia de ideales de $D$
            con $\down d\leq \Sup X$; es decir: $d\in j(\bigcup X)$,
            de modo que tenemos $d=\Sup\{d_1,\dots,d_n\}$,
            donde $d_i\in I_i\in X$ para $i=1,\dots,n$.
            Luego, $d\in j(\bigcup_{i=1}^n I_i)$,
            por lo cual $\down d\leq\Sup\{I_i\mid i=1,\dots,n\}$.
            Esto muestra la compacidad de $\down d$.
        \end{quote}
        Recíprocamente, todo ideal compacto $I$ es principal:
        \begin{quote}
            Observemos que $I\leq\Sup\{\down d\mid d\in I\}$,
            lo cual implica que existen $d_1,\dots,d_n\in I$
            tales que $I\leq\Sup\{\down d_1,\dots,\down d_n\}$,
            pero
            \[
                I
                \leq \Sup\{\down d_1,\dots,\down d_n\}
                = \down(d_1\sup\cdots\sup d_n),
            \]
            con $d_1\sup\dots\sup d_n\in I$, pues $I$ es ideal.
            Luego, $I=\down(d_1\sup\cdots\sup d_n)$,
            de modo que $I$ es principal.
       \end{quote} 
        Luego, la función $\kappa_D:D\to\cal K\cal ID$ dada como
        \[
            \kappa_D(d) = \down d
        \]
        es un isomorfismo.
        Nótese que, como todo ideal compacto $I$ es principal,
        el elemento $d\in D$ que lo genera es su mayor elemento,
        por lo cual $I$ tiene supremo $\Sup I=d$.
        
        Solo falta verificar que $\cal ID$ es coherente,
        pero esto es sencillo:
        Primero, todo ideal $I\in\cal ID$ es supremo
        de ideales coherentes:
        \[
            I = \Sup\{\down d\mid d\in I\}.
        \]
        Además, el ideal total
        \[
            1_{\cal ID}=D=\down 1_D
        \]
        es compacto y, dados ideales compactos
        $\down d,\down e\in\cal K\cal ID$, tenemos que su ínfimo
        $\down d\inf\down e=\down(d\inf e)$ es compacto.
    \end{itemize}
\end{proof}

\begin{definition}
    Sean $A$ y $B$ marcos coherentes.
    Decimos que un morfismo de marcos $f:A\to B$ es coherente
    si $f(\cal KA)\subseteq\cal KB$.
    
    Los marcos coherentes, junto con los morfismos coherentes,
    forman una categoría, a la cual denotamos como $\CohFrm$.
    Nótese que tenemos un funtor de inclusión $\CohFrm\to\Frm$.
\end{definition}

Por definición, un morfismo de marcos coherentes $f:A\to B$
se restringe a un morfismo de retículas (distributivas)
$\cal Kf:\cal KA\to\cal KB$.
Además, dado un morfismo de retículas distributivas $g:D\to E$,
definimos $\cal Ig:\cal ID\to\cal IE$ como
\[
    \cal Ig(I)
    = \Sup\{\down g(d)\mid d\in I\},
\]
lo cual respeta supremos (ya que $I=\Sup\{\down d\mid d\in I\}$)
y respeta ínfimos, pues
\begin{align*}
    \cal Ig(I)\inf\cal Ig(J)
    &= \Sup\{\down g(d)\mid d\in I\}
        \inf \Sup\{\down g(e)\mid e\in J\} \\
    &= \Sup\{\down g(d\inf e)\mid d\in I, e\in J\} \\
    &\leq \Sup\{\down g(x)\mid x\in I\cap J\}.
\end{align*}
(La otra comparación se sigue de la monotonía).

También es claro que las asignaciones $f\mapsto\cal Kf$
y $g\mapsto\cal Ig$ respetan la identidad y la composición
de morfismos, por lo cual obtenemos funtores
$\cal I:\DLat\to\CohFrm$ y $\cal K:\CohFrm\to\DLat$.

\begin{theorem}
    Los funtores
    \[
        \begin{tikzcd}
            \DLat \ar[r,"\cal I",shift left] &
            \CohFrm \ar[l,"\cal K",shift left]
        \end{tikzcd}
    \]
    constituyen una equivalencia.
\end{theorem}
\begin{proof}
    Dado que, para toda retícula distributiva $D$
    y todo marco coherente $A$,  $\kappa_D$ y $\iota_A$
    son isomorfismos, basta demostrar que las familias
    de morfismos
    \begin{align*}
        \kappa_\bullet &= (\kappa_D\mid D\in\DLat) \\
        \iota_\bullet &= (\iota_A\mid A\in\CohFrm)
    \end{align*}
    son transformaciones naturales
    \begin{align*}
        \kappa_\bullet:\id_\DLat &\to\cal K\cal I \\
        \iota_\bullet:\id_\CohFrm &\to\cal I\cal K.
    \end{align*}
    
    Dado un morfismo de retículas distributivas
    $g:D\to E$, debemos verificar la conmutatividad de
    \[
        \begin{tikzcd}
            \cal K\cal ID \ar[d,"\cal K\cal I g"']
                & D \ar[l,"\kappa_D"'] \ar[d,"g"] \\
            \cal K\cal IE
                & E \ar[l,"\kappa_E"]
        \end{tikzcd}
    \]
    En efecto, tenemos
    \begin{align*}
        \cal K\cal Ig(\kappa_D(d))
        &= \cal Ig(\down d) \\
        &= \Sup\{\down g(x)\mid x\in\down d\} \\
        &= \Sup\{\down g(x)\mid x\leq d\} \\
        &= \down g(d) \\
        &= \kappa_E(g(d)).
    \end{align*}
    Esto muestra la naturalidad de $\kappa_\bullet$.
    
    Por otro lado, dado un morfismo de marcos coherentes
    $f:A\to B$ debemos verificar la conmutatividad de
    \[
        \begin{tikzcd}
            \cal I\cal KA
                \ar[d,"\cal I\cal Kf"']
            & A \ar[d,"f"]
                \ar[l,"\iota_A"'] \\
            \cal I\cal KB & B \ar[l,"\iota_B"] 
        \end{tikzcd}
    \]
    Primero notemos que, para $c\in A$ compacto, tenemos
    \begin{align*}
        \cal I\cal Kf(\iota_A(c))
        &= \Sup\{\down f(x)\cap\cal KB\mid x\in\iota_A(c)\} \\
        &= \Sup\{\down f(x)\cap\cal KB\mid x\leq c,x\in\cal KA\} \\
        &= \down f(c)\cap\cal KB \\
        &= \iota_B(f(c)).
    \end{align*}
    Como todo elemento de $A$ es supremo de compactos,
    podemos tomar $a\in A$ y encontrar $X\subseteq A$
    con $a=\Sup X$.
    Luego, como $\cal I\cal Kf$, $\iota_A$, $\iota_B$ y $f$
    son morfismos de marcos, tenemos
    \begin{align*}
        \cal I\cal Kf(\iota_A(a))
        &= \cal I\cal Kf(\iota_A(\Sup X)) \\
        &= \Sup\{\cal I\cal Kf(\iota_A(c))\mid c\in X\} \\
        &= \Sup\{\iota_B(f(c))\mid c\in X\} \\
        &= \iota_B(f(\Sup X)) \\
        &= \iota_B(f(a)).
    \end{align*}
    Esto muestra la naturalidad de $\iota_\bullet$.
\end{proof}

El siguiente lema, junto con el lema de Zorn,
nos servirán para mostrar
que todo marco coherente es espacial.

\begin{lemma}
    Sean $A$ un marco y $d\in A$.
    Denotemos como $Z(d)$ al conjunto de los ideales de $A$
    que no contienen a $d$.
    Entonces
    \begin{enumerate}
        \item el copo $Z(d)$ es cerrado bajo uniones de familias
        dirigidas,
        \item cada elemento máximo de $Z(d)$ es un ideal primo de $A$ y,
        \item si $d$ es compacto, entonces cada máximo de $Z(d)$
        es principal.
    \end{enumerate}
\end{lemma}
\begin{proof}
    \begin{enumerate}
        \item Basta observar que $\cal IA$ es cerrado
        bajo uniones dirigidas.
        \item Sea $M$ un elemento máximo de $Z(d)$.
        Notemos que $M\neq A$, pues $d\nin M$.
        Para mostrar que es primo, veremos que $x,y\nin M$ implica
        $x\inf y\nin M$.
        En efecto, si $x,y\not\in M$,
        entonces tenemos
        \begin{align*}
            M&<M\sup \down x=j(M\cup\{x\}), \\
            M&<M\sup\down y = j(M\cup\{y\}),
        \end{align*}
        donde $j(S)$ es el ideal de $A$ generado por $S\subseteq A$.
        Por la maximalidad de $M$, tenemos que $d\in j(M\cup\{x\})$
        y $d\in j(M\cup\{y\})$.
        Luego, $d\leq m_1\sup x$ y $d\leq m_2\sup y$, para algunos
        $m_1,m_2\in M$, por la construcción de $j$.
        Poniendo $m=m_1\sup m_2$, tenemos $m\in M$, por lo cual
        \begin{align*}
            d   &\leq (m_1\sup x)\inf(m_2\sup y) \\
                &\leq (m\sup x)\inf(m\sup y) \\
                &=m\sup (x\inf y)
                \in j(M\cup\{x\inf y\})
                = M\sup\down(x\inf y).
        \end{align*}
        Luego, $x\inf y\nin M$ pues,
        de otro modo, tendríamos $d\in M\sup\down(x\inf y)=M$,
        lo cual no sucede porque $M\in Z(d)$.
        
        \item 
        Supongamos que $d\in A$ es compacto y sea $M$ un elemento
        máximo de $Z(d)$.
        Si $p=\Sup M$, entonces tenemos $M\subseteq\down p$,
        por lo cual basta mostrar que $p\in M$.
        Si fuera el caso que $p\nin M$,
        entonces $M$ estaría propiamente contenido
        en el ideal $\down p$.
        Por la maximalidad de $M$, tendríamos $d\leq p=\Sup M$
        y la compacidad de $d$ implicaría que $d\in M$.
    \end{enumerate}
\end{proof}

Nótese que, si $d\in A$ es un elemento compacto,
entonces por el lema de Zorn y el punto 1,
todo elemento de $Z(d)$ está contenido en un elemento máximo de
$Z(d)$, el cual es un ideal primo principal de $A$,
por los puntos 2 y 3.


\begin{lemma}
    Todo marco coherente es espacial.
\end{lemma}
\begin{proof}
    Sea $A$ un marco coherente.
    Debemos ver que $U_A:A\to\cal O\pt A$ es inyectiva.
    Tomemos elementos $a,b\in A$ distintos (digamos $a\nleq b$).
    Entonces mostraremos que $U_A(a)\neq U_A(b)$
    probando que existe $p\in U_A(a)$ con $p\nin U_A(b)$;
    es decir: un elemento irreducible $p\in A$
    tal que $a\nleq p$ y $b\leq p$.
    
    Dado que $A$ es coherente, existe un compacto $d\in\cal KA$
    tal que $d\leq a$ y $d\nleq b$
    (de otro modo, tendríamos $a\leq b$, por la coherencia de $A$).
    Por el lema anterior, podemos tomar un ideal primo
    $I=\down p$ que es máximo en $Z(d)$
    y contiene a $\down b\in Z(d)$.
    Notemos que
    \begin{itemize}
        \item $p\in\pt A$, pues $I$ es primo,
        \item $a\nleq p$ pues, de otro modo,
        tendríamos $d\leq a\leq p$,
        \item $b\leq p$, pues $I$ contiene a $\down b$.
    \end{itemize}
    Luego, $p$ es el elemento primo que buscábamos.
\end{proof}

\begin{definition}
    Sea $S$ un espacio sobrio.
    Decimos que $S$ es \emph{coherente}
    si $\cal OS$ es un marco coherente.
    
    Sea $f:S\to T$ una función continua entre espacios coherentes.
    Decimos que $f$ es coherente si $\cal Of:\cal OT\to\cal OS$
    es un morfismo coherente.
    Es decir, si todo abierto compacto de $T$ tiene preimagen
    compacta en $S$.
    
    Los espacios coherentes con sus funciones continuas
    coherentes forman una categoría llamada $\CohTop$.
\end{definition}

De la definición anterior,
es claro que la equivalencia $\Sob\simeq(\Sp)^\op$
se restringe a una equivalencia $\CohTop\simeq(\CohFrm)^\op$.
(Nota: en algunos lugares de la literatura, a los espacios
coherentes se les llama espacios espectrales).

\begin{theorem}[Dualidad de Stone generalizada]
    La categoría de espacios coherentes es dual
    a la categoría de retículas distributivas:
    \[
        \CohTop \simeq (\DLat)^\op
    .\]
\end{theorem}
\begin{proof}
    Se sigue de lo anterior:
    \[
        \CohTop \simeq (\CohFrm)^\op \simeq (\DLat)^\op
    .\]
\end{proof}

Un caso especial del teorema anterior es
la dualidad de Stone clásica.

\begin{definition}
    Decimos que un espacio topológico es cero dimensional
    si existe una base de la topología que consiste en
    conjuntos cerrados-abiertos.
    
    Un espacio cero dimensional, compacto y Hausdorff es llamado
    un espacio de Stone.
    
    Los espacios de Stone y las funciones continuas forman
    una categoría que denotamos como $\Stone$.
\end{definition}

Se puede probar que un espacio topológico es de Stone
si, y solo si, es Hausdorff y coherente,
y que todas las funciones continuas entre espacios de Stone
son coherentes, de modo que la dualidad anterior
se restringe como sigue.

\begin{theorem}[Dualidad de Stone]
    \[
        \Stone \simeq (\Bool)^\op.
    \]
\end{theorem}

%\section*{Fin de la sesión 23 (2 de diciembre)}
