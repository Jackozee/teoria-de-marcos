%\section*{(SESIÓN 9: 7 OCT)}
\chapter{Marcos en teoría de anillos y módulos}
Esto es una prueba jeje
Ahora trataremos de dar una motivación algebraica para
estudiar los núcleos de un marco.
En realidad, el estudio de los núcleos se puede realizar
en un contexto más general que el de marcos.
A continuación damos un ejemplo de cómo se pueden usar
los núcleos para estudiar anillos y módulos.

\paragraph{Localizaciones}
Recordemos que, en teoría de anillos,
podemos localizar un anillo $A$ con respecto a un conjunto
multiplicativo $S\subseteq A$.
Por ejemplo, tomando $A=\Z$ y $S=\Z-\{0\}$,
obtenemos la localización $i:\Z\to\Q$.

El morfismo $i\colon\Z\to\Q$ induce dos funtores
\[
    \Mod_\Z\leftrightarrows\Mod_\Q
\]
llamados restricción y extensión de escalares.
Las categorías, al igual que los anillos,
tienen su propio concepto de localización.
La definición (que no haremos aquí) es tal que
los funtores $\Mod_\Z\leftrightarrows\Mod_\Q$
exhiben que $\Mod_\Q$ es una localización de $\Mod_\Z$.

\paragraph{Teorías de torsión}
La clase de los $\Z$-módulos (es decir, grupos abelianos)
contiene las clases $\scr T_t$ y $\scr F_t$
formadas por los $\Z$-módulos de torsión y los $\Z$-módulos
libres de torsión, respectivamente.
Notemos que $\scr T_t$ y $\scr F_t$ satisfacen:
\begin{itemize}
    \item
    $\Mod_\Z(M,N)=0$ para cualesquiera $M\in\scr T_t$
    y $N\in\scr F_t$.
    \item
    Si $\Mod_\Z(P,N)=0$ para todo $N\in\scr F_t$,
    entonces $P\in\scr T_t$.
    \item
    Si $\Mod_\Z(M,P)=0$ para todo $M\in\scr T_t$,
    entonces $P\in\scr F_t$.
\end{itemize}
Debido a que la pareja $(\scr T_t,\scr F_t)$ satisface estos
axiomas, decimos que es una teoría de torsión.
Además, como $\scr T_t$ es cerrada bajo submódulos,
decimos que $(\scr T_t,\scr F_t)$ es una teoría de torsión hereditaria.

\paragraph{Los ideales de $\Z$ y un colímite}
Consideremos el conjunto $\Lambda(\Z)$ de los ideales de $\Z$.
Como todos los ideales de $\Z$ son principales, tenemos
\[
    \Lambda(\Z)=\{\<n\>\mid n\in \Z\}
.\]
También consideraremos el conjunto $\nti\Z$
de los ideales de $\Z$ distintos de $0$.

Notemos que $\Lambda(\Z)$ es una retícula, con ínfimo $\inf=\cap$
y supremo $\sup=+$.
Además, el subconjunto $\nti\Z\subseteq\Lambda(\Z)$
satisface las siguientes propiedades:
\begin{itemize}
    \item es no vacío,
    \item cerrado bajo intersecciones y
    \item absorbe hacia arriba
    (es decir, si $I\in\nti\Z$ y $I\leq J$,
    entonces $J\in\nti\Z$).
\end{itemize}
Si un subconjunto de una retícula tiene estas propiedades,
decimos que es un filtro de la retícula.
En otras palabras, $\nti\Z$ es un filtro de $\Lambda(\Z)$.

Cada ideal $I$ de $\Z$ es, en particular, un $\Z$-módulo,
por lo cual podemos considerar el $\Z$-módulo $\Mod_\Z(I,\Z)$.
Además, dados ideales $I\leq J$ de $\Z$,
tenemos un morfismo de $\Z$-módulos
\[
    \phi^I_J:\Mod_\Z(J,\Z)\to\Mod_\Z(I,\Z)
\]
dado por la restricción: $\phi^I_J(f)=f|_I$.

Notemos que los morfismos $(\phi^I_J\mid I\leq J\in\Lambda(\Z))$
satisfacen
\begin{itemize}
    \item para todo $I$, tenemos $\phi^I_I=\id_{\Mod_\Z(I,\Z)}$ y
    \item siempre que $I\leq J\leq K$, tenemos
    $\phi^J_I\phi^K_J=\phi^K_I$. Esto es: $f|_J|_I=f|_I$.
\end{itemize}

En otras palabras: viendo al copo $\nti\Z$
como una categoría, tenemos un funtor
\[
    \Mod_\Z(-,\Z):(\nti\Z)^\op\to\Mod_\Z
.\]

En general, dado un conjunto dirigido $\scr I$
y una categoría $\cal C$,
un sistema dirigido en $\cal C$ indicado por $\scr I$
es un funtor $F:\scr I\to\cal C$.
Como $\nti\Z$ es codirigido (ya que es cerrado bajo ínfimos),
se sigue que $(\nti\Z)^\op$ es dirigido.
Es decir: el funtor
\[
    \Mod_\Z(-,\Z):(\nti\Z)^\op\to\Mod_\Z
\]
es un sistema dirigido en $\cal C=\Mod_\Z$,
indicado por $\scr I = (\nti\Z)^\op$.

Sean $\scr I$ un conjunto dirigido
y $F:\scr I\to\cal C$ un sistema dirigido.
Denotemos como $\sqsubseteq$ al orden en $\scr I$ y,
para cada morfismo $\alpha:i\to j$ en $\scr I$
(es decir, $i\sqsubseteq j$) usaremos la notación
$\phi_i^j=F(\alpha):F(i)\to F(j)$.
El límite directo de $F$ es un objeto $\cal C$,
comúnmente denotado $\varinjlim F$ o $\varinjlim_{i\in\scr I}F(i)$,
junto con una familia de morfismos
$(\eta_i:F(i)\to\varinjlim F \mid i\in\cal I)$
que satisfacen las siguientes condiciones:
\begin{itemize}
    \item Para cada $i\sqsubseteq j\in\scr I$,
    el siguiente diagrama es conmutativo:
    \[
        \begin{tikzcd}
            F(i) \ar[rr,"\phi_i^j"] \ar[dr,"\eta_i"']
            && F(j) \ar[dl,"\eta_j"] \\
                & \varinjlim F
        \end{tikzcd}
    \]
    Es decir, $\eta_j\phi_i^j=\eta_i$.
    \item Dado un objeto $X$ de $\cal C$
    y una familia de morfismos $(\mu_i:F(i)\to X\mid i\in\cal I)$
    tales que $\mu_j\phi_i^j=\mu_i$ (siempre que $i\sqsubseteq j$),
    entonces existe un único morfismo $\mu:\varinjlim F\to X$
    tal que el siguiente diagrama conmuta:
    \[
        \begin{tikzcd}
            F(i) \ar[rr,"\phi_i^j"] \ar[dr,"\eta_i"']
            \ar[ddr,"\mu_i"',bend right] 
            && F(j) \ar[dl,"\eta_j"] \ar[ddl,"\mu",bend left] \\
                & \varinjlim F \ar[d,"\mu"] \\
                & X
        \end{tikzcd}
    \]
    (siempre que $i\sqsubseteq j\in\scr I$).
\end{itemize}

Se puede demostrar que, para la categoría $\cal C=\Mod_\Z$,
los límites directos siempre existen
y están dados por
\[
    \varinjlim F = \Big(\bigsqcup_{i\in\scr I}F(i)\Big)/\simr
\]
donde $\simr$ es la relación de equivalencia en
$\bigsqcup_{i\in\scr I}F(i)$ definida, para $f\in F(i)$ y $g\in F(j)$,
como
\[
    f\sim g
    \ssi
    \exists k\geq i,j,\; \phi_i^k(f)=\phi_j^k(g)
\]
y donde los morfismos $\eta_i:F(i)\to\varinjlim_{i\in\scr I}F(i)$
son las proyecciones $f\mapsto [f]$ de $f\in F(i)$ a su clase
de equivalencia.

Ahora consideremos el conjunto dirigido $\scr I=(\nti\Z)^\op$
y el sistema dirigido
\begin{align*}
    \Mod_\Z(-,\Z):(\nti\Z)^\op&\to\Mod_\Z \\
    I&\mapsto \Mod_\Z(I,\Z).
\end{align*}
Para cada ideal $\<n\>\in\nti\Z$, definamos el morfismo
$\mu_n:\Mod_\Z(\<n\>,\Z)\to\Q$ dado por $\mu_n(f)=f(n)/n$.

Entonces, si $\<n\>\sqsubseteq \<m\>$ (es decir, $\<m\>\leq\<n\>$),
el diagrama
\[
    \begin{tikzcd}
      \Mod_\Z(\<n\>,\Z) \ar[rr,"{\phi_n^m}"] \ar[dr,"\mu_n"']
      && \Mod_\Z(\<m\>,\Z) \ar[dl,"\mu_m"] \\
      & \Q
    \end{tikzcd}
\]
es conmutativo.

\begin{exe}%[Alfredo $\checkmark$ ]
  Mostrar que el morfismo inducido en el límite directo
  \begin{align*}
      \mu : \varinjlim_{\<n\>\in\nti\Z}\Mod_\Z(\<n\>,\Z)
        &\to \Q \\
      [f] &\mapsto \frac{f(n)}{n}
  \end{align*}
  es un isomorfismo.
\end{exe}
\begin{sol}
    Primero veamos que $\mu$ está bien definida.
    Tomemos morfismos relacionados
    $f\sim g\in\bigsqcup_{\<n\>\in\nti\Z}\Mod_\Z(\<n\>,\Z)$.
    Es decir, $f:\<m\>\to\Z$ y $g:\<n\>\to\Z$ satisfacen
    \[
        f|_{\<k\>} = g|_{\<k\>}
    \]
    para alguna $k\in\Z$ con $m\mid k$, $n\mid k$.
    Es decir, $k=rm$, $k=sn$ para algunos $r,s\in\Z$.
    Luego, tenemos
    \begin{align*}
        \mu([g])
        &= \frac {g(n)} n
        = \frac {sg(n)}{sn} 
        = \frac {g(sn)}{sn} 
        = \frac {g(k)}{k} \\
        &= \frac {f(k)}{k} 
        = \frac {f(rm)}{rm} 
        = \frac {rf(m)}{rm} 
        = \frac {f(m)}{m} 
        = \mu([f]).
    \end{align*}
    Se sigue que $\mu$ está bien definida
    en las clases de equivalencia;
    es decir: no depende del representante.

    Ahora, dado $r/m\in\Q$, consideremos la función
    $f:\<m\>\to\Z$ dada por $am\mapsto ar$.
    En efecto, $f$ respeta sumas y productos por enteros,
    así que podemos considerar $[f]$.
    
    Queremos ver que la asignación $r/m\mapsto [f]$ no
    depende de $r$ y de $m$, sino solo del cociente $r/m$.
    
    Para esto, tomamos $s/n=r/m$ y
    debemos mostrar que $g:\<n\>\to\Z$
    dada como $an\mapsto as$ cumple $[f]=[g]$.
    Como $s/n=r/m$, entonces $sm=rn$, así que $mn$ cumple
    \begin{align*}
        f|_{\<mn\>}(amn)
        &= f(amn)
        = arn
        = asm
        = g(anm)
        = g|_{\<mn\>}(amn),
    \end{align*}
    por lo cual $f|_{\<mn\>}=g|_{\<mn\>}$ y $[f]=[g]$.
    
    Luego, la asignación $r/m\mapsto [(am\mapsto ar)]$
    es una función
    \[
        \xi:\Q\to\varinjlim_{\<n\>\in\nti\Z}\Mod_\Z(\<n\>,\Z)
    .\]
    Notemos que $\xi$ es $\Z$-lineal, ya que
    \begin{align*}
        \xi\left(a\frac r m + \frac s n\right)
        &= \xi\left(\frac{arn + ms}{mn}\right) \\
        &= \left( xmn \mapsto x(arn+ms) \right) \\
        &= (xmn \mapsto xarn+xms) \\
        &= (xmn \mapsto xarn) + (xmn \mapsto xms) \\
        &= a(xmn \mapsto xrn) + (xmn \mapsto xms) \\
        &= a\xi\left(\frac{rn}{mn}\right)
            + \xi\left(\frac{ms}{mn}\right) \\
        &= a\xi\left(\frac{r}{m}\right)
            + \xi\left(\frac{s}{n}\right).
    \end{align*}
    Finalmente, observemos que
    \begin{align*}
        \xi(\mu(r/m))
        &= \xi(am\mapsto ar) \\
        &= \frac{(am\mapsto ar)(m)}{m} \\
        &= \frac{r}{m} \\
        \mu(\xi([f]))
        &= \mu\left(\frac{f(n)} n \right) \\
        &= [an\mapsto af(n)] \\
        &= [an\mapsto f(an)] \\
        &= [f]
    .\end{align*}
    Esto muestra que $\xi$ es la inversa de $\mu$,
    así que tenemos un isomorfismo
    \[
        \begin{tikzcd}
        \mathbb Q \ar[r,shift left,"\mu"] &
        \displaystyle
        \varinjlim_{\<n\>\in\nti\Z}\Mod_\Z(\<n\>,\Z),
        \ar[l,shift left,"\xi"]
       \end{tikzcd} 
    \]
    Esto es lo que se quería.
\end{sol}

Ahora, ¿qué tiene que ver esto con núcleos?

\paragraph{Filtros de Gabriel}
Recordemos que $\nti\Z$ es un filtro de $\Lambda(\Z)$.
De hecho, tiene otras propiedades interesantes:
Dado cualquier ideal $I\in\nti\Z$, tenemos
\[
    (n:I)=\{k\in\Z\mid nk\in I\}\in\nti\Z
\]
para todo $n\in\Z$.
Además, si $J\in\nti\Z$, entonces
\[
    (\forall n\in J,\; (n:I)\in\nti\Z)\implies I\in\nti\Z
.\]
A un filtro con estas dos propiedades se le llama filtro de Gabriel.

\paragraph{Núcleos deferentes}
Se puede demostrar que el operador $j:\Lambda(\Z)\to\Lambda(\Z)$
dado como
\[
    j(I) = \{n\in\Z\mid(n:I)\in\nti\Z\}
\]
es un núcleo en $\Lambda(\Z)$, en el sentido de que es monótono,
infla, es idempotente y preserva ínfimos.
Más aún, el conjunto $N(\Lambda(\Z))$ de los núcleos en $\Z$
es un marco.
Adicionalmente, el núcleo $j$ satisface
\[
    j(n:I) = (n:j(I))
.\]
A un núcleo con esta propiedad se le llama núcleo deferente.

En particular, tomando el núcleo $j$ definido arriba,
se puede demostrar que su conjunto de puntos fijos es
\[
    \Lambda(\Z)_j = \{0\leq \Z\}
,\]
el cual es isomorfo a la retícula $\Lambda(\Q)$ de ideales de $\Q$.

En términos generales, hay una correspondencia
(que no desarrollaremos aquí)
entre los núcleos deferentes de $\Lambda(\Z)$,
los filtros de Gabriel en $\Lambda(\Z)$,
las teorías de torsión hereditaria de $\Mod_\Z$
y las localizaciones de $\Mod_\Z$.

Por ejemplo, al núcleo $j$ le corresponden
el filtro $\nti\Z$, la teoría de torsión
$(\scr T_t,\scr F_t)$ y la localización
$\Mod_\Z\to\Mod_\Q$.

